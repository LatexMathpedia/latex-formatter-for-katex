\documentclass[11pt,a4paper]{article}

\usepackage[spanish]{babel}
\usepackage{amsmath,amsfonts, amssymb, mathtools, bbm} % Podemos añadir amssymb, amsthm o bm
\usepackage{graphicx, tikz, xparse}
\usepackage[top=2cm,bottom=2cm,left=3cm,right=3cm,marginparwidth=1.75cm]{geometry} % Este paquete permite modificar los márgenes del documento
\usepackage[colorlinks=true, allcolors=blue]{hyperref} % Se indica que los hipervínculos van todos en azul
\usepackage{setspace}
\usepackage{xcolor, tcolorbox}
\usepackage{cancel} %tachar cosas
\tcbuselibrary{breakable}
\usepackage{hyperref}
\usepackage{titlesec}
\usepackage{cancel}
\usepackage{pgfplots}  % Para graficar funciones en tikz
\usepackage{background}
\usetikzlibrary{arrows.meta}
\usepackage[bottom]{footmisc}
\usepackage{extarrows}
\allowdisplaybreaks

% Definir la marca de agua
\backgroundsetup{
  position=current page.west,
  angle=90,
  nodeanchor=west,
  vshift=-1cm,
  hshift=-5.5cm,
  color=gray,
  scale=1,
  contents={\textsf{Created by Diego Díaz Mendaña $|$ Licensed under CC BY-NC-SA 4.0}}
}


\graphicspath{ {images/}}

%Colores
\definecolor{blanco}{HTML}{FFFFFF}
\definecolor{negro}{HTML}{000000}
\definecolor{azulSuave}{HTML}{6ac9d5}
\definecolor{naranjaSuave}{HTML}{d5956a}
\definecolor{verdeSuave}{HTML}{6ad578}
\definecolor{magenta}{HTML}{FF00FF}
\definecolor{dorado}{HTML}{ad8a1f}

\newtcolorbox{dem_box}[1]{
before=\par\smallskip\centering,
colframe=azulSuave!70,
colback=white,
fonttitle=\bfseries,
coltitle=negro,
title=#1,
flushleft title,
width=1\linewidth,
breakable = true
}

\newtcolorbox{ejem_box}[1]{
before=\par\smallskip\centering,
colframe=verdeSuave!70,
colback=white,
fonttitle=\bfseries,
coltitle=negro,
title=#1,
flushleft title,
width=1\linewidth,
breakable = true
}

\newtcolorbox{ej_box}[1]{
before=\par\smallskip\centering,
colframe=naranjaSuave!70,
colback=white,
fonttitle=\bfseries,
coltitle=negro,
title=#1,
flushleft title,
width=1\linewidth,
breakable = true
}


\setstretch{1.2}
\decimalpoint

\title{\textbf{TEMA 3}: Espacio de Medida de Lebesgue \((\mathbb{R}^N, \mathcal{M}_N, \mu_N)\)}
\author{Diego Díaz Mendaña}
%\date{Fecha}

\begin{document}
\NoBgThispage 
\maketitle
\vspace{20ex}
\section*{Disclaimer}
Estos apuntes son un resumen basado en el material proporcionado en la asignatura de ``Análisis III'' de la Universidad de Oviedo. Han sido escritos mediante las definiciones y demostraciones explicadas en clase. Todo el contenido ha sido organizado y formulado con fines educativos y no comerciales.\vspace{2ex}


\vspace{12ex}
\section*{Licencia de uso}
Apuntes Análisis III - Tema 3 © 2025 by Diego Díaz Mendaña is licensed under CC BY-NC-SA 4.0. To view a copy of this license, visit \href{https://creativecommons.org/licenses/by-nc-sa/4.0/}{https://creativecommons.org/licenses/by-nc-sa/4.0/}

\newpage

\hypersetup{linkcolor=black}
\tableofcontents
\hypersetup{linkcolor=blue}
\newpage

\section{\texorpdfstring{Cubos y volúmenes en \(\mathbb{R}^N\)}{Cubos y volúmenes en R^N}}
\subsection{Intervalo degenerado. Definición}
\noindent Llamamos \textbf{intervalo degenerado} a cualquier intervalo vacío o unitario.\vspace{2ex}

\begin{ejem_box}{Ejemplo}
  Algunos intervalos degenerados son:
  \begin{align*}
    \{a\} = [a, a] \quad \mbox{ o } \quad \emptyset = (2, 1)
  \end{align*}
\end{ejem_box}
\vspace{3ex}

\subsection{\texorpdfstring{Cubo degenerado de \(\mathbb{R}^N\). Definición}{Cubo degenerado de R^N. Definición}}
\noindent Sea \(N \in \mathbb{N}\), llamamos \textbf{cubo degenerado de \(\mathbb{R}^N\)} a cualquier cubo:
\begin{align*}
  C = I_1 \times I_2 \times \dots \times I_N
\end{align*}
donde cada \(I_i\) es un intervalo abierto de \(\mathbb{R}\) y alguno de sus intervalos es degenerado. En general, llamamos \textbf{cubo de \(\mathbb{R}^N\)} a cualquier producto cartesiano:
\begin{align*}
  I_1 \times \dots \times I_N
\end{align*}
donde cada \(I_i\) es un intervalo.\vspace{3ex}

\subsection{Volumen de un cubo. Definición}\label{def:vol_cubo}
\noindent Dado un cubo \(C\) de \(\mathbb{R}^N\) con \(N \in \mathbb{N}\), llamamos \textbf{volumen de \(C\)} al número:\footnote{Y hasta aquí llego Riemann. Todo parecía muy bonito, era como que había elecciones en las matemáticas. Luego vino el periodo post-electoral y ya bueno...}
\begin{align*}
  v_N(C) \coloneq \left\{
    \begin{array}{ll}
      0 & \mbox{ si } C \mbox{ es degenerado} \\[2ex]
      \displaystyle \prod_{i = 1}^N |b_i - a_i| & \begin{array}{l}
        \mbox{si } C \mbox{ no es degenerado, } C = I_1 \times \dots \times I_N
        \mbox{ con } I_j \mbox{ invervalo de} \\
        \mbox{extr. izq. } a_j \mbox{ y drch. } b_j 
        \mbox{ tq } - \infty \leq a_j < b_j \leq \infty
      \end{array}
    \end{array}
  \right.\\
\end{align*}
\begin{ej_box}{Nota}
  Al final, tenemos que si algún intervalo es degenerado, el ``espesor'' del cubo en alguna dimensión será 0, por lo que su volumen \(n\)-dimensional sería también cero.\vspace{2ex}

  Para cubos no degenerados, esta idea de volumen no deja de ser la generalización de lo que trabajamos en 3 dimensiones, por ejemplo, con un paralelepípedo:
  \begin{align*}
    C = [0, 1] \times [0, 2] \times [0, 3] \implies v_N(C) = |1 - 0| \cdot |2 - 0| \cdot |3 - 0| = 6
  \end{align*}
\end{ej_box}
\vspace{3ex}

\subsection{Medida exterior. Definición}\label{def:medida_exterior}
\noindent Sea \(A \subseteq \mathbb{R}^N\) se define su \textbf{medida exterior} como el número:
\begin{align*}
  \mu_N^*(A) \coloneq \inf \left\{\displaystyle \sum_{i = 1}^{\infty} v_N(I_i) \colon \{I_i\}_{i \in \mathbb{N}}\mbox{ colección de cubos abiertos tq } A \subseteq \displaystyle \bigcup_{i \in \mathbb{N}}I_i\right\} \in [0, \infty]
\end{align*}

\begin{ej_box}{Advertencia}
  \(\mu_N^*\) no es una medida de verdad\footnote{Lo que habría suspirado Jordan porque lo fuera} ya que carece de \(\sigma\)-aditividad.
\end{ej_box}
\vspace{2ex}

\begin{ej_box}{Nota}
  La idea intuitiva es cubrir \(A\) con una colección numerable de cubos abiertos, calcular su suma de volúmenes y tomar la menor de todas las sumas posibles considerando todos los posibles recubrimientos de \(A\) con cubos abiertos.
\end{ej_box}
\vspace{3ex}

\subsubsection{\texorpdfstring{Técnicas fundamentales de \(\mu_N^*\)}{Técnicas fundamentales de muN*}}\label{sec:tecnicas_muN*}
\noindent Las dos propiedades básicas que caracterizan la medida exterior son:
\begin{itemize}
  \item \textbf{Cota superior por recubrimientos}: Partiendo de la definición:
  \begin{align*}
    \mu_N^*(A) \leq \displaystyle \sum_{i = 1}^{\infty} v_N(I_i) \qquad \forall \{I_i\}_{i \in \mathbb{N}} \mbox{ colección de cubos abiertos tq } A \subseteq \displaystyle \bigcup_{i = 1}^{\infty} I_i\\
  \end{align*}

  \item \textbf{Propiedad de aproximación (\(\varepsilon\)-cercanía):} para todo \(\varepsilon > 0\) existe \(\{I_i\}_{i \in \mathbb{N}}\) recubrimiento con cubos abiertos de \(A\) tal que:
  \begin{align*}
    \mu_N^*(A) + \varepsilon > \sum_{i = 1}^{\infty} v_N(I_i)
  \end{align*}
  Esto es, podemos aproximar la medida exterior tanto como queramos desde arriba.\vspace{2ex}

  \begin{ej_box}{Obsevación}
    \noindent Si \(\mu_N^*(A) < \infty\) esto se interpreta gráficamente como:\vspace{2ex}

  \begin{center}
    \begin{tikzpicture}
      \draw[thick] (-5, 0) -- (5, 0);
      \draw (-1, 0.2) -- (-1, -0.2);
      \draw (1, 0.2) -- (1, -0.2);
      \draw (2, 0.1) -- (2, -0.1);
      \draw (3, 0.2) -- (3, -0.2);
      
      \node[scale=0.8] at (5.3,0) {\(\infty\)};
      \node[scale=0.8] at (-1,-0.4) {0};
      \node[scale=0.8] at (1, 0.4) {\(\mu_N^*(A)\)};
      \node[scale=0.8] at (3, 0.4) {\(\mu_N^*(A) + \varepsilon\)};
      \node[scale=0.8] at (2, -0.6) {\(\displaystyle \sum_{i = 1}^{\infty} v_N(I_i)\)};
    \end{tikzpicture}
  \end{center}
  \end{ej_box}
\end{itemize}
\vspace{3ex}

\begin{ejem_box}{Ejemplo}
  \textit{Se cumple que \(\mu_N^*(\emptyset ) = 0\) ya que:}\vspace{2ex}

  Podemos recubrir \(\emptyset \) con el propio conjunto vacío:
  \begin{align*}
    \emptyset \subseteq \emptyset \cup \emptyset \cup \dots
  \end{align*}
  Como el volumen de \(\emptyset \) es 0 por \hyperref[def:vol_cubo]{definición de volumen de cubo degenerado} entonces:
  \begin{align*}
    0 \leq \mu_N^*(\emptyset ) \leq 0 + 0 + \dots = 0
  \end{align*}
  Y por el criterio del Sandwich se cumple que \(\mu_N^*(\emptyset ) = 0\)
\end{ejem_box}
\vspace{3ex}

\begin{ejem_box}{Ejemplo}
  \textit{Sea \(C\) es cubo degenerado acotado entonces \(\mu_N^*(C) = 0\) ya que:}\vspace{2ex}

  Se dan dos posibles casos:
  \begin{itemize}
    \item Si \(C = \emptyset \) entonces \(\mu_N^*(C) = 0\) (ver ejemplo anterior)
    \item Si \(C \neq \emptyset \), supongamos \(C = I_1 \times I_2 \times \dots \times I_N\) donde cada \(I_j\) es un intervalo acotado y al menos unos de ellos es unitario. Supongamos \(I_1 = \{a\}\) (la demostración es análoga si es otro). Para cualquier \( \varepsilon > 0\) si \(I_k\) es un intervalo de extremos \(a_k\) y \(b_k\) entonces:
    \begin{align*}
      C \subseteq C_\varepsilon \coloneq (a - \varepsilon, a + \varepsilon) \times (a_2 - \varepsilon, b_2 + \varepsilon) \times \dots \times (a_N - \varepsilon, b_N + \varepsilon)
    \end{align*}
    Y como esto es válido \(\forall \varepsilon > 0\) tenemos que:
    \begin{align*}
      \mu_N^*(C) & \leq v_N(C_{\varepsilon}) = 2 \varepsilon \cdot (b_2 - a_2 + 2\varepsilon) \cdot \dots \cdot (b_N - a_N + 2\varepsilon) 
    \end{align*}
    Por lo que si hacemos \(\varepsilon \to 0\) entonces:
    \begin{align*}
      0 \leq \mu_N^*(C) \leq 0 \cdot \underbracket{(b_2 - a_2)}_{\in \mathbb{R}} \cdot \dots \cdot \underbracket{(b_N - a_N)}_{\in \mathbb{R}} = 0
    \end{align*}
    Que se cumple ya que \(C\) es acotado y, por lo tanto, \(b_j - a_j < \infty\) para todo \(j\).\vspace{2ex}

    Ahora, aplicando el criterio del sandwich, tenemos que:
    \begin{align*}
      \mu_N^*(C) = 0
    \end{align*}
  \end{itemize}
\end{ejem_box}
\vspace{3ex}

\begin{ejem_box}{Ejemplo}
  \textit{Sea \(A \subseteq \mathbb{R}^N\) numerable entonces \(\mu_N^*(A) = 0\) ya que:}\vspace{2ex}

  Fijamos \(\varepsilon > 0\) cualquiera con:\footnote{Aunque vosotros ya no hacéis todo eso de la mili, aquí es como en las filas indias, el que está primero es el que se suele llevar la mejor tajada.}
  \begin{align*}
    \varepsilon = \frac{\varepsilon}{2} + \frac{\varepsilon}{4} + \frac{\varepsilon}{8} + \dots = \sum_{i = 1}^{\infty} \frac{\varepsilon}{2^i}
  \end{align*}
  Como \(A \) numerable entonces \(A = \{a_n : n \in \mathbb{N}\}\) y por cada \(a_n\) tomamos un cubo abierto \(I_n\) tal que:
  \begin{align*}
    a_n \in I_n \quad \mbox{ y } \quad v_N(I_n) = \frac{\varepsilon}{2^n}
  \end{align*}
  Entonces:
  \begin{align*}
    A \subseteq \displaystyle \bigcup_{n \in \mathbb{N}} I_n \implies 0 \leq \mu_N^*(A) \leq \displaystyle \sum_{n = 1}^{\infty} v_N(I_n) = \displaystyle \sum_{i = 1}^{\infty} \frac{\varepsilon}{2^i} = \varepsilon 
  \end{align*}
  Y como \(\varepsilon > 0\) es arbitrario, podemos hacer que \(\varepsilon \to 0\), por el criterio del sándwich:
  \begin{align*}
    \mu_N^*(A) = 0
  \end{align*}
\end{ejem_box}
\vspace{3ex}

\subsection{\texorpdfstring{Propiedades de \(\mu_N^*\). Proposición}{Propiedades de muN*. Proposición}}\label{prop:propiedades_muN*}
\noindent Sea \(N \in \mathbb{N}\) entonces \(\mu_N^*\) cumple las siguientes propiedades:
\begin{enumerate}
  \item \textbf{Monotonía}: Sea \(A \subseteq B \subseteq \mathbb{R}^N\) entonces se cumple:
  \begin{align*}
    \mu_N^*(A) \leq \mu_N^*(B)
  \end{align*}
  \item \textbf{Subaditividad numerable}\footnote{Esto era la felicidad de los matemáticos. Estaba Jordan muy contento porque creía que si exigíamos que fueran disjuntos entonces se daba la igualdad hasta que llegó ... con su conjunto y le dijo que no se cumplía. Aquí llego la depresión y todas esas cosas, ya sabéis como son los matemáticos, se ahogan en un vaso}: Sea \(\{A_i\}_{i \in \mathbb{N}} \subseteq \mathcal{P}(\mathbb{R}^N)\) entonces se cumple:
  \begin{align*}
    \mu_N^*\left(\displaystyle \bigcup_{i \in \mathbb{N}} A_i\right) \leq \displaystyle \sum_{i = 1}^{\infty} \mu_N^*(A_i)
  \end{align*}
  \item \(\forall C\) cubo compacto de \(\mathbb{R}^N\) se cumple:\footnote{Y por esto, la integral de Riemann se va por la ventana}
  \begin{align*}
    \mu_N^*(C) = v_N(C) = v_N(C^\circ) = \mu_N^*(C^\circ)
  \end{align*}
\end{enumerate}
\begin{dem_box}{Demostración}
  \begin{enumerate} \leftskip -10pt
    \item Sea \(A \subseteq B \subseteq \mathbb{R}^N\), definimos los conjuntos:
    \begin{align*}
      S_A & = \left\{\displaystyle \sum_{i = 1}^{\infty} v_N(I_i) \colon \{I_i\}_{i \in \mathbb{N}} \mbox{ cubos abiertos, } A \subseteq \displaystyle \bigcup_{i \in \mathbb{N}} I_i\right\}\\[2ex]
      S_B & = \left\{\displaystyle \sum_{i = 1}^{\infty} v_N(I_i) \colon \{I_i\}_{i \in \mathbb{N}} \mbox{ cubos abiertos, } B \subseteq \displaystyle \bigcup_{i \in \mathbb{N}} I_i\right\}
    \end{align*}
    \begin{itemize} \leftskip -12pt
      \item \textit{Ver que \(S_B \subseteq S_A\).} Sea \(s \in S_B\), existe \(\{I_i\}_{i \in \mathbb{N}}\) recubrimiento de \(B\) con cubos abiertos tal que:
      \begin{align*}
        s = \displaystyle \sum_{i = 1}^{\infty} v_N(I_i) \quad \mbox{ y } \quad B \subseteq \displaystyle \bigcup_{i = 1}^{\infty} I_i
      \end{align*}
      Como \(A \subseteq B\) entonces:
      \begin{align*}
        A \subseteq B \subseteq \displaystyle \bigcup_{i = 1}^{\infty} I_i \implies \{I_i\}_{i \in \mathbb{N}} \mbox{ recubrimiento de } A
      \end{align*}
      Por lo tanto \(s \in S_A\) y así \(S_B \subseteq S_A\).\vspace{2ex}

      \item \textit{Ver que \(\mu_N^*(A) \leq \mu_N^*(B)\).} Por \hyperref[def:medida_exterior]{definición de medida exterior} y relación de ínfimos de conjuntos:
      \begin{align*}
        \mu_N^*(A) = \inf S_A \leq \inf S_B = \mu_N^*(B)\\
      \end{align*}
    \end{itemize}

    \item Se consideran dos posibles casos:
    \begin{itemize} \leftskip -12pt
      \item Si \(\sum_{n = 1}^{\infty} \mu_N^*(A_n) = \infty\) entonces la desigualdad se cumple trivialmente.
      \item Si \(\sum_{n = 1}^{\infty} \mu_N^*(A_n) < \infty\) entonces fijamos \(\varepsilon > 0\). Para cada \(i \in \mathbb{N}\), aplicando la \hyperref[sec:tecnicas_muN*]{técnica de \(\varepsilon\)-cercanía}, existe recubrimiento \(\{I_{i,j}\}_{j \in \mathbb{N}}\) con cubos abiertos de \(A_i\) tal que:
      \begin{align*}
        \displaystyle \sum_{j = 1}^{\infty} v_N(I_{i,j}) < \mu_N^*(A_i) + \frac{\varepsilon}{2^i} \quad \mbox{ y } \quad A_i \subseteq \displaystyle \bigcup_{j = 1}^{\infty} I_{i,j}
      \end{align*}
      Como la colección \(\{I_{i,j} : i, j \in \mathbb{N}\}\) es numerable y cubre \(\bigcup_{i \in \mathbb{N}} A_i\) ya que:
      \begin{align*}
        \displaystyle \bigcup_{i = 1}^{\infty} A_i \subseteq \displaystyle \bigcup_{i = 1}^{\infty} \displaystyle \bigcup_{j = 1}^{\infty} I_{i,j}
      \end{align*}
      Entonces, por \hyperref[def:medida_exterior]{definición de medida exterior}:
      \begin{align*}
        \mu_N^*\left(\displaystyle \bigcup_{i = 1}^{\infty} A_i\right) \leq \displaystyle \sum_{\substack{i \in \mathbb{N}\\j \in \mathbb{N}}} v_N(I_{i,j}) = \displaystyle \sum_{i = 1}^{\infty} \displaystyle \sum_{j = 1}^{\infty} v_N(I_{i,j})
      \end{align*}
      Y reordenando la suma doble (ya que los términos son no negativos):
      \begin{align*}
        \displaystyle \sum_{i = 1}^{\infty} \displaystyle \sum_{j = 1}^{\infty} v_N(I_{i,j}) & = \displaystyle \sum_{i = 1}^{\infty}\left(\displaystyle \sum_{j = 1}^{\infty}v_N(I_{i,j})\right) < \displaystyle \sum_{i = 1}^{\infty} \left(\mu_N^*(A_i) + \frac{\varepsilon}{2^i}\right) =\\[2ex]
        & = \displaystyle \sum_{i = 1}^{\infty} \mu_N^*(A_i) + \displaystyle \sum_{i = 1}^{\infty} \frac{\varepsilon}{2^i} = \left(\displaystyle \sum_{i = 1}^{\infty} \mu_N^*(A_i)\right) + \varepsilon
      \end{align*}
      Así, como \(\varepsilon > 0\) es arbitrario, si hacemos que \(\varepsilon \to 0\) entonces:
      \begin{align*}
        0 \leq \mu_N^*\left(\displaystyle \bigcup_{i = 1}^{\infty} A_i\right) \leq \left(\displaystyle \sum_{i = 1}^{\infty} \mu_N^*(A_i)\right) \underbracket{+ \varepsilon}_{\to 0} \implies \mu_N^*\left(\displaystyle \bigcup_{i = 1}^{\infty} A_i\right) \leq \displaystyle \sum_{i = 1}^{\infty} \mu_N^*(A_i)
      \end{align*}
    \end{itemize}   
    
    \item \textit{Para simplificar, la demostración será en \(\mathbb{R}^1\) para el cubo \(C = [a, b] \subseteq \mathbb{R}\) con \( - \infty < a < b < \infty\). La demostración se divide en cuatro desigualdades:}\vspace{2ex}
    \begin{itemize} \leftskip -12pt
      \item \textit{Igualdad de volúmenes: \(v_1(C^{\circ}) = v_1(C)\).} Por definición de volumen:
      \begin{align*}
        v_1(C^\circ) = v_1((a, b)) = b - a = v_1([a, b]) = v_1(C)
      \end{align*}
      \item \textit{Igualdad de medidas exteriores: \(\mu_1^*(C^\circ) = \mu_1^*(C)\).} Como \((a, b) \subseteq [a, b]\):
      \begin{align*}
        \mu_1^*(\overbracket{(a, b)}^{C^\circ}) \overset{\text{\tiny monotonía}}{\leq} \mu_1^*(\overbracket{[a, b]}^{C}) & = \mu_1^*\left((a, b) \cup \{a\} \cup \{b\}\right) \overset{\text{\tiny subaditiv.}}{\leq}\\[2ex]
        & \leq \mu_1^*((a, b)) + \underbracket{\mu_1^*(\{a\})}_{= 0} + \underbracket{\mu_1^*(\{b\})}_{= 0} = \mu_1^*(\underbracket{(a, b)}_{C^\circ})
      \end{align*}
      Por tanto, aplicando el criterio del sandwich se tiene el resultado.
      \item \textit{Cota superior \(\mu_1^*(C) \leq v_1\).} Sea \(\varepsilon > 0\) cualquiera, entonces:
      \begin{align*}
        I_\varepsilon = (a - \varepsilon, b + \varepsilon) \supseteq [a, b] = C
      \end{align*}
      Por tanto, por \hyperref[def:medida_exterior]{definición de medida exterior}:
      \begin{align*}
        \mu_1^*(C) \leq v_1(I_\varepsilon) = b - a + 2\varepsilon \quad \forall \varepsilon > 0
      \end{align*}
      Haciendo \(\varepsilon \to 0\) se tiene:
      \begin{align*}
        \mu_1^*(C) \leq v_1(C) = b - a
      \end{align*}
      \item \textit{Cota inferior \(v_1(C) \leq \mu_1^*(C)\).} Sea \(\varepsilon > 0\) cualquiera, por \hyperref[def:medida_exterior]{definición de medida exterior} sabemos que \(\exists \{I_i\}_{i \in \mathbb{N}}\) colección numerable de cubos abiertos tal que:
      \begin{align*}
        C \subseteq \displaystyle \bigcup_{i = 1}^{\infty} I_i \quad \mbox{ y } \quad \mu_1^*(C) + \varepsilon > \displaystyle \sum_{i = 1}^{\infty} v_1(I_i)
      \end{align*}
      Como \(C\) compacto, existe subrecubrimiento finito, i.e., \(\exists n \in \mathbb{N}\) tal que:
      \begin{align*}
        C \subseteq \displaystyle \bigcup_{i = 1}^{n} I_i
      \end{align*}
      Cada \(I_k\) es un intervalo abierto de la forma:
      \begin{align*}
        I_k = (a_k, b_k) \quad \mbox{ con } - \infty \leq a_k < b_k \leq \infty
      \end{align*}
      Podemos suponer que ningún \(I_k\) está contenido dentro de otro, ya que en ese caso lo podríamos eliminar del recubrimiento sin perder la propiedad de recubrimiento. Además, podemos ordenarlos de forma que sus extremos izquierdos estén ordenados crecientemente:
      \begin{align*}
        a_1 < a_2 < \dots < a_n
      \end{align*}
      Así, podemos notar que para que el recubrimiento cubra todo el intervalo \([a, b]\) se deben cumplir las siguientes condiciones:\vspace{2ex}
      \begin{itemize} \leftskip -12pt
        \item El primer intervalo \(I_1\) debe empezar antes de que empiece el intervalo \([a, b]\), es decir:
        \begin{align*}
          a_1 < a < b_1
        \end{align*}
        \item Para que el recubrimiento cubra todo el intervalo \([a, b]\), el siguiente intervalo \(I_2\) debe empezar antes de que termine \(I_1\), es decir:
        \begin{align*}
          a_2 < b_1
        \end{align*}
        Y en general tendremos que \(a_{k + 1} < b_k\).
        \item Finalmente, para cubrir el extremo derecho \(b\) se debe cumplir:
        \begin{align*}
          b_n > b
        \end{align*}
      \end{itemize}
      Por lo tanto, si calculamos la suma de los volúmenes de los intervalos \(I_k\):
      \begin{align*}
        \displaystyle \sum_{k = 1}^{n} v_1(I_k) &  = \displaystyle \sum_{k = 1}^{n} (b_k - a_k) = (b_1 - a_1) + (b_2 - a_2) + \dots + (b_n - a_n) =\\[2ex]
        & = b_n - (a_n - b_{n - 1}) - \dots - (a_2 - b_1) - a_1 \\[2ex]
        & = b_n - a_1 - \displaystyle \sum_{k = 1}^{n - 1} \underbracket{(a_{k + 1} - b_k)}_{ <  0} \geq b_n - a_1 \geq b - a
      \end{align*}
      ya que hemos visto que:
      \begin{itemize} \leftskip -12pt
        \item \(a_{k + 1} < b_k \implies a_{k + 1} - b_k < 0\) para todo \(1 \leq k \leq n - 1\)
        \item \(b_n > b\) y \(a_1 < a\) entonces \(b_n - a_1 > b - a\)
      \end{itemize}
      Así, continuando con la desigualdad inicial:
      \begin{align*}
        b - a < \displaystyle \sum_{k = 1}^{n} v_1(I_k) \leq \displaystyle \sum_{i = 1}^{\infty} v_1(I_i) < \mu_1^*(C) + \varepsilon
      \end{align*}
      Haciendo \(\varepsilon \to 0\) se tiene:
      \begin{align*}
        v_1(C) = b - a \leq \mu_1^*(C)
      \end{align*}
    \end{itemize}
  \end{enumerate}
\end{dem_box}
\newpage

\section{\texorpdfstring{Espacio de medida de Lebesgue \((\mathbb{R}^N, \mathcal{M}_n, \mu_N)\)}{Espacio de medida de Lebesgue (R^N, M_n, mu_N)}}
\subsection{Conjunto medible de Lebesgue. Definición}
\noindent Sea \(E \subseteq \mathbb{R}^N\) decimos que es \textbf{medible de Lebesgue} si \(\forall C\) cubo abierto acotado se cumple:
\begin{align*}
  v_N(C) = \mu_N^*(C \cap E) + \mu_N^*(C \cap E^C)
\end{align*}
O equivalentemente, como \(v_N(C) = \mu_N^*(C)\) y \(\mu_N^*\) es subaditiva:
\begin{align*}
  v_N(C) & = \mu_N^*(C) \leq \mu_N^*(C \cap E) + \mu_N^*(C \cap E^C)
\end{align*}

\begin{ej_box}{Nota}
  Al conjunto de todos los subconjuntos medibles de Lebesgue en \(\mathbb{R}^N\) lo denotamos por \(\mathcal{M}_N\).
\end{ej_box}
\vspace{3ex}

\subsection{\texorpdfstring{Propiedades de \(\mathcal{M}_N\). Proposición}{Propiedades de MN. Proposición}}\label{prop:propiedades_MN}
\noindent Sea \(E \subseteq \mathbb{R}^N\) entonces:
\begin{enumerate}
  \item Si \(\mu_N^*(E) = 0\) entonces \(E \in \mathcal{M}_N\), en particular, \(\emptyset \in \mathcal{M}_N\)
  \item Si \(E \in \mathcal{M}_N\) entonces \(E^C \in \mathcal{M}_N\)
  \item Todo semiespacio abierto de \(\mathbb{R}^N\) está en \(\mathcal{M}_N\), i.e., \(S = I_1 \times \dots \times I_N \subseteq \mathbb{R}^N\) tal que:
  \begin{align*}
     \exists j \in \{1, \dots, N\} \mbox{ con } \left\{
      \begin{array}{ll}
        I_k = ( - \infty, \infty) & \mbox{ si } k \neq j\\[1ex]
        I_j = ( - \infty, b) \mbox{ ó } I_j = (a, \infty) & \mbox{ con } - \infty < a < b < \infty
      \end{array}
     \right.
  \end{align*}
\end{enumerate}
\begin{dem_box}{Demostración}
  \begin{enumerate} \leftskip -10pt
    \item Sea \(E \subseteq \mathbb{R}^N\) con \(\mu_N^*(E) = 0\) y \(C\) cubo abierto acotado cualquiera. Tenemos:
    \begin{align*}
      v_N(C) \leq \underbracket{\mu_N^*(C \cap E)}_{\overset{\text{\tiny monot.}}{=} 0} + \mu_N^*\left(C \cap E^C\right) \leq \mu_N^*\left(C \cap E^C\right) \overset{\text{\tiny monot.}}{\leq} \mu_N^*(C) = v_N(C)
    \end{align*}
    Por lo tanto, por el criterio del sándwich:
    \begin{align*}
      v_N(C) = \mu_N^*(C \cap E) + \mu_N^*(C \cap E^C) \implies E \in \mathcal{M}_N 
    \end{align*}
    \item Sea \(E \in \mathcal{M}_N\) y \(C\) cubo abierto acotado cualquiera entonces:
    \begin{align*}
      v_N(C) = \mu_N^*(C \cap E) + \mu_N^*(C \cap E^C) = \mu_N^*(C \cap E^C) + \mu_N^*(C \cap (E^C)^C)
    \end{align*}
    Por lo tanto \(E^C \in \mathcal{M}_N\).

    \item Sin pérdida de generalidad, basta probarlo para:
    \begin{align*}
      H = (a, \infty) \times \mathbb{R} \times \overset{N - 1}{\dots } \times \mathbb{R} \quad \mbox{ con } - \infty < a < \infty
    \end{align*}
    Los casos \(\left\{
      \begin{array}{l}
        ( - \infty, a) \times \mathbb{R} \times\dots \times \mathbb{R}\\[2ex]
        H = J_1 \times\dots \times J_N \mbox{ con } \left\{
          \begin{array}{l}
            J_j = ( - \infty, a) \mbox{ ó } (a, \infty)\\
            J_i = \mathbb{R} \mbox{ si } i \neq j
          \end{array}
        \right.
      \end{array}
    \right.\) son análogos.\vspace{2ex}

    Sea \(C = \overbracket{(a_1, b_1)}^{J_1} \times \dots \times \overbracket{(a_N, b_N)}^{J_N}\) un cubo abierto acotado cualquiera. Así:
    \begin{align*}
      C \cap H & = \left\{
        \begin{array}{ll}
          C & \mbox{ si } a \leq a_1 \\[1ex]
          (a, b_1) \times J_2 \times \dots \times J_N & \mbox{ si } a_1 < a < b_1 \\[1ex]
          \emptyset & \mbox{ si } b_1 \leq a
        \end{array}
      \right. \\[3ex]
      C \cap H^C & = \left\{
        \begin{array}{ll}
          \emptyset & \mbox{ si } a \leq a_1 \\[1ex]
          (a_1, a] \times J_2 \times \dots \times J_N & \mbox{ si } a_1 < a < b_1 \\[1ex]
          C & \mbox{ si } b_1 \leq a
        \end{array}
      \right.
    \end{align*}
    Analicemos los 3 casos según el valor de \(a\):
    \begin{itemize}
      \item Si \(a \leq a_1\) entonces \(C \cap H = C\) y \(C \cap H^C = \emptyset\) por lo que:
      \begin{align*}
        \mu_N^*(C \cap H) + \mu_N^*(C \cap H^C) = v_N(C) + v_N(\emptyset ) = \mu_N^*(C) \implies H \in \mathcal{M}_N
      \end{align*}
      \item Si \(a_1 < a < b_1\) entonces se tiene:
      \begin{align*}
        C \cap H & = (a, b_1) \times J_2 \times \dots \times J_N \\[2ex]
        C \cap H^C & = (a_1, a] \times J_2 \times \dots \times J_N
      \end{align*}
      Y aunque \((a_1, a]\) no es un intervalo abierto, como \(\{a\}\) tiene medida cero, entonces \((a_1, a] = (a_1, a) \cup \{a\}\) tiene la misma medida que \((a_1, a)\) así:
      \begin{align*}
        \mu_N^*(C \cap H) & = v_N(C \cap H) = |a - b_1| \cdot v_N(J_2) \cdot \dots \cdot v_N(J_N)\\[2ex]
        \mu_N^*(C \cap H^C) & = v_N(C \cap H^C) = |a_1 - a| \cdot v_N(J_2) \cdot \dots \cdot v_N(J_N)
      \end{align*}
      Por lo que sumando:
      \begin{align*}
        \mu_N^*(C \cap H) + \mu_N^*(C \cap H^C) & = \overbrace{(|a - b_1| + |a_1 - a|)}^{(b_1 - a_1)} \cdot v_N(J_2) \cdot \dots \cdot v_N(J_N) = \\[2ex]
        & = v_N(J_1) \cdot v_N(J_2) \cdot \dots \cdot v_N(J_N) =\\[2ex]
        & = v_N(C) = \mu_N^*(C) \implies H \in \mathcal{M}_N
      \end{align*}
      \item Si \(b_1 \leq a\) entonces \(C \cap H = \emptyset\) y \(C \cap H^C = C\) por lo que:
      \begin{align*}
        \mu_N^*(C \cap H) + \mu_N^*(C \cap H^C) = \mu_N^*(\emptyset) + \mu_N^*(C) = \mu_N^*(C) \implies H \in \mathcal{M}_N
      \end{align*}
    \end{itemize}
  \end{enumerate}
\end{dem_box}
\vspace{3ex}


\subsection{Caracterización de medibilidad de Lebesgue. Proposición}\label{prop:caract_medibilidad_Lebesgue}
\noindent Dado \(E \subseteq \mathbb{R}^N\) los siguientes enunciados son equivalentes:
\begin{enumerate}
  \item \(E \in \mathcal{M}_N\)
  \item \(\forall A \subseteq \mathbb{R}^N\) se cumple:
  \begin{align*}
    \mu_N^*(A) = \mu_N^*(A \cap E) + \mu_N^*(A \cap E^C)
  \end{align*}
  o equivalentemente:
  \begin{align*}
    \mu_N^*(A) \geq \mu_N^*(A \cap E) + \mu_N^*(A \cap E^C)
  \end{align*}
\end{enumerate}

\begin{dem_box}{Demostración}
  \begin{enumerate}
    \item[\(2 \Rightarrow 1)\)] Trivial, tomamos \(A = C\) cubo abierto acotado cualquiera y por la propiedad de la medida exterior \(\mu_N^*(C) = v_N(C)\) entonces se cumple:
    \begin{align*}
      v_N(C) = \mu_N^*(C) = \mu_N^*(C \cap E) + \mu_N^*(C \cap E^C)
    \end{align*}
    Por lo tanto \(E \in \mathcal{M}_N\).
    \item[\(1 \Rightarrow 2)\)] Tomamos \(A \subseteq \mathbb{R}^N\) cualquiera y tenemos dos casos:
    \begin{itemize}
      \item Si \(\mu_N^*(A) = \infty\), por subaditividad de \(\mu_N^*\):
      \begin{align*}
        \infty = \mu_N^*(A) \leq \mu_N^*(A \cap E) + \mu_N^*(A \cap E^C) \leq \infty
      \end{align*}
      Y por el criterio del sándwich se cumple la igualdad.\vspace{2ex}

      \item Si \(\mu_N^*(A) < \infty\) entonces \(\forall \varepsilon > 0\) existe recubrimiento numerable \(\{I_i\}_{i \in \mathbb{N}}\) de cubos abiertos tales que:
      \begin{align*}
        A \subseteq \displaystyle \bigcup_{i = 1}^{\infty} I_i \quad \mbox{ y } \quad \mu_N^*(A) + \varepsilon > \displaystyle \sum_{i = 1}^{\infty} v_N(I_i)
      \end{align*}
      Notar \(\mu_N^*(A) + \varepsilon < \infty\) entonces los cubos \(I_i\) son acotados. Como \(E \in \mathcal{M}_N\):
      \begin{align*}
        \mu_n^*(A) + \varepsilon & \geq \displaystyle \sum_{i = 1}^{\infty} v_N(I_i) \overset{E \in \mathcal{M}_N}{ = }\displaystyle \sum_{i = 1}^{\infty} \left[\mu_N^*(I_i \cap E) + \mu_N^*(I_i \cap E^C)\right] = \\[2ex]
        & = \displaystyle \sum_{i = 1}^{\infty} \mu_N^*(I_i \cap E) + \displaystyle \sum_{i = 1}^{\infty} \mu_N^*(I_i \cap E^C) \overset{\text{\tiny subaditividad}}{\geq} \\[2ex]
        & \geq \mu_N^*\left(E \bigcap \left[\displaystyle \bigcup_{i = 1}^{\infty} I_i\right] \right) + \mu_N^*\left(E^C \bigcap \left[\displaystyle \bigcup_{i = 1}^{\infty} I_i\right]\right) \overset{\text{\tiny monotonia}}{\geq} \\[2ex]
        & \geq \mu_N^*(A \cap E) + \mu_N^*(A \cap E^C)
      \end{align*}
      Tomamos \(\varepsilon > 0\) y hacemos \(\varepsilon \to 0\) para concluir que:
      \begin{align*}
        \mu_N^*(A) + \underbracket{\varepsilon}_{\to 0} \geq  \mu_N^*(E \cap A) + \mu_N^*(E^C \cap A)
      \end{align*}
    \end{itemize}
  \end{enumerate}
\end{dem_box}
\vspace{3ex}

\subsection{Proposición}
\noindent Sea \(E, F \in \mathcal{M}_N\) entonces:
\begin{enumerate}
  \item \(E \cup F \in \mathcal{M}_N\)
  \item \(E \cap F \in \mathcal{M}_N\)
  \item \(E \setminus F \in \mathcal{M}_N\) \vspace{2ex}
\end{enumerate}
\begin{dem_box}{Demostración}
  \begin{enumerate}
    \item Sea \(A \subseteq \mathbb{R}^N\) cualquiera. Entonces:
    \begin{align*}
      \mu_N^*(A) & \overset{E \in \mathbb{M}_N}{=} \mu_N^*(A \cap E) + \mu_N^*(A \cap E^C) \overset{F \in \mathcal{M}_N}{=} \\[2ex]
      & = \mu_N^*(A \cap E) + \mu_N^*(A \cap E^C \cap F) + \mu_N^*(A \cap E^C \cap F^C) = \\[2ex]
      & = \mu_N^*(A \cap E) + \mu_N^*(A \cap [F \setminus E]) + \mu_N^*(A \cap [E \cup F]^C) \overset{\text{\tiny subadit.}}{\geq}\\[2ex]
      & \geq \mu_N^*(A \cap [\underbracket{E \cup(F \setminus E)}_{E \cup F}]) + \mu_N^*(A \cap [E \cup F]^C) = \\[2ex]
      & = \mu_N^*(A \cap [E \cup F]) + \mu_N^*(A \cap [E \cup F]^C) \implies E \cup F \in \mathcal{M}_N\\
    \end{align*}
    \item Como \((E \cap F)^C = E^C \cup F^C\) y como \(E^C, F^C \in \mathcal{M}_N\) entonces:
    \begin{align*}
      E \cap F = (E \cap F)^{C^C} \in \mathcal{M}_N\\
    \end{align*}
    \item Como \(E \setminus F = E \cap F^C\) y como \(F^C \in \mathcal{M}_N\) entonces \(E \setminus F \in \mathcal{M}_N\)
  \end{enumerate}
\end{dem_box}
\vspace{3ex}

\subsection{Corolario}
\noindent Sean \(\{E_i\}_{i \in \mathbb{N}} \subseteq \mathcal{M}_N\) con \(N \in \mathbb{N}\) entonces para \(n \in \mathbb{N}\):
\begin{align*}
  \displaystyle \bigcup_{i = 1}^{n} E_i \in \mathcal{M}_N \quad \mbox{ y } \quad \displaystyle \bigcap_{i = 1}^{n} E_i \in \mathcal{M}_N
\end{align*}
\begin{dem_box}{Demostración}
  Por inducción en \(n\) usando la proposición anterior.
\end{dem_box}
\vspace{3ex}

\subsection{Proposición}\label{prop:3.3_5}
\noindent Sea \(\{E_i\}_{i \in \mathbb{N}} \subseteq \mathcal{M}_N\) donde \(E_i \cap E_j = \emptyset \) si \(i \neq j\) y sea \(A \in \mathbb{R}^N\) con \(n \in \mathbb{N}\) entonces:
\begin{align*}
  \mu_N^*\left(A \cap \left[\displaystyle \bigcup_{i = 1}^{n}E_i\right]\right) = \displaystyle \sum_{i = 1}^{n} \mu_N^*(A \cap E_i)
\end{align*}
\begin{dem_box}{Demostración}
  Por inducción en \(n\):
  \begin{itemize} \leftskip -10pt 
    \item Si \(n = 1\) se cumple trivialmente ya que:
    \begin{align*}
      \mu_N^*\left(A \cap \left[\displaystyle \bigcup_{i = 1}^{1} E_i\right]\right) = \mu_N^*(A \cap E_1) = \displaystyle \sum_{i = 1}^{1} \mu_N^*(A \cap E_i)
    \end{align*}
    \item Por hipótesis de inducción, supongamos que se cumple para \(n\):
    \begin{align*}
      \mu_N^*\left(A \cap \left[\displaystyle \bigcup_{i = 1}^{n} E_i\right]\right) = \displaystyle \sum_{i = 1}^{n} \mu_N^*(A \cap E_i)
    \end{align*}
    Probemos que se cumple para \(n + 1\), es decir:
    \begin{align*}
      \mu_N^*\left(A \cap \left[\displaystyle \bigcup_{i = 1}^{n + 1} E_i\right]\right) = \displaystyle \sum_{i = 1}^{n + 1} \mu_N^*(A \cap E_i)
    \end{align*}
    Como \(E_{n + 1} \in \mathcal{M}_N\) por \hyperref[prop:caract_medibilidad_Lebesgue]{carracterización de medibilidad de Lebesgue} aplicada al conjunto \(A \cap \left[\bigcup_{i = 1}^{n + 1} E_i\right]\) tenemos:
    \begin{align*}\label{eq:prop_3_5}\tag{I}
      \mu_N^*\left(A \cap \left[\displaystyle \bigcup_{i = 1}^{n + 1} E_i\right]\right) = & \mu_N^*\left(\left[A \cap \left[\displaystyle \bigcup_{i = 1}^{n + 1} E_i\right]\right] \cap E_{n + 1}\right) + \\[1ex]
      & \hspace{15ex} + \mu_N^*\left(A \cap \left[\displaystyle \bigcup_{i = 1}^{n + 1} E_i\right]\cap E_{n + 1}^C\right)
    \end{align*}
    Como tenemos que \(E_i \cap E_j = \emptyset \) si \(i \neq j\) entonces el primer término es:
    \begin{align*}
      \left[A \cap \left[\displaystyle \bigcup_{i = 1}^{n + 1} E_i\right]\right] \cap E_{n + 1} = A \cap \left[\left(\displaystyle \bigcup_{i = 1}^{n + 1} E_i\right) \cap E_{n + 1}\right] = A \cap E_{n + 1}
    \end{align*}
    y el segundo término es:
    \begin{align*}
      \left[A \cap \left[\displaystyle \bigcup_{i = 1}^{n + 1} E_i\right]\right] \cap E_{n + 1}^C = A \cap \left[\left(\displaystyle \bigcup_{i = 1}^{n + 1} E_i\right) \cap E_{n + 1}^C\right] = A \cap \left[\displaystyle \bigcup_{i = 1}^{n} E_i\right]
    \end{align*}
    Luego si sustituimos en \eqref{eq:prop_3_5}:
    \begin{align*}
      \mu_N^*\left(A \cap \left[\displaystyle \bigcup_{i = 1}^{n + 1} E_i\right]\right)& = \mu_N^*(A \cap E_{n + 1}) + \mu_N^*\left(A \cap \left[\displaystyle \bigcup_{i = 1}^{n} E_i\right]\right) \overset{\text{\tiny H.I.}}{=} \\[2ex]
      & = \mu_N^*(A \cap E_{n + 1}) + \displaystyle \sum_{i = 1}^{n} \mu_N^*(A \cap E_i) = \\[2ex]
      & = \displaystyle \sum_{i = 1}^{n + 1} \mu_N^*(A \cap E_i)
    \end{align*}
  \end{itemize}
\end{dem_box}
\vspace{3ex}

\subsection{Teorema de Medida de Lebesgue}
\noindent Para \(\mu_N \coloneq {\mu_N^*}_{|_{\mathcal{M}_N}}\) se tiene que \((\mathbb{R}^N, \mathcal{M}_N, \mu_N)\) es un espacio de medida completo tal que:
\begin{align*}
  \tau_{\mathbb{R}^N} \subseteq \mathcal{M}_N \quad \mbox{ y } \quad \mu_N(C) = v_N(C) \quad \forall C \mbox{ cubo acotado}
\end{align*}
\begin{dem_box}{Demostración}
  \begin{enumerate}\leftskip -15pt
    \item \textit{Ver que \(\mathcal{M}_N\) es una \(\sigma\)-álgebra}
    \begin{itemize} \leftskip -15pt
      \item \textit{\(\emptyset \in \mathcal{M}_N\)}: Por \hyperref[prop:propiedades_MN]{las propiedades de \(\mathcal{M}_N\)} sabemos que \(\emptyset \in \mathcal{M}_N\).\vspace{2ex}
      \item \textit{Si \(E \in \mathcal{M}_N\) entonces \(E^C \in \mathcal{M}_N\)}: Por \hyperref[prop:propiedades_MN]{las propiedades de \(\mathcal{M}_N\)} sabemos que si \(E \in \mathcal{M}_N\) entonces \(E^C \in \mathcal{M}_N\). \vspace{2ex}
      \item \textit{\(\mathcal{M}_N\) es cerrada bajo uniones numerables:} Sea \(\{E_i\}_{i \in \mathbb{N}} \subseteq \mathcal{M}_N\) aplicando desjuntificación, obtenemos: 
      \begin{align*}
      F_1 & \coloneq E_1 \in \mathcal{M}_N\\
      F_2 & \coloneq E_2 \setminus E_1 \in \mathcal{M}_N\\
      \vdots \hspace{0.7ex} & \hspace{10ex}\vdots \\
      F_{n + 1} & \coloneq E_{n + 1} \setminus \displaystyle \bigcup_{i = 1}^{n} E_i \in \mathcal{M}_N
    \end{align*}
    Así, tenemos que \(\forall n \in \mathbb{N}\) se cumple:
    \begin{align*}
      \displaystyle \bigcup_{i = 1}^{n} F_i = \displaystyle \bigcup_{i = 1}^{n} E_i \quad \mbox{y} \quad \displaystyle \bigcup_{i = 1}^{\infty} F_i = \displaystyle \bigcup_{i = 1}^{\infty} E_i \quad \mbox{ y } \quad F_i \cap F_j = \emptyset \mbox{ si } i \neq j
    \end{align*}
    Veamos que es medible. Sea \(A \subseteq \mathbb{R}^N\) cualquiera, entonces para cada \(n \in \mathbb{N}\) como \(\bigcup_{i = 1}^{n} E_i \in \mathcal{M}_N\) entonces\footnote{(Aquí tuvo un pequeño lapsus y puso E donde era F y un joven alumno le corrigió). Muchas gracias, gracias a tí no me pilla el toro. Tengo unas pegatinas de matemáticas en casa y tu ya tienes una, solo que no las tengo aquí.}:
    \begin{align*}
      \mu_N^*(A) = \mu_N^*\left(A \cap \left[\displaystyle \bigcup_{i = 1}^{n} E_i\right]\right) + \mu_N^*\left(A \cap \left[\displaystyle \bigcup_{i = 1}^{n}E_i\right]^C\right)
    \end{align*}
    Donde tenemos que:
    \begin{itemize}
      \item \(\bigcup_{i = 1}^n E_i = \bigcup_{i = 1}^n F_i\)
      \item \(\left(\bigcup_{i = 1}^n E_i\right)^C \supseteq \left(\bigcup_{i = 1}^\infty E_i\right)^C\)
    \end{itemize}
    Aplicando monotonía de \(\mu_N^*\) en el segundo término:
    \begin{align*}
      \mu_N^*(A) & \geq \mu_N^*\left(A \cap \left[\displaystyle \bigcup_{i = 1}^{n} F_i\right]\right) + \mu_N^*\left(A \cap \left[\displaystyle \bigcup_{i = 1}^{\infty} E_i\right]^C\right) \xlongequal[\text{\tiny \hyperref[prop:3.3_5]{anterior}}]{\text{\tiny \hyperref[prop:3.3_5]{prop}}} \\[2ex]
      & = \displaystyle \sum_{i = 1}^{n} \mu_N^*(A \cap F_i) + \mu_N^*\left(A \cap \left[\displaystyle \bigcup_{i = 1}^{\infty} E_i\right]^C\right)
    \end{align*}
    Y si hacemos tender \(n \to \infty\) obtenemos:
    \begin{align*}
      \mu_N^*(A) & \geq \displaystyle \sum_{i = 1}^{\infty} \mu_N^*(A \cap F_i) + \mu_N^*\left(A \cap \left[\displaystyle \bigcup_{i = 1}^{\infty} E_i\right]^C\right) \overset{\text{\tiny subaditv}}{\geq}\\[2ex]
      & \geq \mu_N^*\left(A \cap \left[\displaystyle \bigcup_{i = 1}^{\infty} F_i\right]\right) + \mu_N^*\left(A \cap \left[\displaystyle \bigcup_{i = 1}^{\infty} E_i\right]^C\right) = \\[2ex]
      & = \mu_N^*\left(A \cap \left[\displaystyle \bigcup_{i = 1}^{\infty} E_i\right]\right) + \mu_N^*\left(A \cap \left[\displaystyle \bigcup_{i = 1}^{\infty} E_i\right]^C\right)
    \end{align*}
    Como la desigualdad contraria se cumple por subaditividad de \(\mu_N^*\), es decir:
    \begin{align*}
      \mu_N^*(A) \leq \mu_N^*\left(A \cap \left[\displaystyle \bigcup_{i = 1}^{\infty} E_i\right]\right) + \mu_N^*\left(A \cap \left[\displaystyle \bigcup_{i = 1}^{\infty} E_i\right]^C\right)
    \end{align*}
    Por el criterio del sándwich se cumple la igualdad y por lo tanto:
    \begin{align*}
      \mu_N^*(A) = \mu_N^*\left(A \cap \left[\displaystyle \bigcup_{i = 1}^{\infty} E_i\right]\right) + \mu_N^*\left(A \cap \left[\displaystyle \bigcup_{i = 1}^{\infty} E_i\right]^C\right) \implies \displaystyle \bigcup_{i = 1}^{\infty} E_i \in \mathcal{M}_N\\
    \end{align*}
    \end{itemize}
    \item \textit{Ver que \(\tau_{\mathbb{R}^N} \subseteq \mathcal{M}_N\)}: Para cada \(j = 1, \dots, N\) definimos los semiespacios:
    \begin{align*}
      H_j^{ + } & \coloneq (a_j, \infty) \times \mathbb{R}^{N - 1} \\[2ex]
      H_j^{ - } & \coloneq (-\infty, a_j) \times \mathbb{R}^{N - 1}
    \end{align*}

    Vimos en \hyperref[prop:propiedades_MN]{las propiedades de \(\mathcal{M}_N\)} que \(H_j^{ + }, H_j^{ - } \in \mathcal{M}_N\).\vspace{2ex}

    Como todo cubo abierto \(C = (a_1, b_1) \times (a_2, b_2) \times \dots \times (a_N, b_N)\) puede expresarse como intersección finita de semiespacios \(H_j^{ + }\) y \(H_j^{ - }\):
    \begin{align*}
      C = \displaystyle \bigcap_{i = 1}^{N} \left(H_i^{ + } \cap H_i^{ - }\right)
    \end{align*}
    entonces \(C \in \mathcal{M}_N\) por la propiedad de cerradura bajo intersección finita de \(\mathcal{M}_N\).\vspace{2ex}

    Ahora, definimos la base numerable de \(\tau_{\mathbb{R}^N}\).:
    \begin{align*}
      \mathcal{A} \coloneq \left\{(p_1, q_1) \times (p_2, q_2) \times \dots \times (p_N, q_N) : p_i, q_i \in \mathbb{Q}, p_i < q_i\right\}
    \end{align*}
    Al ser base de \(\tau_{\mathbb{R}^N}\), todo abierto \(O \in \tau_{\mathbb{R}^N}\) puede expresarse como unión numerable de elementos de \(\mathcal{A}\). Como \(\mathcal{A} \subseteq \mathcal{M}_N\) y \(\mathcal{M}_N\) es cerrada bajo uniones numerables, entonces \(O \in \mathcal{M}_N\).\vspace{2ex}

    \item \textit{Ver que \(\mu_N\) es medida:} Veamos que cumple las dos propiedades de la definición de medida:
    \begin{itemize} \leftskip -15pt
      \item \textit{\(\mu_N(\emptyset ) = 0\)}: Ya hemos probado\footnote{Como habéis notado, yo no soy de aquí y digo barbaridades como hemos demostrado en vez de demostramos y cosas así porque, aunque demostramos tiene tiempo pasado, también indica presente. Antes decía mucho demostramos en lugar de hemos demostrado y, un alumno me hizo un análisis sintáctico y, por no insistir, empecé a decir hemos demostrado y nos quitamos de problemas.} esto, por definición \(\mu_N\) y por la propiedad de la medida exterior \(\mu_N^*\) tenemos:
      \begin{align*}
        \mu_N(\emptyset ) = \mu_N^*(\emptyset ) = 0
      \end{align*}
      \item \textit{\(\sigma\)-aditividad:} Sea \(\{E_i\}_{i \in \mathbb{N}} \subseteq \mathcal{M}_N\) disjuntos entonces para cada \(n \in \mathbb{N}\):
      \begin{align*}
        \displaystyle \sum_{i = 1}^{n} \mu_N(E_i) & \xlongequal[\text{\tiny \hyperref[prop:3.3_5]{prop. ant}}]{E_i \cap E_j = \emptyset } \mu_N\left(\displaystyle \bigcup_{i = 1}^{n}E_i\right) \overset{\text{\tiny monot.}}{\leq} \mu_N\left(\displaystyle \bigcup_{i = 1}^{\infty} E_i\right) = \\[2ex]
        & = \mu_N^*\left(\displaystyle \bigcup_{i = 1}^{\infty} E_i\right) \overset{\text{\tiny subadit.}}{\leq} \displaystyle \sum_{i = 1}^{\infty} \mu_N^*(E_i) = \displaystyle \sum_{i = 1}^{\infty} \mu_N(E_i)
      \end{align*}
      Y ahora, si hacemos \(n \to \infty\) obtenemos:
      \begin{align*}
        \displaystyle \sum_{i = 1}^{\infty} \mu_N(E_i) \leq \mu_N\left(\displaystyle \bigcup_{i = 1}^{\infty} E_i\right) \leq \displaystyle \sum_{i = 1}^{\infty} \mu_N(E_i)
      \end{align*}
      Por el criterio del sándwich se cumple la igualdad y por lo tanto es \(\sigma\)-aditiva.\vspace{3ex}
    \end{itemize}

    \item \textit{\((\mathbb{R}^N, \mathcal{M}_N, \mu_N)\) es completo:} Sea \(E \in \mathcal{M}_N\) con \(\mu_N(E) = 0\) y \(F \subseteq E\), entonces:
    \begin{align*}
      0 \leq \mu_N^*(F) \overset{\text{\tiny monot.}}{\leq} \mu_N^*(E) = 0 \implies \mu_N^*(F) = 0 \implies F \in \mathcal{M}_N
    \end{align*}

    \item \textit{Ver que \(\mu_N(C) = v_N(C)\) para todo cubo acotado \(C\)}: Sea \(C\) un cubo abierto acotado en \(\mathbb{R}^N\), por las \hyperref[prop:propiedades_muN*]{propiedades de \(\mu_N^*\)} sabemos que \(\mu_N^*(C) = v_N(C)\) y como \(C \in \mathcal{M}_N\) entonces:
    \begin{align*}
      \mu_N(C) = \mu_N^*(C) = v_N(C)\\
    \end{align*}
  \end{enumerate}
\end{dem_box}
\vspace{3ex}

\begin{ej_box}{Nota}
  Ahora, para cerrar este tema, nos va a faltar\footnote{Y aquí algunos estadísticos hacen barbaridades dignas de colgarles de un árbol} ver la relación entre la medida de Lebesgue y la medida de Borel:
  \begin{align*}
    (\mathbb{R}^N, \widetilde{\mathcal{B}_N}, \widetilde{\mu_N}) = (\mathbb{R}^N, \mathcal{M}_N, \mu_N)
  \end{align*}
\end{ej_box}
\vspace{3ex}

\subsection{\texorpdfstring{Caracterización topológica de los conjuntos medibles de Lebesgue de \(\mathbb{R^N}\). Teorema}{Caracterización topológica de los conjuntos medibles de Lebesgue de R\^N. Teorema}}\label{th:caract_topologica_conjuntos_medibles_Lebesgue}
\noindent Para todo \(E \subseteq \mathbb{R}^N\) los siguientes enunciados son equivalentes:
\begin{enumerate}
  \item \(E \in \mathcal{M}_N\)
  \item \(\forall \varepsilon > 0\) existe \(O\) abierto tal que \(E \subseteq O\) y \(\mu_N^*(O \setminus E) < \varepsilon\)
  \item \(\forall \varepsilon > 0\) existe \(C\) cerrado tal que \(C \subseteq E\) y \(\mu_N^*(E \setminus C) < \varepsilon\)
  \item \(\forall \varepsilon > 0\) existe \(O\) abierto y \(C\) cerrado tal que \(C \subseteq E \subseteq O\) y \(\mu_N^*(O \setminus C) < \varepsilon\)
\end{enumerate}
\vspace{2ex}
\begin{ej_box}{Nota}
  La demostración se hará siguiendo el esquema:
  \begin{align*}
    1 & \implies 2 \implies 1\\
    1 & \implies 3 \implies 1\\
    1 & \implies 4 \implies 2 \land 3
  \end{align*}
\end{ej_box}
\vspace{2ex}
\begin{dem_box}{Demostración}
  \begin{enumerate}
    \item[\(1 \Rightarrow 2\))] Sea \(E \in \mathcal{M}_N\) se puede aproximar por abiertos que lo contienen. Así se dan dos casos:
    \begin{itemize} \leftskip -15pt
      \item \textit{Sea \(\mu_N(E) < \infty\)}: Por \hyperref[def:medida_exterior]{definción de medida exterior}, \(\forall \varepsilon > 0\) existe un recubrimiento por cubos abiertos \(\{I_i\}_{i \in \mathbb{N}}\) tales que:
      \begin{align*}
        E \subseteq \displaystyle \bigcup_{i = 1}^{\infty} I_i \quad \mbox{ y } \quad \mu_N(E) + \varepsilon > \displaystyle \sum_{i = 1}^{\infty} v_N(I_i)
      \end{align*}
      Sea \(O \coloneq \bigcup_{i = 1}^\infty I_i\) así \(O\) es abierto y \(E \subseteq O\). Como \(E\) es medible entonces:
      \begin{align*}
        \mu_N(O) = \mu_N(O \cap E) + \mu_N(O \cap E^C) = \mu_N(E) + \mu_N(O \setminus E)
      \end{align*}
      Por lo tanto:
      \begin{align*}
        \mu_N(O \setminus E) & = \mu_N(O) - \mu_N(E) \leq \displaystyle \sum_{i = 1}^{\infty} v_N(I_i) - \mu_N(E) < \varepsilon
      \end{align*}
      \item \textit{Sea \(\mu_N(E) = \infty\)}: Descomponemos \(\mathbb{R}^N\) en conjuntos disjuntos, para ello, consideramos los cubos \(C_n \coloneq [ - n, n]^N\) con \(n \in \mathbb{N}\) y los conjuntos:
      \begin{align*}
        D_1 & \coloneq C_1 \in \mathcal{M}_N\\
        D_2 & \coloneq C_2 \setminus C_1 \in \mathcal{M}_N\\
        \vdots \hspace{0.7ex} & \hspace{10ex}\vdots \\
        D_{n + 1} & \coloneq C_{n + 1} \setminus C_n \in \mathcal{M}_N
      \end{align*}
      Así, \(\mathbb{R}^N = \bigcup_{i = 1}^\infty D_n \):
      \begin{itemize} \leftskip -10pt
        \item \(D_n \cap D_m = \emptyset \) si \(n \neq m\)
        \item \(D_n\) es acotado entonces \(\mu_n(D) < \infty\)
      \end{itemize}
      Entonces definiendo \(E_n \coloneq E \cap D_n \in \mathcal{M}_N\) tenemos:
      \begin{align*}
        E = \displaystyle \bigcup_{n = 1}^{\infty} E_n \quad \mbox{ y } \quad E_n \cap E_m = \emptyset \mbox{ si } n \neq m \quad \mbox{ y } \quad \mu_N(E_n) < \infty
      \end{align*}
      Sea \(\varepsilon > 0\) cualquiera, 
      por el caso anterior para cada \(E_n\), \(\exists O_n\) abierto tq:
      \begin{align*}
        E_n \subseteq O_n \quad \mbox{ y } \quad \mu_N(O_n \setminus E_n) < \frac{\varepsilon}{2^n}
      \end{align*} 
      Sea \(O \coloneq \bigcup_{i = 1}^\infty O_n\) abierto entonces:
      \begin{align*}
        \mu_N(O \setminus E) = \mu_N\left(\displaystyle \bigcup_{n = 1}^{\infty} O_n \setminus \displaystyle \bigcup_{n = 1}^{\infty} E_n\right)  &\leq \mu_N\left(\displaystyle \bigcup_{n = 1}^{\infty} \left[O_n \setminus E_n\right]\right)  \leq \\[2ex]
        & \leq \displaystyle \sum_{n = 1}^{\infty} \mu_N(O_n \setminus E_n) < \displaystyle \sum_{n = 1}^{\infty} \frac{\varepsilon}{2^n} = \varepsilon
      \end{align*}
    \end{itemize}
    \item[\(1 \Rightarrow 3\))] Sea \(E \in \mathcal{M}_N\), entonces queremos aproximarlo por cerrados contenidos en él, así consideramos \(\varepsilon > 0\) entonces aplicando \(1 \Rightarrow 2\) a \(E^C\) existe \(O\) abierto con:
    \begin{align*}
      E^C \subseteq O \quad \mbox{ y } \quad \mu_N(O \setminus E^C) < \varepsilon
    \end{align*}
    Sea \(C \coloneq O^C\) cerrado, así \(C^C = O\) luego:
    \begin{align*}
      \mu_N\left(C^C\setminus E^C\right) = \mu_N(O \setminus E^C) < \varepsilon\\
    \end{align*}
    \item[\(1 \Rightarrow 4\))] Sea \(E \in \mathcal{M}_N\) y consideramos \(\varepsilon > 0\) cualquiera, aplicando \(1 \Rightarrow 2\) y \(1 \Rightarrow 3\) se obtienen \(O\) abierto y \(C\) cerrado tales que:
    \begin{align*}
      C \subseteq E \subseteq O \quad \mbox{ y } \quad \mu_N(O \setminus E) < \frac{\varepsilon}{2}, \quad \mu_N(E \setminus C) < \frac{\varepsilon}{2}
    \end{align*}
    Luego:
    \begin{align*}
      \mu_N(O\setminus C)  = \mu_N\left([O \setminus E] \cup [E \setminus C]\right) \leq \mu_N(O \setminus E) + \mu_N(E \setminus C)< \frac{\varepsilon}{2} + \frac{\varepsilon}{2} = \varepsilon\\
    \end{align*}
    \item[\(4 \Rightarrow 2\))] \textit{(Y \(4 \Rightarrow 3\))} Por hipótesis dado \(\varepsilon > 0\) existen \(C\) cerrado y \(O \) abierto tales que:
    \begin{align*}
      C \subseteq E \subseteq O \quad \mbox{ y } \quad \mu_N(O \setminus C) < \varepsilon
    \end{align*}
    Entonces:
    \begin{align*}
      O\setminus E \subseteq O \setminus C & \implies \mu_N^*(O \setminus E) \leq \mu_N^*(O \setminus C) < \varepsilon\\[2ex]
      E \setminus C \subseteq O \setminus C & \implies \mu_N^*(E \setminus C) \leq \mu_N^*(O \setminus C) < \varepsilon\\
    \end{align*}
    \item[\(2 \Rightarrow 1\))] Por hipótesis para cada \(n \in \mathbb{N}\) existe \(O_n\) abierto tal que:
    \begin{align*}
      E \subseteq O_n \quad \mbox{ y } \quad \mu_N^*(O_n \setminus E) < \frac{1}{n}
    \end{align*}
    Consideramos los abiertos \(G_n \coloneq \bigcap_{i = 1}^n O_i\) así:
    \begin{itemize}
      \item \(G_n \supseteq G_{n + 1}\)
      \item \(E \subseteq G_n\)
      \item \(\mu_N^*(G_n \setminus E) \leq \mu_N^*(O_n \setminus E) < \frac{1}{n}\)
    \end{itemize}
    Sea \(G \coloneq \bigcap_{i = 1}^\infty G_n \in \mathcal{M}_N\) tenemos que \(E \subseteq G\) y :
    \begin{align*}
      \mu_N^*(G \setminus E) \leq \mu_N^*(G_n \setminus E) < \frac{1}{n} \quad \forall n \in \mathbb{N}
    \end{align*}
    Haciendo \(n \to \infty\) obtenemos:
    \begin{align*}
      0 \leq \mu_N^*(G \setminus E) \leq 0 \implies N \coloneq G \setminus E \in \mathcal{M}_N \quad \mbox{ y } \quad \mu_N(N) = 0
    \end{align*}
    Además, como \(G, N \in \mathcal{M}_N\) entonces \(E = G \setminus N \in \mathcal{M}_N\) \vspace{2ex}

    \item[\(3 \Rightarrow 1\))] Por hipótesis para cada \(n \in \mathbb{N}\) existe \(C_n\) cerrado tal que:
    \begin{align*}
      E \supseteq C_n \quad \mbox{ y } \quad \mu_N^*(E \setminus C_n) < \frac{1}{n}
    \end{align*}
    Consideramos los cerrados \(D_n \coloneq \bigcup_{i = 1}^n C_i\) así:
    \begin{itemize}
      \item \(D_n \subseteq D_{n + 1}\)
      \item \(E \supseteq D_n\)
      \item \(\mu_N^*(E \setminus D_n) \leq \mu_N^*(E \setminus C_n) < \frac{1}{n}\)
    \end{itemize}
    Sea \(D \coloneq \bigcup_{i = 1}^\infty D_n \in \mathcal{M}_N\), así:
    \begin{align*}
      E \supseteq D \quad \mbox{ y } \quad \mu_N^*(E \setminus D) \leq \mu_N^*(E \setminus D_n) < \frac{1}{n}
    \end{align*}
    Haciendo \(n \to \infty\) obtenemos:
    \begin{align*}
      0 \leq \mu_N^*(E \setminus D) \leq 0 \implies M \coloneq E \setminus D \in \mathcal{M}_N \quad \mbox{ y } \quad \mu_N(M) = 0
    \end{align*}
    Además, como \(D, M \in \mathcal{M}_N\) entonces \(E = D \cup M \in \mathcal{M}_N\)
  \end{enumerate}
\end{dem_box}
\vspace{3ex}
\begin{ej_box}{Nota}
  Las implicaciones de esta demostración lo que dicen (en castellano) es:
  \begin{itemize}
    \item[\(1 \Rightarrow 2\))] Si \(E\) es medible entonces se puede aproximar por abiertos que lo contienen.
    \item[\(1 \Rightarrow 3\))] Si \(E\) es medible entonces se puede aproximar por cerrados que están contenidos en él.
    \item[\(1 \Rightarrow 4\))] Si \(E\) es medible entonces puede ser ``encajado'' entre un cerrado y un abierto.
    \item[\(4 \Rightarrow 2\))] \textit{(y \(4 \Rightarrow 3\))} La aproximación bilateral implica la aproximación unilateral.
    \item[\(2 \Rightarrow 1\))] Si \(E\) puede ser aproximado por abiertos que lo contienen entonces es medible.
    \item[\(3 \Rightarrow 1\))] Si \(E\) puede ser aproximado por cerrados que están contenidos en él entonces es medible. 
  \end{itemize}
\end{ej_box}

\begin{ej_box}{Observación}
  Todo \(E \in \mathcal{M}_N\) se puede expresar como:
  \begin{align*}
    E = \left(\displaystyle \bigcap_{n = 1}^{\infty} G_n\right) \setminus N \quad \mbox{ o } \quad E = \left(\displaystyle \bigcup_{n = 1}^{\infty} D_n\right) \cup M
  \end{align*}
  donde \((G_n)_n\) es una sucesión decreciente de abiertos, \((D_n)_n\) es una sucesión creciente de cerrados y \(N, M\) son conjuntos de medida nula.\vspace{2ex}

  Además, los conjuntos \(\bigcap_{n = 1}^\infty G_n\) y \(\bigcup_{n = 1}^\infty D_n\) son conjuntos Borelianos.
\end{ej_box}
\vspace{3ex}

\subsection{\texorpdfstring{Medida de Borel y medida de Lebesgue en \(\mathbb{R}^N\). Corolario}{Medida de Borel y medida de Lebesgue en R\^N. Corolario}}\label{th:relacion_medida_Borel_Lebesgue}
\noindent Se tiene que:
\begin{align*}
  \left(\mathbb{R}^N, \mathcal{M}_N, \mu_N\right) = \left(\mathbb{R}^N, \widetilde{\mathcal{B}_N}, \widetilde{\mu_{N|_{\mathcal{B}_N}}}\right)
\end{align*}
\begin{dem_box}{Demostración}
  \begin{itemize} \leftskip -18pt
    \item \textit{Veamos que \(\mathcal{M}_N = \widetilde{\mathcal{B}_N}\)}:
    \begin{itemize} \leftskip -18pt
      \item \textit{Veamos que \(\widetilde{\mathcal{B}_N} \subseteq \mathcal{M}_N\)}: Sea \(E \in \widetilde{\mathcal{B}_N}\) donde:
      \begin{align*}
        \widetilde{\mathcal{B}_N} \coloneq \left\{A \cup M : A \in \mathcal{B}_N, \; M \in \mathcal{N}\right\}
      \end{align*}
      con:
      \begin{align*}
        \mathcal{N} \coloneq \left\{M \subseteq \mathbb{R}^N : \exists P \in \mathcal{B}_N \mbox{ tal que } M \subseteq P \mbox{ y } \mu_N(P) = 0\right\}
      \end{align*}
      Así, \(E = A \cup M\) con \(A \in \mathcal{B}_N\) y \(M \in \mathcal{N}\) entonces:
      \begin{align*}
        A \in \mathcal{B}_N \subseteq \mathcal{M}_N \implies A \in \mathcal{M}_N
      \end{align*}
      Como \(\mu_N\) es completa y \(M \subseteq P\) con \(P \in \mathcal{B}_N \subseteq \mathcal{M}_N\) y \(\mu_N(P) = 0\) entonces:
      \begin{align*}
        M \in \mathcal{M}_N
      \end{align*}
      Finalmente, como \(\mathcal{M}_N\) es \(\sigma\)-álgebra, es cerrada bajo uniones numerables:
      \begin{align*}
        E = A \cup M \in \mathcal{M}_N\\
      \end{align*}
      \item \textit{Veamos que \(\mathcal{M}_N \subseteq \widetilde{\mathcal{B}_N}\)}: Sea \(E \in \mathcal{M}_N\), por el \hyperref[th:caract_topologica_conjuntos_medibles_Lebesgue]{teorema de caracterización topológica de los conjuntos medibles de Lebesgue}, \(E = D \cup M\) con \(D\) conjunto Boreliano y \(\mu_N(M) = 0\).\vspace{10ex}
      
      Como \(M \in \mathcal{M}_N\) y \(\mu_N(M) = 0\) entonces existe una sucesión decreciente de conjuntos abiertos \((O_n)_n\) tal que:
      \begin{align*}
        M \subseteq O_n \quad \mbox{ y } \quad \mu_N(O_n) \xlongequal[\mu_N(M) = 0]{} \mu_N(O_n\setminus M) < \frac{1}{n}
      \end{align*}
      Sea \(G \coloneq \bigcap_{n = 1}^{\infty} O_n\) entonces tenemos que:
      \begin{itemize} \leftskip -12pt
        \item \(G\) es intersección numerable de abiertos entonces \(G \in \mathcal{B}_N\)
        \item \(M \subseteq G\)
        \item \(\mu_N(G) \leq \mu_N(O_n) < \frac{1}{n} \) para todo \(n \in \mathbb{N}\) luego \(\mu_N(G) = 0\)
      \end{itemize}
      Por tanto, \(M \in \mathcal{N}\) y así:
      \begin{align*}
        E = \underbracket{D}_{\in \mathcal{B}_N} \cup \underbracket{M}_{\in \mathcal{N}} \in \widetilde{\mathcal{B}_N}
      \end{align*}
    \end{itemize}
    \item \textit{Veamos que \(\mu_N = \widetilde{{\mu_N}_{|_{\mathcal{B}_N}}}\)}: Sea \(E \in \widetilde{\mathcal{B}_N}\) entonces:
    \begin{align*}
      E = A \cup M \quad \mbox{ con } \left\{
        \begin{array}{l}
          A \in \mathcal{B}_N\\
          M \subseteq P \in \mathcal{B}_N \mbox{ y } \mu_N(P) = 0
        \end{array}
      \right.
    \end{align*}
    Luego, por definición de \(\widetilde{\mu_N}\):
    \begin{align*}
      \widetilde{\mu_N}(E) = \widetilde{\mu_N}(A \cup M) = \mu_N(A)
    \end{align*}
    Como \(E \subseteq A \cup P\) entonces:
    \begin{align*}
      \mu_N(E) \leq \mu_N(A \cup P) \overset{\text{subadit.}}{\leq} \mu_N(A) + \mu_N(P) = \mu_N(A) \leq \mu_N(E)
    \end{align*}
    Por el criterio del sándwich se cumple la igualdad y por lo tanto:
    \begin{align*}
      \mu_N(E) = \widetilde{\mu_N}(E)
    \end{align*}
  \end{itemize}
\end{dem_box}
\end{document}