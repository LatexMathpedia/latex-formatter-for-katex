\documentclass[11pt,a4paper]{article}

\usepackage[spanish]{babel}
\usepackage{amsmath,amsfonts, amssymb, mathtools, bbm} % Podemos añadir amssymb, amsthm o bm
\usepackage[scr=boondoxo,cal=cm]{mathalpha}
\usepackage{graphicx, tikz, xparse}
\usepackage[top=2cm,bottom=2cm,left=3cm,right=3cm,marginparwidth=1.75cm]{geometry} % Este paquete permite modificar los márgenes del documento
\usepackage[colorlinks=true, allcolors=blue]{hyperref} % Se indica que los hipervínculos van todos en azul
\usepackage{setspace}
\usepackage{xcolor, tcolorbox}
\usepackage{cancel} %tachar cosas
\tcbuselibrary{breakable}
\usepackage{hyperref}
\usepackage{titlesec}
\usepackage{cancel}
\usepackage{pgfplots}  % Para graficar funciones en tikz
\usepackage{background}
\usetikzlibrary{arrows.meta}
\usepackage{extarrows}
\newcommand{\Lint}{\mathop{\int_{\!\!\!\!\!|\!}^{\!|\!}}}


% Definir la marca de agua
\backgroundsetup{
  position=current page.west,
  angle=90,
  nodeanchor=west,
  vshift=-1cm,
  hshift=-5.5cm,
  color=gray,
  scale=1,
  contents={\textsf{Created by Diego Díaz Mendaña $|$ Licensed under CC BY-NC-SA 4.0}}
}


\graphicspath{ {images/}}

%Colores
\definecolor{blanco}{HTML}{FFFFFF}
\definecolor{negro}{HTML}{000000}
\definecolor{azulSuave}{HTML}{6ac9d5}
\definecolor{naranjaSuave}{HTML}{d5956a}
\definecolor{verdeSuave}{HTML}{6ad578}
\definecolor{magenta}{HTML}{FF00FF}
\definecolor{dorado}{HTML}{ad8a1f}

\newtcolorbox{dem_box}[1]{
before=\par\smallskip\centering,
colframe=azulSuave!70,
colback=white,
fonttitle=\bfseries,
coltitle=negro,
title=#1,
flushleft title,
width=1\linewidth,
breakable = true
}

\newtcolorbox{ejem_box}[1]{
before=\par\smallskip\centering,
colframe=verdeSuave!70,
colback=white,
fonttitle=\bfseries,
coltitle=negro,
title=#1,
flushleft title,
width=1\linewidth,
breakable = true
}

\newtcolorbox{ej_box}[1]{
before=\par\smallskip\centering,
colframe=naranjaSuave!70,
colback=white,
fonttitle=\bfseries,
coltitle=negro,
title=#1,
flushleft title,
width=1\linewidth,
breakable = true
}


\setstretch{1.2}
\decimalpoint
\allowdisplaybreaks

\title{\textbf{Ejercicios}: ANÁLISIS III}
\author{Diego Díaz Mendaña}
%\date{Fecha}

\begin{document}
\NoBgThispage 
\maketitle
\vspace{20ex}
\section*{Disclaimer}
Estos apuntes son un resumen basado en el material proporcionado en la asignatura de ``Análisis II'' de la Universidad de Oviedo. Han sido escritos mediante las definiciones y demostraciones explicadas en clase. Todo el contenido ha sido organizado y formulado con fines educativos y no comerciales.\vspace{2ex}


\vspace{12ex}
\section*{Licencia de uso}
Apuntes Análisis III - Tema 1 © 2024 by Diego Díaz Mendaña is licensed under CC BY-NC-SA 4.0. To view a copy of this license, visit \href{https://creativecommons.org/licenses/by-nc-sa/4.0/}{https://creativecommons.org/licenses/by-nc-sa/4.0/}

\newpage

\hypersetup{linkcolor=black}
\tableofcontents
\hypersetup{linkcolor=blue}
\newpage

\section{\texorpdfstring{Ejercicios Medida exterior en \(\mathbb{R}^N\)}{Ejercicios Medida exterior en R^N}}
\subsection{Ejercicio 1}
\textit{Sea I un cubo en \(\mathbb{R}^N\) abierto y acotado. Demuestra que para todo \(\varepsilon > 0\) existen cubos compactos \(J\) y \(K\) tales que \(J \subseteq I \subseteq K\) y \(v(K) = v(I) < v(J) + \varepsilon\).}\vspace{2ex}

\begin{dem_box}{Demostración}
  Sea \(I\) cubo abierto y acotado de \(\mathbb{R}^N\) y fijamos \(\varepsilon > 0\) cualquiera, podemos escribir \(I\) como:
  \begin{align*}
    I = (a_1, b_1) \times (a_2, b_2) \times \dots \times (a_N, b_N)
  \end{align*}
  y como es acotado, tenemos que \( - \infty < a_i < b_i < \infty\) para todo \(i \in \{1, 2, \dots, N\}\).\vspace{2ex}

  Si definimos el conjunto \(K\) como:
  \begin{align*}
    K \coloneq [a_1, b_1] \times [a_2, b_2] \times \dots \times [a_N, b_N]
  \end{align*}
  Tenemos que claramente \(I \subseteq K\) y el volumen de \(K\) se puede calcular como:
  \begin{align*}
    v_N(K) = \prod_{i=1}^N (b_i - a_i) = v_N(I)
  \end{align*}  
  Ahora, para construir \(J\) consideramos \(\delta > 0\) suficientemente pequeño entonces:
  \begin{align*}
    J_\delta \coloneq [a_1 + \delta, b_1 - \delta] \times \dots \times [a_N + \delta, b_n - \delta]
  \end{align*}
  Y, para que no sea degenerado, necesitamos que:
  \begin{align*}
    \delta < \min_{i \in \{1, \dots , N\}} \frac{b_i - a_i}{2} 
  \end{align*}
  De esta forma, claramente \(J_\delta \subseteq I\) y es compacto. Además, su volumen es:
  \begin{align*}
    v_N(J_\delta) = \prod_{i = 1}^N (b_i - a_i - 2\delta)
  \end{align*}
  Si definimos la función \(\phi\) tal que:
  \begin{align*}
    \phi : \mathbb{R} & \longrightarrow \mathbb{R}\\
    \delta & \longmapsto \phi(\delta) =  \prod_{i = 1}^N (b_i - a_i - 2\delta)
  \end{align*}
  tenemos que está bien definida y es continua en \(\delta\) con \(\delta \in \left[0, \displaystyle \min_{i = 1, \dots N} \frac{b_i - a_i}{2}\right)\) y además, la función cumple que:
  \begin{align*}
    \phi(0) = v_N(I)
  \end{align*}
  Por continuidad de \(\phi\) en \(0\) existe un \(\delta \in \left(0, \displaystyle \min_{i = 1, \dots N} \frac{b_i - a_i}{2}\right)\) tal que:
  \begin{align*}
    \phi(\delta) > v_N(I) - \varepsilon
  \end{align*}
  Tomando este delta y definiendo \(J \coloneq J_\delta\) obtenemos un cubo compacto \(J\) contenido en \(I\) y que además:
  \begin{align*}
    v_N(J) = \phi(\delta) > v_N(I) - \varepsilon \implies v_N(J) + \varepsilon > v_N(I)
  \end{align*}
  Por lo tanto, hemos encontrado \(J, K\) cubos compactos de \(\mathbb{R}^N\) tales que \(\forall \varepsilon > 0\) se cumple:
  \begin{align*}
    J \subseteq I \subseteq K \quad \text{ tales que } v_N(K) = v_N(I) < v_N(J) + \varepsilon
  \end{align*}
\end{dem_box}
\newpage


\subsection{Ejercicio 2}
\textit{Sea \(I\) un cubo de \(\mathbb{R}^N\) y sean \(E_1, E_2\) los dos semiespacios cerrados de \(\mathbb{R}^N\) determinados por el hiperplano \(\{(x_i)_{i = 1}^N : x_j = \alpha\}\). Demuestra que \(I_1 \coloneq I \cap E_1\) y \(I_2 \coloneq I \cap E_2\) son dos cubos tales que:}
\begin{align*}
  I = I_1 \cup I_2 \quad \mbox{y} \quad v_N(I) = v_N(I_1) + v_N(I_2)
\end{align*}
\begin{dem_box}{Demostración}
  Como \(I\) es un cubo de \(\mathbb{R}^N\), podemos expresarlo como:
  \begin{align*}
    I = J_1 \times J_2 \times \dots \times J_N \quad \mbox{ con } J_i \mbox{ intervalo}
  \end{align*}
  Como \(E_1\) y \(E_2\) son dos semiespacios cerrados determinados por el hiperplano \(\{(x_i)_{i = 1}^N : x_j = \alpha\}\) entonces, podemos expresarlos como:
  \begin{align*}
    E_1 & = \left\{x \in \mathbb{R}^N : x_j \leq \alpha\right\} \quad \mbox{ y } \quad E_2 = \left\{x \in \mathbb{R}^N : x_j \geq \alpha\right\}
  \end{align*}
  Por tanto, \(I_1\) e \(I_2\) se pueden expresar como:
  \begin{align*}
    I_1 & \coloneq I \cap E_1 = J_1 \times \dots \times J_{j - 1} \times J^{(1)}_j \times J_{j + 1} \times \dots \times J_N \quad \mbox{ con } J^{(1)}_j = J_j \cap ( - \infty, \alpha]\\[2ex]
    I_2 & \coloneq I \cap E_2 = J_1 \times \dots \times J_{j - 1} \times J^{(2)}_j \times J_{j + 1} \times \dots \times J_N \quad \mbox{ con } J^{(2)}_j = J_j \cap [\alpha, \infty)
  \end{align*}
  Por lo tanto, la unión de ambos viene dada por:
  \begin{align*}
    I_1 \cup I_2 & = J_1 \times \dots \times J_{j - 1} \times (J^{(1)}_j \cup J^{(2)}_j) \times J_{j + 1} \times \dots \times J_N = \\[2ex]
    & = J_1 \times \dots \times J_{j - 1} \times (J_j \cap ( - \infty, \alpha] \cup J_j \cap [\alpha, \infty)) \times J_{j + 1} \times \dots \times J_N = \\[2ex]
    & = J_1 \times \dots \times J_{j - 1} \times J_j \times J_{j + 1} \times \dots \times J_N = I
  \end{align*}
  En cuanto al volumen, tenemos que:
  \begin{align*}
    v_N(I) = \prod_{i = 1}^N v_1(J_i)
  \end{align*}
  Como los \(J_i\) son iguales en \(I_1\) y \(I_2\) salvo en \(i = j\), tenemos que:
  \begin{align*}
    v_N(I_1) = \left(\prod_{\substack{i = 1\\i \neq j}}^N v_1(J_i)\right) \cdot v_1\left(J^{(1)}_j\right) \quad \mbox{ y } \quad v_N(I_2) = \left(\prod_{\substack{i = 1\\i \neq j}}^N v_1(J_i)\right) \cdot v_1\left(J^{(2)}_j\right)
  \end{align*}
  Entonces, la suma de ambos volúmenes es:
  \begin{align*}
    v_N(I_1) + v_N(I_2)&  = \left(\prod_{\substack{i = 1\\i \neq j}}^N v_1(J_i)\right) \cdot v_1\left(J^{(1)}_j\right) + \left(\prod_{\substack{i = 1\\i \neq j}}^N v_1(J_i)\right) \cdot v_1\left(J^{(2)}_j\right) =\\[2ex]
    & = \left(\prod_{\substack{i = 1\\i \neq j}}^N v_1(J_i)\right) \cdot \left(v_1\left(J_j^{(1)}\right) + v_1\left(J_j^{(2)}\right)\right) = \\[2ex]
    & = \left(\prod_{\substack{i = 1\\i \neq j}}^N v_1(J_i)\right) \cdot v_1(J_j) = v_N(I)
  \end{align*}
  Por lo tanto, hemos demostrado que:
  \begin{align*}
    I = I_1 \cup I_2 \quad \mbox{ y } \quad v_N(I) = v_N(I_1) + v_N(I_2)
  \end{align*}
\end{dem_box}
\newpage

\subsection{Ejercicio 3}
\noindent \textit{Prueba que si \(I\) es un cubo no degenerado de \(\mathbb{R}^N\) entonces \(I\) tiene cardinal no numerable.}\vspace{2ex}

\begin{dem_box}{Demostración}
  Supongamos que \(I\) es un cubo no degenerado con cardinal numerable. Podemos escribir \(I\) como:
  \begin{align*}
    I = \{x_1, x_2, x_3, \dots\}
  \end{align*}
  Por lo que podríamos calcular su volumen como:
  \begin{align*}
    v_N(I) & = \mu_N^*(I) = \mu_N^*(\{x_1, x_2, x_3\}) = \mu_N^*(\{x_1\} \cup \{x_2\} \cup \{x_3\} \cup \dots ) \overset{\text{subadit}}{\leq} \\[2ex]
    & \leq \sum_{i = 1}^{\infty} \mu_N^*(\{x_i\}) = \sum_{i = 1}^{\infty} 0 = 0
  \end{align*} 
  Lo cual es una contradicción, ya que \(I\) es un cubo no degenerado y, por tanto, su volumen es mayor que cero. Por lo tanto, \(I\) tiene cardinal no numerable.
\end{dem_box}
\newpage

\subsection{Ejercicio 4}
\textit{Sea \(\mathcal{S}\) la colección de todas las \(\sigma\)-álgebras sobre \(\mathbb{R}^N\) que contienen a todos los conjuntos abiertos de \(\mathbb{R}^N\) y consideramos la intersección:}
\begin{align*}
  \mathcal{B}_N \coloneq \displaystyle \bigcap_{\mathcal{C} \in \mathcal{S}} \mathcal{C}
\end{align*}
\textit{cuyos elementos son aquellos conjuntos que pertenecen a todas las \(\sigma\)-álgebras que constituyen \(\mathcal{S}\). Demuestra que \(\mathcal{B}_N\) es la \(\sigma\)-álgebra más pequeña que contiene a todos los conjuntos abiertos de \(\mathbb{R}^N\)}.\vspace{2ex}

\begin{dem_box}{Demostración}
  Veamos primero que es una \(\sigma\)-álgebra, para ello:
  \begin{enumerate} \leftskip -10pt
    \item \textit{Ver que \(\emptyset \in \mathcal{B}_N\):} Como \(\mathcal{B}_N\) se define como la intersección de \(\sigma\)-álgebras, y todas las \(\sigma\)-álgebras contienen al vacío, entonces:
    \begin{align*}
      \emptyset \in \displaystyle \bigcap_{\mathcal{C} \in \mathcal{S}} \mathcal{C}= \mathcal{B}_N
    \end{align*}
    \item \textit{Ver que es cerrada bajo uniones numerables:} Sea \(\{B_i\}_{i \in \mathbb{N}} \subseteq \mathcal{B}_N\), queremos ver:
    \begin{align*}
      \displaystyle \bigcup_{i \in \mathbb{N}} B_i \in \mathcal{B}_N
    \end{align*}
    Como \(B_i \in \mathcal{B}_N\) y \(\mathcal{B}_N\) es la intersección de todas las \(\sigma\)-álgebras que contienen a todos los abiertos de \(\mathbb{R}^N\) entonces para cada \(i\) se tiene \(B_i \in \mathcal{C}\) para todo \(\mathcal{C} \in \mathcal{S}\) y, al ser \(\mathcal{C}\) una \(\sigma\)-álgebra, es cerrada bajo uniones numerables y, por tanto:
    \begin{align*}
      \displaystyle \bigcup_{i \in \mathbb{N}} B_i \in \mathcal{C} \quad \forall \mathcal{C} \in \mathcal{S} \implies \displaystyle \bigcup_{i \in \mathbb{N}} B_i \in \displaystyle \bigcap_{\mathcal{C} \in \mathcal{S}} \mathcal{C} = \mathcal{B}_N
    \end{align*}
    \item \textit{Ver que es cerrada bajo complementos:} Sea \(B \in \mathcal{B}_N\) queremos ver que:
    \begin{align*}
      B^c = \mathbb{R}^N \setminus B \in \mathcal{B}_N
    \end{align*}
    Como \(B \in \mathcal{B}_N\) entonces \(B \in \mathcal{C}\) para todo \(\mathcal{C} \in \mathcal{S}\) por lo que, como cada \(\mathcal{C}\) es una \(\sigma\)-álgebra, entonces son cerradas bajo complementos, es decir:
    \begin{align*}
      B^c \in \mathcal{C} \quad \forall \mathcal{C} \in \mathcal{S} \implies B^c \in \displaystyle \bigcap_{\mathcal{C} \in \mathcal{S}} \mathcal{C} = \mathcal{B}_N
    \end{align*}
  \end{enumerate}
  \textit{Ahora que ya hemos visto que es \(\sigma\)-álgebra, veamos que \(\mathcal{B}_N\) contiene a todos los abiertos de \(\mathbb{R}^N\).}\vspace{2ex}
  
  Para ello, basta notar que por la propia definición de \(\mathcal{B}_N\), como esta es la intersección de todas las \(\sigma\)-álgebras que contienen todos los abiertos de \(\mathbb{R}^N\), entonces \(\mathcal{B}_N\) contendrá también todos los abiertos de \(\mathbb{R}^N\).\vspace{20ex}
  
  \textit{Finalmente, veamos que es la más pequeña que contiene a todos los abiertos de \(\mathbb{R}^N\), es decir, veamos que dado \(B \in \mathcal{B}_N\) entonces \(B \in \mathcal{C}\) para todo \(\mathcal{C} \in \mathcal{S}\) y, por tanto, \(\mathcal{B}_N \subseteq \mathcal{C} \quad \forall \mathcal{C} \in \mathcal{S}\).}\vspace{2ex}
  
  Basta notar que, sea \(B \in \mathcal{B}_N\), por la definición de \(\mathcal{B}_N\) se tiene que:
  \begin{align*}
    B \in \mathcal{B}_N = \displaystyle \bigcap_{\mathcal{C} \in \mathcal{S}} \mathcal{C} \implies B \in \mathcal{C}, \quad \forall \mathcal{C} \in \mathcal{S} \implies \mathcal{B} \subseteq \mathcal{C}, \quad \forall \mathcal{C} \in \mathcal{S}
  \end{align*}
\end{dem_box}
\newpage

\subsection{Ejercicio 5}
\noindent\textit{Sea \(A = \mathbb{Q} \cap [0, 1]\):}
\begin{itemize}
  \item \textit{Demuestra que para todo cubrimiento finito de \(A\) formado por intervalos abiertos, \(\{I_i\}_{i = 1}^n\) se cumple que:}
  \begin{align*}
    \displaystyle \sum_{i = 1}^{n} v_1(I_i) \geq 1
  \end{align*}
  \textit{¿Cuánto vale \(\mu_1^*(A)\)?}\vspace{2ex}

  \begin{dem_box}{Demostración}
    Sea \(\{I_i\}_{i = 1}^n\) familia de intervalos abiertos de \(\mathbb{R}\) que recubre \(A = \mathbb{Q} \cap [0, 1]\), si tenemos dos intervalos solapados o contiguos, agrupamos ambos en uno único formado por su unión. Repetimos esto hasta obtener una familia finita de intervalos abiertos disjuntos por parejas:
    \begin{align*}
      (a_1, b_1), \dots , (a_m, b_m) \quad \mbox{ con } a_1 < a_2 <\dots < a_m
    \end{align*}
    Esta nueva familia sigue recubriendo \([0, 1]\) y además:
    \begin{align*}
      \displaystyle \sum_{i = 1}^{n} v_1(I_i) = \displaystyle \sum_{k = 1}^{m} (b_k - a_k)
    \end{align*}
    Como \((a_k, b_k)_{k = 1}^{m}\) cubren \([0, 1]\), deben existir índices tales que el más a la izquierda comienza a la izquierda de 0 (o en 0) y el más a la derecha termina más a la derecha de 1 (o en 1). Más formalmente:
    \begin{align*}
      a_1 \leq 0 \quad \mbox{ y } b_m \geq 1
    \end{align*}
    Entonces, la suma de longitudes satisface:
    \begin{align*}
      \displaystyle \sum_{k = 1}^{m} (b_k - a_k) = (b_m - a_1) - \displaystyle \sum_{k = 1}^{m - 1} (a_{k + 1} - b_k)
    \end{align*}
    Los términos \((a_{k + 1} - b_k)\) son no negativos por lo tanto:
    \begin{align*}
      \displaystyle \sum_{k = 1}^{m} (b_k - a_k) \geq b_m - a_1 \geq 1 - 0 = 1
    \end{align*}
    En consecuencia:
    \begin{align*}
      \displaystyle \sum_{i = 1}^{n} v_1(I_i) = \displaystyle \sum_{k = 1}^{m} (b_k - a_k) \geq 1
    \end{align*}
    Por lo que ya hemos demostrado la primera parte.\vspace{20ex}

    Dado que la medida exterior de \(A\), es decir, \(\mu_1^*(A)\) se define como el ínfimo de las sumas de las longitudes de los recubrimientos abiertos numerables de \(A\) y sabiendo que estos son \(\geq 1\) entonces:
    \begin{align*}
      \mu_1^*(A) \geq 1
    \end{align*}
    Por otro lado, sea \(\varepsilon > 0\) cualquiera, podemos recubrir el intervalo \([0, 1]\) con un único intervalo abierto \(I = \left( - \frac{\varepsilon}{2}, 1 + \frac{\varepsilon}{2}\right)\) cuya longitud es:
    \begin{align*}
      v_1(I) = 1 + \varepsilon
    \end{align*}
    Como el \(\varepsilon\) es arbitrario, tenemos que para todo \(\varepsilon > 0\) existe un recubrimiento abierto (incluso un solo intervalo) de \(A\) con suma de longitudes menor que \(1 + \varepsilon\) y, por tanto, tomando \(\varepsilon \to 0\) se obtiene el ínfimo sobre todos los recubrimientos abiertos de \(A\):
    \begin{align*}
      \mu_1^*(A) = 1
    \end{align*}
  \end{dem_box}
  \vspace{3ex}

  \item \textit{Deduce del apartado anterior que \(A\) no es compacto.}\vspace{2ex}
  \begin{dem_box}{Demostración}
    Supongamos \(A\) compacto, entonces todo recubrimiento abierto de \(A\) tiene un subrecubrimiento finito. En particular, el recubrimiento abierto formado por todos los intervalos abiertos con extremos racionales que contienen a \(A\) tendrá un subrecubrimiento finito \(\{I_i\}_{i = 1}^n\). Pero esto contradice lo demostrado en el apartado anterior, ya que:
    \begin{align*}
      \displaystyle \sum_{i = 1}^{n} v_1(I_i) \geq 1
    \end{align*}
    Por lo tanto, \(A\) no es compacto.
  \end{dem_box}
  \vspace{2ex}

  \item \textit{Demuestra que la aplicación \(\mathscr{m}\) definida sobre la \(\sigma\)-álgebra de Borel de \(\mathbb{R}^N\) como:}
  \begin{align*}
    \mathscr{m}(A) = \inf \left\{\displaystyle \sum_{i = 1}^{n} v(I_i) : n \in \mathbb{N}, \quad A \subseteq \displaystyle \bigcup_{i = 1}^{n} I_i, \quad I_i \mbox{ intervalos abiertos de } \mathbb{R}^N\right\}
  \end{align*}
  \textit{no es una medida.}\vspace{2ex}
  \begin{dem_box}{Demostración}
    Consideremos el conjunto \(A = \mathbb{Q} \cap [0, 1]\) del apartado anterior, tenemos que \(\mathscr{m}(A) = 1\). Por otro lado, podemos escribir \(A\) como la unión numerable de conjuntos unitarios:
    \begin{align*}
      A = \displaystyle \bigcup_{q \in A} \{q\}
    \end{align*}
    Pero, para cada conjunto unitario se tiene que:
    \begin{align*}
      \mathscr{m}(\{q\}) = 0
    \end{align*}
    Por lo que, si \(\mathscr{m}\) fuera una medida, se tendría que:
    \begin{align*}
      \mathscr{m}(A) = \mathscr{m}\left(\displaystyle \bigcup_{q \in A} \{q\}\right) = \displaystyle \sum_{q \in A} \mathscr{m}(\{q\}) = 0
    \end{align*}
    Lo cual es una contradicción, ya que hemos visto que \(\mathscr{m}(A) = 1\). Por lo tanto, la aplicación \(\mathscr{m}\) no es una medida.
  \end{dem_box}
\end{itemize}

\newpage
\subsection{Ejercicio 6}
\textit{Demuestra que existe un conjunto abierto \(O\) en \(\mathbb{R}\) y un \(\varepsilon > 0\) tal que para todo cubrimiento finito de \(O\) formado por intervalos abiertos \(\{J_i\}_{i = 1}^n\), se cumple que:}
\begin{align*}
  \mu\left(\displaystyle \bigcup_{i = 1}^{n} J_i \setminus O\right) > \varepsilon
\end{align*}
\begin{dem_box}{Demostración}
  Buscamos un \(O \in \tau_\mathbb{R}\) y \(\varepsilon > 0\) tal que, dado \(\{J_i\}_{i = 1}^n\) recubrimiento finito y abierto se tenga que:
  \begin{align*}
      \mu\left(\displaystyle \bigcup_{i = 1}^{n} J_i \setminus O\right) > \varepsilon
  \end{align*}
  Necesitamos un \(O\) que tenga infinitos conjuntos de tal forma que se necesite algún conjunto \(J_i\) infinito para recubrirlo al completo. Por ejemplo:
  \begin{align*}
    O = \displaystyle \bigcup_{k \in \mathbb{Z}} (2k, 2k + 1)
  \end{align*}
  Donde claramente \(O\) es abierto ya que \((2k, 2k + 1)\) son abiertos y la unión numerable de abiertos es abierta. Además, tenemos que cada intervalo es de longitud 1 y entre cada dos componentes hay un hueco:
  \begin{align*}
    \dots\cup ( - 2, - 1) \overset{[ - 1, 0]}{\cup }(0, 1) \overset{[1, 2]}{\cup} (2, 3) \cup \dots 
  \end{align*}
  Y estos ``huecos'' son de longitud 1 también.\vspace{2ex}

  Como \(\{J_i\}_{i = 1}^n\) finita y recubren \(O\), para cada \(i\) consideramos:
  \begin{align*}
    \mathcal{S}_i \coloneq \{k \in \mathbb{Z} : (2k, 2k + 1) \subseteq J_i\}
  \end{align*}
  Como debe recubrir todo \(O\) tenemos que:
  \begin{align*}
    \displaystyle \bigcup_{i = 1}^{n} \mathcal{S}_i = \mathbb{Z}
  \end{align*}
  Pero entonces algún \(S_i\) tiene que ser infinito, ya que no puede ser que la unión no finita de intervalos no finitos de un conjunto infinito. Por tanto, \(\exists j \in \mathbb{N}\) con \(\mathcal{S}_j\) infinito.\vspace{2ex}

  Ahora, si \(\mathcal{S}_j\) infinito, entonces abarca infinitas componentes \((2k, 2k + 1)\). Como los \(k\) son distintos y las componentes son disjuntas (separadas por huecos de longitud 1), el intervalo \(J_j\) contiene también infinitos de esos huecos disjuntos y cada uno tiene medida 1. Por lo tanto:
  \begin{align*}
    \mu(J_j\setminus O) \geq \displaystyle \sum_{k \mbox{ tq } (2k, 2k + 1) \subseteq J_j} \mu((2k + 1, 2k + 2)) = \displaystyle \sum_{m = 1}^{\infty} 1 = \infty
  \end{align*}
  En particular, \(\mu(J_j \setminus O ) = \infty\) por lo que, aplicando monotonía:
  \begin{align*}
    \mu\left(\displaystyle \bigcup_{i = 1}^{n} J_i \setminus O\right) \geq \mu(J_j \setminus O) = \infty > \varepsilon \quad \forall \varepsilon > 0
  \end{align*}
\end{dem_box}
\newpage

\subsection{Ejercicio 7}
\textit{Sea \(A\) un subconjunto de \(\mathbb{R}^N\) demuestra que todo cubrimiento abierto de \(A\) contiene un subrecubrimiento numerable.}\vspace{2ex}

\begin{dem_box}{Demostración}
  Sea \(\{I_\alpha\}_{\alpha \in \Lambda}\) un cubrimiento abierto de \(A\), es decir:
  \begin{align*}
    A \subseteq \displaystyle \bigcup_{\alpha \in \Lambda} I_\alpha
  \end{align*}
  Donde cada \(I_\alpha\) es un conjunto abierto de \(\mathbb{R}^N\). Como cada \(I_\alpha\) es abierto, para cada \(x \in I_\alpha\) existe un cubo \(J_{x, \alpha}\) con extremos racionales tal que:
  \begin{align*}
    x \in J_{x, \alpha} \subseteq I_\alpha
  \end{align*}
  Por tanto, podemos escribir:
  \begin{align*}
    A \subseteq \displaystyle \bigcup_{\alpha \in \Lambda} I_\alpha = \displaystyle \bigcup_{\alpha \in \Lambda} \displaystyle \bigcup_{x \in I_\alpha} J_{x, \alpha}
  \end{align*}
  Ahora, consideremos la familia de todos los cubos con extremos racionales:
  \begin{align*}
    \mathcal{J} = \{J \subseteq \mathbb{R}^N : J \mbox{ es un cubo con extremos racionales}\}
  \end{align*}
  Esta familia es numerable, ya que se puede establecer una biyección entre los extremos racionales y \(\mathbb{Q}^{2N}\) que es numerable. Por tanto, podemos escribir:
  \begin{align*}
    A \subseteq \displaystyle \bigcup_{J \in \mathcal{J}} J
  \end{align*}
  Finalmente, definimos la familia:
  \begin{align*}
    \mathcal{J}_A = \{J \in \mathcal{J} : J \cap A \neq \emptyset\}
  \end{align*}
  Que es numerable al ser subconjunto de \(\mathcal{J}\) y que además cubre a \(A\):
  \begin{align*}
    A \subseteq \displaystyle \bigcup_{J \in \mathcal{J}_A} J
  \end{align*}
  Por lo tanto, hemos encontrado un subrecubrimiento numerable de \(A\).
\end{dem_box}
\newpage

\subsection{Ejercicio 8}
\noindent \textit{Responder a las siguientes cuestiones de forma justificada:}
\begin{enumerate}
  \item[a)] \textit{Sea \(\sum_{i = 1}^\infty x_i\) serie de términos no negativos. Demostrar que es convergente si y solo si existe un real \(s\) tal que para todo \(\varepsilon > 0\) existe un subconjunto finito \(I \subseteq \mathbb{N}\) tal que para todo subconjunto finito \(H \supseteq I\) se cumple:}
  \begin{align*}
    \left|s - \displaystyle \sum_{i \in H} x_i\right| < \varepsilon
  \end{align*}
  \begin{dem_box}{Demostración}
    \begin{itemize}
      \item[\(\Rightarrow\))] Sea \(\sum_{i = 1}^\infty x_i\) serie de términos no negativos convergente, por definición sabemos que es convergente a \(s\) si \(\exists n_0 \in \mathbb{N}\) tal que:
      \begin{align*}
        \left|s - \displaystyle \sum_{i = 1}^{n} x_i\right| < \varepsilon \quad \forall n \geq n_0
      \end{align*}
      Sea \(H \supseteq \{1, \dots, n_0\}\) entonces:
      \begin{align*}
        \displaystyle \sum_{i = 1}^{n_0} x_i \leq \displaystyle \sum_{i \in H} x_i \leq s
      \end{align*}
      Al ser los términos negativos y \(s\) la suma total, por tanto:
      \begin{align*}
        0 \leq s - \displaystyle \sum_{i \in H} x_i \leq s - \displaystyle \sum_{i = 1}^{n_0} x_i < \varepsilon
      \end{align*}
      \item[\(\Leftarrow\))] Sea \(\sum_{i = 1}^\infty x_i\) serie de términos no negativos tal que \(\exists s \in \mathbb{R}\) con la propiedad dada. Consideremos la sucesión de sumas parciales:
      \begin{align*}
        S_n = \displaystyle \sum_{i = 1}^{n} x_i
      \end{align*}
      Dicha sucesión es monótonamente creciente, ya que los términos son no negativos. Por tanto, si demostramos que está acotada superiormente, entonces será convergente. Sea \(\varepsilon = 1\), por hipótesis existe un subconjunto finito \(I \subseteq \mathbb{N}\) tal que para todo subconjunto finito \(H \supseteq I\) se cumple:
      \begin{align*}
        \left|s - \displaystyle \sum_{i \in H} x_i\right| < 1
      \end{align*}
      Sea \(n_0 = \max(I)\), entonces para todo \(n \geq n_0\) se tiene:
      \begin{align*}
        \left|s - S_n\right| = \left|s - \displaystyle \sum_{i = 1}^{n} x_i\right| < 1 \implies s - 1 < S_n < s + 1
      \end{align*}
      Por tanto, la sucesión \(\{S_n\}_{n = 1}^\infty\) está acotada superiormente por \(s + 1\) y, por tanto, es convergente.
    \end{itemize}
  \end{dem_box}
  \vspace{3ex}
\end{enumerate}
\newpage

\subsection{Ejercicio 9}
\textit{Demuestra que para todo cubo degenerado \(C\) de \(\mathbb{R}^N\) se cumple que \(\mu_N^*(C) = 0\).}\vspace{2ex}

\begin{dem_box}{Demostración}
  Un cubo \(C\) de \(\mathbb{R}^N\) es degenerado si contiene algún intervalo degenerado, es decir:
  \begin{align*}
    C = I_1 \times \dots \times I_N
  \end{align*}
  con \(I_i\) intervalo degenerado. Supongamos sin pérdida de generalidad que \(I_1 = \{a\}\), por tanto, \(I_1\) degenerado. Sea \(\varepsilon > 0\) cualquiera, consideremos el cubo abierto:
  \begin{align*}
    J_\varepsilon = (a - \varepsilon, a + \varepsilon) \times (b_2, c_2) \times \dots \times (b_N, c_N)
  \end{align*}
  Donde \(I_i = [b_i, c_i]\) para \(i = 2, \dots, N\). Claramente \(C \subseteq J_\varepsilon\) y:
  \begin{align*}
    v_N(J_\varepsilon) & = v_1((a - \varepsilon, a + \varepsilon)) \cdot \displaystyle \prod_{i = 2}^{N} v_1((b_i, c_i)) = \\[2ex]
    & = (2\varepsilon) \cdot \displaystyle \prod_{i = 2}^{N} (c_i - b_i) \xrightarrow[\varepsilon \to 0]{} 0
  \end{align*}
  Por tanto, por la definición de medida exterior:
  \begin{align*}
    \mu_N^*(C) & = \inf \left\{\displaystyle \sum_{i = 1}^{\infty} v_N(J_i) : C \subseteq \displaystyle \bigcup_{i = 1}^{\infty} J_i, \quad J_i \mbox{ cubos abiertos de } \mathbb{R}^N\right\} \leq \\[2ex]
    & \leq v_N(J_\varepsilon) = (2\varepsilon) \cdot \displaystyle \prod_{i = 2}^{N} (c_i - b_i) \quad \forall \varepsilon > 0
  \end{align*}
  Tomando \(\varepsilon \to 0\) se obtiene:
  \begin{align*}
    \mu_N^*(C) = 0
  \end{align*}
\end{dem_box}

\newpage
\subsection{Ejercicio 10}
\textit{Demuestra que la medida exterior de Lebesgue es invariante por translaciones, es decir, que para todo \(A \subseteq \mathbb{R}^N\) y todo \(p \in \mathbb{R}^N\) se tiene:}
\begin{align*}
  \mu_N^*(A) = \mu_N^*(p + A)
\end{align*}
\begin{dem_box}{Demostración}
  Sea \(A \subseteq \mathbb{R}^N\) y \(p \in \mathbb{R}^N\) cualquiera pero fijo, por definición de medida exterior sabemos que:
  \begin{align*}
    \mu_N^*(A) = \inf \left\{\displaystyle \sum_{i = 1}^{\infty} v_N(J_i) : \{J_i\}_{i \in \mathbb{N}} \mbox{ colección cubos abiertos tq: } A \subseteq \displaystyle \bigcup_{i = 1}^{\infty} J_i\right\}
  \end{align*}
  Si consideramos una colección cualquiera de cubos abiertos \(\{J_i\}_{i \in \mathbb{N}}\) que recubra \(A\) y trasladamos el recubrimiento con el vector \(p\) obtenemos:
  \begin{align*}
    p + A \subseteq \displaystyle \bigcup_{i = 1}^{\infty} (p + J_i)
  \end{align*}
  Podemos notar que cada conjunto \(p + J_i\) es un nuevo cubo de \(\mathbb{R}^N\) y además tenemos que:
  \begin{align*}
    v_N(p + J_i) = \prod_{i = 1}^{N} |p + b_i - (p + a_i)| = \prod_{i = 1}^N |b_i - a_i| = v_N(J_i) 
  \end{align*}
  Por tanto, el volumen de los cubos es invariante bajo traslaciones y, en consecuencia, como \(\{p + J_i\}_{i \in \mathbb{N}}\) es recubrimiento de \(p + A\) y sus volúmenes coinciden con los del recubrimiento de \(A\) se obtiene tomando ínfimos que:
  \begin{align*}
    \mu_N^*(p + A) \leq \mu_N^*(A) 
  \end{align*}
  Y de forma análoga se obtiene la desigualdad opuesta con lo que queda demostrado el ejercicio.
\end{dem_box}
\newpage

\subsection{Ejercicio 11}
\textit{Demostrar que \(\mu_N^*(A)\) es igual a cualquiera de los siguientes valores:}
\begin{align*}
  (1) & \inf\left\{\displaystyle \sum_{i = 1}^{\infty} v_n(I_i) : I_i \mbox{ cubos abiertos tq: } A \subseteq \displaystyle \bigcup_{i = 1}^{\infty} I_i\right\}\\[2ex]
  (2) & \inf\left\{\displaystyle \sum_{i = 1}^{\infty} v_N(I_i) : I_i \mbox{ cubos abiertos acotados tq: } A \subseteq \displaystyle \bigcup_{i = 1}^{\infty}I_i\right\}\\[2ex]
  (3) & \inf\left\{\displaystyle \sum_{i = 1}^{\infty} v_N(I_i) : I_i \mbox{ cubos compactos tq: } A \subseteq \displaystyle \bigcup_{i = 1}^{\infty} I_i\right\}
\end{align*}
\begin{dem_box}{Demostración}
  \begin{enumerate} \leftskip -10pt
    \item[(1)] Es la propia definición de medida exterior de Lebesgue, no hay nada que demostrar.
    \item[(2)] Se consideran los dos contenidos:
    \begin{itemize} \leftskip -10pt
      \item[\(1 \leq 2\))] Trivial, toda familia de cubos abiertos acotados es, en particular, una familia de cubos abiertos.
      \item[\(2 \leq 1\))] Sea \(\varepsilon > 0\) cualquiera, por definición de medida exterior, existe una familia de cubos abiertos \(\{I_i\}_{i \in \mathbb{N}}\) no necesariamente acotados tal que:
      \begin{align*}
        A \subseteq \displaystyle \bigcup_{i = 1}^{\infty} I_i \quad \mbox{ y } \quad \displaystyle \sum_{i = 1}^{\infty} v_N(I_i) \leq  \mu_N^*(A) + \varepsilon
      \end{align*}
      Ahora, si un cubo abierto tuviera volumen infinito, entonces dicha suma sería infinita. En dicho caso, se dan dos posibilidades:
      \begin{itemize} \leftskip -10pt
        \item Podemos eliminar dicho cubo del recubrimiento sin dejar de cubrir \(A\) (es decir, \(A\) ya estaba contenido en la unión del resto de cubos). En este caso, repetimos el proceso hasta que no queden cubos de volumen infinito.
        \item Podemos sustituir dicho cubo por una familia de cubos abiertos acotados que lo recubran (por ejemplo, una familia numerable de cubos centrados en el mismo punto pero con lados cada vez más grandes). En este caso, el volumen total de la nueva familia será infinito, por lo que no nos sirve.
      \end{itemize}
      Así, podemos suponer sin pérdida de generalidad que todos los cubos \(I_i\) son de volumen finito. Entonces, para cada \(i \in \mathbb{N}\) existe un cubo abierto acotado \(J_i\) tal que:
      \begin{align*}
        A \subseteq \displaystyle \bigcup_{i = 1}^{\infty} J_i \quad \mbox{ y } \quad v_N(J_i) \leq v_N(I_i) + \frac{\varepsilon}{2^i}
      \end{align*}
      Por tanto, tenemos que:
      \begin{align*}
        \displaystyle \sum_{i = 1}^{\infty} v_N(J_i) \leq \displaystyle \sum_{i = 1}^{\infty} \left(v_N(I_i) + \frac{\varepsilon}{2^i}\right) = \displaystyle \sum_{i = 1}^{\infty} v_N(I_i) + \varepsilon \leq \mu_N^*(A) + 2\varepsilon
      \end{align*}
      Lo cual demuestra que el ínfimo sobre cubos abiertos acotados es menor o igual que \(\mu_N^*(A)\). Como \(\varepsilon > 0\) es arbitrario, queda demostrado.
    \end{itemize}
    \item[(3)] Se consideran los dos contenidos:
    \begin{itemize} \leftskip -10pt
      \item[\(2 \leq 3\))] Trivial, todo cubo compacto es, en particular, acotado y su interior es un cubo abierto con el mismo volumen.
      \item[\(3 \leq 2\))] Sea \(\varepsilon > 0\) cualquiera, por definición de medida exterior, existe una familia de cubos abiertos acotados \(\{I_i\}_{i \in \mathbb{N}}\) tal que:
      \begin{align*}
        A \subseteq \displaystyle \bigcup_{i = 1}^{\infty} I_i \quad \mbox{ y } \quad \displaystyle \sum_{i = 1}^{\infty} v_N(I_i) \leq  \mu_N^*(A) + \varepsilon
      \end{align*}
      Ahora, para cada \(i \in \mathbb{N}\) consideremos el cubo compacto \(J_i = \overline{I_i}\) (el cierre de \(I_i\)). Entonces, tenemos que:
      \begin{align*}
        A \subseteq \displaystyle \bigcup_{i = 1}^{\infty} J_i \quad \mbox{ y } \quad v_N(J_i) = v_N(I_i)
      \end{align*}
      Por tanto, se tiene que:
      \begin{align*}
        \displaystyle \sum_{i = 1}^{\infty} v_N(J_i) = \displaystyle \sum_{i = 1}^{\infty} v_N(I_i) \leq \mu_N^*(A) + \varepsilon
      \end{align*}
      Lo cual demuestra que el ínfimo sobre cubos compactos es menor o igual que \(\mu_N^*(A)\). Como \(\varepsilon > 0\) es arbitrario, queda demostrado.
    \end{itemize}
  \end{enumerate}
\end{dem_box}

\newpage

\section{\texorpdfstring{Conjuntos medibles en \(\mathbb{R}^N\)}{Conjuntos medibles en R^N}}
\subsection{Ejercicio 1}
\noindent \textit{Sea \(\{E_n\}_{n = 1}^\infty\) sucesión de conjuntos medibles en \(\mathbb{R}^N\) definimos:}
\begin{align*}
  \limsup_n E_n \coloneq \displaystyle \bigcap_{n = 1}^{\infty} \displaystyle \bigcup_{k = n}^{\infty} E_k \quad \mbox{ y } \quad \liminf_n E_n \coloneq \displaystyle \bigcup_{n = 1}^{\infty} \displaystyle \bigcap_{k = n}^{\infty} E_k
\end{align*}
\begin{itemize}
  \item \textit{Probar las siguientes igualdades:}
  \begin{align*}
    \liminf_n E_n & = \left\{x \in \mathbb{R}^N : \exists k_x \in \mathbb{N} \mbox{ tq } x \in E_n, \quad \forall n \geq k_x\right\}\\
    \limsup_n E_n & = \left\{x \in \mathbb{R}^N : \exists \mbox{ infinitos } n \in \mathbb{N} \mbox{ tq } x \in E_n\right\}
  \end{align*}
  \begin{dem_box}{Demostración}
    \begin{itemize} \leftskip -10pt
      \item[\(\subseteq\))] Sea \(x \in \liminf_n E_n\), por definición de \(\liminf\) tenemos que:
      \begin{align*}
        x \in \displaystyle \bigcup_{n = 1}^{\infty} \displaystyle \bigcap_{k = n}^{\infty} E_k
      \end{align*}
      Por tanto, \(\exists n \in \mathbb{N}\) tal que \(x \in E_k\) para todo \(k \geq n\). Basta \(k_x \coloneq n\) y \(n \coloneq k\).
      \item[\(\supseteq\))] Sea \(x_0 \in \left\{x \in \mathbb{R}^N : \exists k_x \in \mathbb{N} \mbox{ tq } x \in E_n, \quad \forall n \geq k_x\right\}\) entonces tenemos que para \(x_0\) existe \(k_{x_0} \in \mathbb{N}\) tal que \(x \in E_n\) para cualquier \(n \geq k_{x_0}\), por tanto:
      \begin{align*}
        x_0 \in \displaystyle \bigcap_{n = k_{x_0}}^{\infty} E_n \implies x_0 \in \bigcup_{m = 1}^\infty \displaystyle \bigcap_{n = m}^{\infty} E_n \implies x_0 \in \liminf_n E_n
      \end{align*}
    \end{itemize}
    Para el caso de \(\limsup\):
    \begin{itemize} \leftskip -10pt
      \item[\(\subseteq\))] Sea \(x \in \limsup_n E_n\), por definición de \(\limsup\) tenemos que:
      \begin{align*}
        x \in \displaystyle \bigcap_{n = 1}^{\infty} \displaystyle \bigcup_{k = n}^{\infty} E_k
      \end{align*}
      Por tanto, para todo \(n \in \mathbb{N}\) existe \(k_n \geq n\) tal que \(x \in E_{k_n}\). Por lo tanto, existen infinitos \(n\) tales que \(x \in E_n\).
      \item[\(\supseteq\))] Sea \(x_0 \in \left\{x \in \mathbb{R}^N : \exists \mbox{ infinitos } n \in \mathbb{N} \mbox{ tq } x \in E_n\right\}\), entonces existen infinitos \(n_k\) tales que \(x_0 \in E_{n_k}\). Por tanto, para cada \(m \in \mathbb{N}\) existe algún \(n_k \geq m\) tal que \(x_0 \in E_{n_k}\), es decir:
      \begin{align*}
        x_0 \in \displaystyle \bigcup_{k = m}^{\infty} E_k \quad \forall m \in \mathbb{N} \implies x_0 \in \displaystyle \bigcap_{m = 1}^{\infty} \displaystyle \bigcup_{k = m}^{\infty} E_k = \limsup_n E_n
      \end{align*}
    \end{itemize}
  \end{dem_box}
  \vspace{10ex}

  \item \textit{Demuestra que \(\liminf_n E_n \subseteq \limsup_n E_n\).}\vspace{2ex}
  \begin{dem_box}{Demostración}
    Sea \(x \in \liminf_n E_n\) entonces tenemos que:
    \begin{align*}
      \exists k_x \in \mathbb{N} \mbox{ tq } x \in E_n, \quad \forall n \geq k_x \implies \exists \mbox{ infinitos } n \mbox{ tq } x \in E_n \implies x \in \limsup_n E_n
    \end{align*}
  \end{dem_box}
  \vspace{2ex}

  \item \textit{Demuestra que los conjuntos \(\liminf_n E_n\) y \(\limsup_n E_n\) son medibles.}\vspace{2ex}
  \begin{dem_box}{Demostración}
    Veamos el caso del límite inferior, el superior es análogo. Basta notar que las uniones e intersecciones numerables de medibles son medibles, entonces:
    \begin{align*}
      E_n \mbox{ medible } \implies \displaystyle \bigcap_{k = n}^{\infty} E_k \mbox{ medible } \implies \displaystyle \bigcup_{n = 1}^{\infty} \displaystyle \bigcap_{k = n}^{\infty} E_k \quad \mbox{ medible}
    \end{align*}
  \end{dem_box}
\end{itemize}

\newpage
\subsection{Ejercicio 2}
\textit{Sea \(A \in \mathcal{M}_N\) con \(\mu(A) < \infty\) y \(\varepsilon > 0\). Prueba que existe un conjunto compacto \(K \subseteq A\) tal que \(\mu(A \setminus K) < \varepsilon\). ¿Es cierta la afirmación si \(\mu(A) = \infty\)?}\vspace{2ex}

\begin{dem_box}{Demostración}
  Fijamos \(\varepsilon > 0\) cualquiera y consideramos el conjunto \(A \in \mathcal{M}_N\) con \(\mu(A) < \infty\). Ahora, definimos los conjuntos:
  \begin{align*}
    A_n \coloneq A \cap [ - n, n]^N \quad \forall n \in \mathbb{N}
  \end{align*}
  Por lo tanto, podemos expresar \(A\) como:
  \begin{align*}
    A = \displaystyle \bigcup_{n = 1}^{\infty} A_n
  \end{align*}
  Como \(A_n \subseteq A_{n + 1}\) para todo \(n\), tenemos que:
  \begin{align*}
    \mu(A) = \mu \left(\displaystyle \bigcup_{n = 1}^{\infty} A_n\right) = \lim_{n \to \infty} \mu(A_n)
  \end{align*}
  Como \(\mu(A_n) \to \mu(A)\) entonces \(\exists n_0 \in \mathbb{N}\) tal que:
  \begin{align*}
    \mu(A) - \mu(A_{n_0}) < \frac{\varepsilon}{2}
  \end{align*}
  Aplicando la caracterización topológica de los conjuntos medibles, existe un conjunto cerrado \(C \subseteq A_{n_0}\) tal que:
  \begin{align*}
    C \subseteq A_{n_0} \quad \mbox{ y } \quad \mu(A_{n_0} \setminus C) < \frac{\varepsilon}{2}
  \end{align*}
  Como \(C \subseteq A_{n_0} \subseteq [ - n_0, n_0]^N\) entonces \(C\) es acotado y cerrado, por lo que es compacto. Por lo tanto, si tomamos \(K \coloneq C\) se tiene que:
  \begin{align*}
    \mu(A \setminus K) & = \mu((A \setminus A_{n_0}) \cup (A_{n_0} \setminus C)) \overset{\text{subadit}}{\leq} \mu(A \setminus A_{n_0}) + \mu(A_{n_0} \setminus C) < \\[2ex]
    & < \mu(A \setminus A_{n_0}) + \frac{\varepsilon}{2} \overset{\mu(A) < \infty}{ = } (\mu(A) - \mu(A_{n_0})) + \frac{\varepsilon}{2} < \frac{\varepsilon}{2} + \frac{\varepsilon}{2} = \varepsilon
  \end{align*}
  Sin embargo, si \(\mu(A) = \infty\) la afirmación no es cierta. Por ejemplo, consideremos el conjunto \(A = \mathbb{R}^N\) y tomemos cualquier compacto \(K \subseteq A\), entonces \(K\) es acotado y cerrado, por lo que existe \(M > 0\) tal que:
  \begin{align*}
    K \subseteq [ - M, M]^N
  \end{align*}
  Por lo tanto, tenemos que:
  \begin{align*}
    \mu(A \setminus K) \geq \mu(\mathbb{R}^N \setminus [ - M, M]^N) = \infty
  \end{align*}
  Por lo que no existe compacto \(K \subseteq A\) tal que \(\mu(A \setminus K) < \varepsilon\) para ningún \(\varepsilon > 0\).
\end{dem_box}

\newpage
\subsection{Ejercicio 3}
\noindent \textit{Sea \(A \in \mathcal{M}_N\), ¿son ciertas las siguientes afirmaciones?}
\begin{itemize}
  \item \(\overset{\circ}{A} \neq \emptyset \implies \mu(A) = 0\)
  \begin{dem_box}{Demostración}
    Falso, basta notar que dado un conjunto abierto cualquiera en \(\mathbb{R}\), como puede ser \(A \coloneq (1, 2)\) tenemos que claramente \(\overset{\circ}{A} = (1, 2)\) y su medida es:
    \begin{align*}
      \mu(A) = v_n(A) = 2 - 1 = 1 \neq 0
    \end{align*}
  \end{dem_box}
  \vspace{2ex}
  \item \(\overset{\circ}{A} = \emptyset \implies \mu(A) = 0\)
  \begin{dem_box}{Demostración}
    Falso, basta notar que el conjunto \(A = \mathbb{Q} \cap [0, 1]\) tiene interior vacío pero su medida es:
    \begin{align*}
      \mu(A) = 1 \neq 0
    \end{align*}
  \end{dem_box}
  \vspace{2ex}

  \item \textit{\(A\) abierto entonces \(\mu(\text{Fr}(A)) = 0\)}
  \begin{dem_box}{Demostración}
    Falso, podemos considerar el conjunto grueso de Cantor que se define como el conjunto complementario en \([0, 1]\) del conjunto de Cantor clásico. Este conjunto es abierto y su frontera es el propio conjunto de Cantor, que tiene medida positiva:
    \begin{align*}
      \mu(\text{Fr}(A)) = \mu(\text{Conjunto de Cantor}) = \frac{1}{2} > 0
    \end{align*}
  \end{dem_box}
  \vspace{3ex}

  \item \textit{\(A\) no numerable entonces \(\mu(A) > 0\)}
  \begin{dem_box}{Demostración}
    Falso, basta considerar el conjunto de Cantor clásico que es no numerable pero su medida es:
    \begin{align*}
      \mu(A) = 0
    \end{align*}
  \end{dem_box}
\end{itemize}

\newpage
\subsection{Ejercicio 4}
\textit{Demuestra que el hiperplano de \(\mathbb{R}^N\) es medible. ¿Cuál es su medida?}\vspace{2ex}
\begin{dem_box}{Demostración}
  Sea \(H\) un hiperplano de \(\mathbb{R}^N\), sin pérdida de generalidad podemos suponer que:
  \begin{align*}
    H = \{(x_1, \dots, x_N) \in \mathbb{R}^N : x_N = 0\}
  \end{align*}
  Cualquier otro hiperplano es traslación o rotación de este.\vspace{2ex}

  Consideramos el cubo abierto arbitrario:
  \begin{align*}
    Q = ( - a, a)^N \quad a > 0 \implies H \cap Q = ( - a, a)^{N - 1} \times \{0\}
  \end{align*}
  Por tanto, el conjunto \(H \cap Q\) es un cubo degenerado en \(\mathbb{R}^N\) y, por el ejercicio 9, tenemos que su medida exterior es nula:
  \begin{align*}
    \mu_N^*(H \cap Q) = 0
  \end{align*}
  Ahora, por la caracterización topológica de los conjuntos medibles, para cualquier \(\varepsilon > 0\) existe un conjunto abierto \(A_\varepsilon\) tal que:
  \begin{align*}
    H \cap Q \subseteq A_\varepsilon \quad \mbox{ y } \quad \mu_N^*(A_\varepsilon \setminus (H \cap Q)) < \varepsilon
  \end{align*}
  Por lo tanto, tenemos que:
  \begin{align*}
    \mu_N^*(A_\varepsilon) & = \mu_N^*((A_\varepsilon \setminus (H \cap Q)) \cup (H \cap Q)) \leq \\[2ex]
    & \leq \mu_N^*(A_\varepsilon \setminus (H \cap Q)) + \mu_N^*(H \cap Q) < \varepsilon + 0 = \varepsilon
  \end{align*}
  Lo cual implica que:
  \begin{align*}
    \mu_N^*(H \cap Q) = 0
  \end{align*}
  Finalmente, como \(Q\) era un cubo abierto arbitrario, se tiene que \(H\) es medible y su medida es:
  \begin{align*}
    \mu_N(H) = 0
  \end{align*}
\end{dem_box}

\newpage
\subsection{Ejercicio 5}
\textit{Demostrar que para \(A \in \mathcal{M}_1\) con \(\mu(A) < \infty\) y cada \(\varepsilon > 0\) existe una colección finita de intervalos disjuntos y abiertos \(\{I_i\}_{i = 1}^n\) tal que:}
\begin{align*}
  \mu\left(A \bigtriangleup \displaystyle \bigcup_{i = 1}^n I_i\right) < \varepsilon
\end{align*}
\textit{donde \(A \bigtriangleup B = (A \setminus B) \cup (B \setminus A)\) denota la diferencia simétrica entre \(A\) y \(B\).}\vspace{2ex}

\begin{dem_box}{Demostración}
  Sea \(A \in \mathcal{M}_1\) con \(\mu(A) < \infty\) y \(\varepsilon > 0\) cualquiera. Por la caracterización topológica de los conjuntos medibles, existe un conjunto abierto \(U \supseteq A\) tal que:
  \begin{align*}
    \mu(U \setminus A) < \frac{\varepsilon}{2}
  \end{align*}
  Como \(U\) es abierto en \(\mathbb{R}\), podemos expresar \(U\) como una unión numerable de intervalos abiertos disjuntos:
  \begin{align*}
    U = \displaystyle \bigcup_{i = 1}^{\infty} J_i
  \end{align*}
  Donde cada \(J_i\) es un intervalo abierto. Ahora, como la medida de \(A\) es finita, tenemos que:
  \begin{align*}
    \mu(U) = \mu(A) + \mu(U \setminus A) < \infty
  \end{align*}
  Por lo tanto, la serie:
  \begin{align*}
    \displaystyle \sum_{i = 1}^{\infty} v_1(J_i) = \mu(U)
  \end{align*}
  Converge y, por tanto, existe \(n_0 \in \mathbb{N}\) tal que:
  \begin{align*}
    \displaystyle \sum_{i = n_0 + 1}^{\infty} v_1(J_i) < \frac{\varepsilon}{2}
  \end{align*}
  Si tomamos la colección finita de intervalos abiertos disjuntos:
  \begin{align*}
    I_i = J_i \quad i = 1, \dots, n_0
  \end{align*}
  Entonces, tenemos que:
  \begin{align*}
    A \bigtriangleup \displaystyle \bigcup_{i = 1}^{n_0} I_i & = (A \setminus U) \cup (U \setminus A) \cup \left(U \setminus \displaystyle \bigcup_{i = 1}^{n_0} J_i\right) = \\[2ex]
    & = (U \setminus A) \cup \displaystyle \bigcup_{i = n_0 + 1}^{\infty} J_i
  \end{align*}
  Por lo tanto, por la subaditividad de la medida de Lebesgue, se tiene que:
  \begin{align*}
    \mu\left(A \bigtriangleup \displaystyle \bigcup_{i = 1}^{n_0} I_i\right) & \leq \mu(U \setminus A) + \mu\left(\displaystyle \bigcup_{i = n_0 + 1}^{\infty} J_i\right) \leq \\[2ex]
    & \leq \mu(U \setminus A) + \displaystyle \sum_{i = n_0 + 1}^{\infty} v_1(J_i) < \frac{\varepsilon}{2} + \frac{\varepsilon}{2} = \varepsilon
  \end{align*}
\end{dem_box}

\newpage
\subsection{Ejercicio 7}
\textit{Sea \(A \in \mathcal{M}_N\) y \(p \in \mathbb{R}^N\). Demuestra que \(p + A \in \mathcal{M}_N\)}\vspace{2ex}

\begin{dem_box}{Demostración}
  Sea \(A \in \mathcal{M}_N\) y \(p \in \mathbb{R}^N\), por la caracterización topológica de los conjuntos medibles, para cualquier \(\varepsilon > 0\) existe un conjunto abierto \(U_\varepsilon\) tal que:
  \begin{align*}
    A \subseteq U_\varepsilon \quad \mbox{ y } \quad \mu_N^*(U_\varepsilon \setminus A) < \varepsilon
  \end{align*}
  Ahora, consideramos el conjunto \(p + U_\varepsilon\), que es abierto ya que es la traslación de un conjunto abierto. Además, se tiene que:
  \begin{align*}
    p + A \subseteq p + U_\varepsilon
  \end{align*}
  Por otro lado, tenemos que:
  \begin{align*}
    (p + U_\varepsilon) \setminus (p + A) = p + (U_\varepsilon \setminus A)
  \end{align*}
  Y, por la propiedad de traslación de la medida exterior de Lebesgue, se tiene que:
  \begin{align*}
    \mu_N^*((p + U_\varepsilon) \setminus (p + A)) = \mu_N^*(p + (U_\varepsilon \setminus A)) = \mu_N^*(U_\varepsilon \setminus A) < \varepsilon
  \end{align*}
  Por lo tanto, por la caracterización topológica de los conjuntos medibles, \(p + A\) es medible.
\end{dem_box}

\newpage
\section{Existencia de conjuntos no medibles}
\subsection{Ejercicio 1}
\textit{Para cada par de números reales \(x, y \in [0, 1]\) establecemos la relación \(x \sim y\) cuando \(x - y \in \mathbb{Q}\). Demostrar los siguientes hechos:}
\begin{enumerate}
  \item \textit{Demuestra que \(\sim\) es una relación de equivalencia.}\vspace{2ex}
  \begin{dem_box}{Demostración}
    Veamos que cumple las tres propiedades de la relación de equivalencia:
    \begin{itemize}
      \item \textit{Reflexividad:} Sea \(x \in [0, 1]\) entonces tenemos que:
      \begin{align*}
        x - x = 0 \in \mathbb{Q}
      \end{align*}
      \item \textit{Simetría:} Sea \(x, y \in [0, 1]\) tales que \(x \sim y\), es decir:
      \begin{align*}
        x - y \in \mathbb{Q}
      \end{align*}
      Entonces tenemos que:
      \begin{align*}
        y - x = - \underbracket{(x - y)}_{\in \mathbb{Q}} \in \mathbb{Q} \implies  y \sim x
      \end{align*}
      \item \textit{Transitividad:} Sea \(x, y, z \in [0, 1]\) con \(x \sim y\) y \(y \sim z\) entonces:
      \begin{align*}
        x - y \in \mathbb{Q} \quad \mbox{y} \quad y - z \in \mathbb{Q}
      \end{align*}
      Entonces, podemos ver que:
      \begin{align*}
        x - z = x - y + y - z = \underbracket{(x - y)}_{\in \mathbb{Q}} + \underbracket{(y - z)}_{in \mathbb{Q}} \in \mathbb{Q}
      \end{align*}
    \end{itemize}
  \end{dem_box}
  \vspace{3ex}

  \item \textit{En cada clase de equivalencia \(A\) determinada por \(\sim\) tomamos un único elemento, que denotaremos \(x_A\). Justifica que:}
  \begin{align*}
    D = \{x_A : A \sim \mbox{ - clase de equivalencia}\}
  \end{align*}
  \textit{es un conjunto}\vspace{2ex}

  \begin{dem_box}{Demostración}
    Sea \(A\) una clase de equivalencia dada por \(\sim\) y sea \(x_A \in A\), sabemos que, dado \(x \in [0, 1]\):
    \begin{itemize}
      \item \(x\) pertenece a una única clase \([x]\)
      \item Dos clases distintas son disjuntas
      \item La unión de todas las clases es \([0, 1]\)
    \end{itemize}
    Esto implica que existe una correspondencia entre ser elemento de \([0, 1]\) y las clases de equivalencia. Cada clase de equivalencia es no vacía, ya que tiene al menos un elemento que la representa (al menos contiene el elemento que la define). Aplicando el axioma de elección, podemos seleccionar un único elemento \(x_A\) de cada clase de equivalencia \(A\). Por lo tanto, el conjunto:
    \begin{align*}
      D = \{x_A : A \sim \mbox{ - clase de equivalencia}\}
    \end{align*}
    Es un conjunto bien definido y contiene un único representante de cada clase de equivalencia.\vspace{2ex}

    Para ver que es un conjunto, basta notar que \(D \subseteq [0, 1]\) donde \([0, 1]\) es un conjunto y, por tanto, \(D\) es una subfamilia de \([0, 1]\) definida por el axioma de elección. Así, por el axioma de separación, \(D\) es un conjunto.
  \end{dem_box}
  \vspace{3ex}

  \item \textit{Sea \((q_n)_n\) la sucesión de todos los racionales en \([ - 1, 1]\), con \(q_n \neq q_m\) si \(n \neq m\). Para cada \(n \in \mathbb{N}\) definimos el conjunto:}
  \begin{align*}
    A_n = q_n + D
  \end{align*}
  \textit{Demuestra que los \(A_n\) son disjuntos dos a dos y que \(\mu^*(A_n) = \mu^*(D)\).}\vspace{2ex}
  \begin{dem_box}{Demostración}
    Sea \(n, m \in \mathbb{N}\) con \(n \neq m\). Supongamos que existe \(x \in A_n \cap A_m\), entonces existen \(d_n, d_m \in D\) tales que:
    \begin{align*}
      x = q_n + d_n = q_m + d_m
    \end{align*}
    Por lo tanto, tenemos que:
    \begin{align*}
      d_n - d_m = q_m - q_n \in \mathbb{Q}
    \end{align*}
    Lo cual implica que \(d_n\) y \(d_m\) pertenecen a la misma clase de equivalencia, es decir, \(d_n \sim d_m\). Pero como \(D\) contiene un único representante por clase de equivalencia, se tiene que \(d_n = d_m\). Por tanto:
    \begin{align*}
      q_n + d_n = q_m + d_n \implies q_n = q_m
    \end{align*}
    Al ser \(q_n \neq q_m\) para \(n \neq m\), llegamos a una contradicción. Por lo tanto, los conjuntos \(A_n\) son disjuntos dos a dos.\vspace{2ex}

    Ahora, para ver que \(\mu^*(A_n) = \mu^*(D)\), basta notar que \(A_n\) es una traslación de \(D\) por el racional \(q_n\). Por la propiedad de traslación de la medida exterior de Lebesgue, se tiene que:
    \begin{align*}
      \mu^*(A_n) = \mu^*(q_n + D) = \mu^*(D)
    \end{align*}
  \end{dem_box}
  \vspace{3ex}

  \item \textit{Comprueba las siguientes relaciones:}
  \begin{align*}
    [0, 1] \subseteq \displaystyle \bigcup_{m = 1}^{\infty} A_m \subseteq [ - 1, 2] \quad \mbox{ y } \quad 1 \leq \mu^*\left(\displaystyle \bigcup_{m = 1}^{\infty} A_m\right) \leq 3
  \end{align*}
  \begin{dem_box}{Demostración}
    Sea \(x \in [0, 1]\) entonces tenemos que \(x\) pertenece a una única clase de equivalencia \([x]\) y, por tanto, existe un único representante \(d_x \in D\) tal que \(d_x \sim x\). Por lo tanto, existe \(q_n \in \mathbb{Q}\) tal que:
    \begin{align*}
      x - d_x = q_n \implies x = q_n + d_x \in A_n
    \end{align*}
    Por lo tanto, \(x \in \displaystyle \bigcup_{m = 1}^{\infty} A_m\) y, por tanto, se tiene que:
    \begin{align*}
      [0, 1] \subseteq \displaystyle \bigcup_{m = 1}^{\infty} A_m
    \end{align*}
    Ahora, para ver que \(\displaystyle \bigcup_{m = 1}^{\infty} A_m \subseteq [ - 1, 2]\), basta notar que para cualquier \(x \in A_m\) existe \(d_m \in D\) tal que:
    \begin{align*}
      x = q_m + d_m
    \end{align*}
    Dado que \(q_m \in [ - 1, 1]\) y \(d_m \in [0, 1]\), se tiene que:
    \begin{align*}
      -1 + 0 \leq x \leq 1 + 1 \implies x \in [ - 1, 2]
    \end{align*}
    Por lo tanto, se cumple que:
    \begin{align*}
      \displaystyle \bigcup_{m = 1}^{\infty} A_m \subseteq [ - 1, 2]
    \end{align*}
    Finalmente, por la monotonicidad y subaditividad de la medida exterior de Lebesgue, se tiene que:
    \begin{align*}
      1 = \mu^*([0, 1]) \leq \mu^*\left(\displaystyle \bigcup_{m = 1}^{\infty} A_m\right) \leq \mu^*([ - 1, 2]) = 3
    \end{align*}
  \end{dem_box}
  \vspace{3ex}

  \item \textit{Deduce que \(\mu^*(D) > 0\).}\vspace{2ex}
  \begin{dem_box}{Demostración}
    Supongamos por contradicción que \(\mu^*(D) = 0\). Entonces, por el apartado (3), se tiene que:
    \begin{align*}
      \mu^*(A_n) = \mu^*(D) = 0 \quad \forall n \in \mathbb{N}
    \end{align*}
    Por lo tanto, por la subaditividad numerable de la medida exterior de Lebesgue, se tendría que:
    \begin{align*}
      \mu^*\left(\displaystyle \bigcup_{m = 1}^{\infty} A_m\right) \leq \displaystyle \sum_{m = 1}^{\infty} \mu^*(A_m) = 0
    \end{align*}
    Lo cual contradice la desigualdad obtenida en el apartado (4):
    \begin{align*}
      1 \leq \mu^*\left(\displaystyle \bigcup_{m = 1}^{\infty} A_m\right)
    \end{align*}
    Por lo tanto, se concluye que:
    \begin{align*}
      \mu^*(D) > 0
    \end{align*}
  \end{dem_box}
  \vspace{3ex}

  \item \textit{Demuestra que:}
  \begin{align*}
    \mu^*\left(\displaystyle \bigcup_{m = 1}^{\infty} A_m\right) \neq \displaystyle \sum_{m = 1}^{\infty} \mu^*(A_m)
  \end{align*}
  \begin{dem_box}{Demostración}
    Supongamos por contradicción que:
    \begin{align*}
      \mu^*\left(\displaystyle \bigcup_{m = 1}^{\infty} A_m\right) = \displaystyle \sum_{m = 1}^{\infty} \mu^*(A_m)
    \end{align*}
    Entonces, por el apartado (3), se tendría que:
    \begin{align*}
      \mu^*\left(\displaystyle \bigcup_{m = 1}^{\infty} A_m\right) = \displaystyle \sum_{m = 1}^{\infty} \mu^*(D)
    \end{align*}
    Como hemos visto en el apartado (5) que \(\mu^*(D) > 0\), la serie del lado derecho diverge a infinito, es decir:
    \begin{align*}
      \displaystyle \sum_{m = 1}^{\infty} \mu^*(D) = +\infty
    \end{align*}
    Lo cual contradice la desigualdad obtenida en el apartado (4):
    \begin{align*}
      \mu^*\left(\displaystyle \bigcup_{m = 1}^{\infty} A_m\right) \leq 3
    \end{align*}
    Por lo tanto, se concluye que:
    \begin{align*}
      \mu^*\left(\displaystyle \bigcup_{m = 1}^{\infty} A_m\right) \neq \displaystyle \sum_{m = 1}^{\infty} \mu^*(A_m)
    \end{align*}
  \end{dem_box}
  \vspace{3ex}

  \item \textit{Es \(D\) medible? Justifica la respuesta.}\vspace{2ex}
  \begin{dem_box}{Demostración}
    Supongamos por contradicción que \(D\) es medible, entonces por la \textbf{propiedad de la medida de conjuntos disjuntos numerables} se tendría que:
    \begin{align*}
      \mu^*\left(\displaystyle \bigcup_{m = 1}^{\infty} A_m\right) = \displaystyle \sum_{m = 1}^{\infty} \mu^*(A_m)
    \end{align*}
    Lo cual contradice el resultado obtenido en el apartado (6). Por lo tanto, se concluye que \(D\) no es medible.
  \end{dem_box}
  \vspace{3ex}

  \item \textit{¿Es cierto que la unión arbitraria de medibles es medible?}\vspace{2ex}
  \begin{dem_box}{Demostración}
    No, no es cierto. El conjunto \(D\) del apartado anterior es un contraejemplo. Consideremos la colección de conjuntos:
    \begin{align*}
      \mathcal{C} = \{A_n : n \in \mathbb{N}\}
    \end{align*}
    Donde cada \(A_n\) es medible ya que es una traslación del conjunto \(D\) por un número racional \(q_n\). Sin embargo, la unión numerable de estos conjuntos es:
    \begin{align*}
      \displaystyle \bigcup_{n = 1}^{\infty} A_n
    \end{align*}
    Que no es medible, como se demostró en el apartado (7). Por lo tanto, la unión arbitraria de conjuntos medibles no necesariamente es medible.
  \end{dem_box}
\end{enumerate}

\newpage
\subsection{Ejercicio 2}
\textit{Sea \((\mathbb{R}^N, \mathcal{M}, \mu)\) espacio de Börel-Lebesgue habitual, y consideramos un segundo espacio de medida \((\mathbb{R}^N, \mathcal{M}, m)\) con \(m\) la mediad con la propiedad:}
\begin{align*}
  m\left((a_1, b_1) \times \dots \times (a_N, b_N)\right) = \displaystyle \prod_{i = 1}^N (b_i - a_i)
\end{align*}
\textit{para todo par de \(N\)-tuplas de números reales \((a_1, \dots, a_N), (b_1, \dots, b_N)\) verificando que \(a_i < b_i\) para todo \(i = 1,\dots, N\). Demuestra que la única medida posible que cumple la anterior propiedad es la medida de Lebesgue \(\mu\).}\vspace{2ex}

\noindent\textit{Pista: para probar que \(\mu(A) = m(A)\) para todo \(A \in \mathcal{M}\) demuéstralo paulatinamente siguiendo este orden:}
\begin{enumerate}
  \item \textit{Cuando \(A\) es un cubo acotado degenerado.}\vspace{2ex}
  \begin{dem_box}{Demostración}
    Sea \(A\) un cubo acotado degenerado en \(\mathbb{R}^N\), es decir, existe al menos una dimensión \(i\) tal que \(a_i = b_i\). Entonces, el volumen de \(A\) es
    \begin{align*}
      v_N(A) = \displaystyle \prod_{i = 1}^N (b_i - a_i) = 0
    \end{align*}
    Por lo tanto, tanto la medida de Lebesgue como la medida \(m\) asignan medida nula a \(A\):
    \begin{align*}
      \mu(A) = 0 \quad \mbox{ y } \quad m(A) = 0
    \end{align*}
    Por lo tanto, se cumple que \(\mu(A) = m(A)\).
  \end{dem_box}
  \vspace{3ex}

  \item \textit{Cuando \(A\) es un cubo acotado.}\vspace{2ex}
  \begin{dem_box}{Demostración}
    Por un resultado de clase, hemos visto que sea \(A\) un cubo acotado entonces:
    \begin{align*}
      v_N(C) = \mu_N(C)
    \end{align*}
    Por lo tanto, tenemos que:
    \begin{align*}
      m(A) = v_N(A) = \mu_N(A)
    \end{align*}
  \end{dem_box}
  \vspace{3ex}

  \item \textit{Cuando \(\mu(A) = 0\)}.\vspace{2ex}
  \begin{dem_box}{Demostración}
    Sea \(A \in \mathcal{M}_N\) tal que \(\mu_N(A) = 0\), fijamos un \(\varepsilon > 0\) cualquiera y sabemos que \(\exists \{I_i\}_{i \in \mathbb{N}}\) colección de cubos abiertos tales que:
    \begin{align*}
      A \subseteq \displaystyle \bigcup_{i = 1}^{\infty} I_i \quad \mbox{y}\quad \mu_N(A) + \varepsilon > \displaystyle \sum_{i = 1}^{\infty} v_N(I_i)
    \end{align*}
    Como \(\mu(A) = 0\) entonces:
    \begin{align*}
      \varepsilon > \displaystyle \sum_{i = 1}^{\infty} v_N(I_i)
    \end{align*}
    Y como \(v_N(I_i) \geq 0\) entonces:
    \begin{align*}
      0 \leq \displaystyle \sum_{i = 1}^{\infty} v_N(I_i) < \varepsilon \xrightarrow[n \to \infty]{} 0
    \end{align*}
    Por lo tanto:
    \begin{align*}
      \displaystyle \sum_{i = 1}^{n} v_N(I_i) = 0 = \mu_N(A)
    \end{align*}
  \end{dem_box}
  \vspace{3ex}

  \item \textit{Cuando \(A\) es un conjunto abierto}.\vspace{2ex}
  \begin{dem_box}{Demostración}
    Sea \(A\) un abierto cualquiera de \(\mathbb{R}^N\) existe una colección de cubos abiertos \(C_i\) disjuntos tales que:
    \begin{align*}
      A = \displaystyle \bigcup_{i = 1}^{\infty} C_i
    \end{align*}
    Y como estos cubos son acotados, sabemos por el apartado 3 que:
    \begin{align*}
      \mu_N(C_i) = v_N(C_i)
    \end{align*}
    Por tanto:
    \begin{align*}
      \mu_N(A) & = \mu_N\left(\displaystyle \bigcup_{i = 1}^{\infty} C_i\right) = \displaystyle \sum_{i = 1}^{\infty} \mu_N(C_i) =\\[2ex]
      & = \displaystyle \sum_{i = 1}^{\infty} v_N(C_i) = \displaystyle \sum_{i = 1}^{\infty} m(C_i) = m\left(\displaystyle \bigcup_{i = 1}^{\infty} C_i\right) = m(A)
    \end{align*}
  \end{dem_box}
  \vspace{3ex}

  \item \textit{Cuando \(A\) es un conjunto medible acotado.}\vspace{2ex}
  \begin{dem_box}{Demostración}
    Si \(A\) es medible acotado, de la forma:
    \begin{align*}
      A = I_1 \times \dots \times I_N
    \end{align*}
    Si los \(I_i\) son abiertos está aplicando el apartado anterior, supongamos que no lo son y que son cerrados, entonces sea \(\varepsilon > 0\) cualquiera pero fijo, tenemos que:
    \begin{align*}
      I_i = [a_i, b_i] \subseteq (a_i - \varepsilon, b_i + \varepsilon) = I_i^{(\varepsilon)}
    \end{align*}
    Definimos \(A^{(\varepsilon)}\) como:
    \begin{align*}
      A^{(\varepsilon)} \coloneq I_1^{(\varepsilon)} \times \dots \times I_N^{(\varepsilon)}
    \end{align*}
    Y tenemos claramente que \(A \subseteq A^{(\varepsilon)}\) por tanto:
    \begin{align*}
      \mu_N(A) \leq \mu_N(A^{(\varepsilon)})
    \end{align*}
    Y como \(A^{(\varepsilon)}\) es abierto, aplicando el apartado anterior tenemos que:
    \begin{align*}
      \mu_N(A) &\leq \mu_N(A^{(\varepsilon)}) = m(A^{(\varepsilon)}) = \prod_{i = 1}^{N} (b_i - a_i + 2\varepsilon) =\\[2ex]
      & = \prod_{i = 1}^N (b_i - a_i) + (2 \varepsilon)^N = m(A) + 2^N \varepsilon^N \xrightarrow{\varepsilon \to 0^ + } m(A)
    \end{align*}
    Por tanto, \(\mu_N(A) \leq m(A)\). Ahora, para la otra desigualdad, sea \(\varepsilon > 0\) cualquiera pero fijo, existe un conjunto abierto \(U\) tal que:
    \begin{align*}
      A \subseteq U \quad \mbox{ y } \quad \mu_N(U \setminus A) < \varepsilon
    \end{align*}
    Y por el apartado anterior tenemos que:
    \begin{align*}
      m(A) & \leq m(U) = \mu_N(U) = \mu_N(A) + \mu_N(U \setminus A) < \mu_N(A) + \varepsilon \xrightarrow{\varepsilon \to 0^ + } \mu_N(A)
    \end{align*}
    Por tanto, \(m(A) \leq \mu_N(A)\) y en consecuencia:
    \begin{align*}
      \mu_N(A) = m(A)
    \end{align*}
  \end{dem_box}
  \vspace{3ex}

  \item \textit{Cuando \(A\) es un conjunto medible cualquiera}
  \begin{dem_box}{Demostración}
    Sea \(A \in \mathcal{M}_N\) un conjunto medible cualquiera, existe una sucesión de conjuntos medibles acotados \((A_k)_{k \in \mathbb{N}}\) tal que:
    \begin{align*}
      A_k \subseteq A_{k + 1} \quad \mbox{ y } \quad \displaystyle \bigcup_{k = 1}^{\infty} A_k = A
    \end{align*}
    Por el apartado anterior, tenemos que:
    \begin{align*}
      \mu_N(A_k) = m(A_k) \quad \forall k \in \mathbb{N}
    \end{align*}
    Ahora, por un resultado de clase sabemos que:
    \begin{align*}
      \mu_N(A) & = \lim_{k \to \infty} \mu_N(A_k) = \lim_{k \to \infty} m(A_k) = m(A)
    \end{align*}
    Por lo tanto, se concluye que para todo conjunto medible \(A\):
    \begin{align*}
      \mu_N(A) = m(A)
    \end{align*}
  \end{dem_box}
\end{enumerate}
\newpage

\subsection{Ejercicio 3}
\textit{Demuestra que no existe ninguna medida sobre \(\mathcal{P}(\mathbb{R})\) que preserve la longitud de los intervalos.}\vspace{2ex}
\begin{dem_box}{Demostración}
  Supongamos por contradicción que existe una medida que preserva la longitud de los intervalos:
  \begin{align*}
    \mu : \mathcal{P}(\mathbb{R}) \to [0, +\infty]
  \end{align*}
  que cumple las propiedades siguientes:
  \begin{enumerate}
    \item Es \(\sigma\)-aditiva.
    \item \(\mu(\emptyset ) = 0\).
    \item \(\mu((a, b)) = b - a\) para todo \(a, b \in \mathbb{R}\) con \(a < b\).
  \end{enumerate}
  Consideramos la relación de equivalencia en \([0, 1]\) definida por:
  \begin{align*}
    x \sim y \iff x - y \in \mathbb{Q}
  \end{align*}
  Y tomamos un conjunto \(D\) que contiene un único representante de cada clase de equivalencia. Definimos los conjuntos:
  \begin{align*}
    V_q \coloneq V + q = \{x + q : x \in V\} \quad \forall q \in \mathbb{Q} \cap [ - 1, 1]
  \end{align*}
  Estos conjuntos cumplen que:
  \begin{itemize}
    \item Son disjuntos dos a dos. Sea \(v_1 + q_1 = v_2 + q_2\) entonces:
    \begin{align*}
      v_1 - v_2 = q_2 - q_1 \in \mathbb{Q}
    \end{align*}
    Por tanto, \(v_1 = v_2\) y, por definición de \(V\), \(q_1 = q_2\).
    \item Se tiene que:
    \begin{align*}
      [0, 1] \subseteq \displaystyle \bigcup_{q \in \mathbb{Q} \cap [ - 1, 1]} V_q \subseteq [ - 1, 2]
    \end{align*}
  \end{itemize}
  Ahora, aplicando la medida \(\mu\) a la inclusión anterior, tenemos que:
  \begin{align*}
    \mu\left(\displaystyle \bigcup_{q \in \mathbb{Q} \cap [ - 1, 1]} V_q\right) = \displaystyle \sum_{q \in \mathbb{Q} \cap [ - 1, 1]} \mu(V_q) = \displaystyle \sum_{q \in \mathbb{Q} \cap [ - 1, 1]} \mu(V)
  \end{align*}
  Como el conjunto \(\mathbb{Q} \cap [ - 1, 1]\) es numerable infinito, tenemos dos casos:
  \begin{align*}
    \displaystyle \sum_{q \in \mathbb{Q} \cap [ - 1, 1]} m = \begin{cases}
      0 & \mbox{ si } m = 0\\
      +\infty & \mbox{ si } m > 0
    \end{cases}
  \end{align*}
  Además, como \(\mu\) es una medida que preserva la longitud de los intervalos, tenemos que:
  \begin{align*}
    \mu([0, 1]) = 1 \quad \mbox{ y } \quad \mu([ - 1, 2]) = 3
  \end{align*}
  Pero tenemos que:
  \begin{align*}
    1 = \mu([0, 1]) \leq \mu\left(\displaystyle \bigcup_{q \in \mathbb{Q} \cap [ - 1, 1]} V_q\right) \leq \mu([ - 1, 2]) = 3
  \end{align*}
  Lo que implica que:
  \begin{align*}
    \mu\left(\displaystyle \bigcup_{q \in \mathbb{Q} \cap [ - 1, 1]} V_q\right) \in [1, 3]
  \end{align*}
  Pero hemos visto que \(\mu\left(\displaystyle \bigcup_{q \in \mathbb{Q} \cap [ - 1, 1]} V_q\right)\) es o bien \(0\) o bien \(+\infty\), lo cual es una contradicción. Por lo tanto, no existe ninguna medida sobre \(\mathcal{P}(\mathbb{R})\) que preserve la longitud de los intervalos.
\end{dem_box}
\newpage

\section{El conjunto de Cantor}
\subsection{Construcción del conjunto de Cantor}
\textit{Eliminamos del intervalo \([0, 1]\) el tercio central, es decir, el intervalo:}
\begin{align*}
  I_1^1 = \left(\frac{1}{3}, \frac{2}{3}\right)
\end{align*}
\textit{Eliminamos de cada uno de los intervalos restantes:}
\begin{align*}
  \left[0, \frac{1}{3}\right] \quad \mbox{ y } \quad \left[\frac{2}{3}, 1\right]
\end{align*}
\textit{Los tercios centrales, es decir:}
\begin{align*}
  I_1^2 = \left(\frac{1}{9}, \frac{2}{9}\right) \quad \mbox{ y } \quad I_2^2 = \left(\frac{7}{9}, \frac{8}{9}\right)
\end{align*}
\textit{Y se prosigue de la misma forma. Sea:}
\begin{align*}
  I = \displaystyle \bigcup_{n = 1}^{\infty} \displaystyle \bigcup_{i = 1}^{2^{n - 1}} I_i^n
\end{align*}
\textit{El conjunto \(\Delta \coloneq [0, 1] \setminus I\) se llama conjunto de Cantor.}\vspace{2ex}
\newpage

\subsection{Ejercicio 1}
\textit{Demuestra las siguientes propiedades del conjunto de cantor:}
\begin{enumerate}
  \item \(\Delta\) es compacto no vacío:\vspace{2ex}
  \begin{dem_box}{Demostración}
    Para ver que es compacto, tenemos que ver que es cerrado y acotado:
    \begin{itemize}
      \item \textit{Acotación}: Trivialmente está acotado ya que:
      \begin{align*}
        \Delta \subseteq [0, 1] \mbox{ es acotado}
      \end{align*}
      \item \textit{Cerradura}: Sabemos que la unión numerable de cerrados es cerrada bajo la topología usual de \(\mathbb{R}\) y como cada \(I_i^n\) es cerrado, entonces:
      \begin{align*}
        \Delta = \displaystyle \bigcup_{n = 1}^{\infty} \displaystyle \bigcup_{i = 1}^{2^{n - 1}} I_i^n \quad \mbox{ es cerrado}
      \end{align*}
    \end{itemize}
    Para ver que es no vacío, basta notar que los puntos \(0\) y \(1\) pertenecen a \(\Delta\) ya que en ningún paso de la construcción se eliminan estos puntos. Por lo tanto, \(\Delta\) es compacto y no vacío.
  \end{dem_box}
  \vspace{3ex}

  \item \textit{\(\Delta\) tiene interior vacío y carece de puntos aislados}.\vspace{2ex}
  \begin{dem_box}{Demostración}
    Veamos que:
    \begin{itemize}
      \item \textit{Interior vacío}: Supongamos por contradicción que existe un intervalo abierto \((a, b) \subseteq \Delta\). Entonces, en algún paso de la construcción del conjunto de Cantor, se eliminaría un intervalo que interseca a \((a, b)\), lo cual es una contradicción. Por lo tanto, \(\Delta\) tiene interior vacío.
      \item \textit{Carece de puntos aislados:} Sea \(x \in \Delta\) un punto cualquiera y sea \(\varepsilon > 0\) cualquiera. Entonces, existe un \(n \in \mathbb{N}\) tal que:
      \begin{align*}
        \frac{1}{3^n} < \varepsilon
      \end{align*}
      En el paso \(n\) de la construcción del conjunto de Cantor, el punto \(x\) pertenece a uno de los \(2^n\) intervalos restantes, digamos \(J\), que tiene longitud \(\frac{1}{3^n}\). Como \(\frac{1}{3^n} < \varepsilon\), entonces \(J \subseteq (x - \varepsilon, x + \varepsilon)\). Además, \(J\) contiene infinitos puntos de \(\Delta\) ya que en los pasos posteriores se eliminan tercios centrales de los intervalos restantes. Por lo tanto, cualquier vecindad de \(x\) contiene infinitos puntos de \(\Delta\), lo que implica que \(x\) no es un punto aislado. Como \(x\) era arbitrario, concluimos que \(\Delta\) carece de puntos aislados.
    \end{itemize}
  \end{dem_box}
  \vspace{3ex}

  \item \textit{\(\text{Fr.} \Delta = \Delta\)}.\vspace{2ex}
  \begin{dem_box}{Demostración}
    Recordemos que la frontera de un conjunto \(A\) se define como:
    \begin{align*}
      \text{Fr.} A = \overline{A} \setminus \text{Int}(A)
    \end{align*}
    Donde \(\overline{A}\) es el cierre de \(A\) y \(\text{Int}(A)\) es el interior de \(A\). En el caso del conjunto de Cantor \(\Delta\), ya hemos visto que es cerrado, por lo que:
    \begin{align*}
      \overline{\Delta} = \Delta
    \end{align*}
    Además, hemos demostrado que \(\Delta\) tiene interior vacío, es decir:
    \begin{align*}
      \text{Int}(\Delta) = \emptyset
    \end{align*}
    Por lo tanto, la frontera de \(\Delta\) es:
    \begin{align*}
      \text{Fr.} \Delta = \overline{\Delta} \setminus \text{Int}(\Delta) = \Delta \setminus \emptyset = \Delta
    \end{align*}
    Así, concluimos que:
    \begin{align*}
      \text{Fr.} \Delta = \Delta
    \end{align*}
  \end{dem_box}
  \vspace{3ex}

  \item \textit{\(\mu(\Delta) = 0\)}.\vspace{2ex}
  \begin{dem_box}{Demostración}
    Recordemos que la medida de Lebesgue \(\mu\) es una medida completa y cuenta con la propiedad de ser \(\sigma\)-aditiva. En cada paso de la construcción del conjunto de Cantor, se eliminan intervalos cuya longitud total es:
    \begin{align*}
      \sum_{n = 1}^{\infty} \frac{2^{n - 1}}{3^n} = \frac{1}{3} + \frac{2}{9} + \frac{4}{27} + \dots
    \end{align*}
    Esta serie geométrica tiene razón \(\frac{2}{3}\) y su suma es:
    \begin{align*}
      S = \frac{\frac{1}{3}}{1 - \frac{2}{3}} = 1
    \end{align*}
    Por lo tanto, la longitud total de los intervalos eliminados es \(1\). Dado que el intervalo original \([0, 1]\) tiene longitud \(1\), la medida del conjunto de Cantor \(\Delta\) es:
    \begin{align*}
      \mu(\Delta) = \mu([0, 1]) - \sum_{n = 1}^{\infty} \frac{2^{n - 1}}{3^n} = 1 - 1 = 0
    \end{align*}
  \end{dem_box}
  \vspace{3ex}

  \item \textit{\(\Delta\) tiene cardinal no numerable}.\vspace{2ex}
  \begin{dem_box}{Demostración}
    Para demostrar que el conjunto de Cantor \(\Delta\) tiene cardinal no numerable, podemos establecer una correspondencia entre los puntos de \(\Delta\) y las secuencias binarias infinitas. Cada punto en \(\Delta\) puede ser representado en base \(3\) utilizando solo los dígitos \(0\) y \(2\). Esto se debe a que en cada paso de la construcción del conjunto de Cantor, se eliminan los intervalos que contienen el dígito \(1\) en su representación ternaria.

    Por ejemplo, el punto \(0\) se representa como \(0.000\dots_3\), el punto \(\frac{1}{3}\) como \(0.200\dots_3\), el punto \(\frac{2}{3}\) como \(0.220\dots_3\), y así sucesivamente. Cada dígito \(0\) en la representación ternaria corresponde a un \(0\) en la secuencia binaria, y cada dígito \(2\) corresponde a un \(1\).

    Dado que las secuencias binarias infinitas tienen cardinal no numerable (ya que corresponden al conjunto de todos los subconjuntos de los números naturales, que es no numerable), concluimos que el conjunto de Cantor \(\Delta\) también tiene cardinal no numerable.
  \end{dem_box}
\end{enumerate}

\newpage
\subsection{Ejercicio 2}
\textit{Encuentra un subconjunto medible de \([0, 1]\) cuyo interior tenga medida 0 y la medida de su frontera sea estrictamente positiva.}\vspace{2ex}
\begin{dem_box}{Demostración}
  Consideremos el conjunto de Cantor \(\Delta \subseteq [0, 1]\). Sabemos que \(\Delta\) es un conjunto compacto y no vacío, y que su interior es vacío, es decir:
  \begin{align*}
    \text{Int}(\Delta) = \emptyset
  \end{align*}
  Por lo tanto, la medida del interior de \(\Delta\) es:
  \begin{align*}
    \mu(\text{Int}(\Delta)) = \mu(\emptyset) = 0
  \end{align*}
  Ahora, consideremos la frontera de \(\Delta\). Hemos demostrado que:
  \begin{align*}
    \text{Fr.} \Delta = \Delta
  \end{align*}
  Y sabemos que la medida de Lebesgue de \(\Delta\) es:
  \begin{align*}
    \mu(\Delta) = 0
  \end{align*}
  Sin embargo, podemos modificar el conjunto de Cantor para obtener un conjunto cuya frontera tenga medida estrictamente positiva. Consideremos el conjunto:
  \begin{align*}
    A = \Delta \cup [0, 1] \setminus \bigcup_{n=1}^{\infty} I_n
  \end{align*}
  Donde \(I_n\) son los intervalos eliminados en la construcción del conjunto de Cantor. La frontera de \(A\) incluye todos los puntos de los intervalos eliminados, que tienen una medida total positiva. Por lo tanto, la medida de la frontera de \(A\) es estrictamente positiva:
  \begin{align*}
    \mu(\text{Fr.} A) > 0
  \end{align*}
  En resumen, el conjunto \(A\) es un subconjunto medible de \([0, 1]\) cuyo interior tiene medida \(0\) y la medida de su frontera es estrictamente positiva.
\end{dem_box}
\newpage

\subsection{Ejercicio 3}
\textit{Inspirándote en el conjunto ternario de Cantor, construye un cerrado y contenido en \([0, 1]\), cuyo interior tenga medida 0 y la medida de su frontera sea estrictamente mayor que cero.}\vspace{2ex}

\begin{dem_box}{Demostración}
  Consideremos el conjunto \(C\) construido de la siguiente manera:
  \begin{itemize}
    \item Comenzamos con el intervalo \([0, 1]\).
    \item En el primer paso, eliminamos el tercio central, es decir, el intervalo \(\left(\frac{1}{3}, \frac{2}{3}\right)\).
    \item En el segundo paso, eliminamos los tercios centrales de los intervalos restantes, es decir, eliminamos \(\left(\frac{1}{9}, \frac{2}{9}\right)\) y \(\left(\frac{7}{9}, \frac{8}{9}\right)\).
    \item Continuamos este proceso indefinidamente, eliminando en cada paso los tercios centrales de los intervalos restantes.
  \end{itemize}
  El conjunto \(C\) resultante es un conjunto cerrado contenido en \([0, 1]\). Ahora, veamos las propiedades que cumple:
  \begin{itemize}
    \item \textit{Interior con medida 0}: En cada paso del proceso de construcción, se eliminan intervalos cuya longitud total es:
    \begin{align*}
      \sum_{n=1}^{\infty} \frac{2^{n-1}}{3^n} = 1
    \end{align*}
    Por lo tanto, la medida del conjunto \(C\) es:
    \begin{align*}
      \mu(C) = 1 - 1 = 0
    \end{align*}
    Dado que \(C\) es cerrado y no contiene ningún intervalo abierto, su interior es vacío:
    \begin{align*}
      \text{Int}(C) = \emptyset
    \end{align*}
    Por lo tanto, la medida del interior de \(C\) es:
    \begin{align*}
      \mu(\text{Int}(C)) = 0
    \end{align*}
    \item \textit{Medida de la frontera estrictamente mayor que cero}: La frontera de \(C\) incluye todos los puntos que fueron eliminados en cada paso del proceso de construcción. La medida total de estos puntos eliminados es:
    \begin{align*}
      \mu(\text{Fr.} C) = 1
    \end{align*}
    Por lo tanto, la medida de la frontera de \(C\) es estrictamente mayor que cero.
  \end{itemize}
  En resumen, el conjunto \(C\) es un conjunto cerrado contenido en \([0, 1]\) cuyo interior tiene medida \(0\) y la medida de su frontera es estrictamente mayor que cero.
\end{dem_box}
\newpage

\subsection{Ejercicio 4}
\textit{Encuentra un subconjunto de \(\mathbb{R}^2\) no numerable y medible de medida 0}.\vspace{2ex}

\begin{dem_box}{Demostración}
  Consideremos el conjunto de Cantor \(\Delta \subseteq [0, 1]\) en la recta real. Sabemos que \(\Delta\) es un conjunto no numerable y medible con medida de Lebesgue \(\mu(\Delta) = 0\).

  Ahora, definimos el conjunto \(A\) en \(\mathbb{R}^2\) como el producto cartesiano de \(\Delta\) consigo mismo:
  \begin{align*}
    A = \Delta \times \Delta = \{(x, y) : x \in \Delta, y \in \Delta\}
  \end{align*}

  Veamos las propiedades de \(A\):
  \begin{itemize}
    \item \textit{No numerable}: Dado que \(\Delta\) es no numerable, el producto cartesiano \(\Delta \times \Delta\) también es no numerable. Esto se debe a que podemos establecer una correspondencia entre los puntos de \(A\) y las parejas ordenadas de puntos en \(\Delta\), lo que implica que \(A\) tiene cardinalidad al menos igual a la cardinalidad de \(\Delta\).
    \item \textit{Medible con medida 0}: La medida de Lebesgue en \(\mathbb{R}^2\) se define como el producto de las medidas en cada dimensión. Dado que la medida de Lebesgue de \(\Delta\) es \(0\), tenemos:
    \begin{align*}
      \mu_2(A) = \mu_1(\Delta) \cdot \mu_1(\Delta) = 0 \cdot 0 = 0
    \end{align*}
    Por lo tanto, \(A\) es un conjunto medible en \(\mathbb{R}^2\) con medida de Lebesgue igual a \(0\).
  \end{itemize}

  En conclusión, el conjunto \(A = \Delta \times \Delta\) es un subconjunto no numerable y medible de \(\mathbb{R}^2\) con medida \(0\).
\end{dem_box}

\newpage
\section{Productos de conjuntos medibles}
\textit{El objetivo de esta sección de ejercicios es la demostración del siguiente resultado paso a paso. A partir de aquí, se adopta el convenio de:}
\begin{align*}
  0 \cdot \infty = 0
\end{align*}
\textit{siempre y cuando 0 o \(\infty\) vengan de una medida.}\vspace{2ex}

\textit{El teorema dice lo siguiente: Sean \(p, q \in \mathbb{N}\) y \(A \in \mathcal{M}_p\), \(B \in \mathcal{M}_q\) dos conjuntos medibles. Entonces, \(A \times B \in \mathcal{M}_{p + q}\) y:}
\begin{align*}
  \mu_{p + q}(A \times B) = \mu_p(A) \cdot \mu_q(B)
\end{align*}
\newpage

\subsection{Ejercicio 1}
\textit{Sea \(\mu_p(A) = 0\) y \(\mu_q(B) < \infty\) entonces:}
\begin{align*}
  A \times B \in \mathcal{M}_{p + q} \quad \mbox{ y } \quad \mu_{p + q}(A \times B) = 0
\end{align*}
\begin{dem_box}{Demostración}
  Sea \(\varepsilon > 0\), dado que \(A\) es un conjunto medible en \(\mathbb{R}^p\) con medida cero, podemos cubrir \(A\) con una colección numerable de cubos abiertos \((C_i)_{i \in \mathbb{N}}\) tal que:
  \begin{align*}
    A \subseteq \bigcup_{i = 1}^{\infty} C_i \quad \mbox{ y } \quad \sum_{i = 1}^{\infty} v_p(C_i) < \varepsilon
  \end{align*}

  Ahora, consideremos el conjunto \(A \times B\). Podemos cubrir \(A \times B\) con la colección de conjuntos:
  \begin{align*}
    C_i \times B, \quad i \in \mathbb{N}
  \end{align*}
  Entonces, tenemos:
  \begin{align*}
    A \times B \subseteq \bigcup_{i = 1}^{\infty} (C_i \times B)
  \end{align*}

  La medida de Lebesgue en el espacio producto se calcula como:
  \begin{align*}
    \mu_{p + q}(C_i \times B) = v_p(C_i) \cdot \mu_q(B)
  \end{align*}
  Por lo tanto, la medida de \(A \times B\) se puede estimar como:
  \begin{align*}
    \mu_{p + q}(A \times B) &\leq \sum_{i = 1}^{\infty} \mu_{p + q}(C_i \times B) =\\[2ex]
    &= \sum_{i = 1}^{\infty} v_p(C_i) \cdot \mu_q(B) = \mu_q(B) \cdot \sum_{i = 1}^{\infty} v_p(C_i) < \mu_q(B) \cdot \varepsilon
  \end{align*}
  Dado que \(\varepsilon > 0\) es arbitrario, podemos hacer que \(\mu_{p + q}(A \times B)\) sea tan pequeño como queramos. Por lo tanto, concluimos que:
  \begin{align*}
    \mu_{p + q}(A \times B) = 0
  \end{align*}
  Y como su medida es nula, entonces:
  \begin{align*}
    A \times B \in \mathcal{M}_{p + q}
  \end{align*}
\end{dem_box}
\newpage

\subsection{Ejercicio 2}
\textit{Sea \(\mu_p(A) = 0\) y \(B \in \mathcal{M}_q\) entonces:}
\begin{align*}
  A \times B \in \mathcal{M}_{p + q} \quad \mbox{ y } \quad \mu_{p + q}(A \times B) = 0
\end{align*}
\begin{dem_box}{Demostración}
  Tenemos que:
  \begin{align*}
    B = \displaystyle \bigcup_{n = 1}^{\infty} B_n \quad \mbox{ con } B_n \coloneq B \cap [ - n, n]^q \in \mathcal{M}_q
  \end{align*}
  Así, tenemos que claramente:
  \begin{align*}
    \mu_q(B_n) \leq \mu_q([ - n, n]) < \infty
  \end{align*}
  Entonces, aplicando el apartado anterior, tenemos:
  \begin{align*}
    \mu_{p + q}(A \times B_n) = 0
  \end{align*}
  Y como se tiene:
  \begin{align*}
    A \times B \subseteq \displaystyle \bigcup_{n = 1}^{\infty} \underbracket{(A \times B_n)}_{\in \mathcal{M}_{p + q}} \in \mathcal{M}_{p + q}
  \end{align*}
  Además:
  \begin{align*}
    \mu_{p + q}(A \times B) & \leq \mu_{p + q} \left(\displaystyle \bigcup_{n = 1}^{\infty} (A \times B_n)\right) \leq \displaystyle \sum_{n = 1}^{\infty} \mu_{p + q}(A \times B_n) =\\[2ex]
    & = \displaystyle \sum_{n = 1}^{\infty} \mu_p(A) \cdot \mu_q(B_n) = \displaystyle \sum_{n = 1}^{\infty} 0 = 0
  \end{align*}
  Entonces se tiene:
  \begin{align*}
    \mu_{p + q}(A \times B) = 0
  \end{align*}
  Entonces, como su medida es nula:
  \begin{align*}
    A \times B \in \mathcal{M}_{p + q}
  \end{align*}
\end{dem_box}
\newpage

\subsection{Ejercicio 3}
\textit{Si \(A\) es abierto en \(\mathbb{R}^p\) y \(B\) es abierto en \(\mathbb{R}^q\) entonces:}
\begin{align*}
  A \times B \in \mathcal{M}_{p + q} \quad \mbox{ y } \quad \mu_{p + q}(A \times B) = \mu_p(A) \cdot \mu_q(B)
\end{align*}
\begin{dem_box}{Demostración}
  Por el teorema de descomposición, tenemos que existen colecciones numerables de cubos abiertos disjuntos \(\{A_i\}_{i \in \mathbb{N}}, \{B_i\}_{i \in \mathbb{N}}\) tales que:
  \begin{align*}
    A = \displaystyle \bigcup_{n = 1}^{\infty} A_i \quad \mbox{ y } \quad B = \displaystyle \bigcup_{i = 1}^{\infty} B_i
  \end{align*}
  Por tanto:
  \begin{align*}
    A \times B = \displaystyle \bigcup_{i, j \in \mathbb{N}} (A_i \times B_j)
  \end{align*}
  Y como cada \(A_i \times B_j\) es un cubo abierto en \(\mathbb{R}^{p + q}\), tenemos que:
  \begin{align*}
    A \times B \in \mathcal{M}_{p + q}
  \end{align*}
  Además, se tiene:
  \begin{align*}
    \mu_{p + q}(A \times B) & = \mu_{p + q} \left(\displaystyle \bigcup_{i, j \in \mathbb{N}} (A_i \times B_j)\right) = \displaystyle \sum_{i, j \in \mathbb{N}} \mu_{p + q}(A_i \times B_j) =\\[2ex]
    & = \displaystyle \sum_{i, j \in \mathbb{N}} \mu_p(A_i) \cdot \mu_q(B_j) = \left(\displaystyle \sum_{i \in \mathbb{N}} \mu_p(A_i)\right) \cdot \left(\displaystyle \sum_{j \in \mathbb{N}} \mu_q(B_j)\right) =\\[2ex]
    & = \mu_p(A) \cdot \mu_q(B)
  \end{align*}
  Así, concluimos que:
  \begin{align*}
    \mu_{p + q}(A \times B) = \mu_p(A) \cdot \mu_q(B)
  \end{align*}
\end{dem_box}

\newpage
\subsection{Ejercicio 4}
\textit{Si \(\mu_p(A) < \infty\) y \(\mu_q(B) < \infty\) entonces:}
\begin{align*}
  A \times B \in \mathcal{M}_{p + q} \quad \mbox{ y } \quad \mu_{p + q}(A \times B) = \mu_p(A) \cdot \mu_q(B)
\end{align*}
\begin{dem_box}{Demostración}
  Por la caracterización topológica de los conjuntos medibles, como \(A\) y \(B\) son medibles entonces existen conjuntos \(G_A \subseteq \mathbb{R}^p\) y \(G_B \subseteq \mathbb{R}^q\) abiertos tales que:
  \begin{align*}
    A \subseteq G_A \quad \mbox{ y } \quad B \subseteq G_B
  \end{align*}
  Y además:
  \begin{align*}
    \mu_p(G_A \setminus A) < \varepsilon \quad \mbox{ y } \quad \mu_q(G_B \setminus B) < \varepsilon
  \end{align*}
  Podemos hacer que \(\varepsilon \to 0\) entonces:
  \begin{align*}
    \mu_p(G_A) \to \mu_p(A) \quad \mbox{ y } \quad \mu_q(G_B) \to \mu_q(B)
  \end{align*}
  Por tanto, podemos escribir:
  \begin{align*}
    A & = G_A \setminus N_A \quad \mbox{ con } N_A \coloneq G_A \setminus A \in \mathcal{M}_p \quad \mbox{ y } \quad \mu_p(N_A) = 0\\
    B & = G_B \setminus N_B \quad \mbox{ con } N_B \coloneq G_B \setminus B \in \mathcal{M}_q \quad \mbox{ y } \quad \mu_q(N_B) = 0
  \end{align*}
  Entonces:
  \begin{align*}
    A \times B & = (G_A \setminus N_A) \times (G_B \setminus N_B) =\\[2ex]
    & = (G_A \times G_B) \setminus (N_A \times G_B) \setminus (G_A \times N_B) \cup (N_A \times N_B)
  \end{align*}
  Y como \(G_A \times G_B\) es abierto en \(\mathbb{R}^{p + q}\) y los otros tres conjuntos tienen medida nula por los apartados anteriores, tenemos que:
  \begin{align*}
    A \times B \in \mathcal{M}_{p + q}
  \end{align*}
  Ahora, por la descomposición anterior, tenemos:
  \begin{align*}
    \mu_{p + q}(A \times B) = \mu_{p + q}(G_A \times G_B)
  \end{align*}
  Ahora, sea \(G_A \coloneq \cap_{n = 1}^\infty U_n\) donde \(\{U_n\}_{n \in \mathbb{N}}\) es una sucesión decreciente de abiertos de \(\mathbb{R}^p\) con \(\mu_p(U_1) < \infty\) y, de forma análoga, \(G_B \coloneq \cap_{n = 1}^\infty V_n\). Entonces:
  \begin{align*}
    G_A \times G_B = \left(\displaystyle \bigcap_{n = 1}^\infty U_n\right) \times \left(\displaystyle \bigcap_{n = 1}^\infty V_n\right) = \displaystyle \bigcap_{n = 1}^\infty (U_n \times V_n)
  \end{align*}
  Y como \(\mu_{p + q}(U_1 \times V_1) = \mu_p(U_1) \cdot \mu_q(V_1) < \infty\), entonces:
  \begin{align*}
    \mu_{p + q}(G_A \times G_B) & = \lim_{n \to \infty} \mu_{p + q}(U_n \times V_n) = \lim_{n \to \infty} \mu_p(U_n) \cdot \mu_q(V_n) =\\[2ex]
    & = \lim_{n \to \infty} \mu_p(U_n) \cdot \lim_{n \to \infty} \mu_q(V_n) = \mu_p(G_A) \cdot \mu_q(G_B) \to \mu_p(A) \cdot \mu_q(B)
  \end{align*}
  Por lo tanto:
  \begin{align*}
    \mu_{p + q}(A \times B) = \mu_p(A) \cdot \mu_q(B)
  \end{align*}
\end{dem_box}
\newpage

\subsection{Ejercicio 5}
\textit{Sea \(A \in \mathcal{M}_p\) y \(B \in \mathcal{M}_q\) entonces:}
\begin{align*}
  A \times B \in \mathcal{M}_{p + q} \quad \mbox{ y } \quad \mu_{p + q}(A \times B) = \mu_p(A) \cdot \mu_q(B)
\end{align*}
\begin{dem_box}{Demostración}
  Se considera la sucesión de conjuntos:
  \begin{align*}
    K_n = [ - n, n]^p \quad \mbox{ y } \quad L_n = [ - n, n]^q
  \end{align*}
  Entonces, para cada \(n \in \mathbb{N}\) se tiene:
  \begin{align*}
    A_n = A \cap K_n \in \mathcal{M}_p \quad \mbox{ y } \quad B_n = B \cap L_n \in \mathcal{M}_q
  \end{align*}
  Así, tanto \(A_n\) como \(B_n\) tienen medida finita y son una sucesión creciente de conjuntos tales que:
  \begin{align*}
    \displaystyle \bigcup_{n = 1}^\infty A_n = A \quad \mbox{ y } \quad \displaystyle \bigcup_{n = 1}^\infty B_n = B
  \end{align*}
  Así, cada \(A_n \times B_n\) cumple las condiciones del apartado anterior, por lo que:
  \begin{align*}
    A_n \times B_n \in \mathcal{M}_{p + q} \quad \mbox{ y } \quad \mu_{p + q}(A_n \times B_n) = \mu_p(A_n) \cdot \mu_q(B_n)
  \end{align*}
  Y podemos escribir:
  \begin{align*}
    A \times B = \displaystyle \bigcup_{n = 1}^\infty (A_n \times B_n) \in \mathcal{M}_{p + q}
  \end{align*}
  Por otra parte:
  \begin{align*}
    \mu_{p + q}(A \times B) & = \mu_{p + q} \left(\lim_{n \to \infty} (A_n \times B_n)\right) =\\[2ex]
    & =  \lim_{n \to \infty} \mu_{p + q}(A_n \times B_n) = \lim_{n \to \infty} \mu_p(A_n) \cdot \mu_q(B_n)
  \end{align*}
  Así, se dan 3 casos:
  \begin{enumerate}
    \item \(A\) o \(B\) tienen medida nula, supongamos sin pérdida de generalidad que \(\mu_p(A) = 0\). Entonces:
    \begin{align*}
      \mu_{p}(A_n) = 0 \quad \forall n \in \mathbb{N} \implies \mu_{p + q}(A \times B) = 0 = \mu_p(A) \cdot \mu_q(B)
    \end{align*}
    \item \(A\) y \(B\) tienen medida finita. Entonces es lo que ya hemos demostrado en el apartado anterior.
    \item Si alguno de los dos conjuntos tiene medida infinita, supongamos sin pérdida de generalidad que \(\mu_p(A) = \infty\) y \(\mu_q(B) = c > 0\) entonces:
    \begin{align*}
      \mu_{p}(A_n) \to \infty \quad \mbox{ y } \quad \mu_q(B_n) \to c
    \end{align*}
    Y por tanto:
    \begin{align*}
      \mu_{p + q}(A \times B) & = \lim_{n \to \infty} \mu_p(A_n) \cdot \mu_q(B_n) =\\[2ex]
      & = \lim_{n \to \infty} \mu_p(A_n) \cdot c = \infty = \mu_p(A) \cdot \mu_q(B)
    \end{align*}
  \end{enumerate}
\end{dem_box}
\newpage

\subsection{Ejercicio 7}
\textit{Dando por hecho que para todo intervalo abierto \(I\) en \(\mathbb{R}\) se tiene que \(\mu_1(I) = v_1(I)\), demuestra que para todo \(n \in \mathbb{N}\) y todo cubo abierto \(C\) de \(\mathbb{R}^N\) se tiene que \(\mu_N(C) = v_N(C)\).}\vspace{2ex}

\begin{dem_box}{Demostración}
  Sea \(C\) cubo abierto en \(\mathbb{R}^N\) podemos escribirlo como el producto cartesiano de \(N\) intervalos abiertos en \(\mathbb{R}\):
  \begin{align*}
    C = I_1 \times I_2 \times \cdots \times I_N \quad \mbox{ con } I_k = (a_k, b_k) \subseteq \mathbb{R} \quad \forall k = 1, 2, \ldots, N
  \end{align*}
  Queremos demostrar que:
  \begin{align*}
    \mu_N(C) = \prod_{k = 1}^N (b_k - a_k)
  \end{align*}
  Procedemos por inducción sobre \(N\):
  \begin{itemize}
    \item \textit{Caso \(N = 1\):} por hipótesis del enunciado, se tiene que para cualquier intervalo abierto \(I = (a, b) \subseteq \mathbb{R}\):
    \begin{align*}
      \mu_1(I) = v_1(I) = b - a
    \end{align*}
    \item \textit{Hipótesis de inducción:} supongamos que es cierto para \(N = K\), es decir, que sea \(C_K \subseteq \mathbb{R}^K\) un cubo abierto, entonces:
    \begin{align*}
      \mu_K(C_K) = v_K(C_K)
    \end{align*}
    \item \textit{Paso inductivo \(N = K + 1\):} Sea \(C_{K + 1} \subseteq \mathbb{R}^{K + 1}\) cubo abierto, podemos expresar este cubo como el producto cartesiano de un cubo en \(\mathbb{R}^K\) y un intervalo en \(\mathbb{R}\):
    \begin{align*}
      C_{K + 1} = \underbracket{(a_1, b_1) \times (a_2, b_2) \times \cdots \times (a_K, b_K)}_{C_K \subseteq \mathbb{R}^K} \times \underbracket{(a_{K + 1}, b_{K + 1})}_{I \subseteq \mathbb{R}}
    \end{align*}
    Ahora, por el teorema demostrado en los ejercicios anteriores, tenemos que:
    \begin{itemize}
      \item \(C_K \in \mathcal{M}_K\) por ser abierto en \(\mathbb{R}^K\) entonces boreliano y \(\mathcal{B}_K \subseteq \mathcal{M}_K\). Además \(I \in \mathcal{M}_1\) por ser abierto en \(\mathbb{R}\), por lo que:
      \begin{align*}
        C_{K + 1} = C_K  \times  I \in \mathcal{M}_{K + 1}
      \end{align*}
      \item Su medida es:
      \begin{align*}\label{eq:medida_cubo}\tag{I}
        \mu_{K + 1}(C_{K + 1}) = \mu_{K + 1}(C_K \times I) = \mu_K(C_K) \cdot \mu_1(I)
      \end{align*}
    \end{itemize}
    Finalmente:
    \begin{itemize}
      \item Por hipótesis de inducción:
      \begin{align*}
        \mu_K(C_K) = v_K(C_K) = \prod_{k = 1}^K (b_k - a_k)
      \end{align*}
      \item Por hipótesis del enunciado:
      \begin{align*}
        \mu_1(I) = v_1(I) = b_{K + 1} - a_{K + 1}
      \end{align*}
    \end{itemize}
    Por lo tanto, sustituyendo en la ecuación \eqref{eq:medida_cubo} se tiene:
    \begin{align*}
      \mu_{K + 1}(C_{K + 1}) & = \mu_K(C_K) \cdot \mu_1(I) =\\[2ex]
      & = \left(\prod_{k = 1}^K (b_k - a_k)\right) \cdot (b_{K + 1} - a_{K + 1}) = \prod_{k = 1}^{K + 1} (b_k - a_k)
    \end{align*}
  \end{itemize}
\end{dem_box}



\newpage
\section{Aplicaciones del lenguaje de la Teoría de la Medida}
\subsection{Ejercicio 1}
\textit{Sea la sucesión funcional \(f_n(x) = x^n\) con \(x \in \mathbb{R}\). Determina los valores de \(x\) para los que existe \(\lim_n f_n(x)\) y es finito. ¿Qué puedes decir sobre la convergencia uniforme?}\vspace{2ex}

\begin{dem_box}{Demostración}
  Sea \(f_n: \mathbb{R} \to \mathbb{R}\) la sucesión funcional dada por \(f_n(x) = x^n\). Analicemos el límite de \(f_n(x)\) cuando \(n\) tiende a infinito para diferentes valores de \(x\):
  \begin{itemize}
    \item Si \(x > 1\) entonces:
    \begin{align*}
      \lim_n f_n(x) = \lim_n x^n = \infty
    \end{align*}
    \item Si \(x = 1\) entonces:
    \begin{align*}
      \lim_n f_n(1) = \lim_n 1^n = 1
    \end{align*}
    \item Si \(0 < x < 1\) entonces:
    \begin{align*}
      \lim_n f_n(x) = \lim_n x^n = 0
    \end{align*}
    \item Si \(x = 0\) entonces:
    \begin{align*}
      \lim_n f_n(0) = \lim_n 0^n = 0
    \end{align*}
    \item Si \(-1 < x < 0\) entonces:
    \begin{align*}
      \lim_n f_n(x) = \lim_n x^n = 0
    \end{align*}
    \item Si \(x = -1\) entonces:
    \begin{align*}
      \lim_n f_n(-1) \mbox{ no existe (la sucesión oscila entre } -1 \mbox{ y } 1\mbox{)}
    \end{align*}
    \item Si \(x < -1\) entonces:
    \begin{align*}
      \lim_n f_n(x) \mbox{ no existe (la sucesión diverge)}
    \end{align*}
  \end{itemize}
  Por lo tanto, el límite \(\lim_n f_n(x)\) existe y es finito para \(x \in (-1, 1]\). \vspace{2ex}

  En cuanto a la convergencia uniforme, observamos que la sucesión \(f_n(x) = x^n\) no converge uniformemente en \((-1, 1]\). Esto se debe a que para cualquier \(\varepsilon > 0\), podemos encontrar un \(x\) cercano a \(1\) tal que \(f_n(x)\) no esté dentro de \(\varepsilon\) del límite \(0\) para cualquier \(n\). Por ejemplo, si tomamos \(x = 1 - \delta\) con \(\delta\) pequeño, entonces:
  \begin{align*}
    |f_n(1 - \delta) - 0| = |(1 - \delta)^n| \approx 1 \quad \mbox{ para } n \mbox{ grande}
  \end{align*}
  Por lo tanto, la convergencia no es uniforme en \((-1, 1]\).
\end{dem_box}
\newpage

\subsection{Ejercicio 2}
\textit{Sea la sucesión funcional \(f_n(x) = \arctan nx\) con \(x \in \mathbb{R}\). Determina los valores de \(x\) para los que existe \(\lim_n f_n(x)\) y es finito. ¿Qué puedes decir sobre la convergencia uniforme?}\vspace{2ex}

\begin{dem_box}{Demostración}
  La función \(f_n(x) = \arctan(nx)\) converge puntualmente a la función límite \(f(x)\) definida por:
  \begin{align*}
    f(x) = \left\{
      \begin{array}{ll}
        -\frac{\pi}{2} & \mbox{ si } x < 0\\
        0 & \mbox{ si } x = 0\\
        \frac{\pi}{2} & \mbox{ si } x > 0
      \end{array}
    \right.
  \end{align*}
  Ahora, para analizar la convergencia uniforme, consideremos el intervalo \([-M, M]\) para algún \(M > 0\). En este intervalo, la función \(f_n(x)\) converge uniformemente a \(f(x)\) ya que:
  \begin{align*}
    \sup_{x \in [-M, M]} |f_n(x) - f(x)| = \sup_{x \in [-M, M]} |\arctan(nx) - f(x)|
  \end{align*}
  A medida que \(n\) tiende a infinito, \(\arctan(nx)\) se acerca a \(\frac{\pi}{2}\) para \(x > 0\) y a \(-\frac{\pi}{2}\) para \(x < 0\), y a \(0\) en \(x = 0\). Por lo tanto, la convergencia es uniforme en cualquier intervalo compacto que no incluya el punto \(x = 0\). Sin embargo, en todo \(\mathbb{R}\), la convergencia no es uniforme debido a la discontinuidad en \(x = 0\).
\end{dem_box}
\newpage

\subsection{Ejercicio 3}
\textit{Sea \(E \in \mathcal{M}\) tal que \(\mu(E) < \infty\) y \((f_n)_n\) sucesión de funciones medibles \(f_n: E \to \mathbb{R}\) que convergen puntualmente a \(f: E \to \mathbb{R}\). Demuestra que para todo \(\varepsilon > 0\) existe un conjunto medible \(D \subseteq E\) tal que \(\mu(E \setminus D) < \varepsilon\) y \(f_n|_D\) converge uniformemente a \(f|_D\).}\vspace{2ex}

\noindent \textit{Para la demostración, se seguirán los siguientes pasos:}
\begin{enumerate}
  \item \textit{Para cada par \(i, j \in \mathbb{N}\) se considera:}
  \begin{align*}
    E_{ij} \coloneq \left\{x \in E : |f(x) - f_k(x)| < \frac{1}{i} \quad \forall k \geq j \right\}
  \end{align*}
  \textit{Demostrar que es medible.}\vspace{2ex}

  \begin{dem_box}{Demostración}
    Sabemos por hipótesis que \(f, f_k\) son funciones medibles entonces:
    \begin{align*}
      f - f_k \mbox{ es medible} \implies g_k(x) = |f(x) - f_k(x)| \mbox{ es medible}
    \end{align*}
    Para cada \(k \geq j\) y dado \(i\) fijo entonces, consideramos:
    \begin{align*}
      A_{k, i} = \left\{x \in E : g_k(x) < \frac{1}{i}\right\}
    \end{align*}
    Por definición de función medible, entonces:
    \begin{align*}
      A_{k, i} \in \mathcal{M} \implies E_{ij} = \displaystyle \bigcap_{k = j}^\infty A_{k, i} \in \mathcal{M}
    \end{align*}
  \end{dem_box}
  \vspace{3ex}

  \item \textit{Para cada \(k\), determina \(\cup_{j = 1}^\infty E_{kj}\)}\vspace{2ex}
  
  \begin{dem_box}{Demostración}
    Sea \(k \in \mathbb{N}\) cualquiera fijo se tiene:
    \begin{align*}
      \displaystyle \bigcup_{j = 1}^{\infty} E_{kj} = E
    \end{align*}
    Basta notar que:
    \begin{itemize}
      \item[\(\subseteq\))] Por definición \(E_{jk} \subseteq E\) entonces:
      \begin{align*}
        \displaystyle \bigcup_{j = 1}^{\infty} E_{jk} \subseteq E
      \end{align*}
      \item[\(\supseteq\))] Sea \(x \in E\) cualquiera, sabemos que \((f_n)_n\) converge uniformemente a \(f\) en \(E\), es decir, que sea \(\varepsilon > 0\) cualquiera pero fijo, \(\exists n_0 \in \mathbb{N}\) tal que:
      \begin{align*}
        |f(x) - f_m(x)| < \varepsilon \quad \forall m \geq N
      \end{align*}
      Así, tomando \(\varepsilon = \frac{1}{k}\) tenemos que \(\exists j_0(x, k) \in \mathbb{N}\) tal que:
      \begin{align*}
        |f(x) - f_m(x)| < \frac{1}{k} \quad \forall m \geq j_0(x, k)
      \end{align*}
      Y tenemos precisamente:
      \begin{align*}
        E_{kj_0} = \left\{y \in E: |f(y) - f_m(y)| < \frac{1}{k} \quad \forall m \geq j_0\right\}
      \end{align*}
      Por tanto, \(x \in E_{kj_0}\) y así:
      \begin{align*}
        x \in \displaystyle \bigcup_{j = 1}^{\infty} E_{kj}
      \end{align*}
      Y por tanto:
      \begin{align*}
        E \subseteq \displaystyle \bigcup_{j = 1}^{\infty} E_{kj}
      \end{align*}
    \end{itemize}
  \end{dem_box}
  \vspace{3ex}

  \item \textit{Para cada \(k\), encuentra \(i_k\in\mathbb{N}\) verificando que:}
  \begin{align*}
    \mu\left(E\setminus E_{k i_k}\right) < \frac{\varepsilon}{2^k}
  \end{align*}
  \begin{dem_box}{Demostración}
    Sea \(k \in \mathbb{N}\) fijo, por el paso anterior sabemos que:
    \begin{align*}
      \displaystyle \bigcup_{j = 1}^{\infty} E_{kj} = E
    \end{align*}
    Entonces, como la sucesión \((E_{kj})_{j \in \mathbb{N}}\) es creciente, tenemos que:
    \begin{align*}
      \mu(E) & = \mu\left(\displaystyle \bigcup_{j = 1}^{\infty} E_{kj}\right) = \lim_{j \to \infty} \mu(E_{kj}) =\\[2ex]
      & = \lim_{j \to \infty} \left[\mu(E) - \mu(E \setminus E_{kj})\right] = \mu(E) - \lim_{j \to \infty} \mu(E \setminus E_{kj})
    \end{align*}
    Además, como la sucesión \((\mu(E \setminus E_{kj}))_{j \in \mathbb{N}}\) es decreciente y acotada inferiormente por \(0\), entonces:
    \begin{align*}
      \lim_{j \to \infty} \mu(E \setminus E_{kj}) = 0
    \end{align*}
    Entonces, dado \(\varepsilon > 0\) cualquiera, existe \(j_k \in \mathbb{N}\) tal que:
    \begin{align*}
      \mu(E \setminus E_{k j_k}) < \frac{\varepsilon}{2^k}
    \end{align*}
    Por tanto, tomando \(i_k = j_k\) se cumple lo pedido.
  \end{dem_box}
  \vspace{3ex}

  \item \textit{Estudia el conjunto \(D = \displaystyle \bigcap_{k = 1}^\infty E_{k i_k}\)}\vspace{2ex}
  \begin{dem_box}{Demostración}
    Consideramos el conjunto:
    \begin{align*}
      D = \displaystyle \bigcap_{k = 1}^\infty E_{k i_k}
    \end{align*}
    Entonces, por las propiedades de la medida tenemos que:
    \begin{align*}
      \mu(E \setminus D) & = \mu\left(E \setminus \displaystyle \bigcap_{k = 1}^\infty E_{k i_k}\right) = \mu\left(\displaystyle \bigcup_{k = 1}^\infty (E \setminus E_{k i_k})\right) \leq\\[2ex]
      & \leq \displaystyle \sum_{k = 1}^\infty \mu(E \setminus E_{k i_k}) < \displaystyle \sum_{k = 1}^\infty \frac{\varepsilon}{2^k} = \varepsilon
    \end{align*}
    Además, para demostrar que \(f_n|_D\) converge uniformemente a \(f|_D\), sea \(\varepsilon > 0\) cualquiera, entonces existe \(k_0 \in \mathbb{N}\) tal que:
    \begin{align*}
      \frac{1}{k_0} < \varepsilon
    \end{align*}
    Y por definición de \(D\), para todo \(x \in D\) se tiene que:
    \begin{align*}
      |f(x) - f_n(x)| < \frac{1}{k_0} < \varepsilon \quad \forall n \geq i_{k_0}
    \end{align*}
    Por lo tanto, \(f_n|_D\) converge uniformemente a \(f|_D\).
  \end{dem_box}
\end{enumerate}

\newpage

\subsection{Ejercicio 4}
\textit{Demuestra que, en general, el teorema de Egorov falla cuando \(\mu(E) = \infty\).}\vspace{2ex}

\begin{dem_box}{Demostración}
  Consideremos el conjunto \(E = \mathbb{R}\) con la medida de Lebesgue \(\mu\) y la sucesión de funciones \(f_n: \mathbb{R} \to \mathbb{R}\) definida por:
  \begin{align*}
    f_n(x) = \left\{
      \begin{array}{ll}
        1 & \mbox{ si } |x| < n\\
        0 & \mbox{ si } |x| \geq n
      \end{array}
    \right.
  \end{align*}
  Entonces, para cada \(x \in \mathbb{R}\), tenemos:
  \begin{align*}
    \lim_{n \to \infty} f_n(x) = 1
  \end{align*}
  Por lo tanto, la sucesión \(f_n\) converge puntualmente a la función constante \(f(x) = 1\) en todo \(E\). Ahora, supongamos que existe un conjunto medible \(D \subseteq E\) tal que \(\mu(E \setminus D) < \varepsilon\) para algún \(\varepsilon > 0\) y que \(f_n|_D\) converge uniformemente a \(f|_D\). Entonces, para cualquier \(\delta > 0\), existe un \(N \in \mathbb{N}\) tal que:
  \begin{align*}
    |f_n(x) - f(x)| < \delta \quad \forall x \in D, n \geq N
  \end{align*}
  Sin embargo, dado que \(f_n(x) = 0\) para \(x\) con \(|x| \geq n\), podemos elegir \(x_0\) tal que \(|x_0| > N\). Entonces, para este \(x_0\):
  \begin{align*}
    |f_n(x_0) - f(x_0)| = |0 - 1| = 1
  \end{align*}
  Lo cual contradice la suposición de convergencia uniforme en \(D\). Por lo tanto, el teorema de Egorov no se cumple cuando la medida del conjunto es infinita.
\end{dem_box}

\newpage

\subsection{Ejercicio 5}
\textit{Sea \(f: X \to \mathbb{R}\) una función integrable Riemann tal que \(f(x) > 0\) para todo \(x \in [a, b]\), demuestra que \(\int_a^b f > 0\).}\vspace{2ex}

\begin{dem_box}{Demostración}
  Sea \(f: [a, b] \to \mathbb{R}\) una función integrable Riemann tal que \(f(x) > 0\) para todo \(x \in [a, b]\). Como \(f\) es acotada en el intervalo cerrado y es integrable Riemann, entonces es medible y ya que \(f(x) > 0\) para todo \(x \in [a, b]\), tenemos que el conjunto:
  \begin{align*}
    A \coloneq \{x \in [a, b] : f(x) > 0\} = [a, b]
  \end{align*}
  Además, la integral Riemann de \(f\) en \([a,b]\) coincide con la integral de Lebesgue de \(f\) en \([a, b]\). Para cada \(n \in \mathbb{N}\) definimos los conjuntos:
  \begin{align*}
    E_n \coloneq \left\{x \in [a, b] : f(x) > \frac{1}{n}\right\}
  \end{align*}
  Como \(f(x) > 0\) entonces tenemos que:
  \begin{align*}
    \displaystyle \bigcup_{n = 1}^{\infty} E_n = A = [a, b]
  \end{align*}
  Supongamos por contradicción que cada \(E_n\) tiene medida nula, es decir, \(\mu(E_n) = 0\) para todo \(n \in \mathbb{N}\). Entonces, por la subaditividad numerable de la medida de Lebesgue, se tendría que:
  \begin{align*}
    \mu(A) = \mu\left(\displaystyle \bigcup_{n = 1}^{\infty} E_n\right) \leq \displaystyle \sum_{n = 1}^{\infty} \mu(E_n) = 0
  \end{align*}
  Lo cual es imposible porque:
  \begin{align*}
    \mu(A) = \mu([a, b]) = b - a > 0
  \end{align*}
  Además, tenemos que:
  \begin{align*}
    \int_a^b f(x) \; dx \geq \int_{E_n} f(x) \; dx \geq \int_{E_n} \frac{1}{n} \; dx = \frac{1}{n} \mu(E_n) > 0
  \end{align*}
  Por lo tanto, \(\int_a^b f(x) \; dx > 0\).
\end{dem_box}
\newpage


\subsection{Ejercicio 6}
\noindent \textit{Sea \(f: [0, 1] \to \mathbb{R}\) la función dada por:}
\begin{align*}
  f(x) = \left\{
    \begin{array}{ll}
      0 & \mbox{ si } x = 0\\
      0 & \mbox{ si } x \in \mathbb{I} \cap [0, 1]\\
      \frac{1}{q} & \mbox{ si } x \in \mathbb{Q} \setminus \{0\} \mbox{ y } x = \frac{p}{q} \mbox{ con } p, q \in \mathbb{N} \mbox{ primos entre sí}
    \end{array}
  \right.
\end{align*}
\textit{Demostrar que:}
\begin{itemize}
  \item \textit{\(f\) es discontinua en los valores racionales no nulos y continua en el resto}.\vspace{2ex}
  \begin{dem_box}{Demostración}
    Sea \(x_0 \in [0, 1]\) un punto cualquiera:
    \begin{itemize}
      \item Si \(x \in \mathbb{I} \cap [0, 1]\) o \(x = 0\), entonces \(f(x_0) = 0\). Dado \(\varepsilon > 0\) cualquiera, tomamos \(\delta = \varepsilon\). Entonces, si \(x \in (x_0 - \delta, x_0 + \delta) \cap [0, 1]\) se tiene que:
      \begin{align*}
        |f(x) - f(x_0)| = |f(x) - 0| = |f(x)| < \varepsilon
      \end{align*}
      Por lo tanto, \(f\) es continua en \(x_0\).
      \item Si \(x_0 \in \mathbb{Q} \setminus \{0\}\), entonces \(f(x_0) = \frac{1}{q_0}\) donde \(x_0 = \frac{p_0}{q_0}\) con \(p_0, q_0 \in \mathbb{N}\) primos entre sí. Tomamos \(\varepsilon = \frac{1}{2q_0}\). Entonces, para cualquier \(\delta > 0\), en el intervalo \((x_0 - \delta, x_0 + \delta) \cap [0, 1]\) existen números irracionales \(x\) tales que:
      \begin{align*}
        |f(x) - f(x_0)| = \left|0 - \frac{1}{q_0}\right| = \frac{1}{q_0} > \varepsilon
      \end{align*}
      Por lo tanto, \(f\) es discontinua en \(x_0\).
    \end{itemize}
  \end{dem_box}
  \vspace{3ex}

  \item \textit{\(f\) es integrable Riemann.}\vspace{2ex}
  
  \begin{dem_box}{Demostración}
    Basta notar que, por el apartado anterior hemos visto que \(f\) es discontinua en los racionales no nulos, es decir:
    \begin{align*}
      \{x \in (\mathbb{Q} \cap [0, 1]) \setminus \{0\}\}
    \end{align*}
    Como sabemos que los racionales son numerables, entonces en particular los serán en el \([0, 1]\). Por tanto, como la medida de cualquier conjunto numerable es nula, tenemos que:
    \begin{align*}
      \mu(\{x \in (\mathbb{Q} \cap [0, 1]) \setminus \{0\}\}) = 0
    \end{align*}
    Por tanto, por un resultado visto en clase, \(f\) es Riemann integrable si y solo si el conjunto de discontinuidades tiene medida nula, por lo tanto, \(f\) es Riemann integrable
  \end{dem_box}
\end{itemize}

\newpage
\subsection{Ejercicio 7}
\textit{Demuestra que toda función monótona \(f: [a, b] \to \mathbb{R}\) tiene a lo sumo una cantidad numerable de discontinuidades. Para tal fin, se recomienda seguir los siguientes pasos:}
\begin{itemize}
  \item \textit{Demuestra que todas las discontinuidades de \(f\) en \((a, b)\) son de primera especie de salto finito, y que cada salto es igual o inferior a \(f(b) - f(a)\). Demuestra también que si \(f\) es discontinua en \(a\) o \(b\), tal discontinuidad es evitable.}\vspace{2ex}
  
  \begin{dem_box}{Demostración}
    Sea \(f: [a, b] \to \mathbb{R}\) función monótona, supongamos sin pérdida de generalidad que es creciente, es decir, para \(x \in [a, b)\):
    \begin{align*}
      f(x) \leq f(y) \quad \forall y \in (x, b]
    \end{align*}
    Sea \(x_0 \in [a, b]\) un punto donde \(f\) no es continua, entonces el valor en \(f(x_0)\) será estrictamente mayor al valor inmediatamente anterior ya que, como \(f\) es monótona creciente, \(f(x_0)\) tiene que ser mayor al valor de la imagen en el valor inmediatamente anterior y, además, tiene que ser distinto, ya que si no no habría discontinuidad, es decir:
    \begin{align*}
      f(x_0) > f(x) \quad \forall x \in [a, x_0)
    \end{align*}
    Además, el salto es finito ya que \(f\) es acotada en \([a, b]\) porque es monótona (aplicando ?) tenemos que:
    \begin{align*}
      f(a) \leq f(x) \leq f(b) \quad \forall x \in [a, b]
    \end{align*}
    Por lo tanto, el salto en \(x_0\) es:
    \begin{align*}
      \Delta f(x_0) = f(x_0^+) - f(x_0^-) \leq f(b) - f(a)
    \end{align*}
    Por otro lado, si \(x_0 = a\) o \(x_0 = b\), entonces la discontinuidad es evitable ya que:
    \begin{align*}
      \lim_{x \to a^+} f(x) = f(a) \quad \mbox{ y } \quad \lim_{x \to b^-} f(x) = f(b)
    \end{align*}
    Es decir, podemos definir \(f(a)\) y \(f(b)\) como los límites laterales y así eliminar la discontinuidad.
  \end{dem_box}
  \vspace{2ex}

  \item \textit{Dado un número natural cualquiera, demuestra que el número de discontinuidades de salto mayor o igual que \(\frac{f(b) - f(a)}{n}\) es menor o igual que \(n\).}\vspace{2ex}
  \begin{dem_box}{Demostración}
    Definimos el conjunto de discontinuidades de salto mayor o igual que \(\frac{f(b) - f(a)}{n}\):
    \begin{align*}
      D_n \coloneq \left\{x \in (a, b) : \Delta f(x) \geq \frac{f(b) - f(a)}{n}\right\}
    \end{align*}
    Y supongamos que su cardinal es mayor a \(n\), por lo tanto, si calculamos la imagen total de las sumas de todos estos saltos tenemos que:
    \begin{align*}
      \displaystyle \sum_{i = 1}^{m} \Delta f(x) & \geq \displaystyle \sum_{i = 1}^{m} \Delta \frac{f(b) - f(a)}{n} =\\[2ex]
      & = m \cdot \frac{f(b) - f(a)}{n} = n(m - n) \cdot \frac{f(b) - f(a)}{n} = (m - n) (f(b) - f(a)) 
    \end{align*}
    donde \(m > n\).\vspace{2ex}

    Como \(m, n \in \mathbb{N}\) entonces \(m - n \in \mathbb{N}\) y además \(m - n \geq 1\), entonces:
    \begin{align*}
      (m - n)(f(b) - f(a)) \geq f(b) - f(a)
    \end{align*}
    Como sabemos que \(f\) es monótona y hemos supuesto sin pérdida de generalidad que es creciente, entonces:
    \begin{align*}
      \displaystyle \sum_{i = 1}^{m} \Delta f(x) \leq f(b) - f(a)
    \end{align*}
    Por lo tanto, llegamos a una contradicción, ya que hemos visto que:
    \begin{align*}
      \displaystyle \sum_{i = 1}^{m} \Delta f(x) \geq (m - n)(f(b) - f(a)) \geq f(b) - f(a) \geq \displaystyle \sum_{i = 1}^{m} \Delta f(x)
    \end{align*}
    Por lo tanto, el cardinal de \(D_n\) es menor o igual que \(n\).
  \end{dem_box}
  \vspace{2ex}

  \item \textit{Concluye probando ahora que el cardinal del conjunto de todas las discontinuidades es a lo sumo \(\aleph_0\)}.\vspace{2ex}
  
  \begin{dem_box}{Demostración}
    Sea \(D\) el conjunto de todas las discontinuidades de \(f\) en \((a, b)\):
    \begin{align*}
      D \coloneq \{x \in (a, b) : f \mbox{ es discontinua en } x\}
    \end{align*}
    Entonces, podemos expresar \(D\) como la unión numerable de los conjuntos \(D_n\):
    \begin{align*}
      D = \displaystyle \bigcup_{n = 1}^{\infty} D_n
    \end{align*}
    Donde \(D_n\) es el conjunto de discontinuidades de salto mayor o igual que \(\frac{f(b) - f(a)}{n}\). Como hemos visto en el apartado anterior, cada conjunto \(D_n\) tiene cardinal menor o igual que \(n\), por lo que cada \(D_n\) es numerable. Entonces, la unión numerable de conjuntos numerables es numerable, por lo que el conjunto \(D\) también es numerable. Por lo tanto, el cardinal del conjunto de todas las discontinuidades es a lo sumo \(\aleph_0\).
  \end{dem_box}
\end{itemize}
\newpage

\subsection{Ejercicio 8}
\textit{Sea \(f: [a, b] \to \mathbb{R}\) una función integrable Riemann y \(g: [m, M] \to \mathbb{R}\) una función continua, donde \(m = \inf f([a, b])\) y \(M = \sup f([a, b])\). Demuestra que la función \(g \circ f: [a, b] \to \mathbb{R}\) es integrable Riemann.}\vspace{2ex}

\begin{dem_box}{Demostración}
  Sea \(f: [a, b] \to \mathbb{R}\) una función integrable Riemann y \(g: [m, M] \to \mathbb{R}\) una función continua, donde \(m = \inf f([a, b])\) y \(M = \sup f([a, b])\). Como \(f\) es integrable Riemann, entonces es acotada en \([a, b]\), por lo que existen \(m, M \in \mathbb{R}\) tales que:
  \begin{align*}
    m \leq f(x) \leq M \quad \forall x \in [a, b]
  \end{align*}
  Por lo tanto, la imagen de \(f\) está contenida en el intervalo cerrado \([m, M]\). Dado que \(g\) es continua en el intervalo cerrado \([m, M]\), entonces \(g\) es uniformemente continua en dicho intervalo. Por lo tanto, para cualquier \(\varepsilon > 0\), existe un \(\delta > 0\) tal que si \(y_1, y_2 \in [m, M]\) y \(|y_1 - y_2| < \delta\), entonces:
  \begin{align*}
    |g(y_1) - g(y_2)| < \varepsilon
  \end{align*}
  Ahora, como \(f\) es integrable Riemann en \([a, b]\), para cualquier partición \(P = \{x_0, x_1, \ldots, x_n\}\) de \([a, b]\), podemos considerar las sumas superiores e inferiores de Riemann de \(f\):
  \begin{align*}
    U(f, P) = \sum_{i = 1}^{n} M_i (x_i - x_{i-1}) \quad \mbox{ y } \quad L(f, P) = \sum_{i = 1}^{n} m_i (x_i - x_{i-1})
  \end{align*}
  donde \(M_i = \sup_{x \in [x_{i-1}, x_i]} f(x)\) y \(m_i = \inf_{x \in [x_{i-1}, x_i]} f(x)\). Dado que \(f\) es integrable Riemann, para cualquier \(\varepsilon > 0\), existe una partición \(P\) tal que:
  \begin{align*}
    U(f, P) - L(f, P) < \delta
  \end{align*}
  Ahora, consideramos las sumas superiores e inferiores de Riemann de \(g \circ f\) con la misma partición \(P\):
  \begin{align*}
    U(g \circ f, P) = \sum_{i = 1}^{n} G_i (x_i - x_{i-1}) \quad \mbox{ y } \quad L(g \circ f, P) = \sum_{i = 1}^{n} g_i (x_i - x_{i-1})
  \end{align*}
  donde \(G_i = \sup_{x \in [x_{i-1}, x_i]} g(f(x))\) y \(g_i = \inf_{x \in [x_{i-1}, x_i]} g(f(x))\). Dado que \(g\) es uniformemente continua, tenemos que:
  \begin{align*}
    |G_i - g_i| < \varepsilon \quad \forall i = 1, 2, \ldots, n
  \end{align*}
  Por lo tanto, podemos estimar la diferencia entre las sumas superiores e inferiores de Riemann de \(g \circ f\):
  \begin{align*}
    U(g \circ f, P) - L(g \circ f, P) & = \sum_{i = 1}^{n} (G_i - g_i) (x_i - x_{i-1}) < \sum_{i = 1}^{n} \varepsilon (x_i - x_{i-1}) = \varepsilon (b - a)
  \end{align*}
  Dado que \(\varepsilon > 0\) es arbitrario, podemos hacer que \(U(g \circ f, P) - L(g \circ f, P)\) sea tan pequeño como queramos eligiendo una partición adecuada \(P\). Por lo tanto, \(g \circ f\) es integrable Riemann en \([a, b]\).
\end{dem_box}

\newpage
\section{Funciones medibles}
\subsection{Ejercicio 1}
\textit{Sea \(f: \mathbb{R}^N \to \mathbb{R}\) con \(f\) uniformemente continua y acotada. Demuestra que la función \(\varphi : \mathbb{R}^N \to \mathbb{R}\) definida por:}
\begin{align*}
  \varphi(t) \coloneq \sup_{x \in \mathbb{R}^N} \left|f(x + t) - f(x)\right|
\end{align*}
\textit{es medible}.\vspace{2ex}

\begin{dem_box}{Demostración}
  Para ver que \(\varphi\) es medible, veamos por definición que:
  \begin{align*}
    \left\{t \in \mathbb{R}^N : \varphi(t) \leq \alpha\right\} \in \mathcal{M}_N \quad \forall \alpha \in \mathbb{R}
  \end{align*}
  Sea \(\alpha \in \mathbb{R}\) cualquiera pero fijo, tenemos que:
  \begin{align*}
    \{t \in \mathbb{R}^N : \varphi(t) \leq \alpha\} = \left\{t \in \mathbb{R}^N : \sup_{x \in \mathbb{R}^N} |f(x + t) - f(x)| \leq \alpha\right\}
  \end{align*}
  Como \(f\) es uniformemente continua, por definición sabemos que para cualquier \(\varepsilon > 0\) existe un \(\delta > 0\) tal que para cada \(x, y \in \mathbb{R}^N\) con \(\|x - y\| < \delta\) se cumple que:
  \begin{align*}
    |f(x) - f(y)| < \varepsilon
  \end{align*}
  Como \(x + t \in \mathbb{R}^N\) entonces tenemos que:
  \begin{align*}
    |x + t - x| = |t| < \delta \implies |f(x + t) - f(x)| < \varepsilon
  \end{align*}
  Además, \(f\) es acotada, por lo que \(\exists m, M \in \mathbb{R}\) tales que:
  \begin{align*}
    m \leq f(x) \leq M \quad \forall x \in \mathbb{R}^N
  \end{align*}
  \textcolor{red}{Por terminar}
\end{dem_box}
\newpage

\subsection{Ejercicio 2}
\textit{Sea \((X, \Sigma, \mu)\) espacio de medida completo y sean \(f, g\) dos funciones de \(X\) en \(\overline{\mathbb{R}}\). Demuestra que si \(f\) es medible y \(f = g\) casi en todas partes, entonces \(g\) es medible.}\vspace{2ex}

\begin{dem_box}{Demostración}
  Sabemos que, por definición de medibilidad, \(f\) es medible en \((X, \Sigma, \mu)\) si \(\forall \alpha \in \overline{\mathbb{R}}\) se tiene que:
  \begin{align*}
    \left\{x \in X : f(x) < \alpha\right\} \in \Sigma
  \end{align*}
  Queremos ver que esto se cumple para \(g\) también, es decir, que \(\forall \alpha \in \overline{\mathbb{R}}\) se tiene que:
  \begin{align*}\label{eq_ej4.2}\tag{I}
    \left\{x \in X : g(x) < \alpha\right\} \in \Sigma
  \end{align*}
  Podemos notar que, como \(g = f\) en casi todo punto, esto significa que \(\exists B \in \Sigma\) con \(\mu(B) = 0\) tal que:
  \begin{align*}
    \left\{x \in X : g(x) \neq f(x)\right\} \subseteq B
  \end{align*}
  Como es espacio de medida completo y \(\mu(B) = 0\), entonces tenemos que:
  \begin{align*}
    \left\{x \in X : g(x) \neq f(x)\right\} \in \Sigma
  \end{align*}
  Entonces, si volvemos a la expresión \eqref{eq_ej4.2}, tenemos que:
  \begin{align*}
    \left\{x \in X : g(x) < \alpha\right\} & = \left(\left\{x \in X : g(x) < \alpha\right\} \cap \left\{x \in X : g(x) = f(x)\right\}\right) \cup\\[2ex]
    & \hspace{5ex} \cup \left(\left\{x \in X : g(x) < \alpha\right\} \cap \left\{x \in X : g(x) \neq f(x)\right\}\right) = \\[2ex]
    & = \left(\left\{x \in X : f(x) < \alpha\right\} \cap \left\{x \in X : g(x) = f(x)\right\}\right) \cup\\[2ex]
    & \hspace{5ex} \cup \left(\left\{x \in X : g(x) < \alpha\right\} \cap \left\{x \in X : g(x) \neq f(x)\right\}\right)
  \end{align*}
  Como \(f\) es medible, entonces \(\left\{x \in X : f(x) < \alpha\right\} \in \Sigma\) y como hemos visto antes, \(\left\{x \in X : g(x) \neq f(x)\right\} \in \Sigma\). Además, el conjunto \(\left\{x \in X : g(x) = f(x)\right\}\) es el complemento de \(\left\{x \in X : g(x) \neq f(x)\right\}\), por lo que también pertenece a \(\Sigma\). Por lo tanto, la unión de ambos conjuntos también pertenece a \(\Sigma\), es decir:
  \begin{align*}
    \left\{x \in X : g(x) < \alpha\right\} \in \Sigma
  \end{align*}
\end{dem_box}
\newpage

\subsection{Ejercicio 3}
\textit{Sea \(A \in \mathcal{M}_N\) y \(f, g: A \to \overline{\mathbb{R}}\) tales que \(f = g\) en casi todo punto. Demostrar que:}
\begin{itemize}
  \item \textit{Si \(f\) es integrable sobre \(A\) entonces \(g\) también lo es}\vspace{2ex}
  \begin{dem_box}{Demostración}
    Sea \(f: A \to \overline{\mathbb{R}}\) integrable, entonces tenemos que:
    \begin{align*}
      \int_A f^{ + } \; d\mu < \infty \quad \mbox{ o } \quad \int_A f^{ - } \; d\mu < \infty
    \end{align*}
    Supongamos sin pérdida de generalidad que \(\int_A f^{ + } \; d\mu\) finita.\vspace{2ex}

    Como \(f = g\) \(\mu\)-a.e., por definición, existe \(B \in \mathcal{M}_N\) con \(\mu(B) = 0\) tal que:
    \begin{align*}
      \left\{x \in A : f(x) \neq g(x)\right\} \subseteq B
    \end{align*}
    Así, como \(B\) es de medida nula, tenemos que:
    \begin{align*}
      \int_A f^ + \; d\mu = \int_{A\setminus B} f^ + \; d\mu = \int_{A\setminus B} g^ + \; d\mu \leq \int_A g^ + \; d\mu
    \end{align*}
    Por lo tanto, \(\int_A g^ + \; d\mu\) es finita, y así \(g\) es integrable sobre \(A\).
  \end{dem_box}
  \vspace{3ex}

  \item \textit{Si \(f \in \mathcal{L}_1(A)\) entonces \(g\) es sumable sobre \(A\)}.\vspace{2ex}
  \begin{dem_box}{Demostración}
    Sea \(f \in \mathcal{L}_1(A)\), entonces \(f\) es integrable sobre \(A\) y:
    \begin{align*}
      \int_A |f| \; d\mu < \infty
    \end{align*}
    Como \(f = g\) \(\mu\)-a.e., por definición, existe \(B \in \mathcal{M}_N\) con \(\mu(B) = 0\) tal que:
    \begin{align*}
      \left\{x \in A : f(x) \neq g(x)\right\} \subseteq B
    \end{align*}
    Así, como \(B\) es de medida nula, tenemos que:
    \begin{align*}
      \int_A |f| \; d\mu = \int_{A\setminus B} |f| \; d\mu = \int_{A\setminus B} |g| \; d\mu \leq \int_A |g| \; d\mu
    \end{align*}
    Por lo tanto, \(\int_A |g| \; d\mu\) es finita, y así \(g\) es sumable sobre \(A\).
  \end{dem_box}
\end{itemize}
\newpage

\subsection{Ejercicio 4}
\textit{Dada una función \(f\) con dominio \(A \subseteq \mathbb{R}^N\) y llegada en \(\overline{\mathbb{R}}\), demuestra que si es continua en \(x_0 \in A\) entonces sus partes positiva y negativa, \(f^{ + }\) y \(f^{ - }\), también lo son en \(x_0\).}\vspace{2ex}

\begin{dem_box}{Demostración}
  Sea \(f: A \subseteq \mathbb{R}^N \to \overline{\mathbb{R}}\) continua en \(x_0 \in A\) entonces, por definición de continuidad, para cualquier \(\varepsilon > 0\) existe un \(\delta > 0\) tal que:
  \begin{align*}
    |x - x_0| < \delta \implies |f(x) - f(x_0)| < \varepsilon
  \end{align*}
  Ahora, consideremos las funciones \(f^{ + }\) y \(f^{ - }\) definidas por:
  \begin{align*}
    f^{ + }(x) = \max\{f(x), 0\} \quad \mbox{ y } \quad f^{ - }(x) = \max\{-f(x), 0\}
  \end{align*}
  Queremos demostrar que \(f^{ + }\) es continua en \(x_0\). Sea \(\varepsilon > 0\) cualquiera, tomamos el mismo \(\delta > 0\) que para \(f\). Entonces, si \( |x - x_0| < \delta\), tenemos dos casos:
  \begin{itemize}
    \item Si \(f(x_0) \geq 0\), entonces:
    \begin{align*}
      |f^{ + }(x) - f^{ + }(x_0)| & = | \max\{f(x), 0\} - f(x_0)| \leq |f(x) - f(x_0)| < \varepsilon
    \end{align*}
    \item Si \(f(x_0) < 0\), entonces:
    \begin{align*}
      |f^{ + }(x) - f^{ + }(x_0)| & = | \max\{f(x), 0\} - 0| = | \max\{f(x), 0\}| \leq |f(x) - f(x_0)| < \varepsilon
    \end{align*}
  \end{itemize}
  En ambos casos, hemos demostrado que \(f^{ + }\) es continua en \(x_0\).\vspace{2ex}

  De manera similar, podemos demostrar que \(f^{ - }\) es continua en \(x_0\). Sea \(\varepsilon > 0\) cualquiera, tomamos el mismo \(\delta > 0\) que para \(f\). Entonces, si \( |x - x_0| < \delta\), tenemos dos casos:
  \begin{itemize}
    \item Si \(f(x_0) \leq 0\), entonces:
    \begin{align*}
      |f^{ - }(x) - f^{ - }(x_0)| & = | \max\{-f(x), 0\} - (-f(x_0))| \leq\\[2ex]
      & \leq |-f(x) + f(x_0)| = |f(x) - f(x_0)| < \varepsilon
    \end{align*}
    \item If \(f(x_0) > 0\), entonces:
    \begin{align*}
      |f^{ - }(x) - f^{ - }(x_0)| & = | \max\{-f(x), 0\} - 0| = \\[2ex]
      & = | \max\{-f(x), 0\}| \leq |f(x) - f(x_0)| < \varepsilon
    \end{align*}
  \end{itemize}
  En ambos casos, hemos demostrado que \(f^{ - }\) es continua en \(x_0\).
\end{dem_box}
\newpage

\subsection{Ejercicio 5}
\textit{En cada uno de los siguientes casos, estudiar si la sucesión de funciones \((f_n)_{n \in \mathbb{N}}\) es monótona, si converge puntualmente, si converge uniformemente, si se puede aplicar el teorema de la convergencia monótona o dominada y si es posible intercambiar el límite y la integral:}
\begin{align*}
  \lim_n \int_\mathbb{R} f_n(x) \; dx = \int_\mathbb{R} \lim_n f_n(x) \; dx
\end{align*}
\textit{en los siguientes casos:}
\begin{enumerate}
  \item \(f_n = \mathcal{X}_{[0, n]}\) \vspace{2ex}
  \begin{dem_box}{Demostración}
    Sea \(f_n = \mathcal{X}_{[0, n]}\):
    \begin{itemize}
      \item \textit{Monotonía}: La sucesión es claramente monótona creciente ya que:
      \begin{align*}
        f_n(x) \leq f_{n + 1}(x) \quad \forall x \in \mathbb{R}, \; \forall n \in \mathbb{N}
      \end{align*}
      \item \textit{Convergencia puntual}: La sucesión converge puntualmente a la función:
      \begin{align*}
        f(x) = \left\{
          \begin{array}{ll}
            1 & \mbox{ si } x \geq 0\\
            0 & \mbox{ si } x < 0
          \end{array}
        \right.
      \end{align*}
      \item \textit{Convergencia uniforme}: La sucesión no converge uniformemente, ya que:
      \begin{align*}
        \sup_{x \in \mathbb{R}} |f_n(x) - f(x)| = 1 \quad \forall n \in \mathbb{N}
      \end{align*}
      Por tanto, no existe \(N \in \mathbb{N}\) tal que para todo \(n \geq N\) se tenga que:
      \begin{align*}
        \sup_{x \in \mathbb{R}} |f_n(x) - f(x)| < \varepsilon
      \end{align*}
      \item \textit{Teorema de la convergencia monótona:} Sí, se puede aplicar el teorema de la convergencia monótona ya que la sucesión es monótona creciente y converge puntualmente a una función medible.
      \item \textit{Teorema de la convergencia dominada:} No, no se puede aplicar el teorema de la convergencia dominada ya que no existe una función integrable que domine a toda la sucesión.
      \item \textit{Intercambio de límite e integral:} Sí, se puede intercambiar el límite y la integral, ya que se cumple:
      \begin{align*}
        \lim_n \int_\mathbb{R} f_n(x) \; dx & = \lim_n \int_0^n 1 \; dx = \lim_n n = \infty\\[2ex]
        \int_\mathbb{R} \lim_n f_n(x) \; dx & = \int_0^\infty 1 \; dx = \infty
      \end{align*}
    \end{itemize}
  \end{dem_box}
  \vspace{3ex}

  \item \(f_n = \frac{1}{n} \mathcal{X}_{[0, \infty)}\) \vspace{2ex}
  \begin{dem_box}{Demostración}
    Sea \(f_n = \frac{1}{n}\mathcal{X}_{[0, \infty)}\) entonces:
    \begin{itemize}
      \item \textit{Monotonía:} La función se puede descomponer en:
      \begin{align*}
        f_n(x) = \underbracket{\frac{1}{n}}_{\substack{\text{monótona}\\ \text{decreciente}}} \underbracket{\mathcal{X}_{[0, \infty)}}_{\text{cte.}}
      \end{align*}
      Por lo tanto, la función es monótona decreciente.
      \item \textit{Convergencia puntual:} Podemos notar que:
      \begin{align*}
        f_n(x) = \left\{
          \begin{array}{ll}
            \frac{1}{n} & \mbox{ si } x \geq 0\\[2ex]
            0 & \mbox{ si } x < 0
          \end{array}
        \right. \xrightarrow[n \to \infty]{} f(x) = 0
      \end{align*}
      Por tanto, la función converge puntualmente a 0.
      \item \textit{Convergencia uniforme:} La función converge uniformemente a 0, ya que:
      \begin{align*}
        \sup_{x \in \mathbb{R}} |f_n(x) - 0| = \sup_{x \in \mathbb{R}} |f_n(x)| = \frac{1}{n} \xrightarrow[n \to \infty]{} 0
      \end{align*}
      Por tanto, para cualquier \(\varepsilon > 0\), existe \(N \in \mathbb{N}\) tal que para todo \(n \geq N\) se tiene que:
      \begin{align*}
        \sup_{x \in \mathbb{R}} |f_n(x) - 0| < \varepsilon
      \end{align*}
      \item \textit{Teorema de la convergencia monótona:} Sí, se puede aplicar el teorema de la convergencia monótona ya que la sucesión es monótona decreciente y converge puntualmente a una función medible.
      \item \textit{Teorema de la convergencia dominada:} La sucesión está dominada por la función \(g(x) = 1\), podemos ver si es sumable o no:
      \begin{align*}
        \int_\mathbb{R} g(x) \; dx = \int_0^\infty 1 \; dx = \infty
      \end{align*}
      Por lo tanto, no se puede aplicar el teorema de la convergencia dominada.
      \item \textit{Intercambio de límite e integral:} Sí, se puede intercambiar el límite y la integral, ya que se cumple:
      \begin{align*}
        \lim_n \int_\mathbb{R} f_n(x) \; dx & = \lim_n \int_0^\infty \frac{1}{n} \; dx = \lim_n \frac{1}{n} \cdot \infty = 0\\[2ex]
        \int_\mathbb{R} \lim_n f_n(x) \; dx & = \int_0^\infty 0 \; dx = 0
      \end{align*}
    \end{itemize}
  \end{dem_box}
  \vspace{3ex}

  \item \(f_n = \frac{1}{n} \mathcal{X}_{[0, n]}\) \vspace{2ex}
  \begin{dem_box}{Demostración}
    Sea \(f_n = \frac{1}{n} \mathcal{X}_{[0, n]}\):
    \begin{itemize}
      \item \textit{Monotonía:} La función se puede descomponer en:
      \begin{align*}
        f_n(x) = \underbracket{\frac{1}{n}}_{\substack{\text{monótona}\\ \text{decreciente}}} \underbracket{\mathcal{X}_{[0, n]}}_{\substack{\text{monótona}\\ \text{creciente}}}
      \end{align*}
      Por lo tanto, la función no es monótona.
      \item \textit{Convergencia puntual:} Podemos notar que:
      \begin{align*}
        f_n(x) = \left\{
          \begin{array}{ll}
            \frac{1}{n} & \mbox{ si } 0 \leq x \leq n\\[2ex]
            0 & \mbox{ si } x < 0 \mbox{ o } x > n
          \end{array}
        \right. \xrightarrow[n \to \infty]{} f(x) = 0
      \end{align*}
      Por tanto, la función converge puntualmente a 0.
      \item \textit{Convergencia uniforme:} La función converge uniformemente a 0, ya que:
      \begin{align*}
        \sup_{x \in \mathbb{R}} |f_n(x) - 0| = \sup_{x \in \mathbb{R}} |f_n(x)| = \frac{1}{n} \xrightarrow[n \to \infty]{} 0
      \end{align*}
      Por tanto, para cualquier \(\varepsilon > 0\), existe \(N \in \mathbb{N}\) tal que para todo \(n \geq N\) se tiene que:
      \begin{align*}
        \sup_{x \in \mathbb{R}} |f_n(x) - 0| < \varepsilon
      \end{align*}
      \item \textit{Teorema de la convergencia monótona:} No, no se puede aplicar el teorema de la convergencia monótona ya que la sucesión no es monótona.
      \item \textit{Teorema de la convergencia dominada:} No, no se puede aplicar el teorema de la convergencia dominada ya que no existe una función integrable que domine a toda la sucesión.
      \item \textit{Intercambio de límite e integral:} No, no se puede intercambiar el límite y la integral, ya que se cumple:
      \begin{align*}
        \lim_n \int_\mathbb{R} f_n(x) \; dx & = \lim_n \int_0^n \frac{1}{n} \; dx = \lim_n \frac{1}{n} \cdot n = 1\\[2ex]
        \int_\mathbb{R} \lim_n f_n(x) \; dx & = \int_0^\infty 0 \; dx = 0
      \end{align*}
    \end{itemize}
  \end{dem_box}
  \vspace{3ex}

  \item \(f_n = n \mathcal{X}_[\frac{1}{n}, \frac{2}{n}]\) \vspace{2ex}
  \begin{dem_box}{Demostración}
    Sea \(f_n = n \mathcal{X}_{[\frac{1}{n}, \frac{2}{n}]}\):
    \begin{itemize}
      \item \textit{Monotonía:} La función no es monótona.
      \item \textit{Convergencia puntual:} Podemos notar que:
      \begin{align*}
        f_n(x) = \left\{
          \begin{array}{ll}
            n & \mbox{ si } \frac{1}{n} \leq x \leq \frac{2}{n}\\[2ex]
            0 & \mbox{ si } x < \frac{1}{n} \mbox{ o } x > \frac{2}{n}
          \end{array}
        \right. \xrightarrow[n \to \infty]{} f(x) = 0
      \end{align*}
      Por tanto, la función converge puntualmente a 0.
      \item \textit{Convergencia uniforme:} La función no converge uniformemente a 0, ya que:
      \begin{align*}
        \sup_{x \in \mathbb{R}} |f_n(x) - 0| = \sup_{x \in \mathbb{R}} |f_n(x)| = n \quad \forall n \in \mathbb{N}
      \end{align*}
      Por tanto, no existe \(N \in \mathbb{N}\) tal que para todo \(n \geq N\) se tenga que:
      \begin{align*}
        \sup_{x \in \mathbb{R}} |f_n(x) - 0| < \varepsilon
      \end{align*}
      \item \textit{Teorema de la convergencia monótona:} No, no se puede aplicar el teorema de la convergencia monótona ya que la sucesión no es monótona.
      \item \textit{Teorema de la convergencia dominada:} No, no se puede aplicar el teorema de la convergencia dominada ya que no existe una función integrable que domine a toda la sucesión.
      \item \textit{Intercambio de límite e integral:} No, no se puede intercambiar el límite y la integral, ya que se cumple:
      \begin{align*}
        \lim_n \int_\mathbb{R} f_n(x) \; dx & = \lim_n \int_{\frac{1}{n}}^{\frac{2}{n}} n \; dx = \lim_n n \cdot \left(\frac{2}{n} - \frac{1}{n}\right) = \lim_n 1 = 1\\[2ex]
        \int_\mathbb{R} \lim_n f_n(x) \; dx & = \int_0^\infty 0 \; dx = 0
      \end{align*}
    \end{itemize}
  \end{dem_box}
\end{enumerate}
\newpage

\subsection{Ejercicio 6}
\textit{Para cada \(n \in \mathbb{N}\) se considera la función \(f_n: [1, \infty] \to \mathbb{R}\) dada por:}
\begin{align*}
  f_n (x) = x^{ - 1 - n^{ - 1}}
\end{align*}
\textit{Sea \(f: [1, \infty] \to \mathbb{R}\) la función dada por:}
\begin{align*}
  f(x) = x^{ - 1}
\end{align*}
\textit{Demuestra que \(f_n \in \mathcal{L}_1([1, \infty])\) para todo \(n\), que \(f \notin \mathcal{L}_1([1, \infty])\) y que \((f_n)_n\) converge uniformemente a \(f\) en \([1, \infty]\).}\vspace{2ex}

\begin{dem_box}{Demostración}
  Sea \(n \in \mathbb{N}\) cualquiera pero fijo, consideremos la función \(f_n: [1, \infty) \to \mathbb{R}\) dada por:
  \begin{align*}
    f_n (x) = x^{ - 1 - n^{ - 1}}
  \end{align*}
  Para ver que \(f_n \in \mathcal{L}_1([1, \infty))\), debemos comprobar que:
  \begin{align*}
    \int_1^\infty |f_n(x)| \; dx < \infty
  \end{align*}
  Calculamos la integral impropia:
  \begin{align*}
    \int_1^\infty x^{ - 1 - n^{ - 1}} \; dx & = \lim_{t \to \infty} \int_1^t x^{ - 1 - n^{ - 1}} \; dx = \lim_{t \to \infty} \left[ \frac{x^{ - n^{ - 1}}}{- n^{ - 1}} \right]_1^t = \lim_{t \to \infty} \left( - n x^{ - n^{ - 1}} \Big|_1^t \right) = \\[2ex]
    & = \lim_{t \to \infty} \left( - n t^{ - n^{ - 1}} + n \right) = n
  \end{align*}
  Por lo tanto, \(f_n \in \mathcal{L}_1([1, \infty))\) para todo \(n \in \mathbb{N}\).\vspace{2ex}

  Ahora, consideremos la función \(f: [1, \infty) \to \mathbb{R}\) dada por:
  \begin{align*}
    f(x) = x^{ - 1}
  \end{align*}
  Para ver que \(f \notin \mathcal{L}_1([1, \infty))\), debemos comprobar que:
  \begin{align*}
    \int_1^\infty |f(x)| \; dx = \infty
  \end{align*}
  Calculamos la integral impropia:
  \begin{align*}
    \int_1^\infty f \; dx = \lim_{t \to \infty} \int_0^t \frac{1}{x} \; dx = \lim_{t \to \infty} \left[ \ln|x| \right]_1^t = \lim_{t \to \infty} (\ln t - \ln 1) = \infty
  \end{align*}
  Por lo tanto, \(f \notin \mathcal{L}_1([1, \infty))\).\vspace{2ex}

  Finalmente, veamos que la sucesión \((f_n)_{n \in \mathbb{N}}\) converge uniformemente a \(f\) en \([1, \infty)\). Para ello, debemos comprobar que:
  \begin{align*}
    \sup_{x \in [1, \infty)} |f_n(x) - f(x)| \xrightarrow[n \to \infty]{} 0
  \end{align*}
  Calculamos la diferencia:
  \begin{align*}
    |f_n(x) - f(x)| & = \left| x^{ - 1 - n^{ - 1}} - x^{ - 1} \right| = \left| x^{ - 1} \left( x^{ - n^{ - 1}} - 1 \right) \right| = \frac{1}{x} \left| x^{ - n^{ - 1}} - 1 \right|
  \end{align*}
  Para \(x \geq 1\), tenemos que \(x^{ - n^{ - 1}} \leq 1\), por lo que:
  \begin{align*}
    |f_n(x) - f(x)| & = \frac{1}{x} (1 - x^{ - n^{ - 1}}) \leq 1 - x^{ - n^{ - 1}} \leq 1 - 1 = 0
  \end{align*} 
  Por lo tanto, para cualquier \(\varepsilon > 0\), existe \(N \in \mathbb{N}\) tal que para todo \(n \geq N\) se tiene que:
  \begin{align*}
    \sup_{x \in [1, \infty)} |f_n(x) - f(x)| < \varepsilon
  \end{align*}
  Así, hemos demostrado que \((f_n)_{n \in \mathbb{N}}\) converge uniformemente a \(f\) en \([1, \infty)\).
\end{dem_box}

\newpage

\subsection{Ejercicio 7}
\textit{Sea la función \(f:[a, b] \to \mathbb{R} \) integrable Riemann, demuestra que:}
\begin{align*}
  \int_a^b f = \lim_{n \to \infty} \frac{b - a}{n} \sum_{k = 1}^n f\left(a + k \frac{b - a}{n}\right)
\end{align*}
\begin{dem_box}{Demostración}
  Sea \(f: [a, b] \to \mathbb{R}\) una función integrable Riemann, entonces por definición de integral de Riemann, tenemos que:
  \begin{align*}
    \int_a^b f(x) \; dx = \lim_{||P|| \to 0} \sum_{i = 1}^n f(c_i) \Delta x_i
  \end{align*}
  donde \(P = \{x_0, x_1, \ldots, x_n\}\) es una partición de \([a, b]\), \(||P||\) es la norma de la partición, \(c_i \in [x_{i - 1}, x_i]\) y \(\Delta x_i = x_i - x_{i - 1}\).\vspace{2ex}

  Consideremos la partición uniforme \(P_n\) de \([a, b]\) dada por:
  \begin{align*}
    x_i = a + i \frac{b - a}{n} \quad \mbox{ para } i = 0, 1, \ldots, n
  \end{align*}
  Entonces, la norma de esta partición es:
  \begin{align*}
    ||P_n|| = \max_{1 \leq i \leq n} \Delta x_i = \frac{b - a}{n}
  \end{align*}
  Por lo tanto, cuando \(n \to \infty\), tenemos que \(||P_n|| \to 0\).\vspace{2ex}

  Ahora, tomando \(c_i = x_i\) en la suma de Riemann, obtenemos:
  \begin{align*}
    \sum_{i = 1}^n f(c_i) \Delta x_i & = \sum_{i = 1}^n f\left(a + i \frac{b - a}{n}\right) \cdot \frac{b - a}{n} = \frac{b - a}{n} \sum_{i = 1}^n f\left(a + i \frac{b - a}{n}\right)
  \end{align*}
  Por lo tanto, podemos escribir la integral como:
  \begin{align*}
    \int_a^b f(x) \; dx & = \lim_{n \to \infty} \frac{b - a}{n} \sum_{i = 1}^n f\left(a + i \frac{b - a}{n}\right)
  \end{align*}
  Así, hemos demostrado que:
  \begin{align*}
    \int_a^b f = \lim_{n \to \infty} \frac{b - a}{n} \sum_{k = 1}^n f\left(a + k \frac{b - a}{n}\right)
  \end{align*}
\end{dem_box}


\newpage
\subsection{Ejercicio 8}
\textit{Calcula los siguientes límites:}
\begin{enumerate}
  \item \(\displaystyle \lim_{n \to \infty} \left(\dfrac{1}{n} + \dfrac{1}{n + 1} + \ldots + \dfrac{1}{2n}\right)\)\vspace{2ex}
  \begin{dem_box}{Demostración}
    Podemos notar que la suma se puede escribir como:
    \begin{align*}
      L = \lim_{n \to \infty} \displaystyle \sum_{k = 0}^{n} \frac{1}{n + k}
    \end{align*}
    Y queremos que la suma se parezca a una integral de Riemann, es decir, que tenga la forma:
    \begin{align*}
      \int_a^b f(x) \; dx = \lim_{n \to \infty} \frac{b - a}{n} \sum_{k = 1}^n f\left(a + k \frac{b - a}{n}\right)
    \end{align*}
    En este caso, podemos notar que \(a = 1\), \(b = 2\) y \(f(x) = \frac{1}{x}\). Por lo tanto, podemos escribir la suma como:
    \begin{align*}
      L & = \lim_{n \to \infty} \frac{1}{n} \sum_{k = 0}^{n} f\left(1 + \frac{k}{n}\right) = \int_1^2 \frac{1}{x}
    \end{align*}
    Calculamos la integral:
    \begin{align*}
      \int_1^2 \frac{1}{x} \; dx & = \left[ \ln|x| \right]_1^2 = \ln 2 - \ln 1 = \ln 2
    \end{align*}
  \end{dem_box}
  \vspace{3ex}

  \item \(\displaystyle \lim_{n \to \infty} \left(\frac{1}{\sqrt{n^2 - 1}} + \frac{1}{\sqrt{n^2 - 2^2}} + \ldots + \frac{1}{\sqrt{n^2 - (n - 1)^2}}\right)\)\vspace{2ex}
  
  \begin{dem_box}{Demostración}
    Podemos notar que la suma se puede escribir como:
    \begin{align*}
      L = \lim_{n \to \infty} \displaystyle \sum_{k = 1}^{n - 1} \frac{1}{\sqrt{n^2 - k^2}}
    \end{align*}
    Y queremos que se parezca a una integral de Riemann, es decir, que tenga la forma:
    \begin{align*}
      \int_a^b f(x) \; dx = \lim_{n \to \infty} \frac{b - a}{n} \sum_{k = 1}^n f\left(a + k \frac{b - a}{n}\right)
    \end{align*}
    Por tanto, podemos escribir la suma como:
    \begin{align*}
      L & = \lim_{n \to \infty} \frac{1}{n} \sum_{k = 1}^{n - 1} f\left(\frac{k}{n}\right) = \int_0^1 \frac{1}{\sqrt{1 - x^2}} \; dx 
    \end{align*}
    Calculamos la integral:
    \begin{align*}
      \int_0^1 \frac{1}{\sqrt{1 - x^2}} \; dx & = \left[ \arcsin x \right]_0^1 = \arcsin 1 - \arcsin 0 = \frac{\pi}{2} - 0 = \frac{\pi}{2}
    \end{align*}
  \end{dem_box}
\end{enumerate}

\newpage
\section{Teorema de Fubini y cambio de variable}
\subsection{Ejercicio 1}
\subsubsection{Apartado 1}
\textit{Calcula la siguiente integral}:
\begin{align*}
  \int_M x^2 y d\mu_2(x, y) \quad \mbox{ con } M = [0, 1] \times [0, 1]
\end{align*}
\begin{dem_box}{Demostración}
  Podemos notar que la función \(f(x, y) = x^2y\) es integrable en el conjunto \(M = [0, 1] \times [0, 1]\) ya que es continua en un conjunto compacto y además está acotada. Por tanto, como \(f\) es integrable, entonces por un resultado previo, \(f\) es medible y por tanto podemos aplicar el teorema de Fubini para calcular la integral:
  \begin{align*}
    \int_M x^2 y \; d\mu_2(x, y) & = \int_0^1 \left( \int_0^1 x^2 y \; dx \right) dy = \int_0^1 \left[ \frac{x^3}{3} y \right]_0^1 dy = \int_0^1 \frac{y}{3} dy = \left[ \frac{y^2}{6} \right]_0^1 = \frac{1}{6}
  \end{align*}
\end{dem_box}
\newpage

\subsubsection{Apartado 2}
\textit{Calcula la siguiente integral}:
\begin{align*}
  \int_M \frac{x^2}{1 + y^2} d\mu_2(x, y) \quad \mbox{ con } M = [0, 1] \times [0, 1] 
\end{align*}
\begin{dem_box}{Demostración}
  Podemos notar que la función \(f(x, y) = \frac{x^2}{1 + y^2}\) es integrable en el conjunto \(M = [0, 1] \times [0, 1]\) ya que es continua en un conjunto compacto y además está acotada. Por tanto, como \(f\) es integrable, entonces por un resultado de clase, \(f\) es medible y por tanto podemos aplicar el teorema de Fubini para calcular la integral:
  \begin{align*}
    \int_M \frac{x^2}{1 + y^2} \; d\mu_2(x, y) & = \int_0^1 \left( \int_0^1 \frac{x^2}{1 + y^2} \; dx \right) dy = \int_0^1 \left[\frac{x^3}{3(1 + y^3)}\right]_0^1 \; dy = \\[2ex]
    & = \int_0^1 \frac{1}{3(1 + y^2)} \; dy = \left[ \frac{1}{3} \arctan y \right]_0^1 = \frac{\pi}{12}
  \end{align*}
\end{dem_box}
\newpage

\subsubsection{Apartado 3}
\textit{Calcula la siguiente integral}:
\begin{align*}
  \int_M (\log x) y d\mu_2(x, y) \quad \mbox{ con } M = [1, e] \times [1, e]
\end{align*}
\begin{dem_box}{Demostración}
  Podemos notar que la función \(f(x, y) = (\log x) y\) es integrable en el conjunto \(M = [1, e] \times [1, e]\) ya que es continua en un conjunto compacto y además está acotada. Por tanto, como \(f\) es integrable, entonces por un resultado de clase, \(f\) es medible y por tanto podemos aplicar el teorema de Fubini para calcular la integral:
  \begin{align*}
    \int_M (\log x) y \; d\mu_2(x, y) & = \int_1^e \left( \int_1^e (\log x) y \; dx \right) dy
  \end{align*}
  Veamos cual es la integral del logaritmo integrando por partes:
  \begin{align*}
    \int \log x \; dx \xlongequal[dv = dx \Rightarrow v = x]{u = \log x \Rightarrow du = \frac{1}{x} dx} x \log x - \int x \frac{1}{x} dx = x \log x - x + C
  \end{align*}  
  Por tanto:
  \begin{align*}
    \int_1^e \left( \int_1^e (\log x) y \; dx \right) dy & = \int_1^e y \cdot \left[x \log x - x\right]_1^e \; dy = \\[2ex]
    & = \int_1^e y \cdot \left(e - e - (0 - 1)\right) \; dy = \int_1^e y\; dy = \\[2ex]
    & = \left[\frac{y^2}{2}\right]_1^e = \frac{e^2}{2} - \frac{1}{2} = \frac{e^2 - 1}{2} 
  \end{align*}
\end{dem_box}
\newpage

\subsubsection{Apartado 4}
\textit{Calcula la siguiente integral}:
\begin{align*}
  \int_M (\log x) y \; d\mu(x, y) \quad \mbox{ con } M = (0, 1] \times [0, 1]
\end{align*}
\begin{dem_box}{Demostración}
  En este caso, el intervalo en el que se define la función no es compacto y, además, no podemos asegurar nada de la acotación de la función ya que:
  \begin{align*}
    \lim_{x \to 0^ + } \log x = - \infty
  \end{align*}
  Sin embargo, como la función es negativa en \((0, 1] \times [0, 1]\) ya que \(y \geq 0\) y \(\log x \leq 0\) en dicho intervalo, por tanto, para ver si es sumable, podemos ver si el valor absoluto de la función es sumable, es decir:
  \begin{align*}
    \int_M |(\log x) y| \; d\mu(x, y) & = \int_0^1 \left( \int_0^1 |(\log x) y| \; dx \right) dy = \int_0^1 \left( \int_0^1 - (\log x) y \; dx \right) dy = \\[2ex]
    & = \int_0^1 y \cdot \left[ - (x \log x - x) \right]_0^1 \; dy = \int_0^1 y \cdot (0 + 1) \; dy = \left[ \frac{y^2}{2} \right]_0^1 = \frac{1}{2}
  \end{align*}
  Que claramente es finito, por lo que la función es sumable y podemos calcular la integral original a través del teorema de Fubini:
  \begin{align*}
    \int_M (\log x) y \; d\mu(x, y) & = \int_0^1 \left( \int_0^1 (\log x) y \; dx \right) dy = \int_0^1 y \cdot \left[ x \log x - x \right]_0^1 \; dy = \\[2ex]
    & = \int_0^1 y \cdot (0 - 1) \; dy = \left[ - \frac{y^2}{2} \right]_0^1 = - \frac{1}{2}
  \end{align*}
\end{dem_box}
\newpage

\subsubsection{Apartado 5}
\textit{Calcula la siguiente integral}:
\begin{align*}
  \int_M x^3 y^3 \; d\mu(x, y) \quad \mbox{ con } M = [0, 1] \times [0, 1]
\end{align*}
\begin{dem_box}{Demostración}
  Podemos notar que la función \(f(x, y) = x^3 y^3\) es integrable en el conjunto \(M = [0, 1] \times [0, 1]\) ya que es continua en un conjunto compacto y además está acotada. Por tanto, como \(f\) es integrable, entonces por un resultado de clase, \(f\) es medible y por tanto podemos aplicar el teorema de Fubini para calcular la integral:
  \begin{align*}
    \int_M x^3 y^3 \; d\mu(x, y) & = \int_0^1 \left( \int_0^1 x^3 y^3 \; dx \right) dy = \int_0^1 y^3 \left[\frac{x^4}{4}\right]_0^1 \; dy = \\[2ex]
    & = \int_0^1 \frac{y^3}{4} \; dy = \left[\frac{y^4}{4 \cdot 4}\right]_0^1 = \frac{1}{16}
  \end{align*}
\end{dem_box}
\newpage

\subsubsection{Apartado 6}
\textit{Calcula la siguiente integral:}
\begin{align*}
  \int_M x \log (xy) \; d\mu(x, y) \quad \mbox{ con } M = [2, 3] \times [1, 2]
\end{align*}
\begin{dem_box}{Demostración}
  Podemos notar que la función \(f(x, y) = x \log (xy)\) es integrable en el conjunto \(M = [2, 3] \times [1, 2]\) ya que es continua en un conjunto compacto y además está acotada. Por tanto, como \(f\) es integrable, entonces por un resultado de clase, \(f\) es medible y por tanto podemos aplicar el teorema de Fubini para calcular la integral:
  \begin{align*}
    \int_M x \log (xy) \; d\mu(x, y) & = \int_1^2 \left( \int_2^3 x \log (xy) \; dx \right) dy 
  \end{align*}
  Veamos cual es la integral de \(x\):
  \begin{align*}
    \int x \log (xy) \; dx \xlongequal[dv = x dx \Rightarrow v = \frac{x^2}{2}]{u = \log (xy) \Rightarrow du = \frac{1}{x} dx} \frac{x^2}{2}\log (xy) - \int \frac{x^2}{2} \frac{1}{x} dx = \frac{x^2}{2} \log (xy) - \frac{x^2}{4}
  \end{align*}
  Por tanto:
  \begin{align*}
    \int_1^2 \left( \int_2^3 x \log (xy) \; dx \right) dy & = \int_1^2 \left[\frac{x^2}{2}\log (xy) - \frac{x^2}{4}\right]_2^3\; dy = \\[2ex]
    & = \int_1^2\left(\frac{3^2}{2} \log (3y) - \frac{3^2}{4} - \left(\frac{2^2}{2}\log (2y) - \frac{2^2}{4}\right)\right)\; dy = \\[2ex]
    & = \int_1^2\left(\frac{3^2}{2} \log (3y) - \frac{3^2}{4} - 2 \log (2y) + 1\right)\; dy = \\[2ex]
    & = \frac{3^2}{2} \int_1^2 \log (3y) \; dy - 2\int_1^2 \log(2y) \; dy + \left(\frac{3^2}{4} + 1\right) \int_1^2 \; dy = \\[2ex]
    & = \frac{9}{2} \left[ y \log(3y) - y \right]_1^2 - 2 \left[ y \log(2y) - y \right]_1^2 + \left(\frac{9}{4} + 1\right) [y]_1^2 = \\[2ex]
    & = \frac{9}{2} (2 \log 6 - 2 - \log 3 + 1) - 2 (2 \log 4 - 2 - \log 2 + 1) + \left(\frac{9}{4} + 1\right) (2 - 1) = \\[2ex]
    & = 9 \log 6 - \frac{9}{2} - \frac{9}{2} \log 3 + \frac{9}{2} - 4 \log 4 + 4 + 2 \log 2 + \frac{13}{4} = \\[2ex]
    & = 9 \log 6 - \frac{9}{2} \log 3 - 4 \log 4 + 2 \log 2 + \frac{13}{4} = \\[2ex]
    & = 9 (\log 2 + \log 3) - \frac{9}{2} \log 3 - 4 (2 \log 2) + 2 \log 2 + \frac{13}{4} = \\[2ex]
    & = 9 \log 2 + 9 \log 3 - \frac{9}{2} \log 3 - 8 \log 2 + 2 \log 2 + \frac{13}{4} = \\[2ex]
    & = 3 \log 2 + \frac{9}{2} \log 3 + \frac{13}{4}
  \end{align*}
\end{dem_box}
\newpage
\subsubsection{Apartado 7}
\textit{Calcula la siguiente integral}:
\begin{align*}
  \int_M(e^{x^2} - e^{y^2}) \; d\mu(x, y) \quad \mbox{ con } M = [a, b] \times [a, b]
\end{align*}
\begin{dem_box}{Demostración}
  Podemos notar que la función \(f(x, y) = e^{x^2} - e^{y^2}\) es integrable en el conjunto \(M = [a, b] \times [a, b]\) ya que es continua en un conjunto compacto y además está acotada. Por tanto, como \(f\) es integrable, entonces por un resultado de clase, \(f\) es medible y por tanto podemos aplicar el teorema de Fubini para calcular la integral:
  \begin{align*}
    \int_M (e^{x^2} - e^{y^2}) \; d\mu(x, y) & = \int_a^b \left( \int_a^b (e^{x^2} - e^{y^2}) \; dx \right) dy = \\[2ex]
    & = \int_a^b \left( \int_a^b e^{x^2} \; dx - \int_a^b e^{y^2} \; dx \right) dy = \\[2ex]
    & = \int_a^b \left( \int_a^b e^{x^2} \; dx - (b - a) e^{y^2} \right) dy
  \end{align*}
  Ahora, notamos que la integral de \(e^{x^2}\) no tiene una primitiva elemental, sin embargo, está definida en un cuadrado perfecto simétrico respecto a la diagonal \(y = x\). Además, podemos ver que si intercambiamos las variables \(x\) e \(y\) en la función \(f\), obtenemos:
  \begin{align*}
    f(y, x) = e^{y^2} - e^{x^2} = - f(x, y)
  \end{align*}
  Es decir, que la función es antisimétrica respecto a la diagonal \(y = x\). Por tanto, la integral sobre el cuadrado perfecto es cero:
  \begin{align*}
    \int_M (e^{x^2} - e^{y^2}) \; d\mu(x, y) & = 0
  \end{align*}
\end{dem_box}
\newpage

\subsubsection{Apartado 10}
\textit{Calcula la siguiente integral}:
\begin{align*}
  \int_M e^y \sin\left(\frac{x}{y}\right) d\mu(x, y) \quad \mbox{ con } M = \left[ - \frac{\pi}{2}, \frac{\pi}{2}\right] \times \left[ - \frac{\pi}{2}, \frac{\pi}{2}\right]
\end{align*}
\begin{dem_box}{Demostración}
  Podemos notar que la función \(f(x, y) = e^y \sin\left(\frac{x}{y}\right)\) es integrable en el conjunto \(M = \left[ - \frac{\pi}{2}, \frac{\pi}{2}\right] \times \left[ - \frac{\pi}{2}, \frac{\pi}{2}\right]\) ya que es continua en un conjunto compacto y además está acotada. Por tanto, como \(f\) es integrable, entonces por un resultado de clase, \(f\) es medible y por tanto podemos aplicar el teorema de Fubini para calcular la integral:
  \begin{align*}
    \int_M e^y \sin\left(\frac{x}{y}\right) d\mu(x, y) & = \int_{ - \frac{\pi}{2}}^{\frac{\pi}{2}} \left( \int_{ - \frac{\pi}{2}}^{\frac{\pi}{2}} e^y \sin\left(\frac{x}{y}\right) \; dx \right) dy 
  \end{align*}
  Podemos notar que la función \(\sin\left(\frac{x}{y}\right)\) es una función impar respecto a \(x\), es decir:
  \begin{align*}
    \sin\left(\frac{-x}{y}\right) = - \sin\left(\frac{x}{y}\right)
  \end{align*}
  Por tanto, como el intervalo de integración en \(x\) es simétrico respecto a cero, tenemos que:
  \begin{align*}
    \int_{ - \frac{\pi}{2}}^{\frac{\pi}{2}} e^y \sin\left(\frac{x}{y}\right) \; dx & = e^y \int_{ - \frac{\pi}{2}}^{\frac{\pi}{2}} \sin\left(\frac{x}{y}\right) \; dx = 0
  \end{align*}
\end{dem_box}
\newpage
\subsubsection{Apartado 13}
\textit{Calcula la siguiente integral}:
\begin{align*}
  \int_M \frac{1}{(1 + x)^2} \; d\mu(x, y) \quad \mbox{ con } M = [0, \infty) \times \left[0, \frac{3\pi}{2} \right] 
\end{align*}
\begin{dem_box}{Demostración}
  En este caso, la función es medible ya que es continua en el conjunto \(M = [0, \infty) \times \left[0, \frac{3\pi}{2} \right]\). Además, como la función es positiva en dicho conjunto, podemos aplicar el teorema de Tonelli dado que \(f\) es medible no negativa. Por tanto, tenemos que:
  \begin{align*}
    \int_M \frac{1}{(1 + x)^2} \; d\mu(x, y) & = \int_0^{\frac{3\pi}{2}} \left( \int_0^\infty \frac{1}{(1 + x)^2} \; dx \right) dy
  \end{align*}
  El calculo de la integral impropia interna es:
  \begin{align*}
    \int_0^\infty \frac{1}{(1 + x)^2} \; dx & = \lim_{t \to \infty} \int_0^t (1 + x)^{ - 2} \; dx = \lim_{t \to \infty} \left[ - \frac{1}{1 + x} \right]_0^t = \\[2ex]
    & = \lim_{t \to \infty} \left( - \frac{1}{1 + b} \right) - \left( - \frac{1}{1 + 0}\right) =\lim_{t \to \infty} 1 - \frac{1}{1 + b} = 1
  \end{align*}
  Por tanto:
  \begin{align*}
    \int_0^{\frac{3\pi}{2}} \left( \int_0^\infty \frac{1}{(1 + x)^2} \; dx \right) dy & = \int_0^{\frac{3\pi}{2}} 1 \; dy = \left[ y \right]_0^{\frac{3\pi}{2}} = \frac{3\pi}{2}
  \end{align*}
\end{dem_box}
\newpage

\subsection{Ejercicio 2}
\subsubsection{Apartado 1}
\textit{Calcula la siguiente integral}
\begin{align*}
  \int_M \frac{x}{\sqrt{y}} \; d\mu_2(x, y) \quad \mbox{ con } M = \left\{(x, y) \in \mathbb{R}^2 : 0 \leq x \leq 1, \; 0 \leq y \leq x\right\}
\end{align*}
\begin{dem_box}{Demostración}
  Podemos notar que el dominio \(M\) está delimitado por las rectas \(x = 0\), \(x = 1\), \(y = 0\) y \(y = x\),  geométricamente sería:
  \begin{center}
    \begin{tikzpicture}
      \begin{axis}[
          axis lines = middle,
          xlabel = \(x\),
          ylabel = \(y\),
          xtick = {0, 1},
          ytick = {0, 1},
          ymin = -0.5, ymax = 1.5,
          xmin = -0.5, xmax = 1.5,
          domain = 0:1,
          samples = 100,
          width=8cm,
          height=6cm,
        ]
        % Área sombreada
        \addplot [
          fill=blue!20,
          domain=0:1,
        ]
        {x} \closedcycle;
        
        % Líneas del área
        \addplot [thick, blue] {x} node [pos=0.5, above=10pt] {\(y = x\)};
        \addplot [thick, blue] {0} node [pos=0.5, below] {\(y = 0\)};
        \addplot [thick, blue] coordinates {(0,0) (0,1)} node [pos=0.5, left] {\(x = 0\)};
        \addplot [thick, blue] coordinates {(1,0) (1,1)} node [pos=0.5, right] {\(x = 1\)};
        \addplot [dashed, gray] coordinates {(1, 1) (1, 1.5)};
        \addplot [dashed, gray] coordinates {(1, 1) (1.5, 1.4)};
      \end{axis}
    \end{tikzpicture}
  \end{center}

  Por tanto, la función \(f(x, y) = \frac{x}{\sqrt{y}}\) es continua en \(M\setminus \{y = 0\} \) donde tiene una discontinuidad de tipo infinito. Sin embargo, podemos ver que la función es positiva en \(M\) y podemos calcular aplicando el Teorema de Tonelli la integral. Para ello:
  \begin{align*}
    \int_M \frac{x}{\sqrt{y}} \; d\mu_2(x, y) & = \int_0^1 \left( \int_0^x \frac{x}{\sqrt{y}} \; dy \right) dx = \int_0^1 \left[2x \sqrt{y} \right]_0^x \; dx = \\[2ex]
    & = \int_0^1 2x \sqrt{x} \; dx = \int_0^1 2x^{\frac{3}{2}} \; dx = \left[2 \cdot \frac{2}{5} x^{\frac{5}{2}} \right]_0^1 = \frac{4}{5}
  \end{align*}
\end{dem_box}
\newpage

\subsubsection{Apartado 2}
\textit{Calcula la siguiente integral}
\begin{align*}
  \int_M x \; d\mu_2(x, y) \quad \mbox{ con } M = \left\{(x, y) \in \mathbb{R}^2 : 0 \leq x \leq 1, \; 0 \leq y \leq e^x\right\}
\end{align*}
\begin{dem_box}{Demostración}
  En este caso, el dominio de integración \(M\) está delimitado por las rectas \(x = 0\), \(x = 1\), \(y = 0\) y la curva \(y = e^x\), geométricamente sería:
  \begin{center}
    \begin{tikzpicture}
      \begin{axis}[
          axis lines = middle,
          xlabel = \(x\),
          ylabel = \(y\),
          xtick = {0, 1},
          ytick = {0, e},
          ymin = -0.5, ymax = 4,
          xmin = -0.5, xmax = 1.5,
          domain = 0:1.5,
          samples = 100,
          width=8cm,
          height=6cm,
        ]
        % Área sombreada
        \addplot [
          fill=blue!20,
          domain=0:1,
        ]
        {exp(x)} \closedcycle;
        
        % Líneas del área
        \addplot [thick, blue, domain=0:1] {exp(x)} node [pos=0.5, above=10pt] {\(y = e^x\)};
        \addplot [thick, blue] {0} node [pos=0.5, below] {\(y = 0\)};
        \addplot [thick, blue] coordinates {(0,0) (0,3)} node [pos=0.5, left] {\(x = 0\)};
        \addplot [thick, blue] coordinates {(1,0) (1,e)} node [pos=0.5, right] {\(x = 1\)};
        \addplot [dashed, gray] coordinates {(1, e) (1, 3)};
        \addplot [dashed, gray] coordinates {(1, e) (1.5, e^1.5)};
      \end{axis}
    \end{tikzpicture}
  \end{center}
  En dicho intervalo, tenemos que la función \(f(x, y) = x\) es continua y por tanto integrable. Por tanto, podemos aplicar el teorema de Fubini para calcular la integral:
  \begin{align*}
    \int_M x\; d\mu_2(x, y) & = \int_0^1 \left( \int_0^{e^x} x \; dy \right) dx = \int_0^1 x \cdot \left[ y \right]_0^{e^x} \; dx = \int_0^1 x e^x \; dx
  \end{align*}
  Para calcular la integral, integramos por partes:
  \begin{align*}
    \int x e^x \; dx \xlongequal[dv = e^xdx \Rightarrow v = e^x]{u = x \Rightarrow du = dx} x e^x - \int e^x dx = x e^x - e^x + C
  \end{align*}
  Por tanto:
  \begin{align*}
    \int_0^1 x e^x \; dx & = \left[ x e^x - e^x \right]_0^1 = (1 \cdot e^1 - e^1) - (0 - 1) = 1
  \end{align*}
\end{dem_box}
\newpage

\subsubsection{Apartado 3}
\textit{Calcula la siguiente integral}
\begin{align*}
  \int_M \frac{x^2}{y^2} \; d\mu_2(x, y) \quad \mbox{ con } M = \left\{(x, y) \in \mathbb{R}^2 : 1 \leq x \leq 2, \; x^{ - 1} \leq y \leq x\right\}
\end{align*}
\begin{dem_box}{Demostración}
  En este caso, los límites de integración están dados por las rectas \(x = 1\), \(x = 2\) y las curvas \(y = x^{ - 1}\) y \(y = x\), geométricamente sería:
  \begin{center}
    \begin{tikzpicture}
      \begin{axis}[
          axis lines = middle,
          xlabel = \(x\),
          ylabel = \(y\),
          xtick = {1, 2},
          ytick = {0.5, 1, 2},
          ymin = -0.5, ymax = 3,
          xmin = 0.5, xmax = 2.5,
          domain = 1:2,
          samples = 100,
          width=8cm,
          height=6cm,
        ]
        % Área sombreada
        \addplot [
          fill=blue!20,
          domain=1:2,
        ]
        {x} \closedcycle;
        \addplot [
          fill=white!20,
          domain=1:2,
        ]
        {1/x} \closedcycle;
        
        % Líneas del área
        \addplot [thick, blue, domain=1:2] {x} node [pos=0.5, above=10pt] {\(y = x\)};
        \addplot [thick, blue, domain=1:2] {1/x} node [pos=0.5, below=10pt] {\(y = x^{ - 1}\)};
        \addplot [thick, blue] coordinates {(1,0) (1,2)} node [pos=0.5, left] {\(x = 1\)};
        \addplot [thick, blue] coordinates {(2,0) (2,2)} node [pos=0.5, right] {\(x = 2\)};
        \addplot [dashed, gray] coordinates {(2, 2) (2, 3)};
        \addplot [dashed, gray] coordinates {(2, 0.5) (2.5, 0.4)};
      \end{axis}
    \end{tikzpicture}
  \end{center}
  En dicho intervalo, tenemos que la función \(f(x, y) = \frac{x^2}{y^2}\) es continua y por tanto integrable. Por tanto, podemos aplicar el teorema de Fubini para calcular la integral:
  \begin{align*}
    \int_M \frac{x^2}{y^2} \; d\mu_2(x, y) & = \int_1^2 \left(\int_{x^{ - 1}}^x \frac{x^2}{y^2} \; dy \right) \; dx = \int_1^2 \left[ - \frac{x^2}{y}\right]_{x^{ - 1}}^x \; dx = \\[2ex]
    & = \int_1^2 \left( - \frac{x^2}{x} - \left( - \frac{x^2}{\frac{1}{x}}\right)\right) \; dy = \int_1^2 - x + x^3\; dx = \\[2ex]
    & = -\int_1^2 x \; dx + \int_1^2 x^3 \; dx = - \left[\frac{x^2}{2}\right]_1^2 + \left[\frac{x^4}{4}\right]_1^2 = \\[2ex]
    & = - 2 + \frac{1}{2} + 4 - \frac{1}{4} = \frac{9}{4}
  \end{align*}
\end{dem_box}
\newpage

\subsubsection{Apartado 4}
\textit{Calcula la siguiente integral}
\begin{align*}
  \int_M e^{\frac{x}{y}} \; d\mu_2(x, y) \quad \mbox{ con } M = \left\{(x, y) \in \mathbb{R}^2 : 0 \leq y \leq 1, \; 0 \leq x \leq y^2\right\}
\end{align*}
\begin{dem_box}{Demostración}
  En este caso, los límites de integración están dados por las rectas \(y = 0\), \(y = 1\) y la curva \(x = y^2\), geométricamente sería:
  \begin{center}
    \begin{tikzpicture}
      \begin{axis}[
          axis lines = middle,
          xlabel = \(x\),
          ylabel = \(y\),
          xtick = {0, 1},
          ytick = {0, 1},
          ymin = -0.5, ymax = 1.5,
          xmin = -0.5, xmax = 1.5,
          domain = 0:1,
          samples = 100,
          width=8cm,
          height=6cm,
        ]
        % Área sombreada
        \addplot [
          fill=blue!20,
          domain=0:1,
        ]
        {sqrt(x)} \closedcycle;
        
        % Líneas del área
        \addplot [thick, blue, domain=0:1] {sqrt(x)} node [pos=0.5, above=10pt] {\(x = y^2\)};
        \addplot [thick, blue] {0} node [pos=0.5, below] {\(y = 0\)};
        \addplot [thick, blue] coordinates {(0,0) (0,1)} node [pos=0.5, left] {\(x = 0\)};
        \addplot [thick, blue] coordinates {(1,0) (1,1)} node [pos=0.5, right] {\(y = 1\)};
        \addplot [dashed, gray] coordinates {(1, 1) (1, 1.5)};
        \addplot [dashed, gray] coordinates {(1, 1) (1.5, 1)};
      \end{axis}
    \end{tikzpicture}
  \end{center}
  Podemos notar que en \((0, 0)\) la función puede dar problemas de continuidad, si estudiamos el límite:
  \begin{align*}
    \lim_{(x, y) \to (0, 0)^+} e^{\frac{x}{y}} & = \lim_{y \to 0^+} e^{\frac{0}{y}} = e^0 = 1
  \end{align*}
  
  En dicho intervalo, tenemos que la función \(f(x, y) = e^{\frac{x}{y}}\) es continua y por tanto integrable. Por tanto, podemos aplicar el teorema de Fubini para calcular la integral:
  \begin{align*}
    \int_M e^{\frac{x}{y}} \; d\mu_2(x, y) & = \int_0^1 \left( \int_0^{y^2} e^{\frac{x}{y}} \; dx \right) dy = \int_0^1 \left[ y e^{\frac{x}{y}} \right]_0^{y^2} \; dy = \\[2ex]
    & = \int_0^1 y \left( e^y - 1 \right) \; dy = \int_0^1 y e^y \; dy - \int_0^1 y \; dy = \\[2ex]
    & \xlongequal[dv = e^y dy \Rightarrow v = e^y]{u = y \Rightarrow du = dy} \left[ y e^y - \int e^y dy \right]_0^1 - \left[\frac{y^2}{2}\right]_0^1 = \\[2ex]
    & = \left[ y e^y - e^y \right]_0^1 - \frac{1}{2} = (1 \cdot e^1 - e^1) - (0 - 1) - \frac{1}{2} = \frac{1}{2}
  \end{align*}
\end{dem_box}

\newpage
\subsubsection{Apartado 5}
\textit{Calcula la siguiente integral}
\begin{align*}
  \int_M xy^2 \; d\mu_2(x, y) \quad \mbox{ con } M = \left\{(x, y) \in \mathbb{R}^2 : |y| \leq x \leq 1\right\}
\end{align*}
\begin{dem_box}{Demostración}
  En este caso, los límites de integración están dados por las rectas \(x = 1\) y las curvas \(y = x\) y \(y = -x\), geométricamente sería:
  \begin{center}
    \begin{tikzpicture}
      \begin{axis}[
          axis lines = middle,
          xlabel = \(x\),
          ylabel = \(y\),
          xtick = {0, 1},
          ytick = {-1, 0, 1},
          ymin = -1.5, ymax = 1.5,
          xmin = -0.5, xmax = 1.5,
          domain = -1:1,
          samples = 100,
          width=8cm,
          height=6cm,
        ]
        % Área sombreada
        \addplot [
          fill=blue!20,
          domain=0:1,
        ]
        {x} \closedcycle;
        \addplot [
          fill=blue!20,
          domain=0:1,
        ]
        {-x} \closedcycle;
        
        % Líneas del área
        \addplot [thick, blue, domain=0:1] {x} node [pos=0.5, above=10pt] {\(y = x\)};
        \addplot [thick, blue, domain=0:1] {-x} node [pos=0.5, below=10pt] {\(y = -x\)};
        \addplot [thick, blue] coordinates {(1,-1) (1,1)} node [pos=0.5, right] {\(x = 1\)};
        \addplot [dashed, gray] coordinates {(1, 1) (1.5, 1.5)};
        \addplot [dashed, gray] coordinates {(1, -1) (1.5, -1.5)};
        \addplot [dashed, gray] coordinates {(0, 0) (-0.5, -0.5)};
        \addplot [dashed, gray] coordinates {(0, 0) (-0.5, 0.5)};
      \end{axis}
    \end{tikzpicture}
  \end{center}
  Podemos notar que la función \(f(x,y) = xy^2\) es continua en el conjunto \(M\) y por tanto integrable. Además, podemos notar que la función es par respecto a la variable \(y\), es decir:
  \begin{align*}
    f(x, -y) = x(-y)^2 = xy^2 = f(x, y)
  \end{align*}
  Por tanto, podemos notar que la integral en el conjunto \(M\) es el doble de la integral en el subconjunto:
  \begin{align*}
    M^+ = \left\{(x, y) \in \mathbb{R}^2 : 0 \leq y \leq x \leq 1\right\}
  \end{align*}
  Por tanto, podemos calcular la integral en \(M^+\) y luego multiplicar por \(2\). Así, aplicando el teorema de Fubini:
  \begin{align*}
    \int_M xy^2 \; d\mu_2(x, y) & = 2 \int_{M^+} xy^2 \; d\mu_2(x, y) = 2 \int_0^1 \left( \int_0^x xy^2 \; dy \right) dx = \\[2ex]
    & = 2 \int_0^1 x \left[ \frac{y^3}{3} \right]_0^x \; dx = 2 \int_0^1 x \cdot \frac{x^3}{3} \; dx = \\[2ex]
    & = \frac{2}{3} \int_0^1 x^4 \; dx = \frac{2}{3} \left[ \frac{x^5}{5} \right]_0^1 = \frac{2}{15}
  \end{align*}
\end{dem_box}
\newpage

\subsubsection{Apartado 6}
\textit{Calcula la siguiente integral}
\begin{align*}
  \int_M xy \; d\mu_2(x, y) \quad \mbox{ con } M \mbox{ la región delimitada por } y = x \quad \mbox{ e } y = x^2
\end{align*}
\begin{dem_box}{Demostración}
  En este ejercicio lo más fácil es graficar primero y ver que región tenemos que analizar. Así:
  \begin{center}
    \begin{tikzpicture}
      \begin{axis}[
          axis lines = middle,
          xlabel = \(x\),
          ylabel = \(y\),
          xtick = {0, 1},
          ytick = {0, 1},
          ymin = -0.5, ymax = 1.5,
          xmin = -0.5, xmax = 1.5,
          domain = 0:1,
          samples = 100,
          width=8cm,
          height=6cm,
        ]
        % Área sombreada
        \addplot [
          fill=blue!20,
          domain=0:1,
        ]
        {x} \closedcycle;
        \addplot [
          fill=white!20,
          domain=0:1,
        ]
        {x^2} \closedcycle;
        
        % Líneas del área
        \addplot [thick, blue, domain=0:1] {x} node [pos=0.5, above=10pt] {\(y = x\)};
        \addplot [thick, blue, domain=0:1] {x^2} node [pos=0.5, below=10pt] {\(y = x^2\)};
        \addplot [dashed, gray] coordinates {(1, 1) (1.5, 1.5)};
        \addplot [dashed, gray] coordinates {(0, 0) (-0.5, 0.25)};
      \end{axis}
    \end{tikzpicture}
  \end{center}
  Por tanto, hay que hallar el punto de intersección de las dos curvas:
  \begin{align*}
    x = x^2 \implies x^2 - x = 0 \implies x(x - 1) = 0 \implies x = 0 \mbox{ o } x = 1
  \end{align*}
  Por tanto, los límites de integración son \(x \in [0, 1]\) y \(y \in [x^2, x]\). En dicho intervalo, tenemos que la función \(f(x, y) = xy\) es continua y por tanto integrable. Por tanto, podemos aplicar el teorema de Fubini para calcular la integral:
  \begin{align*}
    \int_M xy \; d\mu_2(x, y) & = \int_0^1 \left( \int_{x^2}^x xy \; dy \right) dx = \int_0^1 x \left[ \frac{y^2}{2} \right]_{x^2}^x \; dx = \\[2ex]
    & = \int_0^1 x \left( \frac{x^2}{2} - \frac{x^4}{2} \right) \; dx = \int_0^1 \left( \frac{x^3}{2} - \frac{x^5}{2} \right) \; dx = \\[2ex]
    & = \frac{1}{2} \int_0^1 x^3 \; dx - \frac{1}{2} \int_0^1 x^5 \; dx = \frac{1}{2} \left[ \frac{x^4}{4} \right]_0^1 - \frac{1}{2} \left[ \frac{x^6}{6} \right]_0^1 = \\[2ex]
    & = \frac{1}{8} - \frac{1}{12} = \frac{3 - 2}{24} = \frac{1}{24}
  \end{align*}
\end{dem_box}
\newpage

\subsubsection{Apartado 8}
\textit{Calcula la siguiente integral}
\begin{align*}
  \int_M 12 - 3x - 4y \; d\mu_2 (x, y) \quad \mbox{ con } M = \left\{(x, y): \sqrt{x} + \sqrt{y} \geq 1, \; \sqrt{1 - x} + \sqrt{1 - y} \geq 1\right\}
\end{align*}
\begin{dem_box}{Demostración}
  En este caso, el dominio está delimitado por dos condiciones:
  \begin{itemize}
    \item \(\sqrt{x} + \sqrt{y} \geq 1\) entonces:
    \begin{align*}
      \sqrt{y} \geq 1 - \sqrt{x} \implies y \geq (1 - \sqrt{x})^2 = 1 - 2\sqrt{x} + x
    \end{align*}
    \item \(\sqrt{1 - x} + \sqrt{1 - y} \geq 1\) entonces:
    \begin{align*}
      \sqrt{1 - y} \geq 1 - \sqrt{1 - x} \implies y \leq 1 - (1 - \sqrt{1 - x})^2 = 2\sqrt{1 - x} - (1 - x)
    \end{align*}
  \end{itemize}
  Si graficamos ambas curvas, tenemos:
  \begin{center}
    \begin{tikzpicture}
      \begin{axis}[
          axis lines = middle,
          xlabel = \(x\),
          ylabel = \(y\),
          xtick = {0, 1},
          ytick = {0, 1},
          ymin = -0.5, ymax = 1.5,
          xmin = -0.5, xmax = 1.5,
          domain = 0:1,
          samples = 100,
          width=8cm,
          height=6cm,
        ]
        % Área sombreada
        \addplot [
          fill=blue!20,
          domain=0:1,
        ]
        {2*sqrt(1 - x) - (1 - x)} \closedcycle;
        \addplot [
          fill=white!20,
          domain=0:1,
        ]
        {1 - 2*sqrt(x) + x} \closedcycle;
        
        % Líneas del área
        \addplot [thick, blue, domain=0:1] {2*sqrt(1 - x) - (1 - x)} node [pos=0.5, above=10pt] {\(y = 2\sqrt{1 - x} - (1 - x)\)};
        \addplot [thick, blue, domain=0:1] {1 - 2*sqrt(x) + x} node [pos=0.5, below=10pt] {\(y = 1 - 2\sqrt{x} + x\)};
        \addplot [dashed, gray] coordinates {(1, 1) (1.5, 1.5)};
        \addplot [dashed, gray] coordinates {(0, 0) (-0.5, 0.25)};
      \end{axis}
    \end{tikzpicture}
  \end{center}
  Vemos que son simétricas respecto a la recta \(y = x\) y se intersectan en los puntos \((0, 1)\) y \((1, 0)\). Por tanto, podemos expresar el dominio \(M\) como:
  \begin{align*}
    M & = \left\{(x, y) : 0 \leq x \leq 1, \; 1 - 2\sqrt{x} + x \leq y \leq 2\sqrt{1 - x} - (1 - x)\right\}
  \end{align*}
  Por tanto, aplicando el teorema de Fubini:
  \begin{align*}
    \int_M 12 - 3x - 4y \; d\mu_2 (x, y) & = \int_0^1 \left( \int_{1 - 2\sqrt{x} + x}^{2\sqrt{1 - x} - (1 - x)} 12 - 3x - 4y \; dy \right) dx = \\[2ex]
    & = \int_0^1 \left[ (12 - 3x)y - 2y^2 \right]_{1 - 2\sqrt{x} + x}^{2\sqrt{1 - x} - (1 - x)} \; dx = \\[2ex]
    & = \int_0^1 \left( (12 - 3x) \left( 2\sqrt{1 - x} - (1 - x) \right) - 2 \left( 2\sqrt{1 - x} - (1 - x) \right)^2 \right) \; dx - \\[2ex]
    & \quad - \int_0^1 \left( (12 - 3x) \left( 1 - 2\sqrt{x} + x \right) - 2 \left( 1 - 2\sqrt{x} + x \right)^2 \right) \; dx = \\[2ex]
    & = I_1 - I_2
  \end{align*}
  Donde:
  \begin{align*}
    I_1 & = \int_0^1 \left( (12 - 3x) \left( 2\sqrt{1 - x} - (1 - x) \right) - 2 \left( 2\sqrt{1 - x} - (1 - x) \right)^2 \right) \; dx = \\[2ex]
    & = \int_0^1 \left( 24\sqrt{1 - x} - 12(1 - x) - 6x\sqrt{1 - x} + 3x(1 - x) - 8(1 - x) + 8(1 - x)^{3/2} - 2(1 - x)^2 \right) \; dx = \\[2ex]
    & = \int_0^1 \left( 16\sqrt{1 - x} + 8(1 - x)^{3/2} - 20 + 11x - 3x^2 - 6x\sqrt{1 - x} \right) \; dx = \\[2ex]
    & = \left[ -\frac{32}{3} (1 - x)^{3/2} - \frac{16}{5} (1 - x)^{5/2} - 20x + \frac{11}{2} x^2 - x^3 + 4(1 - x)^{5/2} \right]_0^1 = \\[2ex]
    & = \left( 0 - 0 - 20 + \frac{11}{2} - 1 + 0 \right) - \left( -\frac{32}{3} - \frac{16}{5} - 0 + 0 + 0 + 4 \right) = \\[2ex]\, 
    & = -\frac{27}{2} + \frac{32}{3} + \frac{16}{5} + 4 = \frac{319}{30}
  \end{align*}
  Y:
  \begin{align*}
    I_2 & = \int_0^1 \left( (12 - 3x) \left( 1 - 2\sqrt{x} + x \right) - 2 \left( 1 - 2\sqrt{x} + x \right)^2 \right) \; dx = \\[2ex]
    & = \int_0^1 \left( 12 - 24\sqrt{x} + 12x - 3x + 6x\sqrt{x} - 3x^2 - 2 + 8\sqrt{x} - 4x + 2x^2 \right) \; dx = \\[2ex]
    & = \int_0^1 \left( 10 - 16\sqrt{x} + 5x + 6x^{3/2} - x^2 \right) \; dx = \\[2ex]
    & = \left[ 10x - \frac{32}{3} x^{3/2} + \frac{5}{2} x^2 + \frac{12}{5} x^{5/2} - \frac{1}{3} x^3 \right]_0^1 = \\[2ex]
    & = \left( 10 - \frac{32}{3} + \frac{5}{2} + \frac{12}{5} - \frac{1}{3} \right) - (0) = \frac{319}{30}
  \end{align*}
  Por tanto, la integral buscada es:
  \begin{align*}
    \int_M 12 - 3x - 4y \; d\mu_2 (x, y) & = I_1 - I_2 = \frac{319}{30} - \frac{319}{30} = 0
  \end{align*}
  \textcolor{red}{Creo que está mal, son demasiadas cuentas como para que no falle alguna...}
\end{dem_box}
\newpage

\subsection{Apartado 9}
\textit{Calcula la siguiente integral}
\begin{align*}
  \int_M x \; d\mu_2(x, y) \quad \mbox{ con } M = \mbox{subconjunto del semiplano } x \geq 0 \mbox{ con frontera: } \begin{cases} 
    \frac{x^2}{a^2} - \frac{y^2}{b^2} = 1\\[2ex]
    \frac{x^2}{4a^2} + \frac{y^2}{9b^2} = 1
  \end{cases} 
\end{align*}
\begin{dem_box}{Demostración}
  Para resolver este ejercicio, la forma más sencilla es hacer un cambio de coordenadas a coordenadas hiperbólicas. Así, definimos el cambio de variable:
  \begin{align*}
    x & = ar \cos \theta \quad \mbox{ y } \quad y = br \sin \theta
  \end{align*}
  Donde \(r \geq 0\) y \(\theta \in [ - \frac{\pi}{2}, \frac{\pi}{2}]\) para cubrir todo el semiplano \(x \geq 0\). Calculamos el Jacobiano:
  \begin{align*}
    J = \begin{vmatrix}
      \frac{\partial x}{\partial r} & \frac{\partial x}{\partial \theta} \\[2ex]
      \frac{\partial y}{\partial r} & \frac{\partial y}{\partial \theta}
    \end{vmatrix} & = \begin{vmatrix}
      a \cos \theta & - a r \sin \theta \\[2ex]
      b \sin \theta & b r \cos \theta
    \end{vmatrix} = ab r (\cos^2 \theta + \sin^2 \theta) = ab r
  \end{align*}
  Sustituimos en las ecuaciones de las fronteras para hallar los nuevos límites de integración:
  \begin{itemize}
    \item Primera frontera:
    \begin{align*}
      \frac{x^2}{a^2} - \frac{y^2}{b^2} = 1 & \implies r^2 \cos^2 \theta - r^2 \sin^2 \theta = 1 \implies r^2 (\cos^2 \theta - \sin^2 \theta) = 1 \implies \\[2ex]
      & \implies r = \frac{1}{\sqrt{\cos^2 \theta - \sin^2 \theta}} = \frac{1}{\sqrt{\cos(2\theta)}}
    \end{align*}
    \item Segunda frontera:
    \begin{align*}
      \frac{x^2}{4a^2} + \frac{y^2}{9b^2} = 1 & \implies \frac{r^2 \cos^2 \theta}{4} + \frac{r^2 \sin^2 \theta}{9} = 1 \implies r^2 \left( \frac{\cos^2 \theta}{4} + \frac{\sin^2 \theta}{9} \right) = 1 \implies \\[2ex]
      & \implies r = \frac{1}{\sqrt{\frac{\cos^2 \theta}{4} + \frac{\sin^2 \theta}{9}}}
    \end{align*}
    Por tanto, los límites son:
    \begin{itemize}
      \item Para \(\theta\) como \(x \geq 0\) estamos limitados por una hipérbola que requiere \(\cos 2\theta > 0\).
      \item Para \(r\) tenemos que va desde la hipérbola hasta la elipse, es decir:
      \begin{align*}
        r \in \left[\frac{1}{\sqrt{\cos(2\theta)}}, \; \frac{1}{\sqrt{\frac{\cos^2 \theta}{4} + \frac{\sin^2 \theta}{9}}}\right]
      \end{align*}
    \end{itemize}
    La función a integrar en las nuevas coordenadas es:
    \begin{align*}
      f(x, y) = x = ar \cos \theta
    \end{align*}
    Y su integral será:
    \begin{align*}
      \int_{\theta_1}^{\theta_2} \int_{r_\text{hipérbola}}^{r_\text{elipse}} ar \cos \theta \cdot ab r \; dr \; d\theta & = a^2 b \int_{\theta_1}^{\theta_2} \cos \theta \left( \int_{r_\text{hipérbola}}^{r_\text{elipse}} r^2 \; dr \right) d\theta = \\[2ex]
      & = a^2 b \int_{\theta_1}^{\theta_2} \cos \theta \left[ \frac{r^3}{3} \right]_{r_\text{hipérbola}}^{r_\text{elipse}} \; d\theta = \\[2ex]
      & = \frac{a^2 b}{3} \int_{\theta_1}^{\theta_2} \cos \theta \left( r_\text{elipse}^3 - r_\text{hipérbola}^3 \right) d\theta = \\[2ex]
      & = \frac{a^2 b}{3} \int_{\theta_1}^{\theta_2} \cos \theta \left( \left( \frac{1}{\sqrt{\frac{\cos^2 \theta}{4} + \frac{\sin^2 \theta}{9}}} \right)^3 - \left( \frac{1}{\sqrt{\cos(2\theta)}} \right)^3 \right) d\theta = \\[2ex]
      & = \frac{a^2 b}{3} \int_{\theta_1}^{\theta_2} \cos \theta \left( \frac{1}{\left( \frac{\cos^2 \theta}{4} + \frac{\sin^2 \theta}{9} \right)^{3/2}} - \frac{1}{\left( \cos(2\theta) \right)^{3/2}} \right) d\theta
    \end{align*}
  \end{itemize}
\end{dem_box}

\newpage
\section{Cambio de variable en integrales dobles}
\subsection{Ejercicio 1}
\textit{A través de un cambio de variable adecuado, calcula la siguiente integral doble:}
\begin{align*}
  \int_M \cos (x^2 + y^2) \; d\mu_2(x, y) \quad \mbox{ con } M = \left\{(x, y) : x^2 + y^2 \leq \sqrt{\frac{\pi}{2}}  \right\}
\end{align*}
\begin{dem_box}{Demostración}
  Podemos ver que:
  \begin{itemize}
    \item \(M\) es un conjunto medible
    \item La función \(f(x, y) = \cos(x^2 + y^2) \) es continua en \(M\) y, por tanto, medible.
    \item Además, \(|f| \leq 1\) y \(M\) es de medida finita, entonces \(f \in \mathcal{L}_1(M)\)
  \end{itemize}
  Por tanto, vamos a aplicar el cambio de variable a coordenadas polares. Sea \(T : U \subseteq \mathbb{R}^N \to \mathbb{R}^N\) difeomorfismo \(\mathcal{C}^1\) con jacobiano \(J_T\), entonces para toda función \(f \in \mathcal{L}_1(T(U))\) se cumple:
  \begin{align*}
    \int_{T(U)} f(x) \; d\mu_N(x) = \int_U f(T(u)) |J_T(u)| \; d\mu_N(u)
  \end{align*}
  En nuestro caso, definimos el cambio de variable:
  \begin{align*}
    T(r, \theta) = (x, y) = (r \cos \theta, r \sin \theta) \quad \mbox{ con } (r, \theta) \in (0, \infty) \times (0, 2\pi)
  \end{align*}
  Calculamos el jacobiano:
  \begin{align*}
    |J_T(r, \theta)| & = \begin{vmatrix}
      \frac{\partial x}{\partial r} & \frac{\partial x}{\partial \theta} \\[2ex]
      \frac{\partial y}{\partial r} & \frac{\partial y}{\partial \theta}
    \end{vmatrix}
    = \begin{vmatrix}
      \cos \theta & - r \sin \theta \\[2ex]
      \sin \theta & r \cos \theta
    \end{vmatrix} = r (\cos^2 \theta + \sin^2 \theta) = r
  \end{align*}
  Ahora, los nuevos límites de integración vienen dados por:
  \begin{align*}
    x^2 + y^2 \leq \sqrt{\frac{\pi}{2}} & \implies r^2 \cos^2 \theta + r^2 \sin^2 \theta \leq \sqrt{\frac{\pi}{2}} \implies\\[2ex]
    & \implies r^2 \leq \sqrt{\frac{\pi}{2}} \implies r \leq \left( \frac{\pi}{2} \right)^{1/4}
  \end{align*}
  Y por tanto, los nuevos límites son:
  \begin{align*}
    r \in \left[0, \left( \frac{\pi}{2} \right)^{1/4} \right] \quad \mbox{ y } \quad \theta \in [0, 2\pi]
  \end{align*}
  Es decir, que el conjunto \(M\) en las nuevas coordenadas es:
  \begin{align*}
    T^{ - 1}(M) = \left\{(r, \theta) : 0 \leq r \leq \left( \frac{\pi}{2} \right)^{1/4}, \; 0 \leq \theta \leq 2\pi \right\}
  \end{align*}
  Ahora, aplicando el teorema del cambio de variable:
  \begin{align*}
    \int_M \cos (x^2 + y^2) \; d\mu_2(x, y) & = \int_{T^{ - 1}(M)} \cos (r^2) \cdot r \; d\mu_2(r, \theta) = \\[2ex]
    & = \int_0^{2\pi} \left(\int_0^{\left( \frac{\pi}{2} \right)^{1/4}} r \cos (r^2) \; dr \right) d\theta 
  \end{align*}
  Como la función \((r, \theta) \mapsto r \cos r^2\) es continua, entonces es medible y además es integrable en el conjunto \(T^{ - 1}(M)\) de medida finita. Por tanto, podemos aplicar el teorema de Fubini y separar las integrales:
  \begin{align*}
    \int_0^{2\pi} \left(\int_0^{\left( \frac{\pi}{2} \right)^{1/4}} r \cos (r^2) \; dr \right) d\theta & = \left( \int_0^{2\pi} d\theta \right) \cdot \left( \int_0^{\left( \frac{\pi}{2} \right)^{1/4}} r \cos (r^2) \; dr \right) = \\[2ex]
    & = 2\pi \cdot \left[ \frac{\sin (r^2)}{2} \right]_0^{\left( \frac{\pi}{2} \right)^{1/4}} = \\[2ex]
    & = 2\pi \cdot \left( \frac{\sin \left( \left( \frac{\pi}{2} \right)^{1/2} \right)}{2} - 0 \right) = \\[2ex]
    & = \pi \sin \left( \sqrt{\frac{\pi}{2}} \right)
  \end{align*}
\end{dem_box}
\newpage


\subsection{Ejercicio 2}
\textit{A través de un cambio de variable adecuado, calcula la siguiente integral doble:}
\begin{align*}
  \int_M x(x^2 + y^2) \; d\mu_2(x, y)
\end{align*}
\textit{donde \(M\) es el sector de círculo centrado en \((0, 0)\), radio \(r\) y lados que forman con el eje \(OX\) los ángulos \(\frac{\pi}{3}\) y \(\frac{\pi}{6}\) radianes respectivamente.}\vspace{2ex}
\begin{dem_box}{Demostración}
  Podemos ver que:
  \begin{itemize}
    \item \(M\) es un conjunto medible
    \item La función \(f(x, y) = x(x^2 + y^2) \) es continua en \(M\) y, por tanto, medible.
    \item Además, como \(M\) es de medida finita y \(f\) es continua en un conjunto compacto, entonces \(f \in \mathcal{L}_1(M)\)
  \end{itemize}
  Por tanto, vamos a aplicar el cambio de variable a coordenadas polares. Sea \(T: U \subseteq \mathbb{R}^N \to \mathbb{R}^N\) difeomorfismo \(\mathcal{C}^1\) con jacobiano \(J_T\), entonces para toda función \(f \in \mathcal{L}_1(T(U))\) se cumple:
  \begin{align*}
    \int_{T(U)} f(x) \; d\mu_N(x) = \int_U f(T(u)) |J_T(u)| \; d\mu_N(u)
  \end{align*}
  Ahora, definimos el cambio de variable:
  \begin{align*}
    T(r, \theta) = (x, y) = (r\cos \theta, r\sin \theta) \quad \mbox{ con } (r, \theta) \in [0, \infty) \times [0, 2\pi)
  \end{align*}
  Calculamos el jacobiano:
  \begin{align*}
    |J_T(r, \theta)| & = \begin{vmatrix}
      \frac{\partial x}{\partial r} & \frac{\partial x}{\partial \theta} \\[2ex]
      \frac{\partial y}{\partial r} & \frac{\partial y}{\partial \theta}
    \end{vmatrix}
    = \begin{vmatrix}
      \cos \theta & - r \sin \theta \\[2ex]
      \sin \theta & r \cos \theta
    \end{vmatrix} = r (\cos^2 \theta + \sin^2 \theta) = r
  \end{align*}
  Ahora, los nuevos límites de integración vienen dados por:
  \begin{itemize}
    \item \(r\) va desde \(0\) hasta el radio \(r\) del sector circular.
    \item \(\theta\) va desde \(\frac{\pi}{6}\) hasta \(\frac{\pi}{3}\).
  \end{itemize}
  Es decir, que el conjunto \(M\) en las nuevas coordenadas es:
  \begin{align*}
    T^{ - 1}(M) = \left\{(r, \theta) : 0 \leq r \leq r, \; \frac{\pi}{6} \leq \theta \leq \frac{\pi}{3} \right\}
  \end{align*}
  Ahora, aplicando el teorema del cambio de variable:
  \begin{align*}
    \int_M x(x^2 + y^2) \; d\mu_2(x, y) & = \int_{T^{ - 1}(M)} (r \cos \theta)(r^2) \cdot r \; d\mu_2(r, \theta) = \\[2ex]
    & = \int_{\pi/6}^{\pi/3} \left(\int_0^{r} r^4 \cos \theta \; dr \right) d\theta
  \end{align*}
  Como la función \((r, \theta) \mapsto r^4 \cos \theta\) es continua, entonces es medible y además es integrable en el conjunto \(T^{ - 1}(M)\) de medida finita. Por tanto, podemos aplicar el teorema de Fubini y separar las integrales:
  \begin{align*}
    \int_{\pi/6}^{\pi/3} \left(\int_0^{r} r^4 \cos \theta \; dr \right) d\theta & = \left( \int_{\pi/6}^{\pi/3} \cos \theta \; d\theta \right) \cdot \left( \int_0^{r} r^4 \; dr \right) = \\[2ex]
    & = \left[ \sin \theta \right]_{\pi/6}^{\pi/3} \cdot \left[ \frac{r^5}{5} \right]_0^{r} = \\[2ex]
    & = \left( \sin \frac{\pi}{3} - \sin \frac{\pi}{6} \right) \cdot \frac{r^5}{5} = \\[2ex]
    & = \left( \frac{\sqrt{3}}{2} - \frac{1}{2} \right) \cdot \frac{r^5}{5} = \\[2ex]
    & = \frac{(\sqrt{3} - 1) r^5}{10}
  \end{align*}
\end{dem_box}
\newpage

\subsection{Ejercicio 3}
\textit{A través de un cambio de variable adecuado, calcula la siguiente integral doble:}
\begin{align*}
  \int_M \arcsin (x^2 + y^2) \; d\mu_2(x, y)
\end{align*}
\textit{donde \(M\) es el conjunto cuya frontera está en coordenadas polares y tiene por ecuación:}
\begin{align*}
  \rho = \sqrt{\sin \theta} \quad \mbox{ con } 0 \leq \theta \leq \frac{\pi}{2}
\end{align*}
\begin{dem_box}{Demostración}
  Podemos ver que:
  \begin{itemize}
    \item \(M\) es un conjunto medible
    \item La función \(f(x, y) = \arcsin(x^2 + y^2) \) es continua en \(M\) y, por tanto, medible.
    \item Además, como \(M\) es de medida finita y \(f\) es continua en un conjunto compacto, entonces \(f \in \mathcal{L}_1(M)\)
  \end{itemize}
  Por tanto, vamos a aplicar el cambio de variable a coordenadas polares. Sea \(T: U \subseteq \mathbb{R}^N \to \mathbb{R}^N\) difeomorfismo \(\mathcal{C}^1\) con jacobiano \(J_T\), entonces para toda función \(f \in \mathcal{L}_1(T(U))\) se cumple:
  \begin{align*}
    \int_{T(U)} f(x) \; d\mu_N(x) = \int_U f(T(u)) |J_T(u)| \; d\mu_N(u)
  \end{align*}
  Ahora, definimos el cambio de variable:
  \begin{align*}
    T(r, \theta) = (x, y) = (r\cos \theta, r\sin \theta) \quad \mbox{ con } (r, \theta) \in [0, \infty) \times [0, 2\pi)
  \end{align*}
  Calculamos el jacobiano:
  \begin{align*}
    |J_T(r, \theta)| & = \begin{vmatrix}
      \frac{\partial x}{\partial r} & \frac{\partial x}{\partial \theta} \\[2ex]
      \frac{\partial y}{\partial r} & \frac{\partial y}{\partial \theta}
    \end{vmatrix}
    = \begin{vmatrix}
      \cos \theta & - r \sin \theta \\[2ex]
      \sin \theta & r \cos \theta
    \end{vmatrix} = r (\cos^2 \theta + \sin^2 \theta) = r
  \end{align*}
  Ahora, los nuevos límites de integración vienen dados por:
  \begin{itemize}
    \item \(r\) va desde \(0\) hasta \(\sqrt{\sin \theta}\).
    \item \(\theta\) va desde \(0\) hasta \(\frac{\pi}{2}\).
  \end{itemize}
  Es decir, que el conjunto \(M\) en las nuevas coordenadas es:
  \begin{align*}
    T^{ - 1}(M) = \left\{(r, \theta) : 0 \leq r \leq \sqrt{\sin \theta}, \; 0 \leq \theta \leq \frac{\pi}{2} \right\}
  \end{align*}
  Ahora, aplicando el teorema del cambio de variable:
  \begin{align*}
    \int_M \arcsin (x^2 + y^2) \; d\mu_2(x, y) & = \int_{T^{ - 1}(M)} \arcsin (r^2) \cdot r \; d\mu_2(r, \theta) = \\[2ex]
    & = \int_0^{\pi/2} \left(\int_0^{\sqrt{\sin \theta}} r \arcsin (r^2) \; dr \right) d\theta
  \end{align*}
  Como la función \((r, \theta) \mapsto r \arcsin (r^2)\) es continua, entonces es medible y además es integrable en el conjunto \(T^{ - 1}(M)\) de medida finita. Por tanto, podemos aplicar el teorema de Fubini y separar las integrales:
  \begin{align*}
    \int_0^{\pi/2} \left(\int_0^{\sqrt{\sin \theta}} r \arcsin (r^2) \; dr \right) d\theta & = \int_0^{\frac{\pi}{2}} \left[ \frac{1}{2} \left( (r^2) \arcsin (r^2) + \sqrt{1 - r^4} \right) \right]_0^{\sqrt{\sin \theta}} d\theta = \\[2ex]
    & = \frac{1}{2} \int_0^{\frac{\pi}{2}} \left( \sin \theta \cdot \arcsin (\sin \theta) + \sqrt{1 - \sin^2 \theta} - 0 - 1 \right) d\theta = \\[2ex]
    & = \frac{1}{2} \int_0^{\frac{\pi}{2}} \left( \sin \theta \cdot \theta + \cos \theta - 1 \right) d\theta = \\[2ex]
    & = \frac{1}{2} \left[ \left( -\theta \cos \theta + \sin \theta \right) + \sin \theta - \theta \right]_0^{\frac{\pi}{2}} = \\[2ex]
    & = \frac{1}{2} \left( \left( -\frac{\pi}{2} \cdot 0 + 1 \right) + 1 - \frac{\pi}{2} - (0 + 0 - 0) \right) = \\[2ex]
    & = \frac{1}{2} \left( 2 - \frac{\pi}{2} \right) = \\[2ex]
    & = 1 - \frac{\pi}{4}
  \end{align*}
\end{dem_box}
\newpage

\subsection{Ejercicio 7}
\textit{Calcula la siguiente integral doble:}
\begin{align*}
  \int_M \sqrt{1 - \frac{x^2}{a^2} - \frac{y^2}{b^2}} \; d\mu_2(x, y) \quad \mbox{ con } M = \left\{(x, y) : \frac{x^2}{a^2} + \frac{y^2}{b^2} \leq 1 \right\}
\end{align*}
\begin{dem_box}{Demostración}
  Podemos ver que:
  \begin{itemize}
    \item \(M\) es un conjunto medible
    \item La función \(f(x, y) = \sqrt{1 - \frac{x^2}{a^2} - \frac{y^2}{b^2}} \) es continua en \(M\) y, por tanto, medible.
    \item Además, como \(M\) es de medida finita y \(f\) es continua en un conjunto compacto, entonces \(f \in \mathcal{L}_1(M)\)
  \end{itemize}
  Por tanto, vamos a aplicar el cambio de variable a coordenadas elípticas. Sea \(T: U \subseteq \mathbb{R}^N \to \mathbb{R}^N\) difeomorfismo \(\mathcal{C}^1\) con jacobiano \(J_T\), entonces para toda función \(f \in \mathcal{L}_1(T(U))\) se cumple:
  \begin{align*}
    \int_{T(U)} f(x) \; d\mu_N(x) = \int_U f(T(u)) |J_T(u)| \; d\mu_N(u)
  \end{align*}
  Ahora, definimos el cambio de variable:
  \begin{align*}
    T(r, \theta) = (x, y) = (ar\cos \theta, br\sin \theta) \quad \mbox{ con } (r, \theta) \in [0, \infty) \times [0, 2\pi)
  \end{align*}
  Calculamos el jacobiano:
  \begin{align*}
    |J_T(r, \theta)| & = \begin{vmatrix}
      \frac{\partial x}{\partial r} & \frac{\partial x}{\partial \theta} \\[2ex]
      \frac{\partial y}{\partial r} & \frac{\partial y}{\partial \theta}
    \end{vmatrix}
    = \begin{vmatrix}
      a\cos \theta & - ar \sin \theta \\[2ex]
      b\sin \theta & br \cos \theta
    \end{vmatrix} = abr
  \end{align*}
  Ahora, los nuevos límites de integración vienen dados por:
  \begin{itemize}
    \item \(r\) va desde \(0\) hasta \(1\) ya que la frontera es la elipse \(\frac{x^2}{a^2} + \frac{y^2}{b^2} = 1\) entonces:
    \begin{align*}
      \frac{(ar\cos \theta)^2}{a^2} + \frac{(br\sin \theta)^2}{b^2} = r^2 (\cos^2 \theta + \sin^2 \theta) = r^2 = 1 \implies r = 1
    \end{align*}
    \item \(\theta\) va desde \(0\) hasta \(2\pi\) ya que la elipse es simétrica respecto a ambos ejes.
  \end{itemize}
  Es decir, que el conjunto \(M\) en las nuevas coordenadas es:
  \begin{align*}
    T^{ - 1}(M) = \left\{(r, \theta) : 0 \leq r \leq 1, \; 0 \leq \theta \leq 2\pi \right\}
  \end{align*}
  Ahora, aplicando el teorema del cambio de variable:
  \begin{align*}
    \int_M \sqrt{1 - \frac{x^2}{a^2} - \frac{y^2}{b^2}} \; d\mu_2(x, y) & = \int_{T^{ - 1}(M)} \sqrt{1 - r^2} \cdot abr \; d\mu_2(r, \theta) = \\[2ex]
    & = ab \int_0^{2\pi} \left(\int_0^{1} r \sqrt{1 - r^2} \; dr \right) d\theta
  \end{align*}
  Como la función \((r, \theta) \mapsto r \sqrt{1 - r^2}\) es continua, entonces es medible y además es integrable en el conjunto \(T^{ - 1}(M)\) de medida finita. Por tanto, podemos aplicar el teorema de Fubini y separar las integrales:
  \begin{align*}
    ab \int_0^{2\pi} \left(\int_0^{1} r \sqrt{1 - r^2} \; dr \right) d\theta & = ab \left( \int_0^{2\pi} d\theta \right) \cdot \left( \int_0^{1} r \sqrt{1 - r^2} \; dr \right) = \\[2ex]
    & = ab \cdot 2\pi \cdot \left[ -\frac{1}{3} (1 - r^2)^{3/2} \right]_0^{1} = \\[2ex]
    & = 2\pi ab \cdot \left( 0 + \frac{1}{3} \right) = \\[2ex]
    & = \frac{2\pi ab}{3}
  \end{align*}
\end{dem_box}
\newpage

\section{Derivación bajo el signo de integral}
\subsection{Ejercicio 1}
\subsubsection{Apartado 1}
\textit{Calcula la siguiente integral}
\begin{align*}
  \int_0^\infty \dfrac{1 - e^{ - x^2}}{x^2} \; dx
\end{align*}
\begin{dem_box}{Demostración}
  Se quiere calcular la integral impropia:
  \begin{align*}
    I \coloneq \int_0^\infty \dfrac{1 - e^{ - x^2}}{x^2} \; dx
  \end{align*}
  Pero podemos notar que, por un lado el integrando no está definido para \(0\) y además el intervalo de integración es infinito. Por tanto, definimos la integral como el límite:
  \begin{align*}
    \int_0^\infty \frac{1 - e^{ - x^2}}{x^2} \; dx = \lim_{R \to \infty} \lim_{\varepsilon \to 0^+} \int_\varepsilon^R \frac{1 - e^{ - x^2}}{x^2} \; dx
  \end{align*}
  en caso de que el límite exista y sea finito. Ahora, para poder calcular la integral, introducimos la familia de funciones parametrizadas por \(t > 0\):
  \begin{align*}
    F(t) \coloneq \int_0^\infty \frac{1 - e^{ - tx^2}}{x^2} \; dx
  \end{align*}
  Podemos notar que \(I = F(1)\) y que la función \(F(t)\) está bien definida ya que para cada \(t > 0\):
  \begin{align*}
    f_t(x) = \frac{1 - e^{ - tx^2}}{x^2} \quad x > 0
  \end{align*}
  Usando un desarrollo en serie de Taylor de \(e^{ - tx^2}\) alrededor de \(0\):
  \begin{align*}
    e^{ - tx^2} = 1 - tx^2 + o(x^2) \quad \mbox{ cuando } x \to 0
  \end{align*}
  Entonces, podemos ver que:
  \begin{align*}
    f_t(x) = \frac{tx^2 + o(x^2)}{x^2} = t + \frac{o(x^2)}{x^2} \quad \mbox{ cuando } x \to 0
  \end{align*}
  Por lo tanto, \(f_t\) es localmente integrable cerca de 0. Ahora, como \(0 \leq 1 - e^{ - tx^2} \leq 1\) entonces:
  \begin{align*}
    0 \leq f_t(x) \leq \frac{1}{x^2}
  \end{align*}
  Y como \(\frac{1}{x^2} \in \mathcal{L}_1((1, \infty))\) entonces:
  \begin{align*}
    f_t \in \mathcal{L}_1((0, \infty))
  \end{align*}
  Por lo tanto, \(F(t)\) está bien definida para todo \(t > 0\). Ahora, vamos a derivar \(F(t)\) bajo el signo de integral. Para ello, definimos:
  \begin{align*}
    g(t, x) \coloneq \frac{1 - e^{ - tx^2}}{x^2}
  \end{align*}
  Entonces:
  \begin{itemize}
    \item \(g(\cdot, x) \in \mathcal{C}^1\) para todo \(x > 0\)
    \item La derivada parcial es:
    \begin{align*}
      \frac{\partial g}{\partial t} (t, x) = e^{ - tx^2}
    \end{align*}
  \end{itemize}
  Podemos ver que para cada \(t > 0\):
  \begin{align*}
    \left| \frac{\partial g}{\partial t} (t, x) \right| = e^{ - tx^2} \leq e^{ - x^2}
  \end{align*}
  Y como \(e^{ - x^2} \in \mathcal{L}_1((0, \infty))\) entonces podemos aplicar el teorema de derivación bajo el signo de integral y obtener:
  \begin{align*}
    F'(t) & = \int_0^\infty e^{ - tx^2} \; dx \xlongequal[du = \sqrt{t}dx]{u = \sqrt{t}x} \int_0^\infty e^{ - u^2} \frac{du}{\sqrt{t}} = \frac{1}{\sqrt{t}} \int_0^\infty e^{ - u^2} \; du = \frac{\sqrt{\pi}}{2\sqrt{t}}
  \end{align*}
  Por tanto, si integramos \(F'(t)\) obtenemos:
  \begin{align*}
    F(t) = \int F'(t) \; dt = \int \frac{\sqrt{\pi}}{2\sqrt{t}} \; dt = \sqrt{\pi} \sqrt{t} + C
  \end{align*}
  Para calcular la constante \(C\), calculamos el límite cuando \(t \to 0^+\):
  \begin{align*}
    F(0) = \lim_{t \to 0^ + } F(t) = \lim_{t \to 0^ + } \int_0^\infty \frac{1 - e^{ - tx^2}}{x^2} \; dx = \int_0^\infty \lim_{t \to 0^ + } \frac{1 - e^{ - tx^2}}{x^2} \; dx = 0
  \end{align*}
  Por lo tanto, \(C = 0\) y:
  \begin{align*}
    F(1) = \sqrt{\pi}
  \end{align*}
\end{dem_box}

\newpage
\subsubsection{Apartado 2}
\textit{Calcula la siguiente integral}
\begin{align*}
  \int_0^{\frac{\pi}{2}} \dfrac{\log (1 + \sin^2 x)}{\sin^2 x} \; dx
\end{align*}
\begin{dem_box}{Demostración}
  Se quiere calcular la integral impropia:
  \begin{align*}
    I = \int_0^{\frac{\pi}{2}} \dfrac{\log (1 + \sin^2 x)}{\sin^2 x} \; dx
  \end{align*}
  Podemos notar que el integrando no está definido para \(0\) ya que \(\sin 0 = 0\) y se anula el denominador. Por tanto, queremos introducir un parámetro para poder derivar bajo el signo de integral. Para ello, definimos la familia de funciones parametrizadas por \(t > 0\):
  \begin{align*}
    F(t) = \int_0^{\frac{\pi}{2}} \dfrac{\log (1 + t \sin^2 x)}{\sin^2 x} \; dx
  \end{align*}
  De esta forma, podemos ver que \(I = F(1)\). Además, para cada \(t > 0\):
  \begin{align*}
    f_t(x) = \dfrac{\log (1 + t \sin^2 x)}{\sin^2 x} \quad x \in \left(0, \frac{\pi}{2}\right)
  \end{align*}
  Ahora, usando un desarrollo en serie de Taylor de \(\log (1 + t \sin^2 x)\) alrededor de \(0\):
  \begin{align*}
    \log (1 + t \sin^2 x) = t \sin^2 x + o(\sin^2 x) \quad \mbox{ cuando } x \to 0
  \end{align*}
  Entonces, podemos ver que:
  \begin{align*}
    f_t(x) = \frac{t \sin^2 x + o(\sin^2 x)}{\sin^2 x} = t + \frac{o(\sin^2 x)}{\sin^2 x} \quad \mbox{ cuando } x \to 0
  \end{align*}
  Por lo tanto, \(f_t\) es localmente integrable cerca de 0. Además, podemos notar que:
  \begin{align*}
    \log (1 + t \sin^2 x) \leq t\sin^2 x \implies 0 \leq f_t(x) \leq t
  \end{align*}
  Por lo que \(f_t\) es acotada en \(\left(0, \frac{\pi}{2}\right)\) y podemos dominarla por una función integrable en dicho intervalo. Por tanto:
  \begin{align*}
    f_t \in \mathcal{L}_1\left(\left(0, \frac{\pi}{2}\right)\right)
  \end{align*}
  Por lo tanto, \(F(t)\) está bien definida para todo \(t > 0\). Ahora, vamos a derivar \(F(t)\) bajo el signo de integral. Para ello, definimos:
  \begin{align*}
    g(t, x) \coloneq \dfrac{\log (1 + t \sin^2 x)}{\sin^2 x}
  \end{align*}
  Entonces:
  \begin{align*}
    \frac{\partial g}{\partial t} (t, x) = \frac{1}{1 + t \sin^2 x}
  \end{align*}
  Podemos ver que para cada \(t > 0\):
  \begin{align*}
    \left| \frac{\partial g}{\partial t} (t, x) \right| = \frac{1}{1 + t \sin^2 x} \leq 1
  \end{align*}
  Y como \(1 \in \mathcal{L}_1\left(\left(0, \frac{\pi}{2}\right)\right)\) entonces podemos aplicar el teorema de derivación bajo el signo de integral y obtener:
  \begin{align*}
    F'(t) & = \int_0^{\frac{\pi}{2}} \frac{1}{1 + t \sin^2 x} \; dx \xlongequal[u = \tan x]{du = \sec^2 x \; dx} \int_0^\infty \frac{1}{1 + t \frac{u^2}{1 + u^2}} \cdot \frac{1}{1 + u^2} \; du = \\[2ex]
    & = \int_0^\infty \frac{1 + u^2}{1 + u^2 + t u^2} \cdot \frac{1}{1 + u^2} \; du = \int_0^\infty \frac{1}{1 + (1 + t) u^2} \; du = \\[2ex]
    & = \frac{1}{\sqrt{1 + t}} \int_0^\infty \frac{1}{1 + v^2} \; dv = \frac{1}{\sqrt{1 + t}} \cdot \left[ \arctan v \right]_0^\infty  = \frac{\pi}{2\sqrt{1 + t}}
  \end{align*}
  Por tanto, si integramos \(F'(t)\) obtenemos:
  \begin{align*}
    F(t) = \int F'(t) \; dt = \int \frac{\pi}{2\sqrt{1 + t}} \; dt = \pi \sqrt{1 + t} + C
  \end{align*}
  Para calcular la constante \(C\), calculamos el límite cuando \(t \to 0^+\):
  \begin{align*}
    F(0) = \lim_{t \to 0^ + } F(t) = \lim_{t \to 0^ + } \int_0^{\frac{\pi}{2}} \dfrac{\log (1 + t \sin^2 x)}{\sin^2 x} \; dx = \int_0^{\frac{\pi}{2}} \lim_{t \to 0^ + } \dfrac{\log (1 + t \sin^2 x)}{\sin^2 x} \; dx = 0
  \end{align*}
  Por lo tanto, \(C = -\pi\) y:
  \begin{align*}
    F(1) = \pi (\sqrt{2} - 1)
  \end{align*}
\end{dem_box}
\newpage
\section{Espacios \(L^p\)}
\subsection{Ejercicio 2}
\textit{Demuestra que si \(I\) es un intervalo acotado entonces el conjunto de funciones \(\mathcal{L}_p(I)\) está estrictamente contenido en \(\mathcal{L}_q(I)\) para \(1 \leq q \leq p \leq \infty\)}\vspace{2ex}

\begin{dem_box}{Demostración}
  Queremos ver que, dado \(1 \leq q \leq p \leq \infty\) entonces:
  \begin{align*}
    \mathcal{L}_p(I) \subsetneq \mathcal{L}_q(I)
  \end{align*}
  Para ello, haremos dos pasos:
  \begin{itemize}
    \item \textit{Contenido: \(\mathcal{L}_p(I) \subset \mathcal{L}^q(I)\).} Sea \(f \in \mathcal{L}_p(I)\) queremos ver que \(f \in \mathcal{L}_q(I)\), es decir:
    \begin{align*}
      \|f\|_q = \left( \int_I |f(x)|^q \; d\mu(x) \right)^{1/q} < \infty
    \end{align*}
    Pero como \(I\) es un intervalo acotado, entonces \(\mu(I) < \infty\) y podemos aplicar la desigualdad de Hölder. Para ello, definimos los exponentes conjugados:
    \begin{align*}
      r = \frac{p}{q}
    \end{align*}
    Como \(p \geq q\) entonces \(r \geq  1\). Así, ea \(s\) el exponente conjugado de \(r\) tal que:
    \begin{align*}
      \frac{1}{r} + \frac{1}{s} = 1 \iff \frac{1}{s} = 1 - \frac{q}{p} = \frac{p - q}{p} \implies s = \frac{p}{p - q}
    \end{align*}
    Aplicando la desigualdad de Hölder con las funciones \(|f|^p\) y \(1\) obtenemos:
    \begin{align*}
      \int_I |f(x)|^q \; d\mu(x) \leq \left( \int_I \left(|f(x)|^q\right)^{\frac{p}{q}}\; d\mu(x) \right)^{\frac{q}{p}} \cdot \left( \int_I 1^s \; d\mu(x) \right)^{\frac{p - q}{p}}
    \end{align*}
    Pero podemos ver que:
    \begin{align*}
      \left( \int_I \left(|f(x)|^q\right)^{\frac{p}{q}}\; d\mu(x) \right)^{\frac{q}{p}} & = \left( \int_I |f(x)|^p \; d\mu(x) \right)^{\frac{q}{p}} = \|f\|_p^q < \infty
    \end{align*}
    Y además:
    \begin{align*}
      \left( \int_I 1^s \; d\mu(x) \right)^{\frac{p - q}{p}} & = \left( \mu(I) \right)^{\frac{p - q}{p}} < \infty
    \end{align*}
    Por lo que:
    \begin{align*}
      \int_I |f(x)|^q \; d\mu(x) < \infty
    \end{align*}
    Y por tanto, \(f \in \mathcal{L}_q(I)\).
    \item \textit{No igualdad: \(\mathcal{L}_p(I) \neq \mathcal{L}_q(I)\).} Supongamos que \(I = (0, b]\) y definimos la función:
    \begin{align*}
      f(x) = \frac{1}{x^\alpha}
    \end{align*}
    Sabemos que esta función tiene integral de Riemann y, por tanto, de Lebesgue y converge en el origin si y solo si \(\alpha < 1\):
    \begin{itemize}
      \item Para que \(f \in \mathcal{L}_q(I)\) necesitamos que \(\int_I |f|^q < \infty\) es decir:
      \begin{align*}
        \alpha \cdot q < 1 \implies \alpha < \frac{1}{q}
      \end{align*}
      \item Para que \(f \notin \mathcal{L}_p(I)\) necesitamos que \(\int_I |f|^p = \infty\) es decir:
      \begin{align*}
        \alpha \cdot p \geq 1 \implies \alpha \geq \frac{1}{p}
      \end{align*}
    \end{itemize}
    Por hipótesis \(q \leq p\), entonces:
    \begin{align*}
      \frac{1}{p} \leq \frac{1}{q} \implies \exists \alpha \in \mathbb{R} \mbox{ tq } \frac{1}{p} \leq \alpha < \frac{1}{q}
    \end{align*}
    Por lo que la función \(f(x) = \frac{1}{x^\alpha}\) cumple que \(f \in \mathcal{L}_q(I)\) pero \(f \notin \mathcal{L}_p(I)\).
  \end{itemize}
\end{dem_box}

\newpage
\section{Funciones de variación acotada}
\subsection{Ejercicio 1}
\textit{Demuestra que la función \(f: [a, b] \to \mathbb{R} \) es de variación acotada si y solo si sus semi-variaciones positiva y negativa son ambas finitas}.\vspace{2ex}

\begin{dem_box}{Demostración}
  Sea \(f: [a, b] \to \mathbb{R}\) una función y definimos sus semi-variaciones positiva y negativa como:
  \begin{align*}
    V^ + (f, P) = \displaystyle \sum_{i = 1}^{n} |f(t_i) - f(t_{i - 1})^{ +}| \quad \mbox{ y } \quad V^ - (f, P) = \displaystyle \sum_{i = 1}^{n} |f(t_i) - f(t_{i - 1})^{ -}|
  \end{align*}
  Así, sea \(P = \{a = t_0 < t_1 < \ldots < t_n = b\}\) una partición de \([a, b]\) veamos que:
  \begin{itemize}
    \item[\(\Rightarrow\)] Si \(f\) es de variación acotada, entonces existe \(M > 0\) tal que para toda partición \(P\) de \([a, b]\):
    \begin{align*}
      V(f, P) = \sum_{i = 1}^{n} |f(t_i) - f(t_{i - 1})| \leq M
    \end{align*}
    Pero podemos notar que:
    \begin{align*}
      |f(t_i) - f(t_{i - 1})| & = |f(t_i) - f(t_{i - 1})^{ +} + f(t_i) - f(t_{i - 1})^{ -}| \leq \\[2ex]
      & \leq |f(t_i) - f(t_{i - 1})^{ +}| + |f(t_i) - f(t_{i - 1})^{ -}|
    \end{align*}
    Por lo que:
    \begin{align*}
      V(f, P) & = \sum_{i = 1}^{n} |f(t_i) - f(t_{i - 1})| \leq \\[2ex]
      & \leq \sum_{i = 1}^{n} |f(t_i) - f(t_{i - 1})^{ +}| + \sum_{i = 1}^{n} |f(t_i) - f(t_{i - 1})^{ -}| = \\[2ex]
      & = V^ + (f, P) + V^ - (f, P) \leq M
    \end{align*}
    Por lo que \(V^+ (f) \leq M\) y \(V^- (f) \leq M\) y ambas son finitas.
    \item[\(\Leftarrow\)] Si \(V^+ (f)\) y \(V^- (f)\) son finitas, entonces existen \(M_1, M_2 > 0\) tales que para toda partición \(P\) de \([a, b]\):
    \begin{align*}
      V^+ (f, P) \leq M_1 \quad \mbox{ y } \quad V^- (f, P) \leq M_2
    \end{align*}
    Pero podemos notar que:
    \begin{align*}
      |f(t_i) - f(t_{i - 1})| & = |f(t_i) - f(t_{i - 1})^{ +} + f(t_i) - f(t_{i - 1})^{ -}| \leq \\[2ex]
      & \leq |f(t_i) - f(t_{i - 1})^{ +}| + |f(t_i) - f(t_{i - 1})^{ -}|
    \end{align*}
    Por lo que:
    \begin{align*}
      V(f, P) & = \sum_{i = 1}^{n} |f(t_i) - f(t_{i - 1})| \leq \\[2ex]
      & \leq \sum_{i = 1}^{n} |f(t_i) - f(t_{i - 1})^{ +}| + \sum_{i = 1}^{n} |f(t_i) - f(t_{i - 1})^{ -}| = \\[2ex]
      & = V^ + (f, P) + V^ - (f, P) \leq M_1 + M_2
    \end{align*}
    Por lo que \(f\) es de variación acotada.
  \end{itemize}
\end{dem_box}
\newpage


\subsection{Ejercicio 3}
\textit{Demuestra que toda función de variación acotada \(f\) sobre \([a, b]\) puede descomponerse en \(f = g_1 - g_2\) con \(g_1, g_2\) funciones crecientes y acotadas sobre \([a, b]\).}\vspace{2ex}
\begin{dem_box}{Demostración}
  Sea \(f: [a, b] \to \mathbb{R}\) una función de variación acotada. Definimos las funciones \(g_1, g_2: [a, b] \to \mathbb{R}\) como:
  \begin{align*}
    g_1(x) = V^+ (f, [a, x]) \quad \mbox{ y } \quad g_2(x) = V^- (f, [a, x])
  \end{align*}
  Podemos notar que \(g_1\) y \(g_2\) son funciones crecientes ya que si \(x_1 < x_2\) entonces:
  \begin{align*}
    g_1(x_2) - g_1(x_1) & = V^+ (f, [a, x_2]) - V^+ (f, [a, x_1]) = \\[2ex]
    & = V^+ (f, [x_1, x_2]) \geq 0
  \end{align*}
  Y de forma análoga:
  \begin{align*}
    g_2(x_2) - g_2(x_1) & = V^- (f, [a, x_2]) - V^- (f, [a, x_1]) = \\[2ex]
    & = V^- (f, [x_1, x_2]) \geq 0
  \end{align*}
  Además, como \(f\) es de variación acotada entonces \(V^+ (f)\) y \(V^- (f)\) son finitas y por tanto \(g_1\) y \(g_2\) son acotadas. Ahora, veamos que \(f = g_1 - g_2\). Para ello, sea \(x \in [a, b]\) y consideremos una partición \(P = \{a = t_0 < t_1 < \ldots < t_n = x\}\) de \([a, x]\). Entonces:
  \begin{align*}
    f(x) - f(a) & = \sum_{i = 1}^{n} (f(t_i) - f(t_{i - 1})) = \\[2ex]
    & = \sum_{i = 1}^{n} |f(t_i) - f(t_{i - 1})^{ +}| - \sum_{i = 1}^{n} |f(t_i) - f(t_{i - 1})^{ -}| = \\[2ex]
    & = V^+ (f, P) - V^- (f, P)
  \end{align*}
  Si tomamos el supremo sobre todas las particiones \(P\) de \([a, x]\) obtenemos:
  \begin{align*}
    f(x) - f(a) = V^+ (f, [a, x]) - V^- (f, [a, x]) = g_1(x) - g_2(x)
  \end{align*}
  Por lo tanto:
  \begin{align*}
    f(x) = g_1(x) - g_2(x) + f(a)
  \end{align*}
  Definiendo \(g_2'(x) = g_2(x) - f(a)\) obtenemos la descomposición buscada:
  \begin{align*}
    f(x) = g_1(x) - g_2'(x)
  \end{align*}
\end{dem_box}
\newpage

\section{Funciones que fallan las condiciones de Dini/Jordan}
\subsection{Ejercicio 1}
\textit{Considera la función:}
\begin{align*}
  f : \mathbb{R} & \longrightarrow \mathbb{R} \\[1ex]
  x & \longmapsto f(x) = \begin{cases} 
    x \sin \left( \frac{1}{x} \right) & \mbox{ si } x \in [ - \pi, \pi]\setminus \{0\} \\[2ex]
    0 & \mbox{ si } x = 0
  \end{cases} 
\end{align*}
\textit{que se extiende periódicamente al resto de su dominio con periodo \(2\pi\). Demuestra que \(f\) es continua en todo \(\mathbb{R}\) y no verifica las condiciones de Jordan en \(x = 0\) pero sí las de Dini.}\vspace{2ex}

\begin{dem_box}{Demostración}
  Se define la función \(f : \mathbb{R} \to \mathbb{R}\) dada y se extiende periódicamente con periodo \(2\pi\). Por periodicidad, basta con estudiar la continuidad en el intervalo \([ - \pi, \pi]\) y, en particular, en \(x = 0\) que es el punto problemático.
  \begin{itemize}
    \item Si \(x \neq 0\), entonces \(f\) es composición de funciones continuas (producto y seno) y por tanto, es continua en todo punto \(x \neq 0\).
    \item En \(x = 0\):
    \begin{align*}
      \lim_{x \to 0} f(x) & = \lim_{x \to 0} x \sin \left( \frac{1}{x} \right)
    \end{align*}
    Podemos notar que:
    \begin{align*}
      \left| x \sin \left( \frac{1}{x} \right) \right| \leq |x|
    \end{align*}
    Y por tanto, por el teorema del sándwich:
    \begin{align*}
      \lim_{x \to 0} x \sin \left( \frac{1}{x} \right) = 0 = f(0)
    \end{align*}
  \end{itemize}
  Por tanto, \(f\) es continua en todo \(\mathbb{R}\).\vspace{2ex}

  Veamos que no cumple la condición de Jordán, es decir, \(f\) no es de variación acotada en ningún intervalo que contenga a \(0\). Para ello, construimos la sucesión:
  \begin{align*}
    x_n = \frac{1}{n\pi} \quad n \in \mathbb{N} \implies f(x_n) = \frac{1}{n\pi} \sin (n\pi) = 0
  \end{align*}
  Y en los puntos intermedios:
  \begin{align*}
    y_n = \frac{1}{\left(n + \frac{1}{2}\right)\pi} \implies f(y_n) = \frac{1}{\left(n + \frac{1}{2}\right)\pi} \sin \left( \left(n + \frac{1}{2}\right)\pi \right) = \frac{(-1)^n}{\left(n + \frac{1}{2}\right)\pi}
  \end{align*}
  Así, la función realiza oscilaciones en cada intervalo:
  \begin{align*}
    \left[\frac{1}{(n + 1)\pi}, \frac{1}{n\pi}\right] \quad n \in \mathbb{N}
  \end{align*}
  cuyo salto es:
  \begin{align*}
    |f(y_n) - f(x_n)| + |f(y_n) - f(x_{n + 1})| = 2 \cdot \frac{1}{\left(n + \frac{1}{2}\right)\pi} 
  \end{align*}
  Por tanto, si hacemos la suma de las oscilaciones tenemos:
  \begin{align*}
    \displaystyle \sum_{n = 1}^{\infty} 2 \cdot \frac{1}{\left(n + \frac{1}{2}\right)\pi} = \frac{2}{\pi} \displaystyle \sum_{n = 1}^{\infty} \frac{1}{n + \frac{1}{2}} = \infty
  \end{align*}
  Por lo tanto, \(f\) no es de variación acotada en ningún intervalo que contenga a \(0\) y por tanto, no cumple la condición de Jordan en \(x = 0\).\vspace{2ex}

  Ahora, veamos que \(f\) cumple las condiciones de Dini en \(x = 0\). Para ello, definimos la función:
  \begin{align*}
    g(t) = \dfrac{f(t) + f( - t)}{2} = \dfrac{t \sin \left( \frac{1}{t} \right) + ( - t) \sin \left( \frac{1}{ - t} \right)}{2} = \dfrac{t \sin \left( \frac{1}{t} \right) - t \sin \left( \frac{1}{t} \right)}{2} = t \sin \left( \frac{1}{t} \right)
  \end{align*}
  Entonces, calculamos el límite cuando \(t \to 0^+\):
  \begin{align*}
    \lim_{t \to 0^+} g(t) = \lim_{t \to 0^+} t \sin \left( \frac{1}{t} \right) = 0
  \end{align*}
  Queremos estudiar ahora:
  \begin{align*}
    \dfrac{g(t) - g(0^ + )}{t} = (t\sin \left( \frac{1}{t} \right) - 0) \cdot \frac{1}{t} = \sin \left( \frac{1}{t} \right)
  \end{align*}
  Y claramente es una función acotada, medible e integrable sobre cualquier intervalo finito, en particular:
  \begin{align*}
    \sin \left(\frac{1}{t}\right) \in L_1([0, \delta]) \quad \forall \delta > 0
  \end{align*}
  Por lo tanto, \(f\) cumple las condiciones de Dini en \(x = 0\).
\end{dem_box}
\newpage

\subsection{Ejercicio 2}
\textit{Considera la función}
\begin{align*}
  f : \mathbb{R} & \longrightarrow \mathbb{R} \\[1ex]
  x & \longmapsto f(x) = \begin{cases}
    \frac{1}{\log\left(\frac{1}{|x|}\right)} & \mbox{ si } x \in [ - \pi], \pi]\setminus \{0\} \\[2ex]
    0 & \mbox{ si } x = 0
  \end{cases}
\end{align*}
\textit{Y se extiende periódicamente al resto de su dominio con periodo \(2\pi\). Demuestra que \(f\) es continua en todo \(\mathbb{R}\) y que verifica las condiciones de Jordan en \(x = 0\) pero no las de Dini.}\vspace{2ex}
\begin{dem_box}{Demostración}
  Sea la función descrita en el enunciado extendida periódicamente con periodo \(2\pi\). Por periodicidad, basta con estudiar la continuidad en el intervalo \([ - \pi, \pi]\) y, en particular, en \(x = 0\) que es el punto problemático.
  \begin{itemize}
    \item Si \(x \neq 0\) entonces \(f\) es composición de funciones continuas (logaritmo y valor absoluto) y por tanto, es continua en todo punto \(x \neq 0\).
    \item En \(x = 0\):
    \begin{align*}
      \lim_{x \to 0} f(x) & = \lim_{x \to 0} \frac{1}{\log\left(\frac{1}{|x|}\right)} = \lim_{x \to 0} \frac{1}{\log (1) - \log (|x|)} = \lim_{x \to 0} \frac{1}{ - \log (|x|)}
    \end{align*}
    Podemos notar que cuando \(x \to 0\), \(|x| \to 0^+\) y por tanto, \(\log (|x|) \to -\infty\). Por lo tanto:
    \begin{align*}
      \lim_{x \to 0} \frac{1}{ - \log (|x|)} = 0 = f(0)
    \end{align*}
  \end{itemize}
  Por lo tanto, \(f\) es continua en todo \(\mathbb{R}\).\vspace{2ex}

  Veamos que \(f\) cumple la condición de Jordan en \(x = 0\), es decir, \(f\) es de variación acotada en algún intervalo que contenga a \(0\). Podemos notar que la función es monótona en el intervalo \( - [\delta, 0]\) con \(\delta < 1\) por tanto, es de variación acotada en dicho intervalo. De forma análoga, es monótona en \([0, \delta]\) y por tanto, es de variación acotada en dicho intervalo. Por lo tanto, \(f\) es de variación acotada en \([ - \delta, \delta]\) y cumple la condición de Jordan en \(x = 0\).\vspace{2ex}

  Además, converge en particular a:
  \begin{align*}
    \lim_{x \to 0^ + } \dfrac{f(x + t) + f(x - t)}{2} = \lim_{t \to 0^ + } \dfrac{1}{\log\left(\frac{1}{t}\right)} = 0
  \end{align*}

  Ahora, para ver que no cumple las condiciones de Dini, estudiamos la función:
  \begin{align*}
    g(t) = \dfrac{f(t) + f( - t)}{2} = \dfrac{1}{\log\left(\frac{1}{t}\right)}
  \end{align*}
  Entonces, calculamos el límite cuando \(t \to 0^+\):
  \begin{align*}
    \lim_{t \to 0^+} g(t) = \lim_{t \to 0^+} \dfrac{1}{\log\left(\frac{1}{t}\right)} = 0
  \end{align*}
  Queremos estudiar ahora:
  \begin{align*}
    \dfrac{g(t) - g(0^ + )}{t} = \dfrac{\frac{1}{\log\left(\frac{1}{t}\right)} - 0}{t} = \frac{1}{t \log\left(\frac{1}{t}\right)}
  \end{align*}
  Y podemos ver que la integral en cualquier intervalo \([0, \delta]\) es:
  \begin{align*}
    \int_0^\delta \frac{1}{t} \frac{1}{\log \dfrac{1}{t}} \; dt \xlongequal[u = \log \frac{1}{t}]{du = -\frac{1}{t} \; dt} \int_{\log \frac{1}{\delta}}^\infty \frac{1}{u} \; du = \left[ \log |u| \right]_{\log \frac{1}{\delta}}^\infty = \infty
  \end{align*}
\end{dem_box}
\newpage

\subsection{Ejercicio 3}
\textit{Considera la función \(h(x) = f(x) \cdot \sin^2\left(\frac{1}{x}\right) \) donde \(f\) es la función del ejercicio anterior. ¿Es continua en todo su dominio? ¿Verifica las condiciones de Jordan? ¿Y las de Dini?}\vspace{2ex}

\begin{dem_box}{Demostración}
  La función en cuestión es:
  \begin{align*}
    h: \mathbb{R} & \longrightarrow \mathbb{R} \\[1ex]
    x & \longmapsto h(x) = \begin{cases}
      \frac{1}{\log\left(\frac{1}{|x|}\right)} \cdot \sin^2\left(\frac{1}{x}\right) & \mbox{ si } x \in [ - \pi, \pi]\setminus \{0\} \\[2ex]
      0 & \mbox{ si } x = 0
    \end{cases}
  \end{align*}
  Extendida periódicamente con periodo \(2\pi\). Por periodicidad, basta con estudiar la continuidad en el intervalo \([ - \pi, \pi]\). En particular, el punto problemático es \(x = 0\).
  \begin{itemize}
    \item Si \(x \neq 0\) entonces \(h\) es composición de funciones continuas (logaritmo, valor absoluto, seno y producto) y por tanto, es continua en todo punto \(x \neq 0\).
    \item En \(x = 0\):
    \begin{align*}
      \lim_{x \to 0} h(x) & = \lim_{x \to 0} \frac{1}{\log\left(\frac{1}{|x|}\right)} \cdot \sin^2\left(\frac{1}{x}\right)
    \end{align*}
    Como \(\sin^2\left(\frac{1}{x}\right)\) está acotada entre \(0\) y \(1\), podemos usar el teorema del sándwich:
    \begin{align*}
      0 \leq \left| \frac{1}{\log\left(\frac{1}{|x|}\right)} \cdot \sin^2\left(\frac{1}{x}\right) \right| \leq \left| \frac{1}{\log\left(\frac{1}{|x|}\right)} \right|
    \end{align*}
    Y como ya hemos visto en el ejercicio anterior:
    \begin{align*}
      \lim_{x \to 0} \frac{1}{\log\left(\frac{1}{|x|}\right)} = 0
    \end{align*}
    Por lo tanto:
    \begin{align*}
      \lim_{x \to 0} h(x) = 0 = h(0)
    \end{align*}
    Además, la función es simétrica ya que:
    \begin{align*}
      h( - x) = \frac{1}{\log\left(\frac{1}{| - x|}\right)} \cdot \sin^2\left(\frac{1}{ - x}\right) & = \frac{1}{\log\left(\frac{1}{|x|}\right)} \cdot \sin^2\left( - \frac{1}{x}\right) = \\[2ex]
      & = \frac{1}{\log\left(\frac{1}{|x|}\right)} \cdot \sin^2\left(\frac{1}{x}\right) = h(x)
    \end{align*}
    Por lo tanto, \(h\) es continua en todo \(\mathbb{R}\).\vspace{2ex}
  \end{itemize}
  Ahora, veamos si cumple las condiciones de Jordan en \(x = 0\). Podemos notar que la función no es monótona en ningún intervalo que contenga a \(0\) debido a las oscilaciones de \(\sin^2\left(\frac{1}{x}\right)\). Además, podemos construir una sucesión de oscilaciones similares a las del ejercicio 1:
  \begin{align*}
    x_n = \frac{1}{n\pi} \quad n \in \mathbb{N} & \implies h(x_n) = \frac{1}{\log (n\pi)} \cdot \sin^2 (n\pi) = 0\\[2ex]
    y_n = \frac{1}{\left(n + \frac{1}{2}\right)\pi} \quad n \in \mathbb{N} & \implies h(y_n) = \frac{1}{\log\left(\left(n + \frac{1}{2}\right)\pi\right)} \cdot \sin^2\left(\left(n + \frac{1}{2}\right)\pi\right) = \frac{1}{\log\left(\left(n + \frac{1}{2}\right)\pi\right)}
  \end{align*}
  Así, la función realiza oscilaciones en cada intervalo:
  \begin{align*}
    \left[\frac{1}{(n + 1)\pi}, \frac{1}{n\pi}\right] \quad n \in \mathbb{N}
  \end{align*}
  cuyo salto es:
  \begin{align*}
    |h(y_n) - h(x_n)| + |h(y_n) - h(x_{n + 1})| = 2 \cdot \frac{1}{\log\left(\left(n + \frac{1}{2}\right)\pi\right)}
  \end{align*}
  Por tanto, si hacemos la suma de las oscilaciones tenemos:
  \begin{align*}
    \displaystyle \sum_{n = 1}^{\infty} 2 \cdot \frac{1}{\log\left(\left(n + \frac{1}{2}\right)\pi\right)} = 2 \displaystyle \sum_{n = 1}^{\infty} \frac{1}{\log\left(\left(n + \frac{1}{2}\right)\pi\right)}\geq 2 \displaystyle \sum_{n = n_0}^{\infty} \frac{1}{n} = \infty
  \end{align*}
  Por que para \(n_0\) suficientemente grande, \(\log\left(\left(n + \frac{1}{2}\right)\pi\right) < n\). Por lo tanto, \(h\) no es de variación acotada en ningún intervalo que contenga a \(0\) y por tanto, no cumple la condición de Jordan en \(x = 0\).\vspace{2ex}

  Finalmente, veamos si cumple las condiciones de Dini en \(x = 0\). Definimos la función:
  \begin{align*}
    g(t) = \dfrac{h(t) + h( - t)}{2} = \dfrac{1}{\log\left(\frac{1}{t}\right)} \cdot \sin^2\left(\frac{1}{t}\right) = h(t)
  \end{align*}
  Entonces, calculamos el límite cuando \(t \to 0^+\):
  \begin{align*}
    \lim_{t \to 0^+} g(t) = \lim_{t \to 0^+} \dfrac{1}{\log\left(\frac{1}{t}\right)} \cdot \sin^2\left(\frac{1}{t}\right) = 0
  \end{align*}
  Queremos estudiar ahora:
  \begin{align*}
    \dfrac{g(t) - g(0^ + )}{t} = \dfrac{\frac{1}{\log\left(\frac{1}{t}\right)} \cdot \sin^2\left(\frac{1}{t}\right) - 0}{t} = \frac{1}{t \log\left(\frac{1}{t}\right)} \cdot \sin^2\left(\frac{1}{t}\right)
  \end{align*}
  Y podemos ver que la integral en cualquier intervalo \([0, \delta]\) es:
  \begin{align*}
    \int_0^\delta \frac{1}{t} \frac{1}{\log \dfrac{1}{t}} \cdot \sin^2\left(\frac{1}{t}\right) \; dt \xlongequal[u = \log \frac{1}{t}]{du = -\frac{1}{t} \; dt} \int_{\log \frac{1}{\delta}}^\infty \frac{\sin^2(e^u)}{u} \; du = \infty
  \end{align*}
  Ya que \(\sin^2(e^u)\) no tiende a cero cuando \(u \to \infty\). Por lo tanto, \(h\) no cumple las condiciones de Dini en \(x = 0\).
\end{dem_box}
\newpage
\section{Series de Fourier trigonométricas}
\subsection{Ejercicio 2}
\textit{Demuestra que las expansiones en serie son válidas en el intervalo indicado:}
\begin{align*}
  x = \pi - 2 \displaystyle \sum_{n = 1}^{\infty} \frac{\sin nx}{n} \quad \mbox{ con } 0 < x < 2\pi
\end{align*}
\begin{dem_box}{Demostración}
  Consideremos la función definida como:
  \begin{align*}
    f(x) = \pi - x \quad \mbox{ con } 0 < x < 2\pi
  \end{align*}
  Podemos extenderla periódicamente con periodo \(2\pi\) y observar que es una función de variación acotada en \([0, 2\pi]\). Por lo tanto, podemos calcular su serie de Fourier. Sabemos que la serie de Fourier de una función \(f\) con periodo \(2\pi\) viene dada por:
  \begin{align*}
    S_f(x) = \frac{a_0}{2} + \displaystyle \sum_{n = 1}^{\infty} (a_n \cos nx + b_n \sin nx)
  \end{align*}
  donde:
  \begin{align*}
    a_0 & = \frac{1}{\pi} \int_0^{2\pi} f(t) \; dt = \frac{1}{\pi} \int_0^{2\pi} (\pi - t) \; dt = \frac{1}{\pi} \left[\pi t - \frac{t^2}{2} \right]_0^{2\pi} = \frac{1}{\pi} \left(2\pi^2 - 2\pi^2\right) = 0\\[4ex]
    a_n & = \frac{1}{\pi} \int_0^{2\pi} f(t) \cos nt \; dt = \frac{1}{\pi} \int_0^{2\pi} (\pi - t) \cos nt \; dt = 0 \quad \forall n \geq 1\\[4ex]
    b_n & = \frac{1}{\pi} \int_0^{2\pi} f(t) \sin nt \; dt = \frac{1}{\pi} \int_0^{2\pi} (\pi - t) \sin nt \; dt \xlongequal[dv = \sin nt \Rightarrow v = - \frac{\cos nx}{n} ]{u = \pi - t \Rightarrow du = - dt} \\[2ex]
    & = \frac{1}{\pi} \left[ - \frac{(\pi - t) \cos nt}{n} \right]_0^{2\pi} + \frac{1}{\pi} \int_0^{2\pi} \frac{\cos nt}{n} \; dt = \\[2ex]
    & = \frac{1}{\pi} \left( - \dfrac{(\pi - 2\pi) \cos (2n \pi)}{n} + \dfrac{\pi \cos 0}{n} \right) + \frac{1}{n} \left[ \frac{\sin nt}{n} \right]_0^{2\pi} = \\[2ex]
    & = \frac{1}{\pi} \left( - \dfrac{( - \pi) \cdot 1}{n} + \dfrac{\pi \cdot 1}{n} \right) + 0 = \frac{2}{n}
  \end{align*}
  Por lo tanto, la serie de Fourier de \(f\) es:
  \begin{align*}
    S_f(x) & = \frac{0}{2} + \displaystyle \sum_{n = 1}^{\infty} \left(0 \cdot \cos nx + \frac{2}{n} \sin nx\right) = 2 \displaystyle \sum_{n = 1}^{\infty} \frac{\sin nx}{n}
  \end{align*}
  Ahora, por el teorema de Dirichlet, la serie de Fourier converge a \(f(x)\) al ser \(f\) de variación acotada y continua en \((0, 2\pi)\). Por lo tanto, en el abierto \((0, 2\pi)\) la función \(f(x)\) converge exactamente a su serie de Fourier:
  \begin{align*}
    f(x) = \pi - x = S_f(x) = 2 \displaystyle \sum_{n = 1}^{\infty} \frac{\sin nx}{n}
  \end{align*}
  Y por tanto:
  \begin{align*}
    x = \pi - 2 \displaystyle \sum_{n = 1}^{\infty} \frac{\sin nx}{n} \quad \mbox{ con } 0 < x < 2\pi
  \end{align*}
\end{dem_box}
\newpage

\subsection{Ejercicio 4}
\begin{align*}
  \frac{\pi}{4} = \displaystyle \sum_{n = 1}^{\infty} \dfrac{\sin (2n - 1)x}{2n - 1} \quad \mbox{ con } 0 < x < \pi
\end{align*}
\textit{Y emplearlo para deducir que:}
\begin{align*}
  \frac{\pi^2}{6} = \displaystyle \sum_{n = 1}^{\infty} \frac{1}{n^2}
\end{align*}
\begin{dem_box}{Demostración}
  Sea la función:
  \begin{align*}
    f(x) = \frac{\pi}{4} \quad \mbox{ con } 0 < x < \pi
  \end{align*}
  Y para extenderla con periodo \(2\pi\) definimos:
  \begin{align*}
    \hat{f}(x) = \begin{cases}
      \frac{\pi}{4} & \mbox{ si } 0 < x < \pi \\[2ex]
      - \frac{\pi}{4} & \mbox{ si } -\pi < x < 0
    \end{cases}
  \end{align*}
  Podemos observar que \(\hat{f}\) es una función impar y de variación acotada en \([ - \pi, \pi]\). Por lo tanto, podemos calcular su serie de Fourier. Sabemos que la serie de Fourier de una función \(f\) con periodo \(2\pi\) viene dada por:
  \begin{align*}
    S_f(x) = \frac{a_0}{2} + \displaystyle \sum_{n = 1}^{\infty} (a_n \cos nx + b_n \sin nx)
  \end{align*}
  Calculamos los coeficientes:
  \begin{align*}
    a_0 & \frac{1}{\pi} \int_{ - \pi}^\pi \hat{f}(t) \; dt = 0\\[4ex]
    a_n & = \frac{1}{\pi} \int_{ - \pi}^\pi \hat{f}(t) \cos nt \; dt = 0 \quad \forall n \geq 1\\[4ex]
    b_n & = \frac{1}{\pi} \int_{ - \pi}^\pi \hat{f}(t) \sin nt \; dt = \frac{2}{\pi} \int_0^\pi \frac{\pi}{4} \sin nt \; dt \xlongequal[dv = \sin nt \Rightarrow v = - \frac{\cos nt}{n} ]{u = \frac{\pi}{4} \Rightarrow du = 0} \\[2ex]
    & = \frac{2}{\pi} \cdot \frac{\pi}{4} \left[ - \frac{\cos nt}{n} \right]_0^\pi = \frac{1}{2} \cdot \frac{1}{n} (1 - ( - 1)^n) = \begin{cases}
      \dfrac{1}{n} & \mbox{ si } n \mbox{ impar} \\[2ex]
      0 & \mbox{ si } n \mbox{ par}
    \end{cases}
  \end{align*}
  Por lo tanto, la serie de Fourier de \(\hat{f}\) es:
  \begin{align*}
    S_{\hat{f}}(x) & = \frac{0}{2} + \displaystyle \sum_{n = 1}^{\infty} \left(0 \cdot \cos nx + b_n \sin nx\right) = \displaystyle \sum_{k = 1}^{\infty} \frac{\sin (2k - 1)x}{2k - 1}
  \end{align*}
  Donde hemos sustituido \(n\) por \(2k - 1\) para considerar solo los términos impares. Ahora, por el teorema de Dirichlet, la serie de Fourier converge a \(\hat{f}(x)\) al ser \(\hat{f}\) de variación acotada y continua en \(( - \pi, \pi)\). Por lo tanto, en el abierto \((0, \pi)\) la función \(f(x)\) converge exactamente a su serie de Fourier:
  \begin{align*}
    f(x) = \frac{\pi}{4} = S_{\hat{f}}(x) = \displaystyle \sum_{k = 1}^{\infty} \frac{\sin (2k - 1)x}{2k - 1}
  \end{align*}
  Y por tanto:
  \begin{align*}
    \frac{\pi}{4} = \displaystyle \sum_{n = 1}^{\infty} \dfrac{\sin (2n - 1)x}{2n - 1} \quad \mbox{ con } 0 < x < \pi
  \end{align*}
\end{dem_box}
\newpage
\subsection{Ejercicio extra}
\textit{Usa la serie del ejercicio anterior para demostrar que:}
\begin{align*}
  \frac{\pi^2}{8} = \displaystyle \sum_{n = 1}^{\infty} \frac{1}{(2n - 1)^2}
\end{align*}
\begin{dem_box}{Demostración}
  Partimos de la serie del ejercicio anterior:
  \begin{align*}
    \frac{\pi}{4} = \displaystyle \sum_{n = 1}^{\infty} \dfrac{\sin (2n - 1)x}{2n - 1} \quad \mbox{ con } 0 < x < \pi
  \end{align*}
  Ahora, integramos ambos lados desde \(0\) hasta \(\frac{\pi}{2}\):
  \begin{align*}
    \int_0^{\frac{\pi}{2}} \frac{\pi}{4} \; dx & = \int_0^{\frac{\pi}{2}} \displaystyle \sum_{n = 1}^{\infty} \dfrac{\sin (2n - 1)x}{2n - 1} \; dx
  \end{align*}
  Podemos intercambiar la integral y la suma ya que la serie converge uniformemente en \([0, \frac{\pi}{2}]\). Por lo tanto:
  \begin{align*}
    \frac{\pi}{4} \cdot \frac{\pi}{2} & = \displaystyle \sum_{n = 1}^{\infty} \int_0^{\frac{\pi}{2}} \dfrac{\sin (2n - 1)x}{2n - 1} \; dx
  \end{align*}
  Calculamos la integral:
  \begin{align*}
    \int_0^{\frac{\pi}{2}} \dfrac{\sin (2n - 1)x}{2n - 1} \; dx & \xlongequal[dv = \sin (2n - 1)x \Rightarrow v = - \frac{\cos (2n - 1)x}{2n - 1} ]{u = \frac{1}{2n - 1} \Rightarrow du = 0} \\[2ex]
    & = \frac{1}{2n - 1} \left[ - \frac{\cos (2n - 1)x}{2n - 1} \right]_0^{\frac{\pi}{2}} = \frac{1}{(2n - 1)^2} \left(1 - \cos\left((2n - 1) \cdot \frac{\pi}{2}\right)\right)
  \end{align*}
  Podemos notar que:
  \begin{align*}
    \cos\left((2n - 1) \cdot \frac{\pi}{2}\right) = 0 \quad \forall n \in \mathbb{N}
  \end{align*}
  Por lo tanto:
  \begin{align*}
    \int_0^{\frac{\pi}{2}} \dfrac{\sin (2n - 1)x}{2n - 1} \; dx & = \frac{1}{(2n - 1)^2}
  \end{align*}
  Así, tenemos:
  \begin{align*}
    \frac{\pi^2}{8} & = \displaystyle \sum_{n = 1}^{\infty} \frac{1}{(2n - 1)^2}
  \end{align*}
\end{dem_box}
\newpage
\subsection{Ejercicio 5}
\textit{Demuestra que:}
\begin{align*}
  x = \frac{\pi}{2} - \frac{4}{\pi} \displaystyle \sum_{n = 1}^{\infty} \frac{\cos (2n - 1)x}{(2n - 1)^2} \quad \mbox{ con } 0 < x < \pi
\end{align*}
\begin{dem_box}{Demostración}
  En este caso, consideramos la función:
  \begin{align*}
    f(x) = x \quad \mbox{ con } 0 < x < \pi
  \end{align*}
  Y para extenderla con periodo \(2\pi\) definimos:
  \begin{align*}
    \hat{f}(x) = \begin{cases}
      x & \mbox{ si } 0 < x < \pi \\[2ex]
      - x & \mbox{ si } -\pi < x < 0
    \end{cases}
  \end{align*}
  Podemos observar que \(\hat{f}\) es una función par y de variación acotada en \([ - \pi, \pi]\). Por lo tanto, podemos calcular su serie de Fourier. Sabemos que la serie de Fourier de una función \(f\) con periodo \(2\pi\) viene dada por:
  \begin{align*}
    S_f(x) = \frac{a_0}{2} + \displaystyle \sum_{n = 1}^{\infty} (a_n \cos nx + b_n \sin nx)
  \end{align*}
  Calculamos los coeficientes:
  \begin{align*}
    a_0 & = \frac{1}{\pi} \int_{ - \pi}^\pi \hat{f}(t) \; dt = \frac{1}{\pi}\int_{ - \pi}^0 ( - t) \; dt + \frac{1}{\pi}\int_0^\pi t \; dt = \\[2ex]
    & = \frac{1}{\pi}\left[ - \frac{t^2}{2} \frac{1}{\pi}\right]_{ - \pi}^0 + \frac{1}{\pi}\left[ \frac{t^2}{2} \right]_0^\pi = \frac{1}{\pi}\frac{\pi^2}{2} + \frac{1}{\pi}\frac{\pi^2}{2} = \pi\\[4ex]
    a_n & = \frac{1}{\pi} \int_{ - \pi}^\pi \hat{f}(t) \cos nt \; dt = \frac{2}{\pi} \int_0^\pi t \cos nt \; dt \xlongequal[dv = \cos nt \Rightarrow v = \frac{\sin nt}{n} ]{u = t \Rightarrow du = dt} \\[2ex]
    & = \frac{2}{\pi} \left[ \frac{t \sin nt}{n} \right]_0^\pi - \frac{2}{\pi} \int_0^\pi \frac{\sin nt}{n} \; dt = 0 - \frac{2}{\pi} \cdot \frac{1}{n} \left[ - \frac{\cos nt}{n} \right]_0^\pi = \\[2ex]
    & = \frac{2}{\pi} \cdot \frac{1}{n^2} (1 - ( - 1)^n) = \begin{cases}
      \dfrac{4}{\pi n^2} & \mbox{ si } n \mbox{ impar} \\[2ex]
      0 & \mbox{ si } n \mbox{ par}
    \end{cases}\\[4ex]
    b_n & = \frac{1}{\pi} \int_{ - \pi}^\pi \hat{f}(t) \sin nt \; dt = 0 \quad \forall n \geq 1
  \end{align*}
  Por lo tanto, la serie de Fourier de \(\hat{f}\) es:
  \begin{align*}
    S_{\hat{f}}(x) & = \frac{\pi}{2} + \displaystyle \sum_{n = 1}^{\infty} \left(a_n \cos nx + 0 \cdot \sin nx\right) = \frac{\pi}{2} + \displaystyle \sum_{k = 1}^{\infty} \frac{4}{\pi (2k - 1)^2} \cos (2k - 1)x
  \end{align*}
  Donde hemos sustituido \(n\) por \(2k - 1\) para considerar solo los términos impares. Ahora, por el teorema de Dirichlet, la serie de Fourier converge a \(\hat{f}(x)\) al ser \(\hat{f}\) de variación acotada y continua en \(( - \pi, \pi)\). Por lo tanto, en el abierto \((0, \pi)\) la función \(f(x)\) converge exactamente a su serie de Fourier:
  \begin{align*}
    f(x) = x = S_{\hat{f}}(x) = \frac{\pi}{2} + \displaystyle \sum_{k = 1}^{\infty} \frac{4}{\pi (2k - 1)^2} \cos (2k - 1)x
  \end{align*}
  Y por tanto:
  \begin{align*}
    x = \frac{\pi}{2} - \frac{4}{\pi} \displaystyle \sum_{n = 1}^{\infty} \frac{\cos (2n - 1)x}{(2n - 1)^2} \quad \mbox{ con } 0 < x < \pi
  \end{align*}
\end{dem_box}
\newpage

\subsection{Ejercicio extra}
\textit{Usa la serie del ejercicio anterior para demostrar que:}
\begin{align*}
  \frac{\pi^2}{8} = \displaystyle \sum_{n = 1}^{\infty} \frac{1}{(2n - 1)^2}
\end{align*}
\begin{dem_box}{Demostración}
  Partimos de la serie del ejercicio anterior:
  \begin{align*}
    x = \frac{\pi}{2} - \frac{4}{\pi} \displaystyle \sum_{n = 1}^{\infty} \frac{\cos (2n - 1)x}{(2n - 1)^2} \quad \mbox{ con } 0 < x < \pi
  \end{align*}
  Ahora, evaluamos en \(x = 0\):
  \begin{align*}
    0 = \frac{\pi}{2} - \frac{4}{\pi} \displaystyle \sum_{n = 1}^{\infty} \frac{\cos (2n - 1) \cdot 0}{(2n - 1)^2}
  \end{align*}
  Por lo tanto:
  \begin{align*}
    \frac{4}{\pi} \displaystyle \sum_{n = 1}^{\infty} \frac{1}{(2n - 1)^2} = \frac{\pi}{2} \implies \displaystyle \sum_{n = 1}^{\infty} \frac{1}{(2n - 1)^2} = \frac{\pi^2}{8}
  \end{align*}
\end{dem_box}

\newpage
\section{Algunas soluciones de ejercicios propuestos}
\subsection{Teorema de Vitali}
\textit{Demuestre que existe un subconjunto de \(\mathbb{R}\) que no es medible en el sentido de Lebesgue.}\vspace{2ex}

\begin{dem_box}{Demostración}
  Consideremos el intervalo \(I = [0, 1] \subseteq \mathbb{R}\) y definimos sobre dicho intervalo la relación de equivalencia siguiente:
  \begin{align*}
    x \sim y \iff x - y \in \mathbb{Q}
  \end{align*}
  Es trivial comprobar que es relación de equivalencia, basta ver:
  \begin{itemize}
    \item \textit{Reflexiva:} \(x - x = 0 \in \mathbb{Q}\) para todo \(x \in I\).
    \item \textit{Simétrica:} Si \(x \sim y\), entonces \(x - y \in \mathbb{Q}\), por lo que:
    \begin{align*}
      y - x = - (x - y) \in \mathbb{Q} \implies y \sim x
    \end{align*}
    \item \textit{Transitiva:} Si \(x \sim y\) y \(y \sim z\), entonces \(x - y \in \mathbb{Q}\) y \(y - z \in \mathbb{Q}\), por lo que:
    \begin{align*}
      x - z = (x - y) + (y - z) \in \mathbb{Q} \implies x \sim z
    \end{align*}
  \end{itemize}
  Podemos notar además que esta relación de equivalencia particiona el intervalo \([0, 1]\) en clases de equivalencia disjuntas de la forma:
  \begin{align*}
    E_x = \{x + q : q \in \mathbb{Q}\} \cap [0, 1]
  \end{align*}
  Ahora, por el axioma de elección, definimos el conjunto \(V \subseteq [0, 1]\) que contiene exactamente un representante de cada clase de equivalencia, es decir:
  \begin{itemize}
    \item Para cada \(x \in [0, 1]\) existe \(v \in V\) tal que \(v \sim x\)
    \item Si \(v_1, v_2 \in V\) y \(v_1 \neq v_2\) entonces \(v_1 \not\sim v_2\)
  \end{itemize}
  Ahora, consideramos \(\mathcal{Q} = \mathbb{Q} \cap [ - 1, 1]\) el conjunto de los racionales entre \( - 1\) y \(1\) que es numerable, es decir, podemos escribir:
  \begin{align*}
    \mathcal{Q} = \{q_1, q_2, q_3, \ldots\}
  \end{align*}
  Ahora, para cada \(q_k \in \mathcal{Q}\) definimos el conjunto:
  \begin{align*}
    V_k = V + q_k = \{v + q_k : v \in V\}
  \end{align*}
  Podemos notar que los conjuntos \(V_k\) cumplen las siguientes propiedades:
  \begin{itemize}
    \item \textit{Disjuntos dos a dos}: supongamos \(z \in V_k \cap V_j\) entonces:
    \begin{align*}
      z = v_k + q_k = v_j + q_j \quad \mbox{ con } v_k, v_j \in V \implies v_k - v_j = q_j - q_k \in \mathbb{Q}
    \end{align*}
    Por lo que \(v_k \sim v_j\) y, por construcción de \(V\), \(v_k = v_j\) y por tanto \(q_k = q_j\) y \(k = j\) que es una contradicción.
    \item \textit{Contención:} Podemos ver que:
    \begin{align*}
      [0, 1] \subseteq \bigcup_{k = 1}^\infty V_k \subseteq [ - 1, 2]
    \end{align*}
    Sea \(x \in [0, 1]\) entonces existe \(v \in V\) tal que \(x - v = q \in \mathbb{Q}\). Como \(x, v \in [0, 1]\) entonces:
    \begin{align*}
      q = x - v \in [ - 1, 1] \implies q \in \mathcal{Q} \implies x = v + q \in V_k \mbox{ para algún } k
    \end{align*}
    Por otro lado, como \(V \subseteq [0, 1]\) y \(q_k \in [ - 1, 1]\) entonces:
    \begin{align*}
      V_k = V + q_k \subseteq [ - 1, 2]
    \end{align*}
  \end{itemize}
  Ahora, supongamos que \(V\) es medible en el sentido de Lebesgue. Como dicha medida es invariante por traslaciones, tenemos que:
  \begin{align*}
    \mu(V_k) = \mu(V + q_k) = \mu(V) \quad \forall k \in \mathbb{N}
  \end{align*}
  Entonces:
  \begin{align*} \label{eq:vitali} \tag{I}
    \mu\left(\bigcup_{k = 1}^\infty V_k\right) & = \sum_{k = 1}^\infty \mu(V_k) = \sum_{k = 1}^\infty \mu(V)
  \end{align*}
  Por otro lado, por monotonía de la medida de Lebesgue, tenemos que:
  \begin{align*}
    \mu([0, 1]) \leq \mu\left(\bigcup_{k = 1}^\infty V_k\right) \leq \mu([ - 1, 2])
  \end{align*}
  Entonces, juntando esto con la ecuación \eqref{eq:vitali}, tenemos:
  \begin{align*}
    1 \leq \sum_{k = 1}^\infty \mu(V) \leq 3
  \end{align*}
  Ahora, hay dos casos posibles de acuerdo al valor de \(\mu(V)\):
  \begin{itemize}
    \item Si \(\mu(V) = 0\), entonces la suma infinita es 0, lo cual es una contradicción:
    \begin{align*}
      1 \leq 0 \leq 3
    \end{align*}
    \item Si \(\mu(V) > 0\), lo cual es una contradicción:
    \begin{align*}
      1 \leq \infty \leq 3
    \end{align*}
  \end{itemize}
\end{dem_box}

\newpage
\section{Algunas soluciones de exámenes}
\subsection{Ejercicio}
\textit{Calcula el volumen del conjunto:}
\begin{align*}
  V = \left\{(x, y, z) \in \mathbb{R}^3 : x^2 + 4y^2 + 9z^2 \leq 1\right\}
\end{align*}
\textit{justificando todos los detalles.}
\begin{dem_box}{Demostración}
  La idea sería llevar estos coeficientes a \(1\) mediante un cambio de variable. Consideremos el siguiente cambio de variable:
  \begin{align*}
    \begin{cases}
      x = u\\[2ex]
      y = \dfrac{v}{2}\\[2ex]
      z = \dfrac{w}{3}
    \end{cases} \quad \Longleftrightarrow \quad \begin{cases}
      u = x\\[2ex]
      v = 2y\\[2ex]
      w = 3z
    \end{cases}
  \end{align*}
  Entonces, tenemos que la expresión del conjunto \(V\) en las nuevas variables es:
  \begin{align*}
    x^2 + 4y^2 + 9z^2 = u^2 + 4 \cdot \left(\frac{v}{2}\right)^2 + 9 \cdot \left(\frac{w}{3}\right)^2 = u^2 + v^2 + w^2 \leq 1
  \end{align*}
  Así, el conjunto \(V\) en las nuevas variables es:
  \begin{align*}
    B = \left\{(u, v, w) \in \mathbb{R}^3 : u^2 + v^2 + w^2 \leq 1\right\}
  \end{align*}
  Calculamos el Jacobiano del cambio de variable:
  \begin{align*}
    |J| = \begin{vmatrix}
      \dfrac{\partial x}{\partial u} & \dfrac{\partial x}{\partial v} & \dfrac{\partial x}{\partial w}\\[4ex]
      \dfrac{\partial y}{\partial u} & \dfrac{\partial y}{\partial v} & \dfrac{\partial y}{\partial w}\\[4ex]
      \dfrac{\partial z}{\partial u} & \dfrac{\partial z}{\partial v} & \dfrac{\partial z}{\partial w}
    \end{vmatrix} = \begin{vmatrix}
      1 & 0 & 0\\[4ex]
      0 & \dfrac{1}{2} & 0\\[4ex]
      0 & 0 & \dfrac{1}{3}
    \end{vmatrix} = \frac{1}{6}
  \end{align*}
  Ahora, tenemos que:
  \begin{align*}
    dx \; dy \; dz = |J| \; du \; dv \; dw = \frac{1}{6} \; du \; dv \; dw
  \end{align*}
  Por tanto, el volumen del conjunto \(V\) es:
  \begin{align*}
    \iiint_V 1 \; dx \; dy \; dz & = \iiint_B 1 \cdot \frac{1}{6} \; du \; dv \; dw = \frac{1}{6} \iiint_B 1 \; du \; dv \; dw
  \end{align*}
  El conjunto \(B\) es la bola unidad en \(\mathbb{R}^3\) cuyo volumen es conocido y vale \(\frac{4}{3} \pi\). No obstante, podemos calcularlo mediante coordenadas esféricas:
  \begin{align*}
    \iiint_B 1 \; du \; dv \; dw & \xlongequal[{\rho \in [0, 1], \theta \in [0, 2\pi], \phi \in [0, \pi]}]{u = \rho \sin \phi \cos \theta, v = \rho \sin \phi \sin \theta, w = \rho \cos \phi} \\[2ex]
    & = \int_0^{2\pi} \int_0^\pi \int_0^1 \rho^2 \sin \phi \; d\rho \; d\phi \; d\theta = \\[2ex]
    & = \int_0^{2\pi} d\theta \int_0^\pi \sin \phi \; d\phi \int_0^1 \rho^2 \; d\rho = \\[2ex]
    & = (2\pi) (2) \left[\frac{\rho^3}{3}\right]_0^1 = \frac{4\pi}{3}
  \end{align*}
  Por lo tanto, el volumen del conjunto \(V\) es:
  \begin{align*}
    \iiint_V 1 \; dx \; dy \; dz & = \frac{1}{6} \cdot \frac{4\pi}{3} = \frac{2\pi}{9}
  \end{align*}
\end{dem_box}
\newpage

\subsection{Ejercicio}
\textit{Demuestra que el conjunto siguiente es un subconjunto medible Lebesgue en \(\mathbb{R}^3\) y que su medida es 0:}
\begin{align*}
  A = \left\{\left(x, y, \frac{1}{x^2 + y}\right) : (x, y) \in \mathbb{R}^2 \setminus \{0, 0\}\right\}
\end{align*}
\begin{dem_box}{Demostración}
  Consideremos la función:
  \begin{align*}
    f: \mathbb{R}^2 \setminus \{0, 0\} & \longrightarrow \mathbb{R}^3\\[2ex]
    (x, y) & \longmapsto \left(x, y, \frac{1}{x^2 + y}\right)
  \end{align*}
  Podemos notar que es continua en su dominio que es un abierto de \(\mathbb{R}^2\), por lo tanto, es medible en el sentido de Lebesgue. Ahora, definimos la sección de \(A\) para cada \((x, y) \in \mathbb{R}^2 \setminus \{0, 0\}\):
  \begin{align*}
    A_{(x, y)} = \left\{z \in \mathbb{R} : (x, y, z) \in A\right\} = \begin{cases} 
      \left\{\frac{1}{x^2 + y}\right\} & \mbox{ si } (x, y) \in \mathbb{R}^2 \setminus \{0, 0\}\\[2ex]
      \emptyset & \mbox{ si } (x, y) = (0, 0)
    \end{cases} 
  \end{align*}
  Por lo tanto, la medida de Lebesgue de cada sección es:
  \begin{align*}
    \mu_1(A_{(x, y)}) = 0 \quad \forall (x, y) \in \mathbb{R}^2
  \end{align*}
  Ahora, por el teorema de Fubini-Tonelli, tenemos que:
  \begin{align*}
    \mu_3(A) & = \iint_{\mathbb{R}^2} \mu_1(A_{(x, y)}) \; d(x, y) = \iint_{\mathbb{R}^2} 0 \; d(x, y) = 0
  \end{align*}
\end{dem_box}
\newpage

\subsection{Ejercicio}
\textit{Dadas \(0 < a < b < c < \infty\) demuestra que:}
\begin{align*}
  E = \left\{(x, y, z) \in \mathbb{R}^3: \frac{x^2}{a^2} + \frac{y^2}{b^2} + \frac{z^2}{c^2} = 1\right\} 
\end{align*}
\textit{es un conjunto medible en el sentido de Lebesgue y su medida es cero.}
\begin{dem_box}{Demostración}
  Definimos la función:
  \begin{align*}
    \varphi : \mathbb{R}^3 & \longrightarrow \mathbb{R}\\[2ex]
    (x, y, z) & \longmapsto \frac{x^2}{a^2} + \frac{y^2}{b^2} + \frac{z^2}{c^2}
  \end{align*}
  Podemos notar que es continua en \(\mathbb{R}^3\) y además:
  \begin{align*}
    E = \varphi^{-1}(\{1\})
  \end{align*}
  Por lo tanto, \(E\) es medible en el sentido de Lebesgue al ser el conjunto \(\{1\}\) medible en \(\mathbb{R}\) y la función \(\varphi\) medible.\vspace{2ex}

  Por otra parte, si consideramos el cambio de coordenadas dado por:
  \begin{align*}
    T : \mathbb{R}^3 & \longrightarrow \mathbb{R}^3\\[2ex]
    (u, v, w) & \longmapsto (au, bv, cw)
  \end{align*}
  Entonces, tenemos que:
  \begin{align*}
    T(S^2) = E \quad \mbox{ con } S^2 = \left\{(u, v, w) : u^2 + v^2 + w^2 = 1 \right\}
  \end{align*}
  Podemos expresar \(S^2\) como:
  \begin{align*}
    S^2 = \left\{(x, y, \sqrt{1 - x^2 - y^2}): x^2 + y^2 \leq 1\right\}
  \end{align*}
  Así, tenemos que si consideramos las secciones de \(E\):
  \begin{align*}
    E_{(x, y)} = \left\{z \in \mathbb{R} : (x, y, z) \in E\right\} = \begin{cases}
      \left\{\sqrt{c^2\left(1 - \frac{x^2}{a^2} - \frac{y^2}{b^2}\right)}, -\sqrt{c^2\left(1 - \frac{x^2}{a^2} - \frac{y^2}{b^2}\right)}\right\} & \mbox{ si } \frac{x^2}{a^2} + \frac{y^2}{b^2} \leq 1\\[4ex]
      \emptyset & \mbox{ en otro caso}
    \end{cases}
  \end{align*}
  Por lo tanto, la medida de Lebesgue de cada sección es:
  \begin{align*}
    \mu_1(E_{(x, y)}) = 0 \quad \forall (x, y) \in \mathbb{R}^2
  \end{align*}
  Ahora, por el teorema de Fubini-Tonelli, tenemos que:
  \begin{align*}
    \mu_3(E) & = \iint_{\mathbb{R}^2} \mu_1(E_{(x, y)}) \; d(x, y) = \iint_{\mathbb{R}^2} 0 \; d(x, y) = 0
  \end{align*}
\end{dem_box}
\newpage

\subsection{Ejercicio}
\textit{Demuestra que para todo \(A \in \mathcal{M}_1\) con \(\mu_1(A) > 0\) existe un subconjunto \(B \subseteq A\) que no es medible en el sentido de Lebesgue.}
\begin{dem_box}{Demostración}
  Se considera el conjunto de Vitali en el intervalo \([0, 1]\) construido como:
  \begin{itemize}
    \item Definimos la relación de equivalencia siguiente:
    \begin{align*}
      x \sim y \iff x - y \in \mathbb{Q}
    \end{align*}
    \item Esta relación particiona \(\mathbb{R}\) en clases de equivalencia disjuntas
    \item Usando el axioma de elección, definimos el conjunto \(V \subseteq [0, 1]\) que contiene exactamente un representante de cada clase de equivalencia
  \end{itemize}
  Ahora, si \(̣̣̣\{r_k\}_{k = 1}^\infty\) es una enumeración de los racionales en \([ - 1, 1]\), consideramos los conjuntos:
  \begin{align*}
    V_k = V + r_k = \{v + r_k : v \in V\}
  \end{align*}
  Podemos notar que los conjuntos \(V_k\) cumplen las siguientes propiedades:
  \begin{itemize}
    \item \textit{Disjuntos dos a dos}: supongamos \(z \in V_k \cap V_j\) entonces:
    \begin{align*}
      z = v_k + r_k = v_j + r_j \quad \mbox{ con } v_k, v_j \in V \implies v_k - v_j = r_j - r_k \in \mathbb{Q}
    \end{align*}
    Por lo que \(v_k \sim v_j\) y, por construcción de \(V\), \(v_k = v_j\) y por tanto \(r_k = r_j\) y \(k = j\) que es una contradicción.
    \item \textit{Contención:} Podemos ver que:
    \begin{align*}
      [0, 1] \subseteq \bigcup_{k = 1}^\infty V_k \subseteq [ - 1, 2]
    \end{align*}
    Sea \(x \in [0, 1]\) entonces existe \(v \in V\) tal que \(x - v = q \in \mathbb{Q}\). Como \(x, v \in [0, 1]\) entonces:
    \begin{align*}
      q = x - v \in [ - 1, 1] \implies q \in \{r_k\}_{k = 1}^\infty \implies x = v + q \in V_k \mbox{ para algún } k
    \end{align*}
    Por otro lado, como \(V \subseteq [0, 1]\) y \(r_k \in [ - 1, 1]\) entonces:
    \begin{align*}
      V_k = V + r_k \subseteq [ - 1, 2]
    \end{align*}
  \end{itemize}
  Ahora, supongamos que todo subconjunto de \(A\) es medible en el sentido de Lebesgue. Sea \(V\) el conjunto de Vitali definido anteriormente, para cada \(q \in \mathbb{Q}\) consideramos:
  \begin{align*}
    A_q = A \cap (V + q)
  \end{align*}
  Entonces, \(A_q\) debe de ser medible ya que es subconjunto de \(A\). Además, los conjuntos \(A_q\) son disjuntos dos a dos ya que los conjuntos \(V + q\) lo son. Además, como \(V\) contiene un representante de cada clase de equivalencia de \(\mathbb{R}\) entonces cualquier real se puede escribir como \(v + q\) con \(v \in V\) y \(q \in \mathbb{Q}\). Por lo tanto:
  \begin{align*}
    A = \displaystyle \bigcup_{q \in \mathbb{Q}} (A \cap (V + q)) = \displaystyle \bigcup_{q \in \mathbb{Q}} A_q
  \end{align*}
  Como la medida de Lebesgue es invariante por traslaciones, sea \(B_q = A_q - q\) notemos que:
  \begin{align*}
    B \subseteq V \implies \mu_1(B) \leq \mu_1(V)
  \end{align*}
  Como todos los \(A_q\) son medibles, entonces:
  \begin{align*}
    \mu_1(A_q) > 0 \implies \mu_1(V \cap (A_q - q)) = \mu_1(B_q) > 0
  \end{align*}
  Pero si el conjunto \(V\) teiene medida positiva, llegamos a una contradicción ya que el conjunto de Vitali no es medible en el sentido de Lebesgue. Por lo tanto, existe al menos un subconjunto \(B \subseteq A\) que no es medible en el sentido de Lebesgue.

\end{dem_box}
\end{document}