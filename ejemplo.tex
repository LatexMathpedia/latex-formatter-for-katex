\documentclass[11pt,a4paper]{article}

\usepackage[spanish]{babel}
\usepackage{amsmath,amsfonts, amssymb, mathtools, bbm} % Podemos añadir amssymb, amsthm o bm
\usepackage{graphicx, tikz, xparse}
\usepackage[top=2cm,bottom=2cm,left=3cm,right=3cm,marginparwidth=1.75cm]{geometry} % Este paquete permite modificar los márgenes del documento
\usepackage[colorlinks=true, allcolors=blue]{hyperref} % Se indica que los hipervínculos van todos en azul
\usepackage{setspace}
\usepackage{xcolor, tcolorbox}
\usepackage{cancel} %tachar cosas
\tcbuselibrary{breakable}
\usepackage{hyperref}
\usepackage{titlesec}
\usepackage{cancel}
\usepackage{pgfplots}  % Para graficar funciones en tikz
\usepackage{background}
\usetikzlibrary{arrows.meta}
\usepackage[bottom]{footmisc}

% Definir la marca de agua
\backgroundsetup{
  position=current page.west,
  angle=90,
  nodeanchor=west,
  vshift=-1cm,
  hshift=-5.5cm,
  color=gray,
  scale=1,
  contents={\textsf{Created by Diego Díaz Mendaña $|$ Licensed under CC BY-NC-SA 4.0}}
}


\graphicspath{ {images/}}

%Colores
\definecolor{blanco}{HTML}{FFFFFF}
\definecolor{negro}{HTML}{000000}
\definecolor{azulSuave}{HTML}{6ac9d5}
\definecolor{naranjaSuave}{HTML}{d5956a}
\definecolor{verdeSuave}{HTML}{6ad578}
\definecolor{magenta}{HTML}{FF00FF}
\definecolor{dorado}{HTML}{ad8a1f}

\newtcolorbox{dem_box}[1]{
before=\par\smallskip\centering,
colframe=azulSuave!70,
colback=white,
fonttitle=\bfseries,
coltitle=negro,
title=#1,
flushleft title,
width=1\linewidth,
breakable = true
}

\newtcolorbox{ejem_box}[1]{
before=\par\smallskip\centering,
colframe=verdeSuave!70,
colback=white,
fonttitle=\bfseries,
coltitle=negro,
title=#1,
flushleft title,
width=1\linewidth,
breakable = true
}

\newtcolorbox{ej_box}[1]{
before=\par\smallskip\centering,
colframe=naranjaSuave!70,
colback=white,
fonttitle=\bfseries,
coltitle=negro,
title=#1,
flushleft title,
width=1\linewidth,
breakable = true
}


\setstretch{1.2}
\decimalpoint

\title{\textbf{TEMA 1}: Preliminares}
\author{Diego Díaz Mendaña}
%\date{Fecha}

\begin{document}
\NoBgThispage 
\maketitle
\vspace{20ex}
\section*{Disclaimer}
Estos apuntes son un resumen basado en el material proporcionado en la asignatura de ``Análisis III''\footnote{Posiblemente la única asignatura de análisis que tenga bien puesto el nombre} de la Universidad de Oviedo. Han sido escritos mediante las definiciones y demostraciones explicadas en clase\footnote{Obviamente, no en las TGs ya que no hay porque ``nos fastidian la digestión''}. Todo el contenido ha sido organizado y formulado con fines educativos y no comerciales.\footnote{Recomendamos seguir estos apuntes ya que, según palabras textuales: ``Los del año pasado son malísimos, es más, no había errores sino horrores.''}\vspace{2ex}


\vspace{12ex}
\section*{Licencia de uso}
Apuntes Análisis III - Tema 1 © 2025 by Diego Díaz Mendaña is licensed under CC BY-NC-SA 4.0. To view a copy of this license, visit \href{https://creativecommons.org/licenses/by-nc-sa/4.0/}{https://creativecommons.org/licenses/by-nc-sa/4.0/}

\newpage

\hypersetup{linkcolor=black}
\tableofcontents
\hypersetup{linkcolor=blue}
\newpage

\section*{Notación y Preliminares}
\addcontentsline{toc}{section}{Notación y Preliminares}
Durante el curso, se usarán preferiblemente las siguientes notaciones durante las demostraciones y ejercicios\footnote{Porque aquí la mayoría no seréis Einstein sino obreros de las matemáticas. Que aquí no es como con la princesa, que llega allí y la suben en el avión directamente. Vosotros primero tenéis que limpiar el avión y apretar tornillos.}:
\begin{itemize}
  \item En clase se empleará la abreviatura \textit{c.} para referirse a \textit{conjuntos}.
  \item Se empleará preferiblemente la \(N\) para referirse al natural en \(R^N\)
  \item En análisis en general consideraremos que los naturales empiezan en \(1\), aunque en álgebra se consideren desde \(0\).\footnote{Aunque discutir esto es ``bizantino''} Por lo tanto, emplearemos la notación \(N \cup \{0\}\) cuando queramos incluir el \(0\) en los naturales.
\end{itemize}

\newpage
\section{Cardinalidad de un conjunto}
\subsection{Cardinalidad. Idea intuitiva}
Podemos entender la \textbf{cardinalidad} de un conjunto \(A\) como la cantidad de elementos de \(A\). Además, dentro de los conjuntos, podemos distinguir entre dos tipos:
\begin{itemize}
  \item \textbf{Conjuntos finitos:} Aquellos conjuntos que tienen un número finito de elementos, es decir, \(\exists n \in\mathbb{N} \cup \{0\}\) tal que \(\exists \) biyección entre \(A\) y \(\{1, 2, \dots, n\} \), esto es:
  \begin{center}
    \begin{tabular}{rcl}
    \(\exists \varphi : \{1, 2, \dots, n\} \) & \( \longrightarrow\) & \(A \neq \emptyset \) \\
    \(i\) & \(\longmapsto\) & \(\varphi(i)\)
  \end{tabular}
  \end{center}

  \item \textbf{Conjuntos infinitos:} Aquellos conjuntos para los cuales existe una aplicación inyectiva \(\varphi : \mathbb{N} \longrightarrow A\). De esta forma, podemos distinguir entre:
  \begin{itemize}
    \item \textbf{Conjuntos infinitos numerables:} Aquellos conjuntos \(A\) para los cuales existe una biyección \(\varphi : \mathbb{N} \longrightarrow A\).
    \item \textbf{Conjuntos infinitos no numerables:} Aquellos conjuntos \(A\) para los cuales existe una aplicación inyectiva \(\varphi : \mathbb{N} \longrightarrow A\) pero no existe ninguna biyección entre ambos conjuntos.
  \end{itemize}
\end{itemize}
\vspace{3ex}

\subsection{Conjuntos numerables y no numerables. Ejemplos}
\noindent Podemos plantear algunos ejemplos de conjuntos numerables y no numerables comunes:

\subsubsection*{Conjuntos numerables}
\noindent Los siguientes conjuntos son numerables:
\begin{enumerate}
  \item El conjunto de los naturales \(\mathbb{N}\).
  \begin{dem_box}{Demostración}
    Trivial, basta considerar la aplicación identidad:
    \begin{align*}
      \varphi : \mathbb{N} & \longrightarrow \mathbb{N} \\
      n & \longmapsto \varphi(n) = n
    \end{align*}
  \end{dem_box}
  \vspace{2ex}

  \item El grupo de los enteros \(\mathbb{Z} = \left\{ - \infty, \dots, - 2, - 1, 0, 1, 2, \dots , \infty\right\}\).
  \begin{dem_box}{Demostración}
    Podemos emplear la siguiente idea para definir una biyección entre \(\mathbb{N}\) y \(\mathbb{Z}\):
    \begin{center}
      \begin{tabular}{c|cccccccc}
        \(\mathbb{N}\) & 1 & 2 & 3 & 4 & 5 & 6 & 7 & \(\dots\) \\
        & \(\downarrow\) & \(\downarrow\) & \(\downarrow\) & \(\downarrow\) & \(\downarrow\) & \(\downarrow\) & \(\downarrow\) & \\
        \(\mathbb{Z}\) & 0 & 1 & -1 & 2 & -2 & 3 & -3 & \(\dots\)
      \end{tabular}\vspace{2ex}
    \end{center}
    De esta forma, podemos definir la aplicación:
    \begin{align*}
      \varphi : \mathbb{N} & \longrightarrow \mathbb{Z} \\
      n & \longmapsto \varphi(n) = \begin{cases}
        \frac{n}{2}, & n \mbox{ par} \\[1ex]
        -\frac{n-1}{2}, & n \mbox{ impar}
      \end{cases}
    \end{align*}
    Y así, tenemos una biyección entre ambos conjuntos.
  \end{dem_box}
  \vspace{2ex}
  
  \item El conjunto \(\mathbb{Q} \coloneq \left\{\dfrac{z}{n} : z \in \mathbb{Z}, n \in \mathbb{N}\right\}\) \vspace{1ex}
  \begin{dem_box}{Demostración}
    Como tenemos que:
    \begin{align}\label{eq:cardinalidadQ}
      \mbox{Card } \mathbb{Z} = \mbox{Card } \mathbb{N} \implies  \mbox{Card } \mathbb{Q} = \mbox{Card } \left\{\frac{z}{n} : z \in \mathbb{N}, n \in \mathbb{N}\right\}
      \tag{I}
    \end{align}
    Para construir una biyección entre \(\mathbb{N}\) y \(\left\{\frac{z}{n} : z, n \in \mathbb{N}\right\}\) podemos ver que si disponemos los naturales en filas y columnas, obtenemos la siguiente tabla expresando los cocientes:
    \begin{center}
        \begin{tabular}{c|ccccccc}
          & 1 & 2 & 3 & 4 & 5 & 6 & \(\dots\) \\
          \hline
          1 & \(\frac{1}{1}\) & \(\frac{1}{2}\) & \(\frac{1}{3}\) & \(\frac{1}{4}\) & \(\frac{1}{5}\) & \(\frac{1}{6}\) & \(\dots\) \\[1ex]
          2 & \(\frac{2}{1}\) & \(\frac{2}{2}\) & \(\frac{2}{3}\) & \(\frac{2}{4}\) & \(\frac{2}{5}\) & \(\frac{2}{6}\) & \(\dots\) \\[1ex]
          3 & \(\frac{3}{1}\) & \(\frac{3}{2}\) & \(\frac{3}{3}\) & \(\frac{3}{4}\) & \(\frac{3}{5}\) & \(\frac{3}{6}\) & \(\dots\) \\[1ex]
          4 & \(\frac{4}{1}\) & \(\frac{4}{2}\) & \(\frac{4}{3}\) & \(\frac{4}{4}\) & \(\frac{4}{5}\) & \(\frac{4}{6}\) & \(\dots\) \\[1ex]
          5 & \(\frac{5}{1}\) & \(\frac{5}{2}\) & \(\frac{5}{3}\) & \(\frac{5}{4}\) & \(\frac{5}{5}\) & \(\frac{5}{6}\) & \(\dots\) \\[1ex]
          6 & \(\frac{6}{1}\) & \(\frac{6}{2}\) & \(\frac{6}{3}\) & \(\frac{6}{4}\) & \(\frac{6}{5}\) & \(\frac{6}{6}\) & \(\dots\) \\[1ex]
          \(\vdots\) & \(\vdots\) & \(\vdots\) & \(\vdots\) & \(\vdots\) & \(\vdots\) & \(\vdots\) & \(\ddots\)
        \end{tabular}
      \end{center}
      Y recorriendo la tabla en diagonal tenemos la siguiente sucesión:
      \begin{align*}
        \frac{1}{1}, \, \frac{2}{1}, \, \frac{1}{2}, \, \frac{3}{1}, \, \frac{2}{2}, \, \frac{1}{3}, \, \frac{4}{1}, \, \frac{3}{2}, \, \frac{2}{3}, \, \frac{1}{4}, \, \frac{5}{1}, \, \frac{4}{2}, \, \frac{3}{3}, \, \frac{2}{4}, \, \frac{1}{5}, \, \frac{6}{1}, \, \dots
      \end{align*}
      Como podemos notar, en esta sucesión aparecen algunos números repetidos (por ejemplo, \(\frac{2}{2}\) y \(\frac{1}{1}\)). Pero podemos eliminar estos números de la sucesión sin que deje de ser numerable. Finalmente, a cada número de esta sucesión le asignamos un natural, es decir:
      \begin{center}
        \begin{tabular}{c|cccccccccc}
          \(\mathbb{N}\) & 1 & 2 & 3 & 4 & 5 & 6 & 7 & 8 & 9 & \(\dots\) \\
          & \(\downarrow\) & \(\downarrow\) & \(\downarrow\) & \(\downarrow\) & \(\downarrow\) & \(\downarrow\) & \(\downarrow\) & \(\downarrow\) & \(\downarrow\) & \\
          \(\left\{\frac{z}{n} \colon z, n \in \mathbb{N}\right\}\) & \(\frac{1}{1}\) & \(\frac{2}{1}\) & \(\frac{1}{2}\) & \(\frac{3}{1}\) & \(\frac{1}{3}\) & \(\frac{4}{1}\) & \(\frac{3}{2}\) & \(\frac{2}{3}\) & \(\frac{1}{4}\) & \(\dots\)
        \end{tabular}
      \end{center}
      \vspace{10ex}

      Así, podemos definir una aplicación \(\varphi : \mathbb{N} \longrightarrow \left\{\frac{z}{n} : z, n \in \mathbb{N}\right\}\) suprayectiva y, como sabemos que:
      \begin{align*}
        \mbox{Card } \left\{\frac{z}{n} : z, n \in \mathbb{N}\right\} \leq \mbox{Card } \mathbb{N} \quad \mbox{y} \quad \mathbb{N} \subseteq \left\{\frac{z}{n} : z, n \in \mathbb{N}\right\}
      \end{align*}
      Tenemos que:
      \begin{align*}
        \mbox{Card } \mathbb{N} \leq \mbox{Card } \left\{\frac{z}{n} : z, n \in \mathbb{N}\right\} \leq \mbox{Card } \mathbb{N}
      \end{align*}
      Por lo tanto, por \eqref{eq:cardinalidadQ} concluimos que:
      \begin{align*}
        \mbox{Card } \left\{\frac{z}{n} : z, n \in \mathbb{N}\right\} = \mbox{Card } \mathbb{N} = \mbox{Card } \mathbb{Q}
      \end{align*}
  \end{dem_box}
\end{enumerate}

\subsubsection*{Conjuntos no numerables}
\noindent Los siguientes conjuntos no son numerables:
\begin{enumerate}
  \item El conjunto \((0, 1)\)
  \begin{dem_box}{Demostración}
    Procederemos por reducción al absurdo. Supongamos que \((0, 1)\) sí es numerable y, por tanto, existe una biyección \(\varphi \colon \mathbb{N} \longrightarrow(0, 1)\). De esta manera, podríamos hacer una lista infinita que contendría todos los elementos de \((0, 1)\), es decir, \(\varphi(\mathbb{N}) = (0, 1)\).\vspace{2ex}

    Ahora, podríamos escribir todos los números del intervalo \((0, 1)\) como \(\varphi(n)\) con \(n \in \mathbb{N}\) de la siguiente forma:
    \begin{align*}
      \varphi(1) & = 0.\textbf{7}00681003\dots \\
      \varphi(2) & = 0.4\textbf{0}0100200\dots \\
      \varphi(3) & = 0.90\textbf{0}100300\dots \\
      \varphi(4) & = 0.100\textbf{1}00400\dots \\
      \vdots \quad & \hspace{10ex} \vdots
    \end{align*}
    Ahora queremos construir un nuevo número \(b \in (0, 1)\). Para ello, nos fijamos en los decimales remarcados de la lista anterior y hacemos lo siguiente:
    \begin{align*}
      b_n = \left(n\text{-ésimo decimal de } \varphi(n) + 1\right) \mod 10
    \end{align*}
    De esta forma en nuestra sucesión \(b\) tenemos:
    \begin{align*}
      b & = 0.b_1 b_2 b_3 b_4 \dots = 0.8112\dots
    \end{align*}
    Y por construcción, \(b \in (0, 1)\) y \(b \neq \varphi(n) \; \forall n \in \mathbb{N}\) ya que difiere en al menos un decimal de cada uno de ellos. Por lo tanto, hemos llegado a una contradicción y concluimos que \((0, 1)\) no es numerable.
  \end{dem_box}
  \vspace{2ex}

  \item El conjunto de los reales \(\mathbb{R}\)
  \begin{dem_box}{Demostración}
    Para esta demostración emplearemos el Teorema de los intervalos encajados de Cantor, que dice que si tenemos una sucesión infinitas de intervalos cerrados y acotados \([a_n, b_n]\) tal que \(a_n < b_n\) donde cada intervalo está contenido en el anterior, es decir:
    \begin{align*}
      [a_1, b_1] \supseteq [a_2, b_2] \supseteq [a_3, b_3] \supseteq \dots \implies \bigcap_{n \in \mathbb{N}} [a_n, b_n] \neq \emptyset
    \end{align*}
    Entonces, procederemos por reducción al absurdo. Supongamos que \(\mathbb{R}\) es numerable y, por tanto, existe una biyección \(\varphi \colon \mathbb{N} \longrightarrow \mathbb{R}\). \vspace{2ex}

    Queremos una sucesión de intervalos encajados que excluyan sistemáticamente a cada número de nuestra lista \(\varphi(\mathbb{N}) = \mathbb{R}\):
    \begin{enumerate}
      \item[1)] Como \(\varphi(1)\) es un punto en la recta real, podemos tomar fácilmente un intervalo que no lo contenga, es decir, \([a_1, b_1]\) tal que:
      \begin{align*}
        \varphi(1) \not\in [a_1, b_1] \; \mbox{ con } a_1 < b_1
      \end{align*}
      \item[2)] Para excluir a \(\varphi(2)\) tomamos otro intervalo \([a_2, b_2]\) tal que:
      \begin{align*}
        \varphi(2) \not\in [a_2, b_2] \subseteq [a_1, b_1] \; \mbox{ con } a_2 < b_2
      \end{align*}
      que es posible ya que \([a_1, b_1]\) es un intervalo que podemos dividir en dos subintervalos y elegir aquel que no contenga a \(\varphi(2)\).
      \item[3)] De forma análoga, para \(n > 2\) tomamos \([a_n, b_n]\) tal que:
      \begin{align*}
        \varphi(n) \not\in [a_n, b_n] \subseteq [a_{n-1}, b_{n-1}] \; \mbox{ con } a_n < b_n
      \end{align*}
    \end{enumerate}

    Así, hemos construido una sucesión de intervalos encajados \([a_n, b_n]\) que además excluyen a cada número de la lista, es decir, \(\forall n \in \mathbb{N}\) tenemos \(\varphi(n) \not\in [a_n, b_n]\).\vspace{2ex}

    Ahora, aplicando el Teorema de los intervalos encajados sabemos que existe al menos un número real \(x\) tal que:
    \begin{align*}
      x \in \bigcap_{n \in \mathbb{N}} [a_n, b_n]
    \end{align*}
    Pero entonces debería existir \(k \in \mathbb{N}\) tal que \(x = \varphi(k)\). Sin embargo, por construcción nos habíamos asegurado de que:
    \begin{align*}
      \varphi(k) \not\in [a_k, b_k] \implies x \not\in [a_k, b_k] \implies x \not\in \bigcap_{n \in \mathbb{N}} [a_n, b_n] \quad \#
    \end{align*}
  \end{dem_box}
\end{enumerate}
\vspace{3ex}

\begin{ej_box}{Observación}
  Podemos ver las siguientes igualdades:
  \begin{align*}
    \mbox{Card } \mathbb{R} = \mbox{Card } \left( -\frac{\pi}{2} , \frac{\pi}{2} \right)
  \end{align*}
  ya que existe una biyección entre ambos conjuntos dada por la función arctan:
  \begin{align*}
    \arctan : \mathbb{R} \longrightarrow \left( - \frac{\pi}{2}, \frac{\pi}{2}   \right)
  \end{align*}
  Y también tenemos que:
  \begin{align*}
    \mbox{Card } \mathbb{C} = \mbox{Card } \mathbb{R}
  \end{align*}
\end{ej_box}
\vspace{3ex}

\subsection{Procesos que dan lugar a conjuntos numerables}
\begin{enumerate}
  \item Sea \(A\) finito y \(B\) infinito numerable tal que \(A \cap B = \emptyset \)\footnote{realmente esto no es necesario, simplemente ``facilita la vida''} entonces:
  \begin{align*}
    \left\{
      \begin{array}{l}    
        A \cup B \mbox{ es infinito numerable}\\
        A \times B \mbox{ es infinito numerable o } \emptyset \mbox{ si } A = \emptyset 
      \end{array}
    \right.
  \end{align*}
  \begin{dem_box}{Demostración}
    Se plantean dos casos:
    \begin{enumerate}
      \item Si \(A = \emptyset \) entonces:
      \begin{align*}
        \left\{
          \begin{array}{l}
            A \cup B = B \mbox{ es infinito numerable}\\
            A \times B = \emptyset \times B = \emptyset
          \end{array}
        \right.
      \end{align*}
      \item Si \(A \neq \emptyset\) entonces:
      \begin{align*}
        \left\{
          \begin{array}{l}
            A = \{a_1, a_2, \dots, a_n\} \quad n \in \mathbb{N} \mbox{ y } a_i \neq a_j \mbox{ si } i \neq j\\[1ex]
            B = \{b_1, b_2, b_3, \dots\} = \{b_n : n \in \mathbb{N}\} \quad b_i \neq b_j \mbox{ si } i \neq j
          \end{array}
        \right.
      \end{align*}
      Así tenemos que la unión natural es:
      \begin{align*}
        A \cup B = \left\{\underbracket{a_1}_{1}, \underbracket{a_2}_{2}, \dots, \underbracket{a_n}_{n}, \underbracket{b_1}_{n+1}, \underbracket{b_2}_{n+2}, \underbracket{b_3}_{n+3}, \dots\right\}
      \end{align*}
      Y podemos definir la biyección:
      \begin{align*}
        \varphi : \mathbb{N} & \longrightarrow A \cup B \\
        m & \longmapsto \varphi(m) = \begin{cases}
          a_m, & 1 \leq m \leq n\\
          b_{m-n}, & m > n
        \end{cases}
      \end{align*}
      Y por otra parte, si consideramos el producto \( A \times B\) tenemos:
      \begin{center}
        \begin{tabular}{c|cccc}
          & \(b_1\) & \(b_2\) & \(b_3\) & \(\dots\) \\
          \hline
          \(a_1\) & \((a_1, b_1)\) & \((a_1, b_2)\) & \((a_1, b_3)\) & \(\dots\) \\[1ex]
          \(a_2\) & \((a_2, b_1)\) & \((a_2, b_2)\) & \((a_2, b_3)\) & \(\dots\) \\[1ex]
          \(a_3\) & \((a_3, b_1)\) & \((a_3, b_2)\) & \((a_3, b_3)\) & \(\dots\) \\[1ex]
          \(\vdots\) & \(\vdots\) & \(\vdots\) & \(\vdots\) & \(\vdots\)\\
          \(a_n\) & \((a_n, b_1)\) & \((a_n, b_2)\) & \((a_n, b_3)\) & \(\dots\)
        \end{tabular}
      \end{center}
      \vspace{2ex}

      Así, recorriendo la tabla columna a columna obtendríamos la sucesión:
      \begin{align*}
        A \times B = \left\{(a_1, b_1), (a_2, b_1), \dots , (a_n, b_1), (a_1, b_2), (a_2, b_2), \dots, (a_n, b_2), \dots  \right\}
      \end{align*}
    \end{enumerate}
  \end{dem_box}
  \vspace{3ex}

  \item Sea \(A, B\) infinitos numerables con \(A \cup B = \emptyset \) entonces:
  \begin{align*}
    \left\{
      \begin{array}{l}
        A \cup B \mbox{ es infinito numerable}\\
        A \times B \mbox{ es infinito numerable }
      \end{array}
    \right.
  \end{align*}
  \begin{dem_box}{Demostración}
    Supongamos ambos conjuntos \(A\) y \(B\) infinitos numerables, es decir:
    \begin{align*}
      A & = \{a_n : n \in \mathbb{N}\} \mbox{ con } a_n \neq a_m \mbox{ si } n \neq m \\[1ex]
      B & = \{b_n : n \in \mathbb{N}\} \mbox{ con } b_n \neq b_m \mbox{ si } n \neq m
    \end{align*}
    Entonces tenemos que la unión natural es:
    \begin{align*}
      A \cup B = \left\{\underbracket{a_1}_1, \underbracket{b_1}_2, \underbracket{a_2}_3, \underbracket{b_2}_4, \underbracket{a_3}_5, \underbracket{b_3}_6, \dots\right\}
    \end{align*}
    Y podemos definir la biyección:
    \begin{align*}
      \varphi : \mathbb{N} & \longrightarrow A \cup B \\
      m & \longmapsto \varphi(m) = \begin{cases}
        a_{\frac{m+1}{2}}, & m \mbox{ impar} \\[1ex]
        b_{\frac{m}{2}}, & m \mbox{ par}
      \end{cases}
    \end{align*}
    \vspace{20ex}

    Y por otra parte, si consideramos el producto \( A \times B\) tenemos:
    \begin{center}
      \begin{tabular}{c|cccc}
        & \(b_1\) & \(b_2\) & \(b_3\) & \(\dots\) \\
        \hline
        \(a_1\) & \((a_1, b_1)\) & \((a_1, b_2)\) & \((a_1, b_3)\) & \(\dots\) \\[1ex]
        \(a_2\) & \((a_2, b_1)\) & \((a_2, b_2)\) & \((a_2, b_3)\) & \(\dots\) \\[1ex]
        \(a_3\) & \((a_3, b_1)\) & \((a_3, b_2)\) & \((a_3, b_3)\) & \(\dots\) \\[1ex]
        \(\vdots\) & \(\vdots\) & \(\vdots\) & \(\vdots\) & \(\vdots\)\\
      \end{tabular}
    \end{center}
    Por lo que, recorriendo la tabla en diagonal obtendríamos la sucesión:
    \begin{align*}
      A \times B = \left\{
        (a_1,b_1), (a_1, b_2), (a_2, b_1), (a_1, b_3), \dots 
      \right\}
    \end{align*}
  \end{dem_box}
\end{enumerate}
\vspace{3ex}

\begin{ejem_box}{Ejercicio}
  \textit{Sea \(A_i\) conjunto infinito numerable \(\forall i \in \mathbb{N}\) tal que \(A_i \cap A_j \neq \emptyset \, \, \forall i \neq j\) (no necesario). Probar que \(\cup_{i \in \mathbb{N}} A_i\) es numerable.}\vspace{2ex}

  Sea \(A_i\) un conjunto infinito numerable cualquiera, será de la forma:
  \begin{align*}
    A_i = \{a_{i_n} : n \in \mathbb{N}\} \mbox{ con } a_{i_n} \neq a_{i_m} \mbox{ si } n \neq m
  \end{align*}
  Y definimos la aplicación \(f_i\) como:
  \begin{align*}
    f_i : \mathbb{N} & \longrightarrow A_i \\
    n & \longmapsto f_i(n) = a_{i_n}
  \end{align*}
  Colocando los elementos \(f_i(j)\) en una matriz donde la fila \(i\) contiene la enumeración de \(A_i\) tenemos:
  \begin{center}
    \begin{tabular}{c|ccccc}
      & \(j = 1\) & \(j = 2\) & \(j = 3\) & \(j = 4\) & \(\dots\) \\
      \hline
      \(i = 1\) & \(f_1(1)\) & \(f_1(2)\) & \(f_1(3)\) & \(f_1(4)\) & \(\dots\) \\[1ex]
      \(i = 2\) & \(f_2(1)\) & \(f_2(2)\) & \(f_2(3)\) & \(f_2(4)\) & \(\dots\) \\[1ex]
      \(i = 3\) & \(f_3(1)\) & \(f_3(2)\) & \(f_3(3)\) & \(f_3(4)\) & \(\dots\) \\[1ex]
      \(i = 4\) & \(f_4(1)\) & \(f_4(2)\) & \(f_4(3)\) & \(f_4(4)\) & \(\dots\) \\[1ex]
      \(\vdots\) & \(\vdots\) & \(\vdots\) & \(\vdots\) & \(\vdots\) & \(\ddots\)      
    \end{tabular}
  \end{center}
  Ahora, recorriendo las entradas por diagonales según la suma \(i + j = k\) tenemos:
  \begin{align*}
    k = 2 & \longrightarrow f_1(1) \\
    k = 3 & \longrightarrow f_1(2), f_2(1) \\
    k = 4 & \longrightarrow f_1(3), f_2(2), f_3(1) \\
    k = 5 & \longrightarrow f_1(4), f_2(3), f_3(2), f_4(1) \\
    \vdots \hspace{2ex} & \hspace{15ex} \vdots
  \end{align*}
  Este recorrido visita todas las parejas \((i,j) \in \mathbb{N} \times \mathbb{N}\) por lo que si definimos la aplicación:
  \begin{align*}
    \varphi : \mathbb{N} & \longrightarrow \bigcup_{i \in \mathbb{N}} A_i \\
    m & \longmapsto \varphi(m) = f_i(j) \mbox{ donde } m = \frac{(i+j-2)(i+j-1)}{2} + i
  \end{align*}
  Tenemos una aplicación suprayectiva y, por lo tanto, la unión es a lo sumo numerable.\vspace{3ex}
  
  Si los \(A_i\) no fueran disjuntos, podemos repetir la misma idea pero teniendo en cuenta que se podría repetir un mismo elemento. Para obtener una biyección, simplemente:
  \begin{enumerate}
    \item Recorremos las entradas como antes formando la lista: \(\varphi(1), \varphi(2), \varphi(3), \dots\)
    \item Definimos una nueva sucesión \(t\) como:
    \begin{itemize}
      \item \(n = 0\) entonces \(t(0 ) = \varphi(0)\)
      \item \(n \geq 1\) entonces \(t(n)\) es el primer elemento \(\varphi(k)\) con \(k\) mayor que los usados previamente y tal que \(\varphi(k) \notin \left\{t(0), \dots, t(n - 1)\right\}\)
    \end{itemize}
  \end{enumerate}
  Así, tenemos que cada \(t(n)\) es distinto por construcción y todo \(x \in \cup_i A_i\) aparece en \(\varphi\) en alguna posición, por lo que su primera aparición será tomada en algún \(t(n)\).
\end{ejem_box}
\vspace{3ex}

\subsection{Partes de un conjunto. Definición}
Sea \(A\) un conjunto, se llama \textbf{partes de \(A\)} al conjunto cuyos elementos son todos los subconjuntos de \(A\). Se denota por \(\mathcal{P}(A)\).\vspace{2ex}

\begin{ej_box}{Nota}
  En particular, tenemos que \(\emptyset \in \mathcal{P}(A)\) y \(A \in \mathcal{P}(A)\).
\end{ej_box}
\vspace{2ex}

\begin{ejem_box}{Ejemplo}
  Sea \(A = \{1, 2\}\) entonces:
  \begin{align*}
    \mathcal{P}(A) = \left\{\emptyset, \{1\}, \{2\}, \{1, 2\}\right\}
  \end{align*}
  Donde tenemos que:
  \begin{align*}
    \mbox{Card } A = 2 \quad \mbox{y} \quad \mbox{Card } \mathcal{P}(A) = 4
  \end{align*}
\end{ejem_box}
\vspace{3ex}
\begin{ejem_box}{Ejemplo}
  Si consideramos el conjunto vacío \(\emptyset \) entonces:
  \begin{align*}
    \mathcal{P}(\emptyset ) = \{\emptyset\}
  \end{align*}
  Por lo que:
  \begin{align*}
    \mbox{Card } \emptyset = 0 \quad \mbox{y} \quad \mbox{Card } \mathcal{P}(\emptyset) = 1
  \end{align*}
\end{ejem_box}
\vspace{3ex}

\begin{ej_box}{Observación}
  En general, si \(A\) es finito entonces se cumple que:
  \begin{align*}
    \mbox{Card } \mathcal{P}(A) = 2^{\text{Card } A} \in \mathbb{N}
  \end{align*}
  En efecto, podemos ver que:
  \begin{itemize}
    \item Si \(A = \emptyset \) entonces:
    \begin{align*}
      \mbox{Card } A = 0 \quad \mbox{ y } \quad \mbox{Card } \mathcal{P}(A) = 1 = 2^0
    \end{align*}
    \item Si \(A \neq \emptyset \), supongamos que \(\mbox{Card } A = n \in \mathbb{N}\) tal que:
    \begin{align*}
      \exists \varphi : \mathcal{P}(A) & \longrightarrow C \coloneq \left\{(\varepsilon_1, \varepsilon_2, \dots, \varepsilon_n ) \mbox{ tq } \varepsilon_i = 0  \mbox{ o } 1\right\}\\
      B & \longmapsto (\varepsilon_1, \varepsilon_2, \dots, \varepsilon_n) \quad \mbox{ con } \varepsilon_i = \left\{
        \begin{array}{ll}
          1 \mbox{ si } a_i \in B \\[1ex]
          0 \mbox{ si } a_i \not\in B
        \end{array}
      \right.
    \end{align*}
    La idea aquí es que estamos ``codificando'' cada subconjunto \(B\) de \(\mathcal{P}(A)\) mediante una sucesión de ceros y unos, donde el \(i\)-ésimo dígito nos indica si el elemento \(a_i\) pertenece o no a \(B\).\footnote{Similar a la idea de ``one-hot encoding'' en machine learning.}\vspace{2ex}

    Así, tenemos que si \(A = \{a_1, a_2, \dots, a_n\}\) entonces:
    \begin{align*}
      C = \{(0,0,\dots,0), (1,0,\dots,0), (0,1,\dots,0), \dots, (1,1,\dots,1)\}
    \end{align*}
    Y por tanto, \(\mbox{Card } C = 2^n\) ya que cada dígito puede tomar dos valores y hay \(n\) dígitos.\vspace{2ex}
  \end{itemize}
\end{ej_box}
\vspace{3ex}

\subsection{\texorpdfstring{\(\mbox{Card } \mathcal{P}(A) > \mbox{Card } A\). Proposición}{Card P(A) > Card A. Proposición}}
\noindent Sea \(A\) un conjunto cualquiera, entonces se cumple que:
\begin{align*}
  \mbox{Card } \mathcal{P}(A) > \mbox{Card } A
\end{align*}
\begin{dem_box}{Demostración}
  Sea un conjunto \(B\) cualquiera, entonces claramente tenemos que:
  \begin{align*}
    \mbox{Card } B \leq \mbox{Card } \mathcal{P}(B)
  \end{align*}
  Supongamos que existe un conjunto \(A\) tal que:
  \begin{align*}
    \mbox{Card } \mathcal{P}(A) = \mbox{Card } A
  \end{align*}
  Entonces, existiría una biyección \(\varphi: A \longrightarrow \mathcal{P}(A)\), es decir, que cada elemento de \(A\) se corresponde exactamente con un subconjunto de \(\mathcal{P}(A)\). \vspace{2ex}

  Consideremos el conjunto de todos los elementos que no pertenecen al subconjunto con el que están asociados mediante \(\varphi\):
  \begin{align*}
    B \coloneq \left\{x \in A : x \not\in \varphi(x)\right\}
  \end{align*}
  Como \(\varphi\) es supuestamente sobreyectiva, debe existir \(z \in A\) tal que \(\varphi(z) = B\), donde se presentan dos casos:
  \begin{itemize}
    \item \(z \in B = \varphi(z)\) entonces por definición de \(B\): \(z \not\in \varphi(z) \quad \# \)
    \item \(z \not\in \varphi(z) = B\) entonces por definición de \(B\): \(z \in B \quad \#\) 
  \end{itemize}
  Como hemos llegado inevitablemente a una contradicción, concluimos que \(\mbox{Card } A \neq \mbox{Card } \mathcal{P}(A)\) y por tanto:
  \begin{align*}
    \mbox{Card } A < \mbox{Card } \mathcal{P}(A)
  \end{align*}
\end{dem_box}
\vspace{3ex}

\begin{ej_box}{Nota}
  La proposición anterior establece la existencia de una jerarquía bastante grande de cardinales no finitos. Por ello, aprovechamos esto para introducir la siguiente notación:
  \begin{align*}
    \underbracket{0}_{\text{\tiny \(\mbox{Card } \emptyset\)} } < \overbrace{ \underbracket{1}_{\text{\tiny \(\mbox{Card } \mathcal{P}(\emptyset )\)}} < \underbracket{2}_{\text{\tiny \(\mbox{Card } \mathcal{P}(\mathcal{P}(\emptyset ))\)}} < 3 < 4 < \dots}^{\mathbb{N}} < \underbracket{\aleph_0 < 2^{\aleph_0} < 2^{2^{\aleph_0}}}_{\text{Cardinales transfinitos}} < \dots
  \end{align*}
  Donde hemos definido \(\aleph_0\) como \(\mbox{Card } \mathbb{N}\).
\end{ej_box}
\vspace{3ex}

\begin{ejem_box}{Ejercicio}
  \textit{Demostrar que \(\mbox{Card } \mathbb{R} = \mbox{Card } \mathcal{P}(\mathbb{N})\)}\vspace{2ex}

  Consideremos la aplicación \(\varphi\) dada por:
  \begin{align*}
    \varphi : \mathcal{P}(\mathbb{N}) & \longrightarrow C = \left\{\left(\varepsilon_1, \varepsilon_2, \dots , \varepsilon_n, \dots \right) \colon \varepsilon_i \in \mathbb{N}\right\}\\
    B & \longmapsto \left(\varepsilon_1, \varepsilon_2, \dots, \varepsilon_n, \dots \right) \mbox{ con } \varepsilon_i = \left\{
      \begin{array}{ll}
        a_i & \mbox{ si } a_i \in B \\[1ex]
        0 & \mbox{ si } a_i \not\in B
      \end{array}
    \right.
  \end{align*}
  Así, cada subconjunto \(B\) de \(\mathcal{P}(\mathbb{N})\) se corresponde con una sucesión infinita de números naturales, donde el \(i\)-ésimo término es \(a_i\) si \(a_i\) pertenece a \(B\) y \(0\) en caso contrario y bastaría ver que \(\phi\) es biyectiva.

  \begin{itemize}
    \item \textbf{Inyectividad:} Supongamos que existen \(B_1, B_2 \in \mathcal{P}(\mathbb{N})\) tales que \(\varphi(B_1) = \varphi(B_2)\). Entonces, sus sucesiones asociadas son iguales:
    \begin{align*}
      \left(\varepsilon_1^{(1)}, \varepsilon_2^{(1)}, \dots\right) = \left(\varepsilon_1^{(2)}, \varepsilon_2^{(2)}, \dots\right)
    \end{align*}
    Por lo que cada término de ambas sucesiones coincide, es decir, \(\forall i \in \mathbb{N}\):
    \begin{align*}
      \varepsilon_i^{(1)} = \varepsilon_i^{(2)}
    \end{align*}
    Y por tanto, \(B_1 = B_2\).
    \item \textbf{Sobreyectividad:} Sea una sucesión cualquiera en \(C\):
    \begin{align*}
      \left(\varepsilon_1, \varepsilon_2, \dots\right) \quad \mbox{ con } \varepsilon_i \in \mathbb{N}
    \end{align*}
    Entonces, podemos definir el subconjunto \(B\) como:
    \begin{align*}
      B = \left\{a_i : \varepsilon_i = a_i \neq 0\right\} \subseteq \mathbb{N}
    \end{align*}
    Por lo que tenemos que:
    \begin{align*}
      \varphi(B) = \left(\varepsilon_1, \varepsilon_2, \dots\right)
    \end{align*}
  \end{itemize}
\end{ejem_box}
\newpage

\section{\texorpdfstring{Descomposición de subconjuntos abiertos de \(\mathbb{R}^N\) en cubos diádicos}{Descomposición de subconjuntos abiertos de R^N en cubos diádicos}}

\subsection{Intervalo diádico. Definición}
\noindent Se llama \textbf{intervalo diádico de orden \(n \in \mathbb{N}\)} a cualquier intervalo de la forma:
\begin{align*}
  \left[\frac{j - 1}{2^n}, \frac{j}{2^n}\right) \quad \mbox{ con } j \in \mathbb{Z}
\end{align*}

\begin{ejem_box}{Ejemplo}
  En otras palabras, estamos dividiendo la recta real en trozos de tamaño \(\frac{1}{2^n}\) y cada trozo empieza en un múltiplo de \(\frac{1}{2^n}\) y termina justo antes del siguiente múltiplo de \(\frac{1}{2^n}\). \vspace{2ex}

  Por ejemplo, para \(n = 1\) los intervalos de longitud \(\frac{1}{2}\) son:
  \begin{align*}
    \left[\frac{j - 1}{2}, \frac{j}{2}\right) = \left\{\dots, \left[-1, -\frac{1}{2}\right), \left[-\frac{1}{2}, 0\right), \left[0, \frac{1}{2}\right), \left[\frac{1}{2}, 1\right), \left[1, \frac{3}{2}\right), \dots\right\}
  \end{align*}
  Y para \(n = 2\) los intervalos de longitud \(\frac{1}{4}\) son:
  \begin{align*}
    \left[\frac{j - 1}{4}, \frac{j}{4}\right) = \left\{\dots, \left[-\frac{1}{2}, -\frac{1}{4}\right), \left[-\frac{1}{4}, 0\right), \left[0, \frac{1}{4}\right), \left[\frac{1}{4}, \frac{1}{2}\right), \dots\right\}
  \end{align*}
\end{ejem_box}
\vspace{3ex}

\begin{ej_box}{Observación}
  Podemos notar que:
  \begin{align*}
    \displaystyle \bigcup_{j\in\mathbb{Z}} \left[\frac{j - 1}{2^n}, \frac{j}{2^n}\right) = \mathbb{R}
  \end{align*}
  Y también podemos ver que se tiene que si \(i \neq j\) entonces:
  \begin{align*}
    \left[\frac{i - 1}{2^n}, \frac{i}{2^n}\right) \cap \left[\frac{j - 1}{2^n}, \frac{j}{2^n}\right) = \emptyset
  \end{align*}
  Además, la colección de todos los intervalos diádicos de orden \(n\) es numerable.
\end{ej_box}
\vspace{3ex}
\begin{ej_box}{Observación}
  Si \(I\) es un intervalo diádico de orden \(n\) y \(J\) es un intervalo diádico de orden \(m\) con \(m \leq n\) entonces:
  \begin{align*}
    I \subseteq J \quad \mbox{ o bien } \quad I \cap J = \emptyset
  \end{align*}
\end{ej_box}
\vspace{3ex}

\subsection{Cubo diádico. Definición}
\noindent Se llama \textbf{cubo diádico de \(\mathbb{R}^N\) de orden \(n \in \mathbb{N}\)} a cualquier conjunto de la forma:
\begin{align*}
  I_1 \times I_2 \times \dots \times I_n \subseteq \mathbb{R}^N
\end{align*}
donde \(I_k\) es un intervalo diádico de orden \(n\) para \(k = 1, 2, \dots, n\).\vspace{3ex}

\begin{ejem_box}{Ejemplo}
  Por ejemplo, en \(\mathbb{R}^2\) un cubo diádico de orden \(n\) es un cuadrado cuyos lados miden \(\frac{1}{2^n}\) y están alineados con los ejes coordenados. \vspace{2ex}

  En particular, un cubo diádico de \(\mathbb{R}^2\) de orden \(1\) podría ser:
  \begin{align*}
    \left[-\frac{1}{2}, 0\right) \times \left[0, \frac{1}{2}\right)
  \end{align*}
  Es decir, el cuadrado de lado \(\frac{1}{2}\) que tiene un vértice en el origen y se extiende hacia el segundo cuadrante.\vspace{2ex}

  \begin{center}
    \begin{tikzpicture}[scale=4]
      \draw[-{Latex}] (-0.7,0) -- (0.7,0) node[right] {\(x\)};
      \draw[-{Latex}] (0,-0.2) -- (0,0.7) node[above] {\(y\)};
      \draw[thick] (-0.5,0) -- (-0.5,0.5) -- (0,0.5) -- (0,0) -- cycle;
      \fill[gray!30] (-0.5,0) -- (-0.5,0.5) -- (0,0.5) -- (0,0) -- cycle;
      \node at (-0.5, -0.1) {\(\frac{-1}{2}\)};
      \node at (0.05, 0.5) {\(\frac{1}{2}\)};
      \node at (0.05, -0.05) {\(0\)};
    \end{tikzpicture}
  \end{center}

  En \(\mathbb{R}^3\), un cubo diádico de orden \(n\) podría ser un cubo cuyos lados miden \(\frac{1}{2^n}\) y están alineados con los ejes coordenados. \vspace{2ex}

  Por ejemplo, un cubo diádico de \(\mathbb{R}^3\) de orden \(1\) podría ser:
  \begin{align*}
    \left[ - \frac{1}{2}, 0 \right) \times \left[0, \frac{1}{2}\right) \times \left[0, \frac{1}{2}\right)
  \end{align*}
  Es decir, el cubo de lado \(\frac{1}{2}\) que tiene un vértice en el origen y se extiende hacia el octante negativo en \(x\) y positivo en \(y\) y \(z\).\vspace{2ex}

  \begin{center}
    \begin{tikzpicture}[scale=4]
      \draw[-{Latex}] (-0.8,0,0) -- (0.7,0,0) node[right] {\(x\)};
      \draw[-{Latex}] (0,-0.2,0) -- (0,0.7,0) node[above] {\(y\)};
      \draw[-{Latex}] (0,0,-0.2) -- (0,0,0.8) node[above] {\(z\)};
      \fill[gray!30] (-0.5,0,0) -- (-0.5,0.5,0) -- (0,0.5,0) -- (0,0,0) -- cycle;
      \fill[gray!30] (-0.5,0,0) -- (-0.5,0,0.5) -- (0,0,0.5) -- (0,0,0) -- cycle;
      \fill[gray!30] (-0.5,0.5,0) -- (-0.5,0.5,0.5) -- (0,0.5,0.5) -- (0,0.5,0) -- cycle;
      \fill[gray!30] (-0.5,0,0.5) -- (-0.5,0.5,0.5) -- (0,0.5,0.5) -- (0,0,0.5) -- cycle;
      
      \draw(-0.5,0,0) -- (-0.7, 0, 0);
      
      \draw[thick] (-0.5,0,0) -- (-0.5,0.5,0) -- (0,0.5,0) -- (0,0,0) -- cycle;
      \draw[thick] (-0.5,0,0) -- (-0.5,0,0.5) -- (0,0,0.5) -- (0,0,0) -- cycle;
      \draw[thick] (-0.5,0.5,0) -- (-0.5,0.5,0.5) -- (0,0.5,0.5) -- (0,0.5,0) -- cycle;
      \draw[thick] (-0.5,0,0.5) -- (-0.5,0.5,0.5) -- (0,0.5,0.5) -- (0,0,0.5) -- cycle;

      \node at (0.1, 0, 0.6) {\(\frac{1}{2}\)};
      \node at (0.07, 0.5, 0) {\(\frac{1}{2}\)};
      \node at (-0.47, -0.08, 0) {\(\frac{-1}{2}\)};
    \end{tikzpicture}
  \end{center}
\end{ejem_box}

\begin{ej_box}{Observación}
  Podemos notar que:
  \begin{align*}
    \displaystyle \bigcup_{C \in \mathcal{F}_n} C = \mathbb{R}^N
  \end{align*}
  donde \(\mathcal{F}_n\) es la colección de todos los cubos diádicos de \(\mathbb{R}^N\) de orden \(n\).\vspace{3ex}

  Así, tenemos que si \(C_1, C_2 \in \mathcal{F}_n\) y \(C_1 \neq C_2\) entonces:
  \begin{align*}
    C_1 \cap C_2 = \emptyset
  \end{align*}
  Además, la colección de todos los cubos diádicos de orden \(n\), esto es, \(\mathcal{F}_n\), es numerable.\vspace{2ex}

  Por último, si \(C \in \mathcal{F}_n\), \(D \in \mathcal{F}_m\) con \(m \leq n\) entonces:
  \begin{align*}
    C \subseteq D \quad \mbox{ o } \quad C \cap D = \emptyset 
  \end{align*}
\end{ej_box}
\vspace{3ex}

\subsection{Teorema de descomposición}
Para todo subconjunto abierto \(O \subseteq \mathbb{R}^N\) no vacío, se tiene que existe una colección numerable de cubos diádicos \(\{C_i\}_{i \in \mathbb{N}}\) disjuntos dos a dos (posiblemente de órdenes distintos) tales que:
\begin{enumerate}
  \item \(O\) es exactamente la unión de todos esos cubos, es decir: \(O = \bigcup_{i \in \mathbb{N}} C_i\)
  \item Cada cubo está completamente dentro de \(O\), incluso su clausura: \(\overline{C_i} \subseteq O \quad \forall i \in \mathbb{N}\)
  \item Son disjuntos entre sí: \(C_i \cap C_j = \emptyset \quad \mbox{ si } i \neq j\)
\end{enumerate}
\vspace{2ex}

\begin{dem_box}{Demostración}
  Recordemos que para cada \(n \in \mathbb{N}\) se tiene que:
  \begin{align*}
    \mathcal{F}_n = \left\{C : C \mbox{ cubo diádico de orden } n\right\}
  \end{align*}
  Entonces definimos la subfamilia de cubos diádicos de orden \(n\) cuya clausura está contenida en \(O\), esto es:
  \begin{align*}
    \mathcal{G}_n \coloneq \left\{C \in \mathcal{F}_n : \overline{C} \subseteq O\right\}
  \end{align*}
  Y construimos recursivamente las familias \(\mathcal{H}_n\) de cubos disjuntos:
  \begin{itemize}
    \item Si \(n = 1\) entonces:
    \begin{align*}
      \mathcal{H}_1 \coloneq \mathcal{G}_1
    \end{align*}
    Es decir, tomamos todos los cubos de orden 1 cuya clausura está en \(O\).
    \item Si \(n > 1\) entonces:
    \begin{align*}
      \mathcal{H}_n \coloneq \left\{C \in \mathcal{G}_n : C \not\subseteq \displaystyle \bigcup_{k = 1}^{n - 1} \bigcup_{D \in \mathcal{H}_k} D\right\} = \left\{C \in \mathcal{G}_n : C \cap \left[\displaystyle \bigcup_{k = 1}^{n - 1} \displaystyle \bigcup_{D \in \mathcal{H}_k} D \right] = \emptyset \right\}
    \end{align*}
    Es decir, tomamos solo los cubos de orden \(n\) que \textbf{no} están ya cubiertos por los cubos de órdenes menores.
  \end{itemize}
  Así, probaremos por doble contenido que:
  \begin{align*}
    O = \bigcup_{n = 1}^\infty \bigcup_{C \in \mathcal{H}_n} C
  \end{align*}
  \begin{itemize}
    \item[\(\supseteq\))] Trivial, todo cubo \(C \in \mathcal{H}_n\) satisface que \(C \subseteq O\) ya que \(C \in \mathcal{G}_n\) y \(\overline{C} \subseteq O\).\vspace{2ex}
    
    \item[\(\subseteq\))] Sea \(x_0 = (x_1, x_2, \dots, x_N) \in O \subseteq \mathbb{R}^N\), como \(O\) es abierto, existe un cubo abierto centrado en \(x_0\) contenido en \(O\), esto es \(\exists n \in \mathbb{N}\) tal que:
    \begin{align*}
      x_0 \in \prod_{i = 1}^N \left(x_i - \frac{1}{2^n}, x_i + \frac{1}{2^n}\right) \subseteq \prod_{i = 1}^N \left[x_i - \frac{1}{2^n}, x_i + \frac{1}{2^n}\right] \subseteq O
    \end{align*}
    Entonces, para cada coordenada \(1 \leq i \leq N\) tenemos que \(\exists! j_i \in \mathbb{Z}\) tal que:
    \begin{align*}
      x_i \in \left[\frac{j_i - 1}{2^n}, \frac{j_i}{2^n} \right)
    \end{align*}
    Así, tenemos que para cada \(1 \leq i \leq N\):
    \begin{align*}
      \left.
        \begin{array}{rl}
          \frac{j_i - 1}{2^n} \leq x_i & \implies \frac{j_i}{2^n} \leq x_i + \frac{1}{2^n}\\[2ex]
          x_i < \frac{j_i}{2^n} & \implies x_i - \frac{1}{2^n} < \frac{j_i - 1}{2^n} 
        \end{array}
      \right\} \implies \left[\frac{j_i - 1}{2^n}, \frac{j_i}{2^n} \right) \subseteq \left(x_i - \frac{1}{2^n}, x_i + \frac{1}{2^n}\right)
    \end{align*}
    Luego:
    \begin{align*}
      C \coloneq \prod_{i = 1}^N \left[\frac{j_i - 1}{2^n}, \frac{j_i}{2^n} \right) \subseteq \prod_{i = 1}^N \left(x_i - \frac{1}{2^n}, x_i + \frac{1}{2^n}\right) \subseteq O
    \end{align*}
    Y por lo tanto se cumple que:
    \begin{align*}
      x_0 \in C \subseteq \overline{C} \subseteq O
    \end{align*}
    Esto prueba que:
    \begin{align*}
      \mathcal{A}_{x_0} = \left\{n \in \mathbb{N} : \exists C \in \mathcal{G}_n \mbox{ con }x_0 \in C\right\} \neq \emptyset 
    \end{align*}
    Y aplicando el principio de buena ordenación de \(\mathbb{N}\):
    \begin{align*}
      \exists \min \mathcal{A}_{x_0} \coloneq m
    \end{align*}
    Veamos que el correspondiente \(C \in \mathcal{G}_m\) con \(x_0 \in C \subseteq O\) cumple que \(C \in \mathcal{H}_m\). En efecto, si \(C \notin \mathcal{H}_m\) entonces:
    \begin{align*}
      C \in \mathcal{G}_n \mbox{ con } n < m = \min \mathcal{A}_{x_0} \quad \#
    \end{align*}
    Por tanto, hemos probado que \(C \in \mathcal{H}_m\) lo que implica que:
    \begin{align*}
      x_0 \in C \subseteq \bigcup_{n = 1}^\infty \bigcup_{D \in \mathcal{H}_n} D \implies O \subseteq \bigcup_{n = 1}^\infty \bigcup_{C \in \mathcal{H}_n} C
    \end{align*}
  \end{itemize}

  Para ver que \( \mathcal{H}_n\) son disjuntos, basta notar que se cumple por construcción:
  \begin{itemize}
    \item Dentro de cada \(\mathcal{H}_n\) los cubos diádicos del mismo orden son disjuntos.
    \item Entre diferentes órdenes, si \(C \in \mathcal{H}_n\) y \(D \in \mathcal{H}_m\) con \(n \neq m\) son disjuntos ya que si \(n > m\) entonces:
    \begin{align*}
      C \not\subseteq \displaystyle \bigcup_{k = 1}^{n - 1} \displaystyle \bigcup_{E \in \mathcal{H}_k} E
    \end{align*}
    es decir, que en particular, no está contenido en \(D\) y por tanto, \(C \cap D = \emptyset\).
  \end{itemize} 

  Finalmente, cada \(\mathcal{H}_n\) es numerable ya que es una subcolección de \(\mathcal{F}_n\) que es numerable, y la unión numerable de conjuntos numerables es numerable.
\end{dem_box}

\newpage
\section{Series Dobles}
El objetivo de esta sección es estudiar qué significa ``sumar todos los elementos'' de una trabla infinita de números reales dispuestos como \(a_n,m\) donde \(n, m \in \mathbb{N}\):
\begin{center}
  \begin{tabular}{c|cccccc}
    \(\mathbb{N}\) & &&&&&\\
    & \(\vdots\) & \(\vdots\) & \(\vdots\) & \(\vdots\) & & \\
    & \(a_{31}\) & \(a_{32}\) & \(a_{33}\) & \(a_{34}\) & \(\dots\) & \\
    & \(a_{21}\) & \(a_{22}\) & \(a_{23}\) & \(a_{24}\) & \(\dots\) & \\
    & \(a_{11}\) & \(a_{12}\) & \(a_{13}\) & \(a_{14}\) & \(\dots\) & \\ \hline
    & 1 & 2 & 3 & 4 & \(\dots\) & \(\mathbb{N}\)
  \end{tabular}
\end{center}
\vspace{3ex}

\subsection{Sucesión doble. Definición}
\noindent Llamamos \textbf{sucesión doble} a una función de la forma:
\begin{align*}
  \varphi : \mathbb{N} \times \mathbb{N} & \longrightarrow \overline{\mathbb{R}} = \mathbb{R} \cup \{\pm \infty\}\\
  (n, m) & \longmapsto \varphi(n, m) = a_{n,m}
\end{align*}
Cada número \(a_{n,m}\) es el \textbf{término general} de la fila \(n\) y columna \(m\) de la sucesión doble.\footnote{Aquí los físicos dicen que si sumas los naturales te da \( - \frac{1}{12}\) o algo así. Digo yo que lo del \( - \frac{1}{12}\) será por los \(12\) apóstoles, porque si no yo no lo veo.}\vspace{3ex}

\subsection{Serie doble. Definición}
Dada una sucesión doble \(\varphi : \mathbb{N} \times \mathbb{N} \longrightarrow \overline{\mathbb{R}}\), llamamos \textbf{serie doble de término general \(a_{n,m}\)} a la expresión:
\begin{align*}
  \displaystyle \sum_{n,m} a_{n, m} = \displaystyle \sum_{\substack{n = 1\\m = 1}} a_{n, m}\\
\end{align*}

\subsubsection{Serie doble convergente. Definición}\label{def:serie_doble_convergente}
\noindent Sea \(\displaystyle \sum_{n,m} a_{n, m}\) serie doble, diremos que es \textbf{convergente} si \(\exists s \in \mathbb{R}\) tal que:
\begin{align*}
  \forall \varepsilon > 0, \quad \exists n_0 \in \mathbb{N} \quad \mbox{tal que} \quad  \forall n, m \geq n_0,
\end{align*}
se cumple que:
\begin{align*}
  \left|s - \displaystyle \sum_{\substack{1 \leq i \leq n\\1 \leq j \leq m}} a_{i,j}\right| < \varepsilon
\end{align*}

\begin{ejem_box}{Ejemplo}
  Podemos ver esta convergencia como que los bloques rectangulares crecientes de la tabla se aproximan a un valor límite \(s\):
  \begin{center}
    \begin{tabular}{c|cccccc}
      \(\mathbb{N}\) & &&&&&\\
      & \(\vdots\) & \(\vdots\) & \(\vdots\) & \(\vdots\) & & \\
      & \(a_{41}\) & \(a_{42}\) & \(a_{43}\) & \(a_{44}\) & \(\dots\) & \\
      & \(a_{31}\) & \(a_{32}\) & \(a_{33}\) & \(a_{34}\) & \(\dots\) & \\
      & \(a_{21}\) & \(a_{22}\) & \(a_{23}\) & \(a_{24}\) & \(\dots\) & \\
      & \textcolor{red}{\(a_{11}\)} & \(a_{12}\) & \(a_{13}\) & \(a_{14}\) & \(\dots\) & \\ \hline
      & 1 & 2 & 3 & 4 & \(\dots\) & \(\mathbb{N}\)
    \end{tabular} \hspace{10ex} \(\displaystyle \sum_{\substack{1 \leq i \leq 1\\1 \leq j \leq 1}} a_{i,j}\)\vspace{3ex}

    \begin{tabular}{c|cccccc}
    \(\mathbb{N}\) & &&&&&\\
      & \(\vdots\) & \(\vdots\) & \(\vdots\) & \(\vdots\) & & \\
      & \(a_{41}\) & \(a_{42}\) & \(a_{43}\) & \(a_{44}\) & \(\dots\) & \\
      & \(a_{31}\) & \(a_{32}\) & \(a_{33}\) & \(a_{34}\) & \(\dots\) & \\
      & \textcolor{red}{\(a_{21}\)} & \textcolor{red}{\(a_{22}\)} & \(a_{23}\) & \(a_{24}\) & \(\dots\) & \\
      & \textcolor{red}{\(a_{11}\)} & \textcolor{red}{\(a_{12}\)} & \(a_{13}\) & \(a_{14}\) & \(\dots\) & \\ \hline
      & 1 & 2 & 3 & 4 & \(\dots\) & \(\mathbb{N}\)
    \end{tabular} \hspace{10ex} \(\displaystyle \sum_{\substack{1 \leq i \leq 2\\1 \leq j \leq 2}} a_{i,j}\)\vspace{3ex}

    \begin{tabular}{c|cccccc}
      \(\mathbb{N}\) & &&&&&\\
      & \(\vdots\) & \(\vdots\) & \(\vdots\) & \(\vdots\) & & \\
      & \(a_{41}\) & \(a_{42}\) & \(a_{43}\) & \(a_{44}\) & \(\dots\) & \\
      & \textcolor{red}{\(a_{31}\)} & \textcolor{red}{\(a_{32}\)} & \textcolor{red}{\(a_{33}\)} & \(a_{34}\) & \(\dots\) & \\
      & \textcolor{red}{\(a_{21}\)} & \textcolor{red}{\(a_{22}\)} & \textcolor{red}{\(a_{23}\)} & \(a_{24}\) & \(\dots\) & \\
      & \textcolor{red}{\(a_{11}\)} & \textcolor{red}{\(a_{12}\)} & \textcolor{red}{\(a_{13}\)} & \(a_{14}\) & \(\dots\) & \\ \hline
      & 1 & 2 & 3 & 4 & \(\dots\) & \(\mathbb{N}\)
    \end{tabular} \hspace{10ex} \(\displaystyle \sum_{\substack{1 \leq i \leq 3\\1 \leq j \leq 3}} a_{i,j}\)\vspace{3ex}
  \end{center}
  Y así sucesivamente.
\end{ejem_box}
\vspace{3ex}

\subsubsection{Serie doble divergente. Definición}\label{def:serie_doble_divergente}
\noindent Sea \(\displaystyle \sum_{n,m} a_{n, m}\) serie doble, diremos que \textbf{es divergente} a \( + \infty\) (resp. \( - \infty\)) si:
\begin{align*}
  \forall K \in \mathbb{R}, \quad \exists n_0 \in \mathbb{N} \quad \mbox{tal que} \quad  \forall n, m \geq n_0,
\end{align*}
se cumple que:
\begin{align*}
  \displaystyle \sum_{\substack{1 \leq i \leq n\\1 \leq j \leq m}} a_{i,j} > K \quad \left(\text{resp.} < K\right)
\end{align*}

\begin{ej_box}{Nota}
  Tanto si la serie doble es convergente como si es divergente, se llama valor de la suma de dicha serie doble al número \(s\) (si es convergente) o al número \(+\infty\) o \(-\infty\) (si es divergente) y se denota por:
  \begin{align*}
    s = \displaystyle \sum_{n, m} a_{n, m} \in \mathbb{R} \qquad \mbox{ y } \qquad \pm\infty = \displaystyle \sum_{n,m} a_{n,m} \in \overline{\mathbb{R}}
  \end{align*}
\end{ej_box}
\vspace{3ex}

\begin{ej_box}{Nota}
  Por la ambigüedad del lenguaje, dos series diferentes pueden tener el mismo valor de la suma pero no ser las mismas series. \vspace{2ex}
  
  Por ejemplo, consideremos las siguientes dos series:
  \begin{align*}
    \displaystyle \sum_{n = 1}^{\infty} a_n = 1 + 0 + 0 + 0 + \dots \qquad \mbox{y} \qquad
    \displaystyle \sum_{n = 1}^{\infty} b_n = \frac{1}{2} + \frac{1}{4} + \frac{1}{8} + \dots 
  \end{align*}
  Y se comete el error de escribir \(\displaystyle \sum_n a_n = \displaystyle \sum_n b_n\) porque ambas series convergen a \(1\)\footnote{Y esto lamentablemente ya no lo va a cambiar ni el papa}.\vspace{2ex}
  
  En realidad, ambas valen 1 como número pero no como series, es decir:
  \begin{align*}
    \displaystyle \sum_{n = 1}^{\infty} a_n \neq \displaystyle \sum_{n = 1}^{\infty} b_n
  \end{align*}
  aunque sí se cumple que:
  \begin{align*}
    \displaystyle \sum_{n = 1}^{\infty} a_n = \displaystyle \sum_{n = 1}^{\infty} b_n = 1\\
  \end{align*}
\end{ej_box}
\vspace{3ex}

\subsection{Teorema}
\noindent Sea \(\varphi : \mathbb{N} \times \mathbb{N} \longrightarrow [0, +\infty]\) entonces para cualquier biyección \(g : \mathbb{N} \longrightarrow \mathbb{N} \times \mathbb{N}\) tenemos:
\begin{align*}
  \displaystyle \sum_{n, m} a_{n,m} = \displaystyle \sum_{n = 1}^{\infty} a_{g(n)} = \displaystyle \sum_{m = 1}^{\infty} \displaystyle \sum_{n = 1}^{\infty} a_{n,m} = \displaystyle \sum_{n = 1}^{\infty} \displaystyle \sum_{m = 1}^{\infty} a_{n,m} \in [0, + \infty]\\
\end{align*}
\vspace{3ex}

\begin{dem_box}{Demostración}
  Veamos la demostración por los dos casos posibles:
  \begin{enumerate}
    \item Algún coeficiente \(a_{n,m}\) es igual a \( + \infty\). Entonces, trivialmente:
    \begin{align*}
      \displaystyle \sum_{n,m} a_{n,m} = + \infty & \qquad \mbox{ y } \qquad \displaystyle \sum_{n = 1}^\infty a_{g(n)} = + \infty\\[2ex]
      \displaystyle \sum_{m = 1}^{\infty}\displaystyle \sum_{n = 1}^{\infty} a_{n,m} = + \infty & \qquad \mbox{ y } \qquad \displaystyle \sum_{n = 1}^{\infty} \displaystyle \sum_{m = 1}^{\infty} a_{n,m} = + \infty\\
    \end{align*}
    \item \(\forall n,m\) se tiene \(a_{n,m} \in [0, \infty)\) entonces \(\forall n\) se tiene que \(a_{g(n)} \in [0, \infty)\) y, como consecuencia, se dan dos subcasos:
    \begin{align*}
      \displaystyle \sum_{n = 1}^{\infty} a_{g(n)} = s \in [0, +\infty) \qquad \mbox{ o } \qquad \displaystyle \sum_{n = 1}^{\infty} a_{g(n)} = + \infty
    \end{align*}
    Veamos ambos casos por separado:
    \begin{itemize}
      \item Si \(\sum a_{g(n)} = s\) entonces \(\forall \varepsilon > 0, \quad \exists n_0\in \mathbb{N}\) tal que \(\forall k \geq  n_0\) se tiene:
      \begin{align*}
        0 \leq s - \displaystyle \sum_{n = 1}^{k} a_{g(n)} < \varepsilon
      \end{align*}
      Como \(g\) biyectiva, \(\exists m_0 \in \mathbb{N}\) donde los primeros \(n_0\) términos de \(a_{g(n)}\) corresponden a los índices en el conjunto \(\left\{1, \dots, m_0\right\} \times \left\{1, \dots, m_0\right\}\), i.e.:
      \begin{align*}
        g\left(\left\{1, 2, \dots, n_0\right\}\right) \subseteq \left\{1, 2, \dots, m_0\right\} \times \left\{1, 2, \dots, m_0\right\}
      \end{align*}
      Así, \(\forall n,m \geq m_0\) tenemos:
      \begin{align*}
        \displaystyle \sum_{\substack{1 \leq i \leq n\\1 \leq j \leq m}} a_{i,j} \geq \displaystyle \sum_{i = 1}^{n_0} a_{g(i)} & \xRightarrow[\text{invertir } \geq]{\text{añadir } - } - \displaystyle \sum_{\substack{1 \leq i \leq n\\1 \leq j \leq m}} a_{i,j} \leq - \displaystyle \sum_{i = 1}^{n_0} a_{g(i)} \implies \\[2ex]
        & \xRightarrow{\text{añadir } s} s - \displaystyle \sum_{\substack{1 \leq i \leq n\\1 \leq j \leq m}} a_{i,j} \leq s - \displaystyle \sum_{i = 1}^{n_0} a_{g(i)} < \varepsilon
      \end{align*}
      Y además, sabemos que:
      \begin{align*}
        \displaystyle \sum_{\substack{1 \leq i \leq n\\1 \leq j \leq m}} a_{i,j} \mbox{ es suma parcial de alguna reordenación de } \displaystyle \sum_{n = 1}^{\infty} a_{g(n)}
      \end{align*}
      \vspace{10ex}

      Como todas las reordenaciones de una serie simple de términos positivos tienen la misma suma\footnote{Esto lo dais en Introducción al Análisis o algo así, bueno, en Análisis -1.} por lo que:
      \begin{align*}
        0 \leq s - \displaystyle \sum_{\substack{1 \leq i \leq n\\1 \leq j \leq m}} a_{i,j} < s - \displaystyle \sum_{i = 1}^{n_0} a_{g(i)} < \varepsilon
      \end{align*}
      Y por tanto, como \(\varepsilon\) era arbitrario, tenemos que:
      \begin{align*}
        \displaystyle \sum_{n,m} a_{n,m} = s\\
      \end{align*}
      \item Si \(\sum a_{g(n)} = + \infty\) entonces \(\textcolor{blue}{\forall K \in \mathbb{R}} \quad \exists n_0\in \mathbb{N}\) tal que \(\forall k \geq  n_0\) se tiene:
      \begin{align*}
        \displaystyle \sum_{n = 1}^{k} a_{g(n)} \textcolor{blue}{\; > K}
      \end{align*}
      Como \(g\) es biyectiva, \(\exists m_0 \in \mathbb{N}\) tal que:
      \begin{align*}
        g\left(\left\{1, 2, 3, \dots, n_0\right\}\right) \subseteq \left\{1, 2, 3, \dots, m_0\right\} \times \left\{1, 2, 3, \dots, m_0\right\}
      \end{align*}
      Así, \(\forall n,m \geq m_0\) tenemos:
      \begin{align*}
        \displaystyle \sum_{\substack{1 \leq i \leq n\\1 \leq j \leq m}} a_{i,j} \geq \displaystyle \sum_{i = 1}^{n_0} a_{g(i)} \textcolor{blue}{\; > K}
      \end{align*}
      Y por tanto, como \(K\) era arbitrario, tenemos que:
      \begin{align*}
        \displaystyle \sum_{n,m} a_{n,m} = + \infty
      \end{align*}
    \end{itemize}
    Así, tenemos que \(\displaystyle \sum_{n,m} a_{n,m}\) es convergente o divergente a \(+\infty\) y además:
    \begin{align*}
      \displaystyle \sum_{n,m} a_{n,m} = \displaystyle \sum_{n = 1}^{\infty} a_{g(n)}
    \end{align*}

    Ahora nos falta ver que:
    \begin{align*}
      \displaystyle \sum_{n, m} a_{n, m} = \displaystyle \sum_{m = 1}^{\infty} \displaystyle \sum_{n = 1}^{\infty} a_{n, m} = \displaystyle \sum_{n = 1}^{\infty} \displaystyle \sum_{m = 1}^{\infty} a_{n, m}
    \end{align*}
    \vspace{20ex}

    Diferenciaremos dos casos:
    \begin{itemize}
      \item Si \(\displaystyle \sum_{n, m} a_{n,m} = s \in [0, \infty)\), es decir, si es convergente.\vspace{2ex}
      
      \hyperref[def:serie_doble_convergente]{Por definición} \(\forall \varepsilon > 0, \quad \exists n_0 \in \mathbb{N} \) tal que \(\forall n,m \geq n_0\) se cumple que:
      \begin{align*}
        \left|s - \displaystyle \sum_{\substack{1 \leq i \leq n\\1 \leq j \leq m}} a_{i,j}\right| < \varepsilon
      \end{align*}
      donde tenemos que:
      \begin{align*}
        \displaystyle \sum_{\substack{1 \leq i \leq n\\1 \leq j \leq m}} a_{i,j} = \displaystyle \sum_{j = 1}^{m} \displaystyle \sum_{i = 1}^{n} a_{i,j} = \displaystyle \sum_{i = 1}^{n} \displaystyle \sum_{j = 1}^{m} a_{i,j}
      \end{align*}
      Así, si desarrollamos la desigualdad anterior, tenemos:
      \begin{align*}
        \left|s - \displaystyle \sum_{\substack{1 \leq i \leq n\\1 \leq j \leq m}} a_{i,j}\right| & = \left|s - \displaystyle \sum_{j = 1}^{m} \displaystyle \sum_{i = 1}^{n} a_{i,j}\right| = \left|s - \displaystyle \sum_{i = 1}^{n} \displaystyle \sum_{j = 1}^{m} a_{i,j}\right|
      \end{align*}
      Veamos que ocurre con cada una de las expresiones, empezando por:
      \begin{align*}
        \left|s - \displaystyle \sum_{j = 1}^{m} \displaystyle \sum_{i = 1}^{n} a_{i,j}\right| < \varepsilon \implies 
      \end{align*}
      \begin{align*}
        \implies & \varepsilon \geq \lim_n \left|s - \displaystyle \sum_{j = 1}^{m} \displaystyle \sum_{i = 1}^{n} a_{i,j}\right| = \left|s - \lim_n \displaystyle \sum_{j = 1}^{m} \displaystyle \sum_{i = 1}^{n} a_{ij}\right| = \\[1ex] 
        & \hspace{10ex} = \left|s - \displaystyle \sum_{j = 1}^{m}\left(\lim_n \displaystyle \sum_{i = 1}^{n} a_{ij}\right)\right| = \left|s - \displaystyle \sum_{j = 1}^{m} \left(\displaystyle \sum_{i = 1}^{\infty} a_{ij}\right)\right| \xRightarrow[\forall m \geq n_0]{\text{Válido}} \\[4ex]
        \implies & \varepsilon \geq \lim_{m} \left|s - \displaystyle \sum_{j = 1}^{m}\left(\displaystyle \sum_{i = 1}^{\infty} a_{i,j}\right)\right| = \left|s - \lim_m \displaystyle \sum_{j = 1}^{m}\underbracket{\left(\displaystyle \sum_{i = 1}^{\infty}a_{ij}\right)}_{ \geq  0}\right| =\\[1ex]
        & \hspace{10.5ex} = \left|s - \displaystyle \sum_{j = 1}^{\infty} \left(\displaystyle \sum_{i = 1}^{\infty} a_{i,j}\right)\right| \geq 0 \xRightarrow[\forall \varepsilon > 0]{\text{Válido}} s - \displaystyle \sum_{j = 1}^{\infty} \left(\displaystyle \sum_{i = 1}^{\infty} a_{i,j}\right) = 0
      \end{align*}
      Y ahora de forma análoga, veamos qué ocurre con:
      \begin{align*}
        \left|s - \displaystyle \sum_{i = 1}^{n} \displaystyle \sum_{j = 1}^{m} a_{i,j}\right| < \varepsilon \implies
      \end{align*}
      \begin{align*}
        \implies & \varepsilon \geq \lim_m \left|s - \displaystyle \sum_{i = 1}^{n} \displaystyle \sum_{j = 1}^{m} a_{i,j}\right| = \dots  = \left|s - \displaystyle \sum_{i = 1}^{n} \left(\displaystyle \sum_{j = 1}^{\infty} a_{i,j}\right)\right| \xRightarrow[\forall n \geq n_0]{\text{Válido}} \\[4ex]
        \implies & \varepsilon \geq \lim_{n} \left|s - \displaystyle \sum_{i = 1}^{n}\left(\displaystyle \sum_{j = 1}^{\infty} a_{i,j}\right)\right| = \dots = \left|s - \displaystyle \sum_{i = 1}^{\infty} \left(\displaystyle \sum_{j = 1}^{\infty} a_{i,j}\right)\right| \geq 0 \xRightarrow[\forall \varepsilon > 0]{\text{Válido}} \\[2ex]
      \implies & s - \displaystyle \sum_{i = 1}^{\infty} \left(\displaystyle \sum_{j = 1}^{\infty} a_{i,j}\right) = 0
      \end{align*}
      Por tanto, si \(\displaystyle \sum_{n,m} a_{n,m} = s \in [0, +\infty)\) entonces:
      \begin{align*}
        \displaystyle \sum_{n,m} a_{n,m} = \displaystyle \sum_{m = 1}^{\infty} \left(\displaystyle \sum_{n = 1}^{\infty} a_{n,m}\right) = \displaystyle \sum_{n = 1}^{\infty} \left(\displaystyle \sum_{m = 1}^{\infty} a_{n,m}\right) = s\\
      \end{align*}
      
      \item Si \(\displaystyle \sum_{n, m} a_{n,m} = \infty\), es decir, si es divergente a \(+\infty\).\vspace{2ex}
      
      Si para algún \(\left\{
        \begin{array}{ll}
          n \in \mathbb{N} & \sum_{m = 1}^{\infty} a_{n,m} = + \infty\\[1ex]
          m \in \mathbb{N} & \sum_{n = 1}^{\infty} a_{n,m} = + \infty
        \end{array}
      \right.\) entonces trivialmente:
      \begin{align*}
        \displaystyle \sum_{n = 1}^{\infty} \displaystyle \sum_{m = 1}^{\infty} a_{n,m} = + \infty & \qquad \mbox{ o } \qquad \displaystyle \sum_{m = 1}^{\infty} \displaystyle \sum_{n = 1}^{\infty} a_{n,m} = + \infty \quad \mbox{respectivamente}
      \end{align*}
      Supongamos que \(\left\{
        \begin{array}{l}
          \sum_{m = 1}^{\infty} a_{n, m} < \infty \quad \forall n\\[1ex]
          \sum_{n = 1}^{\infty} a_{n, m} < \infty \quad \forall m
        \end{array}
      \right.\) \vspace{2ex}
      
      \hyperref[def:serie_doble_divergente]{Por divergencia} \(\forall K \in \mathbb{R}, \quad \exists n_0 \in \mathbb{N}\) tal que \(\forall n,m \geq n_0\) se cumple que:
      \begin{align*}
        \displaystyle \sum_{\substack{1 \leq i \leq n\\1 \leq j \leq m}} a_{i,j} > K
      \end{align*}
      Y aplicando pasos similares al caso de la convergencia, tenemos:
      \begin{align*}
        \displaystyle \sum_{j = 1}^{m} \displaystyle \sum_{i = 1}^{n} a_{i,j} > K \implies 
      \end{align*}
      \begin{align*}
        \implies & K \leq \lim_n \displaystyle \sum_{j = 1}^{m} \displaystyle \sum_{i = 1}^{n} a_{i,j} = \displaystyle \sum_{j = 1}^{m} \lim_n \displaystyle \sum_{i = 1}^{n}a_{i,j} = \displaystyle \sum_{j = 1}^{m} \left(\displaystyle \sum_{i = 1}^{\infty} a_{i,j}\right) \xRightarrow[\forall m \geq n_0]{\text{Válido}} \\[4ex]
        \implies & K \leq \lim_m \displaystyle \sum_{j = 1}^{m} \left(\displaystyle \sum_{i = 1}^{\infty} a_{i, j}\right) = \dots = \displaystyle \sum_{j = 1}^{\infty} \left(\displaystyle \sum_{i = 1}^{\infty} a_{i,j}\right) \geq 0 \xRightarrow[\forall K \in \mathbb{R}]{\text{Válido}} \\[2ex]
        \implies & \displaystyle \sum_{j = 1}^{\infty} \left(\displaystyle \sum_{i = 1}^{\infty} a_{i,j}\right) = + \infty
      \end{align*}
      Y de forma análoga, tenemos:
      \begin{align*}
        \displaystyle \sum_{i = 1}^{n} \displaystyle \sum_{j = 1}^{m} a_{i,j} > K \implies
      \end{align*}
      \begin{align*}
        \implies & K \leq \lim_m \displaystyle \sum_{i = 1}^{n} \displaystyle \sum_{j = 1}^{m} a_{i,j} = \dots = \displaystyle \sum_{i = 1}^{n} \left(\displaystyle \sum_{j = 1}^{\infty} a_{i,j}\right) \xRightarrow[\forall n \geq n_0]{\text{Válido}} \\[4ex]
        \implies & K \leq \lim_n \displaystyle \sum_{i = 1}^{n} \left(\displaystyle \sum_{j = 1}^{\infty} a_{i,j}\right) = \dots = \displaystyle \sum_{i = 1}^{\infty} \left(\displaystyle \sum_{j = 1}^{\infty} a_{i,j}\right) \geq 0 \xRightarrow[\forall K \in \mathbb{R}]{\text{Válido}} \\[2ex]
        \implies & \displaystyle \sum_{i = 1}^{\infty} \left(\displaystyle \sum_{j = 1}^{\infty} a_{i,j}\right) = + \infty
      \end{align*}
      Por tanto, si \(\displaystyle \sum_{n,m} a_{n,m} = + \infty\) entonces:
      \begin{align*}
        \displaystyle \sum_{n,m} a_{n,m} = \displaystyle \sum_{m = 1}^{\infty} \left(\displaystyle \sum_{n = 1}^{\infty} a_{n,m}\right) = \displaystyle \sum_{n = 1}^{\infty} \left(\displaystyle \sum_{m = 1}^{\infty} a_{n,m}\right) = + \infty\\
      \end{align*}
    \end{itemize}
  \end{enumerate}
  Y con esto, queda demostrado que:
  \begin{align*}
    \displaystyle \sum_{n, m} a_{n,m} = \displaystyle \sum_{n = 1}^{\infty} a_{g(n)} = \displaystyle \sum_{m = 1}^{\infty} \displaystyle \sum_{n = 1}^{\infty} a_{n,m} = \displaystyle \sum_{n = 1}^{\infty} \displaystyle \sum_{m = 1}^{\infty} a_{n,m} \in [0, + \infty]\\
  \end{align*}
\end{dem_box}
\end{document}