\documentclass[11pt,a4paper]{article}

\usepackage[spanish]{babel}
\usepackage{amsmath,amsfonts, amssymb, mathtools, bbm} % Podemos añadir amssymb, amsthm o bm
\usepackage{graphicx, tikz, xparse}
\usepackage[top=2cm,bottom=2cm,left=3cm,right=3cm,marginparwidth=1.75cm]{geometry} % Este paquete permite modificar los márgenes del documento
\usepackage[colorlinks=true, allcolors=blue]{hyperref} % Se indica que los hipervínculos van todos en azul
\usepackage{setspace}
\usepackage{xcolor, tcolorbox}
\usepackage{cancel} %tachar cosas
\tcbuselibrary{breakable}
\usepackage{hyperref}
\usepackage{titlesec}
\usepackage{cancel}
\usepackage{pgfplots}  % Para graficar funciones en tikz
\usepackage{background}
\usetikzlibrary{arrows.meta}
\usepackage[bottom]{footmisc}
\usepackage{extarrows}
\allowdisplaybreaks

% Definir la marca de agua
\backgroundsetup{
  position=current page.west,
  angle=90,
  nodeanchor=west,
  vshift=-1cm,
  hshift=-5.5cm,
  color=gray,
  scale=1,
  contents={\textsf{Created by Diego Díaz Mendaña $|$ Licensed under CC BY-NC-SA 4.0}}
}


\graphicspath{ {images/}}

%Colores
\definecolor{blanco}{HTML}{FFFFFF}
\definecolor{negro}{HTML}{000000}
\definecolor{azulSuave}{HTML}{6ac9d5}
\definecolor{naranjaSuave}{HTML}{d5956a}
\definecolor{verdeSuave}{HTML}{6ad578}
\definecolor{magenta}{HTML}{FF00FF}
\definecolor{dorado}{HTML}{ad8a1f}

\newtcolorbox{dem_box}[1]{
before=\par\smallskip\centering,
colframe=azulSuave!70,
colback=white,
fonttitle=\bfseries,
coltitle=negro,
title=#1,
flushleft title,
width=1\linewidth,
breakable = true
}

\newtcolorbox{ejem_box}[1]{
before=\par\smallskip\centering,
colframe=verdeSuave!70,
colback=white,
fonttitle=\bfseries,
coltitle=negro,
title=#1,
flushleft title,
width=1\linewidth,
breakable = true
}

\newtcolorbox{ej_box}[1]{
before=\par\smallskip\centering,
colframe=naranjaSuave!70,
colback=white,
fonttitle=\bfseries,
coltitle=negro,
title=#1,
flushleft title,
width=1\linewidth,
breakable = true
}


\setstretch{1.2}
\decimalpoint

\title{\textbf{TEMA 4}: Funciones medibles}
\author{Diego Díaz Mendaña}
%\date{Fecha}

\begin{document}
\NoBgThispage 
\maketitle
\vspace{20ex}
\section*{Disclaimer}
Estos apuntes son un resumen basado en el material proporcionado en la asignatura de ``Análisis III'' de la Universidad de Oviedo. Han sido escritos mediante las definiciones y demostraciones explicadas en clase. Todo el contenido ha sido organizado y formulado con fines educativos y no comerciales.\vspace{2ex}


\vspace{12ex}
\section*{Licencia de uso}
Apuntes Análisis III - Tema 4 © 2025 by Diego Díaz Mendaña is licensed under CC BY-NC-SA 4.0. To view a copy of this license, visit \href{https://creativecommons.org/licenses/by-nc-sa/4.0/}{https://creativecommons.org/licenses/by-nc-sa/4.0/}

\newpage

\hypersetup{linkcolor=black}
\tableofcontents
\hypersetup{linkcolor=blue}
\newpage

\section{Funciones medibles}
\subsection{Función medible. Definición}\label{def:funcion_medible}
\noindent Sea \((X, \Sigma, \mu)\) espacio de medida y \(f \colon X \to \overline{\mathbb{R}}\) función, se dice \textbf{medible} (o \(\mu\)-medible) si:
\begin{align*}\tag{1}
  \forall \alpha \in \mathbb{R} \quad \mbox{ se cumple } \quad \left\{x \in X : f(x) < \alpha\right\} \in \Sigma
\end{align*}
O, equivalentemente:
\begin{align*}\tag{2}
  \forall \alpha \in \mathbb{R} \quad \mbox{ se cumple } \quad \left\{x \in X : f(x) \leq \alpha\right\} \in \Sigma
\end{align*}
\begin{align*}\tag{3}
  \forall \alpha \in \mathbb{R} \quad \mbox{ se cumple } \quad \left\{x \in X : f(x) > \alpha\right\} \in \Sigma
\end{align*}
\begin{align*}\tag{4}
  \forall \alpha \in \mathbb{R} \quad \mbox{ se cumple } \quad \left\{x \in X : f(x) \geq \alpha\right\} \in \Sigma
\end{align*}
También se puede considerar \(\alpha \in \overline{\mathbb{R}}\) en cualquiera de las definiciones anteriores (se denotarán como (1'), (2'), (3') y (4')).\vspace{2ex}

\begin{dem_box}{Demostración}
  Veamos que son equivalentes:
  \begin{itemize}
    \item[\(1 \Rightarrow 2\))] \textit{Ver que \(\{x : f(x) \leq \alpha\} \in \Sigma\)}: empleando una aproximación por sucesiones decrecientes. Dado \(\alpha \in \mathbb{R}\) podemos notar que la condición \(f(x) \leq \alpha\) es equivalente a que \(f(x) < \alpha + \varepsilon\) para todo \(\varepsilon > 0\) (tomando \(\varepsilon = \frac{1}{n}\)). Por tanto:
    \begin{align*}
      \left\{x : f(x) \leq \alpha\right\} = \displaystyle \bigcap_{n = 1}^{\infty} \left\{x : f(x) < \alpha + \frac{1}{n}\right\} \in \Sigma
    \end{align*}
    que se cumple ya que:
    \begin{itemize}
      \item Cada conjunto \(\left\{x : f(x) < \alpha + \frac{1}{n}\right\} \in \Sigma\) por (1).
      \item \(\Sigma\) es cerrada para intersecciones numerables al ser una \(\sigma\)-álgebra.
    \end{itemize}
    \item[\(2 \Rightarrow 3\))] \textit{Ver que \(\{x : f(x) > \alpha\} \in \Sigma\)}: empleando que las desigualdades estrictas y no estrictas son complementarias cuando se considera el mismo valor. Dado \(\alpha \in \mathbb{R}\) entonces:
    \begin{align*}
      \left\{x : f(x) > \alpha\right\} = \{\underbracket{x :f(x) \leq \alpha}_{\in \Sigma}\}^c \in \Sigma
    \end{align*}
    \item[\(3 \Rightarrow 4\))] \textit{Ver que \(\{x : f(x) \geq \alpha\} \in \Sigma\)}: de forma análoga a \([1 \Rightarrow 2]\) empleando sucesiones crecientes, dado \(\alpha \in \mathbb{R}\) entonces:
    \begin{align*}
      \left\{x : f(x) \geq \alpha\right\} = \displaystyle \bigcap_{n = 1}^{\infty} \underbracket{\left\{x : f(x) > \alpha - \frac{1}{n}\right\}}_{\in \Sigma \; \text{ por (3)}} \in \Sigma
    \end{align*}
    \item[\(4 \Rightarrow 1\))] \textit{Ver que \(\{x : f(x) < \alpha\} \in \Sigma\)}: de forma análoga a \([2 \Rightarrow 3]\) empleando el complemento, dado \(\alpha \in \mathbb{R}\) entonces:
    \begin{align*}
      \left\{x : f(x) < \alpha\right\} = \{\underbracket{x : f(x) \geq \alpha}_{\in \Sigma}\}^c \in \Sigma
    \end{align*}
  \end{itemize}
\end{dem_box}
\begin{dem_box}{Demostración}
  Veamos que son equivalentes las definiciones con \(\alpha \in \overline{\mathbb{R}}\) y las definiciones con \(\alpha \in \mathbb{R}\):
  \begin{itemize} \leftskip -12pt
    \item[1)] \textit{Ver que \((1 \Leftrightarrow 1')\)}
    \begin{itemize} \leftskip -15pt
      \item[\(\Rightarrow\))] Trivial ya que \(\mathbb{R} \subset \overline{\mathbb{R}}\).
      \item[\(\Leftarrow\))] Dado \(\alpha = + \infty\) entonces:
      \begin{align*}
        \left\{x \in X: f(x) < \infty\right\} = \displaystyle \bigcup_{n = 1}^{\infty} \underbracket{ \left\{x \in X: f(x) < n\right\}}_{\in \Sigma} \in \Sigma
      \end{align*}
      Dado \(\alpha = - \infty\) entonces\footnote{Este también existe, como Teruel}:
      \begin{align*}
        \left\{x \in X \colon f(x) < - \infty\right\} = \emptyset \in \Sigma
      \end{align*}
    \end{itemize}
    Y ya estaría\footnote{Pues sí que va a ser como Teruel, vacío.}.
  \item[2)] \textit{Ver que \((2 \Leftrightarrow 2')\)}
    \begin{itemize} \leftskip -15pt
      \item[\(\Rightarrow)\))] Trivial ya que \(\mathbb{R} \subset \overline{\mathbb{R}}\).
      \item[\(\Leftarrow)\))] Dado \(\alpha = + \infty\) entonces:
      \begin{align*}
        \left\{x : f(x) \leq + \infty\right\} = X \in \Sigma
      \end{align*}
      \begin{align*}
        \left\{x : f(x) \leq - \infty\right\} = \displaystyle \bigcap_{n = 1}^{\infty} \underbracket{\left\{x: f(x) \leq - n\right\}}_{\in \Sigma} \in \Sigma
      \end{align*}
    \end{itemize}
  \item[3)] \textit{Ver que \((3 \Leftrightarrow 3')\)}
    \begin{itemize} \leftskip -15pt
      \item[\(\Rightarrow\))] Trivial ya que \(\mathbb{R} \subset \overline{\mathbb{R}}\).
      \item[\(\Leftarrow\))] Dado \(\alpha = + \infty\) entonces:
      \begin{align*}
        \left\{x : f(x) > + \infty\right\} = \emptyset \in \Sigma
      \end{align*}
      Dado \(\alpha = - \infty\) entonces:
      \begin{align*}
        \left\{x : f(x) > - \infty\right\} = \displaystyle \bigcup_{n = 1}^{\infty} \underbracket{\left\{x : f(x) > - n\right\}}_{\in \Sigma} \in \Sigma
      \end{align*}
    \end{itemize}
    \vspace{4ex}

  \item[4)] \textit{Ver que \((4 \Leftrightarrow 4')\)}
    \begin{itemize} \leftskip -15pt
      \item[\(\Rightarrow\))] Trivial ya que \(\mathbb{R} \subset \overline{\mathbb{R}}\).
      \item[\(\Leftarrow\))] Dado \(\alpha = + \infty\) entonces:
      \begin{align*}
        \left\{x : f(x) \geq + \infty\right\} = \displaystyle \bigcap_{n = 1}^{\infty} \underbracket{\left\{x : f(x) \geq n\right\}}_{\in \Sigma} \in \Sigma
      \end{align*}
      Dado \(\alpha = - \infty\) entonces:
      \begin{align*}
        \left\{x : f(x) \geq - \infty\right\} = X \in \Sigma
      \end{align*}
    \end{itemize}
  \end{itemize}
\end{dem_box}
\vspace{5ex}

\subsection{Funciones medibles. Ejemplos}
\noindent Se tienen los siguientes ejemplos de funciones medibles:
\begin{enumerate}
  \item Toda función \(f : \mathbb{N} \longrightarrow \overline{\mathbb{R}}\) es \((\mathbb{N}, \mathcal{P}(\mathbb{N}), m)\)-medible, donde \(m\) es la medida de conteo.
  \begin{dem_box}{Demostración}
    Dado \(\alpha \in \mathbb{R}\), se tiene que:
    \begin{align*}
      \left\{n \in \mathbb{N} : f(n) < \alpha\right\} = \displaystyle \bigcup_{\substack{ k \in \mathbb{N} \\ f(k) < \alpha}} \{k\} \in \mathcal{P}(\mathbb{N})
    \end{align*}
    Por lo tanto, \(f\) es medible.
  \end{dem_box}
  \vspace{3ex}

  \item Sea \(f \colon \mathbb{R}^N \to \mathbb{R}\) continua, es (\(\mathbb{R}^N, \mathcal{M}_N, \mu_N\))-medible y \((\mathbb{R}^N, \mathcal{B}_N, \mu_N)\)-medible.
  \begin{dem_box}{Demostración}
    El intervalo \([ - \infty, \alpha)\) es abierto en \(\overline{\mathbb{R}}\) con la topología usual ya que su complemento es \([\alpha, \infty]\) que es cerrado. Dado que \(f\) es continua, la preimagen de un abierto es un abierto, es decir:
    \begin{align*}
      A_\alpha \coloneq f^{ - 1}([ - \infty, \alpha)) = \left\{x \in \mathbb{R}^N : f(x) < \alpha\right\} \quad \mbox{ es abierto}
    \end{align*}
    Como todo abierto de \(\mathbb{R}^N\) es de Borel y además \(\mathcal{B}_N \subseteq \mathcal{M}_N\) entonces:
    \begin{align*}
      A_\alpha \in \mathcal{B}_N \subseteq \mathcal{M}_N
    \end{align*}
  \end{dem_box}
  \vspace{8ex}

  \item Dado un espacio de medida \((X, \Sigma, \mu)\) y \(A \in \Sigma\) y \(f \colon X \to \overline{\mathbb{R}}\) medible entonces \(f_{|A}\) es medible respecto del espacio de medida \((A, \Sigma_A, \mu_{|_{\Sigma(A)}})\), donde:
  \begin{itemize}
    \item \(\Sigma_A = \left\{A \cap E: E \in \Sigma\right\}\)
    \item \(\mu_{|_{\Sigma(A)}}\) es la restricción de \(\mu\) a \(\Sigma_A\)
  \end{itemize}
  \begin{dem_box}{Demostración}
    Como \(f_{|_A}(x) = f(x)\) para \(x \in A\), podemos reescribir:
    \begin{align*}
      \left\{x \in A: f_{|_A}(x) < \alpha\right\} = A \cap \left\{x \in X : f(x) < \alpha\right\}
    \end{align*}
    Dado que \(f\) es medible, entonces \(\left\{x \in X : f(x) < \alpha\right\} \in \Sigma\) y por lo tanto:
    \begin{align*}
      \left\{x \in A: f_{|_A}(x) < \alpha\right\} \in \Sigma_A
    \end{align*}
  \end{dem_box}
\end{enumerate}
\vspace{3ex}

\subsection{Caracterización de funciones medibles. Proposición}\label{prop:caracterizacion_funciones_medibles}
\noindent Dada una función \(f : \to \overline{\mathbb{R}} \) y un espacio de medida \((X, \Sigma, \mu)\) se cumple que:
\begin{align*}
  f \mbox{ es } \mu \mbox{-medible} \iff \forall O \subseteq \overline{\mathbb{R}} \mbox{ abierto,} \quad f^{ - 1}( O) \in \Sigma
\end{align*}
\begin{ej_box}{Nota}
  En \(\overline{\mathbb{R}}\) se considera la topología usual extendida, que tiene como base los intervalos abiertos \((a, b)\) y, además, los intervalos \((a, \infty]\) y \([ - \infty, b)\). Por tanto, cualquier abierto \(O \subseteq \overline{\mathbb{R}}\) se puede escribir como una unión (a lo sumo numerable) de intervalos abiertos disjuntos dos a dos de las formas previamente indicadas.
\end{ej_box}
\begin{dem_box}{Demostración}
  \begin{itemize} \leftskip -10pt
    \item[\(\Leftarrow\))] Si \(f^{ - 1}(O) \in \Sigma\) para todo abierto \(O \subseteq \overline{\mathbb{R}}\) entonces, en particular, para cada \(a \in \mathbb{R}\) el intervalo \([ - \infty, a)\) es abierto en \(\overline{\mathbb{R}}\) por lo que:
    \begin{align*}
      \left\{x \in X : f(x) < \alpha\right\} = f^{ - 1}([ - \infty, \alpha)) \in \Sigma
    \end{align*}
    Por lo tanto, \(f\) es medible por \hyperref[def:funcion_medible]{definición de función medible (1)}.
    
    \item[\(\Rightarrow\))] Sea \(f\) medible y \(O \subseteq \overline{\mathbb{R}}\) abierto, como todo abierto de \(\overline{\mathbb{R}}\) es unión numerable de intervalos abiertos disjuntos dos a dos, es decir:
    \begin{align*}
      f^{ - 1}(O) = f^{ - 1}\left(\bigcup_{i \in I} J_i\right) = \bigcup_{i \in I} \underbracket{f^{ - 1}(J_i)}_{\in \Sigma \; ?}
    \end{align*}
    Basta ver que \(f^{ - 1}(J_i) \in \Sigma\) para todo intervalo abierto \(J_i\) de \(\overline{\mathbb{R}}\).\vspace{2ex}
    
    Todo intervalo abierto \(J_i\) de \(\overline{\mathbb{R}}\) es de la forma:
    \begin{align*}
      J_i = (\alpha, \beta) = [ - \infty, \beta) \cap (\alpha, + \infty]
    \end{align*}
    Por lo tanto:
    \begin{align*}
      f^{ - 1}(J_i) = \underbracket{f^{ - 1}([ - \infty, \beta))}_{\in \Sigma} \cap \underbracket{f^{ - 1}((\alpha, + \infty])}_{\in \Sigma} \in \Sigma
    \end{align*}
    ya que \(f\) es medible por \hyperref[def:funcion_medible]{definición de función medible (1') y (3')}.
  \end{itemize}
\end{dem_box}
\vspace{3ex}

\subsection{Proposición}
Dado un espacio de medida \((X, \Sigma, \mu)\) y \((f_n: X \to \overline{\mathbb{R}})_{n \in \mathbb{N}}\) una sucesión de funciones medibles entonces se cumplen los siguientes enunciados:
\begin{enumerate}
  \item[i)] \(\max \left\{f_1, \dots , f_n\right\}\) es medible para todo \(n \in \mathbb{N}\).
  \item[ii)] \(\min \left\{f_1, \dots , f_n\right\}\) es medible para todo \(n \in \mathbb{N}\).
  \item[iii)] \(\sup \left\{f_1, f_2, \dots \right\}\) es medible.
  \item[iv)] \(\inf \left\{f_1, f_2, \dots \right\}\) es medible.
  \item[v)] \(\displaystyle \liminf_{n \to \infty} f_n\) es medible.
  \item[vi)] \(\displaystyle \limsup_{n \to \infty} f_n\) es medible.
  \item[vii)] Si \(\exists \displaystyle \lim_{n \to \infty} f_n(x) \in \overline{\mathbb{R}}\) en todo \(x \in X\) entonces \(\displaystyle \lim_{n \to \infty} f_n\) es medible.
\end{enumerate}
\vspace{2ex}

\begin{dem_box}{Demostración}
  Dado \(\alpha \in \mathbb{R}\) cualquiera:
  \begin{enumerate} \leftskip -10pt
    \item[i)] Como el máximo es menor que \(\alpha\) si y solo si todos los elementos son menores que \(\alpha\), se tiene que:
    \begin{align*}
      \left\{x : \max \left\{f_1(x), \dots , f_n(x)\right\} < \alpha\right\} = \bigcap_{i = 1}^{n} \underbracket{\left\{x : f_i(x) < \alpha\right\}}_{\in \Sigma} \in \Sigma
    \end{align*}
    \item[ii)] Como el mínimo es menor que \(\alpha\) si y solo si al menos uno de los elementos es menor que \(\alpha\), se tiene que:
    \begin{align*}
      \left\{x : \min \left\{f_1(x), \dots , f_n(x)\right\} < \alpha\right\} = \bigcup_{i = 1}^{n} \underbracket{\left\{x : f_i(x) < \alpha\right\}}_{\in \Sigma} \in \Sigma
    \end{align*}

    \vspace{20ex}

    \item[iii)] Como el supremo es \(\leq \alpha\) si y solo si todos los elementos son \(\leq \alpha\), se tiene que:
    \begin{align*}
      \left\{x : \sup_{k \in \mathbb{N}} f_k(x) \leq  \alpha\right\} = \bigcap_{n = 1}^{\infty} \underbracket{\left\{x : f_n(x) \leq  \alpha\right\}}_{\in \Sigma} \in \Sigma
    \end{align*}
    \item[iv)] Como el ínfimo es \(\geq \alpha\) si y solo si todos los elementos son \(\geq \alpha\), se tiene que:
    \begin{align*}
      \left\{x : \inf_{k \in \mathbb{N}} f_k(x) \geq  \alpha\right\} = \bigcap_{n = 1}^{\infty} \underbracket{\left\{x : f_n(x) \geq  \alpha\right\}}_{\in \Sigma} \in \Sigma
    \end{align*}
    \item[v)] Definimos \(g_n(x) = \displaystyle \inf_{k \geq n} f_k(x)\) y notamos que:
    \begin{align*}
      \liminf_{k \to \infty} f_k(x) = \sup_{n \in \mathbb{N}} \left(\inf_{k \geq n} f_k(x)\right) = \sup_{n \in \mathbb{N}} g_n(x)
    \end{align*}
    ya que \(g_n\) es medible por iv) (al ser ínfimo de funciones medibles) y aplicando iii) sobre el supremo de las \(g_n\) se concluye que \(\liminf_{k \to \infty} f_k\) es medible.
    \item[vi)] De forma análoga a v) definimos \(h_n(x) = \displaystyle \sup_{k \geq n} f_k(x)\) y notamos que:
    \begin{align*}
      \limsup_{k \to \infty} f_k(x) = \inf_{n \in \mathbb{N}} \left(\sup_{k \geq n} f_k(x)\right) = \inf_{n \in \mathbb{N}} h_n(x)
    \end{align*}
    ya que \(h_n\) es medible por iii) (al ser supremo de funciones medibles) y aplicando iv) sobre el ínfimo de las \(h_n\) se concluye que \(\limsup_{k \to \infty} f_k\) es medible.

    \item[vii)] Si \(\exists \lim_{n \to \infty} f_n(x) \in \overline{\mathbb{R}}\) en todo \(x \in X\) entonces:
    \begin{align*}
      \lim_{n \to \infty} f_n(x) = \liminf_{n \to \infty} f_n(x) = \limsup_{n \to \infty} f_n(x)
    \end{align*}
    Por lo tanto, \(\lim_n f_n\) es medible (por v) y vi)).
  \end{enumerate}
\end{dem_box}
\vspace{3ex}

\begin{ej_box}{Observación}
  Respecto de vii) si tenemos una sucesión de funciones medibles \((f_n)_{n \in \mathbb{N}}\) y definimos:
  \begin{align*}
    A \coloneq \left\{x \in X : \nexists \lim_{n \to \infty} f_n(x) \in \overline{\mathbb{R}}\right\} \neq \emptyset 
  \end{align*}
  Es decir, el conjunto de puntos donde la sucesión no converge (notar que no converger en \(\overline{\mathbb{R}}\) implica que no converge ni a un real ni a \(\pm \infty\)).\vspace{2ex}

  Entonces, se tiene que \(A\) es medible y podemos considerar el espacio de medida donde el límite sí existe:
  \begin{align*}
    (X \setminus A, \Sigma_{X \setminus A}, \mu_{|_{\Sigma(X \setminus A)}})
  \end{align*}
\end{ej_box}
\vspace{3ex}

\subsection{Operaciones con funciones medibles. Proposición}
Sean \(f, g\) funciones medibles respecto de un espacio de medida \((X, \Sigma, \mu)\) se cumplen los siguientes enunciados:
\begin{enumerate}
  \item[i)] \(f + g\) es medible (si \(f + g\) está bien definida en todo \(X\))
  \item[ii)] \(\forall \lambda \in \mathbb{R}\) se tiene que \(\lambda f\) es medible (si \(\lambda f\) está bien definida en todo \(X\))
  \item[iii)] \(f^2\) es medible
  \item[iv)] \(fg\) es medible (si \(fg\) está bien definida en todo \(X\))
\end{enumerate}
\begin{dem_box}{Demostración}
  \begin{enumerate} \leftskip -15pt
    \item[i)] Fijamos \(\alpha \in \mathbb{R}\) cualquiera. Entonces:
    \begin{align*}
      \left\{x : f(x) + g(x) < \alpha\right\} & = \left\{x : f(x) < \alpha - g(x)\right\} \xlongequal[]{\text{\tiny \(\mathbb{Q}\) denso en \(\mathbb{R}\)}} \\[2ex]
      & = \displaystyle \bigcup_{q \in \mathbb{Q}} \left\{x : f(x) < q < \alpha - g(x)\right\} = \\[2ex]
      & = \displaystyle \bigcup_{q \in \mathbb{Q}} \left\{ \left\{x : f(x) < q\right\} \cap \left\{x : q < \alpha - g(x)\right\}\right\} = \\[2ex]
      & = \displaystyle \bigcup_{q \in \mathbb{Q}} \underbracket{\{ \underbracket{\left\{x : f(x) < q\right\} }_{\in \Sigma\text{ (\(f\) medible)}}\cap \underbracket{\left\{x : g(x) < \alpha - q\right\}}_{\in \Sigma\text{ (\(g\) medible)}}\}}_{\in \Sigma} \in \Sigma
    \end{align*}
    \item[ii)] Se dan dos casos:
    \begin{itemize}  \leftskip -18pt
      \item Si \(\lambda = 0\) entonces \(\lambda f(x) = 0\) es medible.
      \item Si \(\lambda > 0\) entonces, dado \(\alpha \in \mathbb{R}\) se tiene:
      \begin{align*}
        \left\{x : \lambda f(x) < \alpha\right\} = \left\{x : f(x) < \frac{\alpha}{\lambda}\right\} \overset{\text{\tiny \(f\) medible}}{\in} \Sigma
      \end{align*}
      \item Si \(\lambda < 0\), análogamente, dado \(\alpha \in \mathbb{R}\) se tiene:
      \begin{align*}
        \left\{x : \lambda f(x) < \alpha\right\} = \left\{x : f(x) > \frac{\alpha}{\lambda}\right\} \overset{\text{\tiny \(f\) medible}}{\in} \Sigma
      \end{align*}
    \end{itemize}
    \item[iii)] Sea \(\alpha \in \mathbb{R}\) cualquiera. Entonces:
    \begin{itemize}  \leftskip -18pt
      \item Si \(\alpha < 0\) se tiene:
      \begin{align*}
        \left\{x : f(x)^2 \leq \alpha\right\} = \emptyset \in \Sigma
      \end{align*}
      \item Si \(\alpha \geq 0\) se tiene:
      \begin{align*}
        \left\{x : f(x)^2 \geq \alpha\right\} & = \left\{x : - \sqrt{\alpha} \leq f(x) \leq \sqrt{\alpha}\right\} =\\[2ex]
        & = \underbracket{\left\{x : f(x) \leq \sqrt{\alpha}\right\}}_{\text{\tiny \(f\) medible \( \implies \in \Sigma\)}} \cap \underbracket{\left\{x : f(x) \geq - \sqrt{\alpha}\right\}}_{\text{\tiny \(f\) medible \( \implies \in \Sigma\)}} \overset{\text{\tiny \(f\) medible}}{\in} \Sigma
      \end{align*}
    \end{itemize}
    \item[iv)] Tenemos que la siguiente fórmula es válida \(\forall x \in X\):
    \begin{align*}
      f \cdot g = \frac{1}{2}\left[(f + g)^2 - f^2 - g^2\right]
    \end{align*}
    excepto si \(x\) está en alguno de los siguientes conjuntos:
    \begin{align*}
      A_1 & \coloneq \left\{x : f(x) = \infty \mbox{ y } g(x) \in [ - \infty, 0)\right\}\\
      A_2 & \coloneq \left\{x : f(x) = \infty \mbox{ y } g(x) \in (0, \infty]\right\}\\
      A_3 & \coloneq \left\{x : f(x) = - \infty \mbox{ y } g(x) \in [ - \infty, 0)\right\}\\
      A_4 & \coloneq \left\{x : f(x) = - \infty \mbox{ y } g(x) \in (0, \infty]\right\}\\
      A_5 & \coloneq \left\{x : g(x) = \infty \mbox{ y } f(x) \in (- \infty, 0)\right\}\\
      A_6 & \coloneq \left\{x : g(x) = \infty \mbox{ y } f(x) \in (0, \infty]\right\}\\
      A_7 & \coloneq \left\{x : g(x) = - \infty \mbox{ y } f(x) \in (- \infty, 0)\right\}\\
      A_8 & \coloneq \left\{x : g(x) = - \infty \mbox{ y } f(x) \in (0, \infty]\right\}
    \end{align*}
    Se tiene que \(A_i \in \Sigma\) para todo \(i = 1, \dots, 8\) ya que se definen como la intersección de dos conjuntos medibles (por ser \(f\) y \(g\) medibles). Definimos además:
    \begin{align*}
      A_9 \coloneq X \setminus \displaystyle \bigcup_{i = 1}^{8} A_i \in \Sigma
    \end{align*}
    Notamos entonces que \(A_i\) son disjuntos dos a dos y que su unión es todo \(X\), así tenemos que:
    \begin{itemize} \leftskip -18pt
      \item Para \(i = 1, \dots, 8\) se cumple que \(f \cdot g|_{A_i} \equiv \infty\) o bien \( f \cdot g|_{A_i} \equiv - \infty\) luego:
      \begin{align*}
        f \cdot g|_{A_i} \mbox{ medible respecto } \left(A_i, \Sigma(A_i), \mu|_{\Sigma(A_i)}\right)
      \end{align*}
      \item Para \(i = 9\) se tiene:
      \begin{align*}
        f \cdot g|_{A_9} = \frac{1}{2}\left[(f + g)^2 - f^2 - g^2\right] \quad \mbox{ medible resp. } \left(A_9, \Sigma(A_9), \mu|_{\Sigma(A_9)}\right)
      \end{align*}
    \end{itemize}
    Así \(\forall i \in \{1, \dots, 9\}\), la función \(f\cdot g|_{A_i}\) es medible respecto de \(\left(A_i, \Sigma(A_i), \mu|_{\Sigma(A_i)}\right)\) luego, \(\forall i \in \{1, \dots, 9\}\) tenemos:
    \begin{align*}
      f\cdot g \cdot \mathcal{X}_{A_i} \quad \mbox{ es medible respecto } (X, \Sigma, \mu)
    \end{align*}
    Entonces:
    \begin{align*}
      \displaystyle \sum_{i = 1}^{9} f \cdot g \cdot \mathcal{X}_{A_i} = f \cdot g \quad \mbox{ medible respecto } (X, \Sigma, \mu)
    \end{align*}
    Y esto está bien definido en todo \(X\) ya que \(A_i \cap A_j = \emptyset\) si \(i \neq j\).
  \end{enumerate}
\end{dem_box}
\vspace{3ex}

\begin{ej_box}{Observación}
  En general, la composición de funciones medibles no tiene sentido salvo en casos especiales. Existen funciones medibles \(fg\) respecto de \((\mathbb{R}, \mathcal{M}_1, \mu_1)\) tales que \(g \circ f\) no es medible respecto de \((\mathbb{R}, \mathcal{M}_1, \mu_1)\).
\end{ej_box}
\vspace{3ex}

\subsection{Composición de funciones medibles y continuas. Proposición}\label{prop:composicion_funciones_medibles_continuas}
Sea \((X, \Sigma, \mu)\) espacio de medida, \(f: X \to \overline{\mathbb{R}}\) medible y \(g: A \to \overline{\mathbb{R}}\) continua con \(f(X) \subseteq A\) se tiene que:
\begin{align*}
  g \circ f \quad \mbox{ es medible}
\end{align*}
\begin{dem_box}{Demostración}
  Sea \(\alpha \in \mathbb{R}\) cualquiera entonces:
  \begin{align*}
    (g \circ f)^{ - 1}([ - \infty, \alpha)) = f^{ - 1}\left(g^{ - 1}([ - \infty, \alpha))\right)
  \end{align*}
  Como \([ - \infty, \alpha)\) es abierto de \(\overline{\mathbb{R}}\) y \(g\) continua, entonces:
  \begin{align*}
    g^{ - 1}([ - \infty, \alpha)) \subseteq A \mbox{ es abierto de } A
  \end{align*}
  Como \(f\) medible, por la \hyperref[prop:caracterizacion_funciones_medibles]{caracterízación de funciones medibles} se tiene que:
  \begin{align*}
    f^{ - 1}\left(g^{ - 1}([ - \infty, \alpha))\right) \in \Sigma
  \end{align*}
\end{dem_box}
\vspace{3ex}

\subsection{Corolario}
Sea \((X, \Sigma, \mu)\) espacio de medida y \(f: X \to \overline{\mathbb{R}}\) medible tal que \(f(x) \neq 0\) para todo \(x \in X\) entonces:
\begin{align*}
  \frac{1}{f} \quad \mbox{ es medible}
\end{align*}
\begin{dem_box}{Demostración}
  Definimos la función \(g\) como:
  \begin{align*}
    g : \overline{\mathbb{R}} \setminus \{0\} & \longrightarrow \overline{\mathbb{R}} \\[1ex]
    y & \longmapsto g(y) = \frac{1}{y}
  \end{align*}
  Como \(g\) continua en \(\overline{\mathbb{R}} \setminus \{0\}\) y \(f(X) \subseteq \overline{\mathbb{R}} \setminus \{0\}\), por la \hyperref[prop:composicion_funciones_medibles_continuas]{proposición anterior} se tiene:
  \begin{align*}
    g \circ f = \frac{1}{f} \quad \mbox{ es medible}
  \end{align*}
\end{dem_box}
\vspace{3ex}

\subsection{Proposición}
\noindent Sean las funciones:
\begin{align*}
  \left.
    \begin{array}{l}
      f \colon \mathbb{R} \to \mathbb{R} \mbox{ medible respecto de } (\mathbb{R}, \mathcal{B}_1, \mu_1) \\[2ex]
      g \colon \mathbb{R} \to \mathbb{R} \mbox{ medible respecto de } (\mathbb{R}, \mathcal{B}_1, \mu_1)
    \end{array}
  \right\} \implies g \circ f \mbox{ medible respecto de } (\mathbb{R}, \mathcal{B}_1, \mu_1)
\end{align*}
\begin{dem_box}{Demostración}
  Se sigue de los siguientes pasos:
  \begin{enumerate} \leftskip -10pt
    \item[1)] Se define \(\mathcal{A} \coloneq \left\{B \in \mathcal{B}_1 : f^{ - 1}(B) \in \mathcal{B}_1\right\}\). Es decir, \(\mathcal{A}\) es el conjunto de todos los borelianos \(B\) de \(\mathbb{R}\) cuya preimagen por \(f\) es también boreliano.\vspace{2ex}
    
    Notar que \(\tau_\mathbb{R} \subseteq \mathcal{A}\) ya que \(f\) es medible respecto de \((\mathbb{R}, \mathcal{B}_1, \mu_1)\) implica que:
    \begin{align*}
      B \in \tau_\mathbb{R} \implies f^{ - 1} (B) \in \mathcal{B}_1
    \end{align*}
    \item[2)] \textit{Ver que \(\mathcal{A}\) es una \(\sigma\)-álgebra:}
    \begin{itemize}
      \item[i)] \textit{Ver que \(\emptyset \in \mathcal{A}\)}: trivial ya que:
      \begin{align*}
        \emptyset \in \tau_\mathbb{R} \subseteq \mathcal{A}
      \end{align*}
      \item[ii)] \textit{Ver que es cerrada para uniones numerables}: Sea \(\{A_i\}_{i \in \mathbb{N}} \subseteq \mathcal{A}\) entonces:
      \begin{align*}
        f^{ - 1}\left(\bigcup_{i \in \mathbb{N}} A_i\right) = \bigcup_{i \in \mathbb{N}} \underbracket{f^{ - 1}(A_i)}_{\in \mathcal{B}_1} \in \mathcal{B}_1 \implies \bigcup_{i \in \mathbb{N}} A_i \in \mathcal{A}
      \end{align*}
      Donde hemos usado que \(A_i \in \mathcal{A} \implies f^{ - 1}(A_i) \in \mathcal{B}_1\).
      \item[iii)] \textit{Ver que es cerrada para complementos}: Sea \(A \in \mathcal{A}\) entonces:
      \begin{align*}
        f^{ - 1}(\mathbb{R} \setminus A) = \mathbb{R} \setminus \underbracket{f^{ - 1}(A)}_{\in \mathcal{B}_1} \in \mathcal{B}_1 \implies \mathbb{R} \setminus A \in \mathcal{A}
      \end{align*}
      Donde hemos usado que \(A \in \mathcal{A} \implies f^{ - 1}(A) \in \mathcal{B}_1\).
    \end{itemize}
    \item[3)] \textit{Ver que \(\mathcal{A} = \mathcal{B}_1\):}
    \begin{align*}
      \left.
        \begin{array}{l}
          \tau_\mathbb{R} \subseteq \mathcal{A} \subseteq \mathcal{B}_1\\
          \mathcal{A} \mbox{ es } \sigma\mbox{-álgebra} \implies \mathcal{B}_1 \subseteq \mathcal{A}
        \end{array}
      \right\} \implies A = \mathcal{B}_1
    \end{align*}
    \item[4)] \textit{Caracterización de medibilidad:} Sea \(f : \mathbb{R} \to\mathbb{R}\), por la \hyperref[prop:caracterizacion_funciones_medibles]{caracterización de funciones medibles} se tiene que \(f\) es medible respecto de \((\mathbb{R}, \mathcal{B}_1, \mu_1)\) si y solo si:
    \begin{align*}
      \forall B \in \mathcal{B}_1 : f^{ - 1}(B) \in \mathcal{B}_1
    \end{align*}
    Como \(\mathcal{A} = \mathcal{B}_1\) se tiene la equivalencia.
    \item[5)] \textit{Medibilidad de \(g \circ f\) respecto de \((\mathbb{R}, \mathcal{B}_1, \mu_1)\):} dado \(O \subseteq \mathbb{R}\) abierto cualquiera:
    \begin{align*}
      (g \circ f)^{ - 1}(O) = f^{ - 1}(\underbracket{g^{ - 1}(O)}_{\in \mathcal{B}_1 \text{ \(g \, \)medible}}) \in \mathcal{B}_1
    \end{align*}
    Por lo tanto, \(g \circ f\) es medible respecto de \((\mathbb{R}, \mathcal{B}_1, \mu_1)\).
  \end{enumerate}
\end{dem_box}
\vspace{3ex}

\subsection{Función característica. Definición}
\noindent Sea \(X\) conjunto y \(A \subseteq X\), se dice \textbf{función característica} o \textbf{indicadora} de \(A\) a la función:
\begin{align*}
  \mathcal{X}_A : X & \longrightarrow \{0,1\} \subseteq \mathbb{R} \\[1ex]
  x & \longmapsto \mathcal{X}_{A}(x) = \left\{
    \begin{array}{ll}
      1 & \mbox{ si } x \in A \\[1ex]
      0 & \mbox{ si } x \notin A
    \end{array}
  \right. \\
\end{align*}

\subsection{Medibilidad de la función característica. Proposición}
\noindent Si \((X, \Sigma, \mu)\) es un espacio de medida entonces se tiene:
\begin{align*}
  \mathcal{X}_A \mbox{ es } \mu\mbox{-medible} \iff A \in \Sigma\\
\end{align*}
\begin{dem_box}{Demostración}
  \begin{itemize} \leftskip -10pt
    \item[\(\Rightarrow\))] Sea \(\mathcal{X}_A\) medible, entonces \(\forall \alpha \in \mathbb{R}\) se tiene:
    \begin{align*}
      \left\{x : \mathcal{X}_A(x) < \alpha\right\} \in \Sigma
    \end{align*}
    En particular, tomando \(\alpha = 1\) se tiene:
    \begin{align*}
      \left\{x :  \mathcal{X}_A  < 1 \right\} = \left\{x : \mathcal{X}_A(x) = 0\right\} = X \setminus A \in \Sigma \implies A = (A^c)^c \in \Sigma
    \end{align*} \leftskip -10pt
    \item[\(\Leftarrow\))] Sea \(A \in \Sigma\), entonces:
    \begin{itemize}
      \setlength{\itemsep}{1.5ex}
      \item \textit{Si \(\alpha \leq 0\)} entonces: \(\{x : \mathcal{X}_A(x) < \alpha\} = \emptyset \in \Sigma\)
      \item \textit{Si \(0 < \alpha \leq 1\)} entonces: \(\{x : \mathcal{X}_A(x) < \alpha\} = X \setminus A \in \Sigma\)
      \item \textit{Si \(\alpha > 1\)} entonces: \(\{x : \mathcal{X}_A(x) < \alpha\} = X \in \Sigma\)
    \end{itemize}
    Por lo tanto, \(\mathcal{X}_A\) es medible.
  \end{itemize}
\end{dem_box}

\subsection{Función simple. Definición}
\noindent Se dice que \(s : X \to \overline{\mathbb{R}}\) es una \textbf{función simple} si su imagen es finita, es decir:
\begin{align*}
  s(X) \quad \mbox{ es un conjunto finito}
\end{align*}
O, equivalentemente, si existe \(n \in \mathbb{N}\) tal que:
\begin{align*}
  s(X) = \left\{\lambda_1, \dots, \lambda_n\right\} \quad \mbox{ con } \lambda_i \in \overline{\mathbb{R}}\\
\end{align*}

\subsection{Expresión canónica. Definición}
\noindent Sea \(s(X) = \left\{\lambda_1 < \lambda_2 < \dots < \lambda_n \right\}\) con \(n \in \mathbb{N}\), se dice \textbf{expresión canónica} de \(s\) a:
\begin{align*}
  s = \displaystyle \sum_{i = 1}^{n} \lambda_i \mathcal{X}_{A_i} \quad \mbox{ donde } A_i = s^{ - 1}(\{\lambda_i\}) = \left\{x \in X : s(x) = \lambda_i\right\}\\
\end{align*}

\begin{ejem_box}{Ejercicio}
  \textit{Demostrar que dado \((X, \Sigma, \mu)\) espacio de medida, \(s : X \to \overline{\mathbb{R}} \) es una función simple medible sii para todo \(i = 1, \dots , n\) se tiene que \(A_i \in \Sigma\).}\vspace{2ex}

  Para esto, consideramos los dos sentidos de la implicación:
  \begin{itemize}
    \item[\(\Rightarrow\))] Basta notar que para todo \(i = 1, \dots , n\):
    \begin{align*}
      A_i = s^{ - 1}(\lambda_i) = \underbracket{\left\{x : s(x) \leq \lambda_i\right\}}_{\in \Sigma} \cap \underbracket{\left\{x : s(x) \geq \lambda_i\right\}}_{\in \Sigma} \in \Sigma
    \end{align*}
    \item[\(\Leftarrow\))] Como los \(A_i\) son disjuntos dos a dos, el valor \(\sum_{i = 1}^n \lambda_i \mathcal{X}_{A_i}(x)\) está bien definida para todo \(x \in X\). Para todo \(i\) se tiene que \(A_i\) medible entonces:
    \begin{align*}
      \displaystyle \sum_{i = 1}^{n} \lambda_i \underbracket{\mathcal{X}_{A_i}(x)}_{\text{\tiny medible}} \quad \mbox{ es } \mbox{ medible}
    \end{align*}
  \end{itemize}
\end{ejem_box}
\vspace{3ex}

\subsection{Parte positiva y negativa de una función. Definición}
\noindent Sea \(f : X \to \overline{\mathbb{R}}\) una función, definimos la \textbf{parte positiva} como la función:
\begin{align*}
  f^{ + }(x) = \max \{f(x), 0\} \quad \forall x \in X
\end{align*}
\noindent y llamamos \textbf{parte negativa} a la función:
\begin{align*}
  f^{ - }(x) = - \left[f(x) - f^{ + }(x)\right] \quad \forall x \in X
\end{align*}
Así, se tiene que:
\begin{align*}
  f = f^{ + } - f^{ - } \quad \mbox{ y } \quad |f| = f^{ + } + f^{ - }\\
\end{align*}

\begin{ej_box}{Nota}
  Sea \(f: X \to \overline{\mathbb{R}}\), para cada \(x_0 \in X\) al menos uno de los valores \(f^{ + }(x_0)\) y \(f^{ - }(x_0)\) es no nulo.
\end{ej_box}
\vspace{2ex}

\begin{ej_box}{Nota}
  Si \(f : X \to \overline{\mathbb{R}}\) es medible entonces \(f^{ + }\) y \(f^{ - }\) son medibles.
\end{ej_box}
\vspace{2ex}


\subsection{Teorema}
Sea \((X, \Sigma, \mu)\) espacio de medida, para toda función medible \(f: X \to [0, \infty]\) existe una sucesión de funciones simples medibles \(\{s_n: X \to [0, \infty]\}_{n \in \mathbb{N}}\) tal que:
\begin{enumerate}
  \item \(\forall n \in \mathbb{N}, \quad \forall x \in X\) se tiene que \(0 \leq s_n(x) \leq s_{n + 1}(x) \leq f(x)\)
  \item \(\forall n \in \mathbb{N}, \quad \forall x \in X\) se tiene que \(\exists \displaystyle  \lim_{n \to \infty} s_n(x) = f(x)\)
\end{enumerate}
\begin{dem_box}{Demostración}
  Para cada \(n \in \mathbb{N}\) consideramos los conjuntos medibles:
  \begin{align*}
    E_n \coloneq f^{ - 1}([n, \infty]) \in \Sigma
  \end{align*}
  Y definimos los conjuntos disjuntos dos a dos:
  \begin{align*}
    E_{n, i} \coloneq f^{ - 1}\left(\left[\frac{i - 1}{2^n}, \frac{i}{2^n}\right)\right) \in \Sigma \quad \forall i = 1, \dots , n 2^n
  \end{align*}
  Definimos la sucesión de funciones simples medibles:
  \begin{align*}
    s_n(x) \coloneq \left[\displaystyle \sum_{i = 1}^{n2^n} \frac{i - 1}{2^n} \mathcal{X}_{E_{n, i}}(x)\right] + n \mathcal{X}_{E_n}(x) \quad \forall x \in X
  \end{align*}
  Para ver la convergencia, tomamos \(x_0 \in X\) cualquiera y entonces se dan dos casos:
  \begin{itemize} \leftskip -15pt
    \item Si \(f(x_0) = \infty\) entonces:
    \begin{align*}
      s_n(x_0) = n \xrightarrow[n \to \infty]{} \infty = f(x_0)
    \end{align*}
    \vspace{20ex}

    \item Si \(0 \leq f(x_0) < \infty\) entonces, sea \(m \in \mathbb{N}\) el natural más pequeño tal que \(f(x_0) < m\):
    \begin{center}
      \begin{tabular}{ccccc}
        \(s_1(x_0)\) & \(s_2(x_0)\) & \(\ldots\) & \(s_{m - 1}(x_0)\) & \( \leq s_m(x_0)\) \\
        \rotatebox{90}{\( = \)} & \rotatebox{90}{\( = \)} & & \rotatebox{90}{\( = \)} &  \\
        \(1\) & \( < 2\) & \( < \dots \) & \( < m - 1\) & 
      \end{tabular}
    \end{center}
    Y para \(n \geq m\) denotando \(i_n\) el único natural \(1 \leq i_n \leq n 2^n\) tal que:
    \begin{align*}
      f(x_0) \in \left[\frac{i_n - 1}{2^n}, \frac{i_n}{2^n}\right) = \left[\frac{2i_n - 2}{2^{n + 1}}, \frac{2i_n - 1}{2^{n + 1}}\right) \cup \left[\frac{2i_n - 1}{2^{n + 1}}, \frac{2i_n}{2^{n + 1}}\right)
    \end{align*}
    se tiene que:
    \begin{align*}
      s_n(x_0) = \frac{i_n - 1}{2^n} & \implies s_{n + 1}(x) \in \left\{\frac{2i_n - 2}{2^{n + 1}}, \frac{2i_n - 1}{2^{n + 1}} \right\} \implies \\[3ex]
      & \implies s_{n + 1}(x_0) \geq \frac{2i_n - 2}{2^{n + 1}} = s_n(x_0) \implies \\[3ex]
      & \implies \frac{i_n - 1}{2^n} = s_n(x_0) \leq f(x_0) < \frac{i_n}{2^n} = \frac{i_n - 1}{2^n} + \frac{1}{2^n} 
    \end{align*}
    Como \((s_n(x_0))_{n \in \mathbb{N}}\) es creciente entonces:
    \begin{align*}
      \exists \lim_{n \to \infty} s_n(x_0) = l
    \end{align*}
    Si \(n \to \infty\) entonces:
    \begin{align*}
      l \leq f(x_0) \leq l + 0
    \end{align*}
  \end{itemize}
\end{dem_box}
\vspace{3ex}

\begin{ej_box}{Observación}
  Si \(f\) es acotada (en \(\mathbb{R}\)), la convergencia de \(s_n\) a \(f\) en la demostración es uniforme.\footnote{Y entonces me podrías decir, ``¿y por qué no nos quedamos con Riemann?'' Y yo os diré: ``NO pedazo de brutos''.}
\end{ej_box}
\vspace{3ex}


\end{document}